%!TEX program = XeLaTeX 
% 下周: 东主1820, 2:30-4:30
%%%%%%%%%%%%%%%%%%%%%%%%%%%%%%%%%%%%%%%%%%%%%%%%%%%%%%%%%%%%%%%%%%%%%%%%%%%%%%%%%%%%5%%%%%%%%%%%%%%%
%  本文档可在安装了CTEX宏包, CTEX字体下的TEX系统运行,
%  访问http://www.ctex.org, 可以获得最新的宏包与字体安装包
%
%  请使用PDFLATEX对模板编译2次, 可得正确结果, 由于hyperref的设置中不支持DVI-PDF,
%  用LATEX编译时需要替换相应的命令, 详见相应注释.
%
% 文档是在原来李湛、何力同学的模板的基础上修改的, 主要包括以下几个地方:
%
%1.修正了原模板使用hyperref宏包中的设置, 使文档更加美观, 对设置作出了说明, 可以进一步修改
%2.修正了定理的样式, 原定理标题是黑体加粗, 现改为黑体, 原定理正文为倾斜楷体, 现改为楷体, 符合一般论文的格式
%3.对导言区的少部分命令修改, 删去了一些默认的重复的设置
%4.对模板的少部分正文进行充实
%5.对部分原来模板中的注释进行了修改, 删去了不必要的, 加入了一些中文的注释, 方便查阅
%
%  by 张越 Apr.12, 有问题请发送你的问题到:frank_melody@hotmail.com
%%%%%%%%%%%%%%%%%%%%%%%%%%%%%%%%%%%%%%%%%%%%%%%%%%%%%%%%%%%%%%%%%%%%
%%%%%%%%%%%%%%%%%%%%%%%%%%%%%%%%%%%%%%%%%%%%%%%%%%%%%%%%%%%%%%%%%%%%%%
% documentclass can be ctexart, ctexrep, ctexbook, 推荐使用模板中的CTEXREP
% cs4size - 默认的字体大. ∷
% punct - 对中文标点的位置(宽度)进行调整
% twoside - if you want to print on both side of the paper, or else you should omit this

\documentclass[notitlepage,cs4size,punct,oneside]{ctexrep}
% \documentclass[fleqn]{article}
\usepackage{subfig}

% \usepackage{ccmap}
% \usepackage{fontspec}
% \setmainfont{Times New Roman}
% \setmainfont{Centaur}
% % \setmainfont{serif}
% \setsansfont{Times New Roman}
% \setmonofont{Times New Roman}


% \usepackage{times}\usepackage[mtbold,mtpluscal,mtplusscr]{mathtime}

% default paper settings, change it according to your word
\usepackage[a4paper,hmargin={2.54cm,2.54cm},vmargin={3.17cm,3.17cm}]{geometry}

\usepackage{amsmath,amssymb,amsthm}
% \usepackage[fleqn]{amsmath}
% 公式编号的计数格式, 在章内计数
\numberwithin{equation}{section}

% set the abstract format, need abstract package

\usepackage[runin]{abstract}

%使用hyperref宏包, 对目录, 公式引用, 文献引用做超链接, 超链接方便电子版的阅读, 但不影响打印
% pdfborder对超链接的边框大小进行设置, 模板中默认边框大小为0
% colorlinks=true, 表示超链接对应的文字采用超链接边框的颜色, =false时保持原字体颜色
% linkcolor=blue, 设置超链接边框的颜色, 可以改为red,green等等.
% CJKbookmarks=true, 生成PDF中文书签,
% 非CTEX套装用户可能发现即便如此设置, 生成的PDF书签也是乱码, 需要用GBK2UNI.EXE解决
\usepackage[pdfborder={0 0 0},colorlinks=true,linkcolor=blue,CJKbookmarks=true]{hyperref}
%若要用LATEX编译, 请用下面的命令替代上述命令:
%\usepackage[dvipdfm,pdfborder={0 0 0},colorlinks=true,linkcolor=blue,CJKbookmarks=true]{hyperref}

\setlength{\absleftindent}{1.5cm} \setlength{\absrightindent}{1.5cm}
\setlength{\abstitleskip}{-\parindent}
\setlength{\absparindent}{0cm}

% Theorem style
\newtheoremstyle{mystyle}{3pt}{3pt}{\kaishu}{0cm}{\heiti}{}{1em}{}
\theoremstyle{mystyle}

\newtheorem{definition}{\hspace{2em}定义}[section]
% 如果没有章, 只有节, 把上面的[chapter]改成[section]
\newtheorem{theorem}[definition]{\hspace{2em}定理}
\newtheorem{axiom}[definition]{\hspace{2em}公理}
\newtheorem{lemma}[definition]{\hspace{2em}引理}
\newtheorem{proposition}[definition]{\hspace{2em}命题}
\newtheorem{corollary}[definition]{\hspace{2em}推论}
\newtheorem{remark}{\hspace{2em}注}[section]
%类似地定义其他“题头”. 这里“注”的编号与定义、定理等是分开的

\def\theequation{\arabic{section}.\arabic{equation}}
\setcounter{equation}{1}
\def\thedefinition{\arabic{section}.\arabic{definition}}

\newcommand{\nq}{\\[5pt]}
\newcommand{\nw}{\\[10pt]}

% title - \zihao{1} for size requirement \heiti for font family requirement

\title{{\zihao{1}\heiti{} 随机参数化Kalman滤波测试模型解的统计量}}

\author{卢川\\学号:13300180056\\专业:信息与计算科学}

\date{}

\usepackage{fontspec}
% \setmainfont{Times New Roman}
\setmainfont{Lucida Bright} 
% \setmainfont{Segoe Script}
\setsansfont{Lucida Bright} 
\setmonofont{Lucida Bright} 
% \setsansfont{Segoe Script}
% \setmonofont{Segoe UI }

%%%%%%%%%%%%%%%%%%%导言区设置完毕
%%%%%%%%%%%%%%%%%%%%%%%%%%%%%%%%%%%%%%%%%%%%%%%%%%%%%%%%%%%%%%%%%%%%%
\begin{document}

%Styles for chapters/section
%若要将章标题左对齐, 用下面这个语句替换相应的设置
%\CTEXsetup[nameformat={\raggedright\zihao{3}\bfseries},%
\CTEXsetup[nameformat={\zihao{3}\heiti},%
           titleformat={\zihao{3}},%
           beforeskip={0.8cm},afterskip={1.2cm}]{chapter}
\CTEXsetup[nameformat={\zihao{4}\bfseries},%
           titleformat={\zihao{4}},%
           name={第~,~节},number={\arabic{section}},%
           beforeskip={0.4cm},afterskip={0.4cm}]{section}
\CTEXsetup[format={\zihao{-4}\bfseries},%
           titleformat={\zihao{-4}},%
           number={\arabic{section}.\arabic{subsection}},%
           beforeskip={0.4cm},afterskip={0.4cm}]{subsection}
\CTEXsetup[format={\zihao{-4}\bfseries},%
           titleformat={\zihao{-4}},%
           number={\arabic{section}.\arabic{subsection}.\arabic{subsubsection}},%
           beforeskip={0.4cm},afterskip={0.4cm}]{subsubsection}
\CTEXoptions[abstractname={摘要:}]
\CTEXoptions[bibname={\heiti 参考文献}]

\renewcommand{\thepage}{\roman{page}}
\setcounter{page}{1}
\tableofcontents\clearpage

\maketitle\renewcommand{\thepage}{\arabic{page}}
\thispagestyle{empty}\setcounter{page}{1}
%%%  论文的页码从正文开始计数, 摘要页不显示页码
% 撰写论文的摘要


\newcommand{\Var}{\text{Var}}
\newcommand{\E}{\text{E}}
\newcommand{\Cov}{\text{Cov}}
\newcommand{\ii}{\text{i}}
\newcommand{\dotW}{\dot{W}}

\begin{abstract}
本文研究了随机参数化Kalman滤波中得到广泛应用的测试模型的解, 利用It\^o积分和特征函数补全了解的一阶和二阶统计量的计算过程, 并对理论结果进行了数值检验. \\

\noindent{\heiti 关键字: } 随机参数化Kalman滤波; 随机微分方程组; 统计量; It\^o积分
\end{abstract}

\newpage
\section{导言}
\subsection{问题简介}
滤波是从自然信号中得到最佳统计估计的过程. 在实际应用中, 研究者需要对自然界中的物理过程进行参数化表示, 而在此过程中, 不合适的采样率和不完全的物理认知会造成模型的系统误差, 最终使得对混乱信号的滤波结果受到较大的影响. 为此, B. Gershgorin等提出了随机参数化扩展Kalman滤波模型\cite{gershgorin2010improving}\cite{gershgorin2010test}, 以降低模型的系统误差, 改进滤波结果.

自然界中的信号可近似的用Langevin方程的解进行模拟\cite{langevin1908theorie}:
\[
\frac{\mathrm{d}u(t)}{\mathrm{d}t} = -\gamma(t)u(t) + \ii\omega u(t)+\sigma \dotW(t)+f(t),
\]
其中$W(t)$表示复值白噪声, $f(t)$表示外界的驱动力. 在\cite{gershgorin2010test}中, Gershgorin等提出了一个测试模型, 将Langevin方程中的各个项进行随机参数化, 以改进模型的系统误差:

\begin{equation} \label{E0}
\left\{
\begin{split}
& \frac{du(t)}{dt} = (-\gamma(t)+\ii\omega)u(t)+b(t)+f(t)+\sigma \dotW(t), \\[5pt]
% \end{equation}
% \begin{equation}
& \frac{db(t)}{dt} = (-\gamma_b+\ii\omega_b)(b(t)-\hat{b})+\sigma_b \dotW_b(t), \\[5pt]
% \end{equation}
% \begin{equation}
& \frac{d\gamma(t)}{dt} = -d_\gamma(\gamma(t)-\hat{\gamma})+\sigma_\gamma \dotW_\gamma(t)
\end{split}
\right.
\end{equation}

此随机微分方程组解的一阶和二阶统计量在大量文献中作为测试标准被广泛应用, 但是Gershgorin在\cite{gershgorin2010test}中没有给出这些统计量的推导过程. 本文利用了It\^o积分和特征函数, 补全了这些统计量的计算过程, 并对其理论结果进行了数值验证.

\subsection{基本假设}

在(\ref{E0})中, $\omega$为$u(t)$的振荡频率, $f(t)$为外部驱动力, $\sigma$表示白噪声$\dotW(t)$的强度. 此外, 参数$\gamma_b$和$d_\gamma$表示振荡阻尼, $\sigma_b$和$\sigma_\gamma$分别表示加性和乘性修正((\ref{E0})中第2, 3式)中白噪声的强度. $\hat{b}$和$\hat{\gamma}$分别表示$b(t)$和$\gamma(t)$的固定平均偏差修正, $\omega_b$表示加性噪声的频率. 白噪声$\dotW_\gamma(t)$是实值函数, 而$\dotW(t)$及$\dotW_b(t)$均为复值, 且其实部和虚部均为独立的白噪声. 

一般认为方程组(\ref{E0})有初值
\begin{equation} \label{init_values}
\left\{
\begin{split}
& u(t_0) = u_0, \\[5pt]
& b(t_0) = b_0, \\[5pt]
& \gamma(t_0) = \gamma_0,
\end{split}
\right.
\end{equation}
且$u_0$, ~$b_0$, ~$\gamma_0$均为独立的高斯随机变量, 其统计量
$\E[u_0]$, ~$\E[\gamma_0]$, ~$\E[b_0]$, ~$\Var(u_0)$, ~$\Var(\gamma_0)$, ~$\Var(b_0)$, ~$\Cov(u_0, u_0^*)$, ~$\Cov(u_0, \gamma_0)$, ~$\Cov(u_0, b_0)$, ~$\Cov(u_0, b_0^*)$均假设为已知.

(\ref{E0})的第二和第三项均为线性方程, 由求解线性常微分方程组的相关知识\cite{fulinjin1984ordinary}可知其通解有以下的形式:
\begin{equation} \label{Sb}
\begin{split}
&b(t) = \hat{b}+(b_0-\hat{b})e^{\lambda_b(t-t_0)}+\sigma_b\int_{t_0}^t e^{\lambda_b(t-s)}dW_b(s), \\[5pt]
&\gamma(t) = \hat{\gamma}+(\gamma_0-\hat{\gamma})e^{-d_{\gamma}(t-t_0)}+\sigma_{\gamma}\int_{t_0}^t e^{-d_{\gamma}(t-s)}dW_{\gamma}(s).
\end{split}
\end{equation}

其中$\lambda_b = -\gamma_b+\ii\omega_b$.
如果我们记
\begin{equation} \label{def J(s, t)}
\begin{split}
&\hat{\lambda} = -\hat{\gamma}+\ii\omega, \\[5pt]
&J(s, t) = \int_s^t (\gamma(s')-\hat{\gamma})ds'
\end{split}
\end{equation}

则(\ref{E0})的通解可以表示为
\begin{equation} \label{Su}
\begin{split}
u(t) &= e^{-J(t_0, t)+\hat{\lambda}(t-t_0)}u_0 + \int_{t_0}^t (b(s)+f(s))e^{-J(s, t)+\hat{\lambda}(s-t_0)}ds \\[5pt]
& + \sigma\int_{t_0}^{t}e^{-J(s, t)+\hat{\lambda}(s-t_0)}dW(s).
\end{split}
\end{equation}

本文在第三和第四节中将对$u, ~b, ~\gamma$, 分别计算其一阶和二阶统计量, 以及各个变量间的相关性.\\[5pt]

\section{一些前置引理}
% \subsection{布朗运动的性质}
% 在应用中, 往往将布朗运动置于随机微分方程中, 以模拟白噪声的性质\cite{hida1980brownian}. 
% 为了简化计算过程, 我们在这里引入布朗运动的一些基本性质\cite{nelson1967dynamical}.
% \begin{theorem}\cite{oksendal2003stochastic} \label{brownian1}
% 布朗运动$B_t$是一个Gaussian过程. 即对于所有的$0 \leqslant t_1 \leqslant\cdots\leqslant t_k$, 随机向量$Z = (B_{t_1}, B_{t_2}\cdots ,B_{t_k})\in \mathbb{R}^{nk}$ 服从多维正态分布. 如果设$B_{t_i}$的初值均为$x$, 则此随机向量的期望为 
% $$M = \E[Z] = (x, x, \cdots, x) \in \mathbb{R}^{nk},$$
% 协方差矩阵为
% $$C = \left(\begin{matrix}
% 			t_1I_n & t_1I_n & \cdots & t_1I_n \\
% 			t_1I_n & t_2I_n & \cdots & t_2I_n \\
% 			\vdots & \vdots & \ddots & \vdots \\
% 			t_1I_n & t_2I_n & \cdots & t_kI_n 
% 			\end{matrix}
% 			\right).$$
% \end{theorem}

% 此定理的证明利用了正态分布特征函数的性质, 具体过程可见\cite{oksendal2003stochastic}的附录A. 以下是定理(\ref{brownian1})的直接而显然的推论:
% \begin{theorem}\cite{oksendal2003stochastic}
% 假设布朗运动$B_t$满足定理(\ref{brownian1})中的条件, 那么当$t\geqslant 0$时,
% $$\E[B_t] = x, ~\E\left[(B_t - x)^2\right] = nt, ~\E[(B_t - x)(B_s - x)] = n\min(s, t).$$ 
% 而且如果$t \geqslant s$, 有$\E\left[(B_t - B_s)^2\right] = n(t-s)$.
% \end{theorem}

% 以下的定理也是布朗运动的一个重要性质, 其证明见\cite{oksendal2003stochastic}的(2.2.12)式.
% \begin{theorem}\cite{oksendal2003stochastic}
% $B_t$有独立的增量, 即对满足$\forall 0 \leqslant t_1 < t_2 \cdots < t_k$的$(t_1, t_2, \cdots t_k)$,
% $$B_{t_1}, ~B_{t_2}-B_{t_1}, ~\cdots, ~B_{t_k}-B_{t_{k-1}}$$相互独立.
% \end{theorem}

\subsection{It\^{o}积分和It\^o等距}
在求解随机微分方程时, 需要对布朗运动进行积分. 由\cite{oksendal2003stochastic}的例3.1.1, 我们可以定义多种不同的随机积分形式, 其中It\^o积分是应用最广的一种:
\begin{definition}\cite{oksendal2003stochastic}(It\^{o}积分)	设$f\in\mathcal{V}(S, T)$. 则$f$的It\^{o}积分定义为
$$\int_S^T f(t, \omega)dB_t(\omega) = \lim_{n\to\infty}\int_S^T\phi_n(t, \omega)dB_t(\omega),$$
其中${\phi_n}$为基本函数序列, 且满足当$n \to \infty$时
$$\E\left[\int_S^T (f(t, \omega)-\phi_n(t, \omega))^2dt\right] \to 0.$$
\end{definition}

由上述定义即可得到It\^{o}积分的一个重要性质:
\begin{theorem}\cite{oksendal2003stochastic} \label{Ito isometric} (It\^{o}等距) 对于$\forall f\in\mathcal{V}(S, T)$有
$$\E\left[\left(\int_S^T f(t, \omega)dB_t\right)^2\right] = \E\left[\int_S^T f^2(t, \omega)dt\right].$$
\end{theorem}

这里$\mathcal{V}(S, T)$见\cite{oksendal2003stochastic}中的定义3.1.2至3.1.4. \\
利用上述方法, 即利用基本函数序列来逼近$\mathcal{V}(S, T)$中的函数, 我们能够得到It\^{o}积分的线性性质:
\begin{theorem}\cite{oksendal2003stochastic} \label{Itoint_property}
设$f, g \in \mathcal{V}(0, T)$, $0 \leqslant S < U < T$. 那么
\begin{flalign*}
& (1)\quad \int_S^T fdB_t = \int_S^U fdB_t + \int_U^T fdB_t, ~\text{a.e.}\\[5pt]
& (2)\quad \int_S^T (cf+g)dB_t = c\int_S^T fdB_t + \int_S^T gdB_t, ~\text{a.e.}\\[5pt]
& (3)\quad \E\left[\int_S^T fdB_t\right] = 0.
\end{flalign*}
\end{theorem}

% 我们还需要引入以下的一维It\^{o}公式\cite{jiangangying2016stochastic}来计算一维情形的It\^o积分.
% \begin{theorem}\cite{oksendal2003stochastic}(It\^{o}公式) \label{Ito formula} 设$X_t$为一个如下的It\^{o}过程:
% $$dX_t = udt+vdB_t,$$
% $g(t, x) \in C^2\left([0, \infty) \times \mathbb{R}\right)$, 则$Y_t = g(t, X_t)$也是一个It\^{o}过程, 且满足:
% $$dY_t = \frac{\partial g}{\partial t}(t, X_t)dt + \frac{\partial g}{\partial x}(t, X_t)dX_t + \frac{1}{2}\frac{\partial^2 g}{\partial x^2}(t, X_t)\cdot(dX_t)^2,$$
% 其中
% $$dt\cdot dt = dt\cdot dB_t = dB_t\cdot dt = 0, ~dB_t\cdot dB_t = dt.$$
% \end{theorem}

% 以下是上述It\^{o}公式的积分形式.
% \begin{theorem}\cite{oksendal2003stochastic} \label{Itoint}(分部积分) 设$f(s, \omega)$对几乎所有的$\omega$关于$s\in [0, t]$是连续的, 且为有界变差函数, 则有
% $$\int_0^t f(s)dB_s = f(t)B_t - \int_0^t B_s df_s.$$
% \end{theorem}
上述几个定理的证明请见\cite[Page 26-31]{oksendal2003stochastic}. \\

\subsection{高斯分布与特征函数}
\begin{proposition}\cite{shuyuanhe2006probability} \label{multiVariable gaussian pdf}
如果随机向量$\textbf{X}$服从多元正态分布
$$\textbf{X} \sim \mathcal{N}(\mu, \Sigma),$$
那么$\textbf{X}$的概率密度为
$$f_\textbf{X}(X_1, \cdots, X_k) = \frac{\exp\left(-\frac{1}{2}(\textbf{X}-\mu)^\top\right)\Sigma^{-1}(\textbf{X}-\mu)}{\sqrt{(2\pi)^k|\Sigma|}}$$
\end{proposition}

\begin{proposition}
对于一个复值高斯变量$X = a+b\ii$, $a$与$b$均服从高斯分布, 那么其期望和方差为
\[
\left\{
\begin{split}
&\E[X] = \E[a] + \ii\E[b], \\[5pt]
&\Var(X) = \E[(X-\E[X])(X-\E[X])^*].
\end{split}
\right.
\]
\end{proposition}

除此之外, 我们还需要引入随机变量的特征函数:
\begin{definition}\cite{shuyuanhe2006probability}(特征函数) 如果$X$是实值随机变量, $\E[\sin(tX)], ~\E[\cos(tX)]$均存在, 那么称
$$\phi(t) = \E\left[e^{\ii tX}\right] = \E[\cos(tX)] + \ii\E[\sin(tX)], ~t\in \mathbb{R}$$
为$X$的特征函数, 其中$\ii$为虚数单位.
\end{definition}
对于随机向量, 其特征函数定义为
\begin{definition} 设$\textbf{X} = (X_1, X_2, \cdots, X_n)$为随机向量, 则$\textbf{X}$的特征函数为
$$\phi(t) = \E\left[\exp(\ii\textbf{t}\textbf{X}^\top)\right], ~\textbf{t} = (t_1, t_2, \cdots, t_n)\in \mathbb{R}^n.$$
\end{definition}
以下为高斯分布的特征函数.
\begin{proposition} \label{Characteristic function of Gaussian Distribution}
高斯分布$N(\mu, \sigma^2)$的特征函数为
$$\phi(t) = \exp(\ii\mu t - \frac{1}{2}\sigma^2t^2)$$
\end{proposition}
此命题的证明见\cite{shuyuanhe2006probability}第五章例2.4.
一个更一般的定义如下:
\begin{definition}\cite{jiangangying2013probability} \label{characteristic function definition 2} 设$\xi$为d-维随机变量, $F$为$\xi$的分布函数. 那么
$$\hat{F}(x) = \int_{\mathbb{R}^d} e^{\ii x\cdot y}dF(y), ~x\in \mathbb{R}^d$$
(其中$(\cdot)$为$\mathbb{R}^d$空间中的内积) 称为分布函数$F$的特征函数.
\end{definition}
由此定义即可看出, 特征函数即为概率密度函数的Fourier变换:
\begin{equation} \label{characteristic function definition 3}
\hat{F}(x) = \int_{\mathbb{R}^d}e^{\ii x\cdot y}f(y)dy.
\end{equation}

以下是Fourier变换的一个基本性质, 在求$u(t)$的统计量时将会用到\cite{boggess2015first}.
\begin{proposition}(Fourier变换的微分关系) \label{Fourier transform property 1}
如果
$$\lim_{|x|\to\infty} f(x) = 0,$$
且$f'(x)$的Fourier变换存在, 那么
$$\mathcal{F}[f'(x)] = \ii\omega\mathcal{F}[f(x)].$$
更一般的, 若
$f(\infty) = f'(\infty) = \cdots = f^{(k-1)}(\infty) = 0,$ 且$\mathcal{F}[f^{(k)}(x)]$存在, 则
$$\mathcal{F}[f^{(k)}(x)] = (\ii\omega)^k \mathcal{F}[f].$$\\[5pt]
\end{proposition}

\section{$b(t)$与$\gamma(t)$的统计量}
\subsection{均值}
当$t$固定时, 由(\ref{Sb})式可知$b(t)$的通解第一项为常数, 第二项中只有$b_0$是一个随机变量. 那么由(\ref{init_values}), ~$b(t)$通解第二项的期望为$(\E[b_0]-\hat{b})e^{\lambda_b(t-t_0)}$. 而由于
$$f = f(s) = e^{\lambda_b(t-s)}\in\mathcal{V}(S, T),$$
故由(\ref{Itoint_property})的(3)式,
$$\E\left[\sigma_b\int_{t_0}^{t} e^{\lambda_b(t-s)dW_b(s)}\right] = 0.$$
从而
\begin{equation} \label{b_stat_1}
\E(b(t)) = \hat{b} + (\E[b_0] - \hat{b})e^{\lambda_b(t-t_0)}.
\end{equation}

对于$\gamma$亦有相同的结论
\begin{equation} \label{gamma_stat_1}
\E(\gamma(t)) = \hat{\gamma} + (\E[\gamma_0] - \hat{\gamma})e^{-d_{\gamma}(t-t_0)}.
\end{equation}

\subsection{方差}
当$t$固定时, $b(t)$为一个复值随机变量, 由\cite{shuyuanhe2006probability}的第四章相关知识,
 % 考虑到$(b_0-\hat{b})e^{\lambda_b(t-t_0)}$与$\sigma_b\int_{t_0}^t e^{\lambda_b(t-s)}dW_b(s)$相互独立(由(\ref{init_values})可知),
\begin{equation} \label{b_stat_2_process}
\begin{split}
\text{Var}&(b(t)) = \E\left[(b(t)-\E[b(t)])(b(t)-\E[b(t)])^*\right] \\[10pt]
&= \E\left[\left((b_0-\E[b_0])e^{\lambda_b(t-t_0)}+\sigma_b\int_{t_0}^t e^{\lambda_b(t-s)}dW_b(s)\right)\times\right. \\[5pt]
&\quad\quad\left.\left((b_0-\E[b_0])e^{\lambda_b(t-t_0)}+\sigma_b\int_{t_0}^t dW_b(s) \right)^*\right] \\[5pt] 
&= e^{-2\gamma_b(t-t_0)}\Var(b_0) + \E\left[\sigma_b^2\int_{t_0}^t e^{\lambda_b(t-s)}dW_b(s)\left(\int_{t_0}^t e^{\lambda_b(t-s)}dW_b(s)\right)^*\right] \\[5pt]
&= e^{-2\gamma_b(t-t_0)}\Var(b_0) + \sigma_b^2\E\left[\int_{t_0}^t e^{-2\gamma_b(t-s)}ds\right]
\end{split}
\end{equation}

上面的最后一式右边利用了It\^{o}等距(\ref{Ito isometric}).
计算右边即可得到
\begin{equation} \label{b_stat_2} 
\Var(b(t)) = e^{-2\gamma_b(t-t_0)}\Var(b_0) + \frac{\sigma_b^2}{2\gamma_b}(1-e^{-2\gamma_b(t-t_0)}).
\end{equation}

同样的, 我们也有
\begin{equation} \label{gamma_stat_2}
\Var(\gamma(t)) = e^{-2d_\gamma(t-t_0)}\Var(\gamma_0) + \frac{\sigma_\gamma^2}{2d_\gamma}(1-e^{-2d_\gamma(t-t_0)}).
\end{equation}

\subsection{协方差}
考虑到$b(t)$为复值函数, 由协方差的定义\cite{shuyuanhe2006probability}即可知
\[
\begin{split}
\Cov&(b(t), b(t)^*) = \E\left[(b(t) - \E[b(t)])(b(t)^* - \E[b(t)^*])\right] \\[10pt]
&= \E\left[(b_0 - \E[b_0]e^{\lambda_b (t-t_0)}+\sigma_b\int_{t_0}^{t} e^{\lambda_b (t-s)}dW_b(s))\right. \times \\[5pt]
&\quad \left.((b_0^* - \E[b_0^*])e^{\lambda_b(t-t_0)}+\sigma_b\int_{t_0}^t e^{\lambda_b (t-s)}dW_b(s))\right] \quad\quad\quad\quad\quad\quad\quad\cdots(\ref{b_stat_1}) \\[5pt]
&= \E\left[(b_0 - \E[b_0])(b_0^* - \E[b_0^*])e^{2\lambda_b (t-t_0)}\right] + \sigma_b\E\left[(b_0 - \E[b_0])\int_{t_0}^{t}e^{\lambda_b(t-s)}dW_b(s)\right]\\[10pt]
&\quad+ \sigma_b\E\left[(b_0^*-\E[b_0^*])e^{\lambda_b(t-t_0)}\int_{t_0}^{t}e^{\lambda_b(t-s)}dW_b(s)\right] + \sigma_b^2\E\left[\left(\int_{t_0}^{t}e^{\lambda_b(t-s)}dW_b(s)\right)^2\right]
\end{split}
\]

在上式的第二项中, 由于$b_0$与$\int_{t_0}^{t} e^{\lambda_b (t-s)}dW_b(s)$相互独立, 故由It\^{o}积分的性质(\ref{Itoint_property})可知
\[
\sigma_b\E\left[(b_0 - \E[b_0])\int_{t_0}^{t}e^{\lambda_b(t-s)}dW_b(s)\right] = \sigma_b\E\left[b_0 - \E[b_0]\right]\E\left[\int_{t_0}^{t}e^{\lambda_b(t-s)}dW_b(s)\right] = 0
\]

同样的, 上式中第三项也为0, 而第四项由It\^{o}等距(\ref{Ito isometric})即知为0. 故
\begin{equation} \label{b_b*_Cov}
\Cov(b(t), b(t)^*) = \E[(b_0-\E[b_0])(b_0^*-\E[b_0^*])]e^{2\lambda_b(t-t_0)} = \Cov(b_0, b_0^*)e^{2\lambda_b(t-t_0)}
\end{equation}

与之类似的,
\begin{equation} \label{b_gamma_Cov}
% \begin{split}
\Cov(b(t), \gamma(t)) = \E[(b(t) - \E[b(t)])(\gamma(t)-\E[\gamma(t)])]
% &= \E\left[\left(b_0 - \E[b_0]e^{\lambda_b (t-t_0)}+\sigma_b\int_{t_0}^{t} e^{\lambda_b (t-s)}dW_b(s)\right)\right. \cdot \\
% &\quad \left.\left((\gamma_0-\E[\gamma_0])e^{-d_\gamma(t-t_0)}+\sigma_\gamma \int_{t_0}^{t} e^{-d_\gamma(t-s)}dW_\gamma(s)\right)\right] \\
% &= \E\left[(b_0-\E[b_0])(\gamma_0-\E[\gamma_0])\right]e^{(\lambda_b-d_\gamma)(t-t_0)} \\
= \Cov(b_0, \gamma_0)e^{(\lambda_b-d_\gamma)(t-t_0)}
% \end{split}
\end{equation}
\\[5pt]
% 上述的第三个等式也用到了It\^{o}等距(\ref{Ito isometric}).

\section{$u(t)$的统计量}
\subsection{均值}
在$u(t)$的表达式(\ref{Su})中, $e^{\hat{\lambda}(t-t_0)}$, $f(t)$和$\sigma$均为常值. 那么由数学期望的线性性质(\cite{shuyuanhe2006probability}第四章定理2.2)和It\^{o}积分的性质(\ref{Itoint_property})可知
\begin{equation} \label{u_stat_1}
\begin{split}
\E[u(t)] &= e^{\hat{\lambda}(t-t_0)}\E\left[e^{-J_0(t_0, t)}u_0\right] + \int_{t_0}^t e^{\hat{\lambda}(t-s)}\E\left[b(s)e^{-J(s, t)}\right]ds  \\[5pt]
&+ \sigma\int_{t_0}^{t} e^{\hat{\lambda}(t-s)}f(s)\E\left[e^{-J(s, t)}\right]ds
\end{split}
\end{equation}

下面我们利用高斯随机过程$J(s, t)$的特征函数来计算上式的右边. \\
首先我们有以下的命题:
\begin{proposition}\cite{gershgorin2008nonlinear}\cite{gershgorin2010filtering} \label{multiVariable gaussian exp 1}
$$\E\left[ze^{\ii bx}\right] = (\E[z]+\ii b\Cov(z, x))e^{\ii b\E[x]-\frac{1}{2}b^2\Var(x)}$$
其中z为复值高斯随机变量, x为实值高斯随机变量. 
\end{proposition}

\begin{proof}[证明]
不妨记
$$z = y + \ii w, \quad y, w\in \mathbb{R}. \label{z substitution}$$
那么我们只要计算出$\E[ye^{\ii bx}]$和$\E[we^{\ii bx}]$, 然后利用(\ref{z substitution})将其组合起来. 令
$\mathbf{v} = (x, y, w)$, 那么由于$\mathbf{v}$为一个满足多元Gauss分布的随机向量, 其特征函数由(\ref{Characteristic function of Gaussian Distribution})即可给出:
\[
\phi_\mathbf{v}(\mathbf{s}) = \exp(\ii\mathbf{s}^\top \E[\mathbf{v}]-\frac{1}{2}\mathbf{s}^\top\Sigma\mathbf{s}),
\]

其中$\Sigma$为协方差矩阵. 记$g(\mathbf{v})$为$\mathbf{v}$的概率密度函数, 那么由特征函数的定义(\ref{characteristic function definition 3}), 即特征函数为概率密度的Fourier变换即可知道,
\[
\phi_\mathbf{v}(\mathbf{s}) = \frac{1}{(2\pi)^3}\int e^{\ii\mathbf{s}^\top\mathbf{v}}g(\mathbf{v})d\mathbf{v}
\]

利用Fourier变换的性质(\ref{Fourier transform property 1}), 我们不妨对$s_2$求偏导\cite{gershgorin2008nonlinear}, 有
\[
\frac{\partial \phi_\mathbf{v}(\mathbf{s})}{\partial s_2} = \frac{1}{(2\pi)^3}\int \ii y_0e^{\ii\mathbf{s}^\top\mathbf{v}}g(\mathbf{v})d\mathbf{v} = i\E\left[y_0e^{\ii\mathbf{s}^\top\mathbf{v}}\right].
\]

令$\mathbf{v} = (b, 0, 0)^\top$,
\[
\left.\E[y_0e^{\ii bx_0}] = -\ii\frac{\partial \phi_\mathbf{v}(s)}{\partial s_2}\right|_{\mathbf{s} = (b, 0, 0)^\top}
\]

同样的, 我们有
\[
\left.\E[w_0e^{\ii bx_0}] = -\ii\frac{\partial \phi_\mathbf{v}(s)}{\partial s_3}\right|_{\mathbf{s} = (b, 0, 0)^\top}
\]

由多元高斯分布的概率密度函数(\ref{multiVariable gaussian pdf})有
\[
\begin{split}
&\frac{\partial \phi_\mathbf{v}(\mathbf{s})}{\partial s_2} = (\ii\E[y_0]-\Var(y_0)s_2-\Cov(x_0, y_0)s_1-\Cov(y_0, w_0)s_3)\phi_{\mathbf{v}}(s) \\[5pt]
&\frac{\partial \phi_\mathbf{v}(\mathbf{s})}{\partial s_3} = (\ii\E[w_0]-\Var(w_0)s_3-\Cov(x_0, w_0)s_1-\Cov(y_0, w_0)s_2)\phi_{\mathbf{v}}(s)
\end{split}
\]

分别计算这两个偏导数在$\mathbf{s} = (b, 0, 0)^\top$处的值, 有
\[
\begin{split}
&\E\left[y_0e^{\ii bx_0}\right] = (\E[y_0]+\ii\Cov(x_0, y_0)b)\exp(\ii b\E[x_0] - \frac{1}{2}\Var(x_0)b^2) \\[5pt]
&\E\left[w_0e^{\ii bx_0}\right] = (\E[w_0]+\ii\Cov(x_0, w_0)b)\exp(\ii b\E[x_0] - \frac{1}{2}\Var(x_0)b^2)
\end{split}
\]

那么
$$\E\left[ze^{\ii bx}\right] = (\E[z]+\ii b\Cov(z, x))e^{\ii b\E[x]-\frac{1}{2}b^2\Var(x)}.$$
\end{proof}

由此我们立刻可以得到
\begin{corollary} \label{multiVariable gaussian exp 2} 在命题\ref{multiVariable gaussian exp 1}的条件下,
$$\E\left[ze^{bx}\right] = (\E[z]+b\Cov(z, x))e^{b\E[x]+\frac{1}{2}b^2\Var(x)}.$$
\end{corollary}

利用推论\ref{multiVariable gaussian exp 2}即有
\begin{equation} \label{E u(t)}
\begin{split}
\E&[u(t)] = e^{\hat{\lambda}(t-t_0)}(\E[u_0]-\Cov(u_0, J(t_0, t)))e^{-\E[J(t_0, t)]+\frac{1}{2}\Var(J(t_0, t))} \\[10pt]
&+ \int_{t_0}^t e^{\hat{\lambda}(t-s)}(\hat{b}+e^{\lambda_b(s-t_0)}(\E[b_0]-\hat{b}-\Cov(b_0, J(s, t))))e^{-\E[J(s, t)]+\frac{1}{2}\Var(J(s, t))}ds \\[5pt]
&+ \int_{t_0}^t e^{\hat{\lambda}(t-s)}f(s)e^{-\E[J(s, t)]+\frac{1}{2}\Var(J(s, t))}ds
\end{split}
\end{equation}

下面我们计算
$$\Cov(u_0, J(s, t)), ~\Cov(b_0, J(s, t)), ~\E[J(s, t)], ~\Var(J(s, t)).$$

首先,
\begin{equation} \label{J(s, t) all}
\begin{split}
J(s, t) &= \int_s^t \left[(\gamma_0-\hat{\gamma})e^{-d_\gamma(s'-t_0)}+\sigma_\gamma\int_{t_0}^{s'}e^{-d_\gamma(s'-x)}dW_\gamma(x)\right]ds' \\[5pt]
& = \int_s^t(\gamma_0-\hat{\gamma})e^{-d_\gamma(s'-t_0)}ds' +\int_s^t \sigma_\gamma\int_{t_0}^{s'} e^{-d_\gamma(s'-x)}dW_\gamma(x)ds'
\end{split}
\end{equation}

那么
\begin{equation} \label{Cov u0 J(s, t)}
\begin{split}
\Cov&(u_0, J(s, t)) = \Cov\left(u_0, \frac{1}{d_\gamma}(e^{-d_\gamma(s-t_0)}-e^{-d_\gamma(t-t_0)})(\gamma_0 - \hat{\gamma})\right) \\[5pt]
&+ \Cov\left(u_0, \int_s^t \sigma_\gamma \int_{t_0}^{s'} e^{-d_\gamma(s'-x)} dW_\gamma(x)ds'\right)\\[5pt]
&= \frac{1}{d_\gamma}(e^{-d_\gamma(s-t_0)}-e^{-d_\gamma(t-t_0)})[\Cov(u_0, \gamma_0)-\Cov(u_0, \hat{\gamma})] \\[5pt]
&= \frac{1}{d_\gamma}(e^{-d_\gamma(s-t_0)}-e^{-d_\gamma(t-t_0)})\Cov(u_0, \gamma_0).
\end{split}
\end{equation}

这是因为$u_0$与$\hat{\gamma}$相互独立. 同理有
\begin{equation} \label{Cov b0 J(s, t)}
\Cov(b_0, J(s, t)) = \frac{1}{d_\gamma}(e^{-d_\gamma(s-t_0)}-e^{-d_\gamma(t-t_0)})\Cov(b_0, \gamma_0).
\end{equation}

为了计算$\E[J(s, t)]$, 我们对(\ref{gamma_stat_1})求积分, 有
\begin{equation} \label{E J(s, t)}
\begin{split}
\E[&J(s, t)] = \E\left[\int_s^t (\gamma(s')-\hat{\gamma})ds'\right] = \int_s^t (\E[\gamma(s')]-\hat{\gamma})ds' \\[5pt]
&= \int_s^t (\E[\gamma_0]-\hat{\gamma})e^{-d_\gamma(s'-t_0)}ds' = \frac{1}{d_\gamma}(e^{-d_\gamma(s-t_0)}-e^{-d_\gamma(t-t_0)})(\E[\gamma_0]-\hat{\gamma}))
\end{split}
\end{equation}

结合(\ref{E J(s, t)})和(\ref{J(s, t) all})可知,
\[
\E\left[\sigma_\gamma\int_s^t\int_{t_0}^{s'}e^{-d_\gamma(s'-x)}dW_\gamma(x)ds'\right] = 0.
\]
此外, 由定义即有
\[
\begin{split}
\Var&(J(s, t)) = \E\left[J^2(s, t)\right] - \E[J(s, t)]^2 \\[10pt]
&= \E\left[\left(\int_s^t (\gamma_0-\hat{\gamma})e^{-d_\gamma(s'-t_0)}ds'\right)^2\right] - \E[J(s, t)]^2 + 2\E\left[\left(\int_s^t(\gamma_0-\hat{\gamma})e^{-d_\gamma(s'-t_0)}ds'\right)\times\right. \\[5pt]
&\quad\left.\left(\sigma_\gamma\int_s^t\int_{t_0}^{s'}e^{-d_\gamma(s'-x)}dW_\gamma(x)ds'\right)\right] + \sigma_\gamma^2\E\left[\left(\int_s^t\int_{t_0}^{s}e^{-d_\gamma(s'-s)}dW_\gamma(s)ds'\right)^2\right] \\[5pt]
&= \frac{1}{d_\gamma^2}\left(e^{-d_\gamma(s-t_0)}-e^{-d_\gamma(t-t_0)}\right)^2(\E[\gamma_0^2]-\E[\gamma_0]^2) + \sigma_\gamma^2\E\left[\left(\int_{t_0}^t \frac{1}{d_\gamma}\left(1-e^{-d_\gamma}(t-x)\right)dW_\gamma(x)\right)^2\right]
\end{split}
\]

由It\^{o}等距(定理\ref{Ito isometric})知上式的第二项为
\[
\begin{split}
\sigma_\gamma^2\E&\left[\left(\int_{t_0}^t \frac{1}{d_\gamma}\left(1-e^{-d_\gamma}(t-x)\right)dW_\gamma(x)\right)^2\right] \\[5pt]
&= \sigma_\gamma^2\E\left[\left(\int_{t_0}^{s}\int_{t_0}^{t}e^{-d_\gamma(s'-x)}ds'dW_\gamma(x)+\int_s^t\int_x^t e^{-d_\gamma(s'-x)}ds'dW_\gamma(x)\right)^2\right] \\[5pt]
% &= \frac{\sigma_\gamma^2}{d_\gamma^2}E\left[\left(\int_{t_0}^t \left(e^{-d_\gamma(s-x)}-e^{-d_\gamma(t-x)}dx\right)+\int_{t_0}^t \left(1-e^{-d_\gamma(t-x)\right)^2dx\right)\right]
% &= \frac{\sigma_\gamma^2}{d_\gamma^2}E\left[\left(\int_{t_0}^{t} \left(e^{-d_\gamma(s-x)}-e^{-d_\gamma(t-x)}\right)dW_\gamma(x)+\int_{s}^{t} \left(1-e^{-d_\gamma(t-x)}\right)dW_\gamma(x)\right)^2\right] \\
&= \frac{\sigma_\gamma^2}{d_\gamma^2}\left\{\int_{t_0}^t \left(e^{-d_\gamma(s-x)}-e^{-d_\gamma(t-x)}\right)^2dx+\int_{s}^t\left(1-e^{-d_\gamma(t-x)}\right)^2dx\right. \\[5pt]
&\quad+\left.2\E\left[\left(\int_{t_0}^t \left(e^{-d_\gamma(s-x)}-e^{-d_\gamma(t-x)}\right)dW_\gamma(x)\right)\left(\int_{s}^t\left(1-e^{-d_\gamma(t-x)}\right)dW_\gamma(x)\right)\right]\right\} \\[5pt]
&= \frac{\sigma_\gamma^2}{d_\gamma^3}\left(-1+d_\gamma(t-s)+e^{-d_\gamma(s+t-2t_0)}+e^{-d_\gamma(t-s)}-\frac{1}{2}e^{-2d_\gamma(t-t_0)}-\frac{1}{2}e^{-2d_\gamma(s-t_0)}\right. \\[5pt]
&\quad+ \left.\frac{1}{2}e^{-2d_\gamma(s-t)}-\frac{1}{2}e^{-2d_\gamma(t-s)}\right) + 2\E\left[\int_{\min\{s, t_0\}}^t\left(e^{-d_\gamma(s-x)}-e^{-d_\gamma(t-x)}\right)\left(1-e^{-d_\gamma(t-x)}\right)dt\right]
\end{split}
\]

这里我们将两个积分区域均扩展至$[\min\{s, t_0\}, ~t]$上, 并设积分区域较小的被积函数在延拓的区间上为0. 那么
\begin{equation} \label{Var J(s, t) final}
\begin{split}
\Var&(J(s, t)) = \frac{1}{d_\gamma^2}\left(e^{-d_\gamma(s-t_0)}-e^{-d_\gamma(t-t_0)}\right)^2\Var(\gamma_0)+ \frac{\sigma_\gamma^2}{d_\gamma^3}(-1+d_\gamma(t-s) \\[5pt]
&\left.+e^{-d_\gamma(s+t-2t_0)}+e^{-d_\gamma(t-s)}-\frac{1}{2}e^{-2d_\gamma(t-t_0)}-\frac{1}{2}e^{-2d_\gamma(s-t_0)}\right) 
\end{split}
\end{equation}
将(\ref{Var J(s, t) final}), (\ref{Cov u0 J(s, t)}), (\ref{Cov b0 J(s, t)})代入(\ref{E u(t)})式即得$u(t)$的均值.

\subsection{方差}
利用定义$\Var(u(t)) = \E\left[|u(t)|^2\right] - \left|\E[u(t)]\right|^2,$
我们记$u(t) = A + B + C,$

其中
\[\left\{
\begin{split}
&A = e^{-J(t_0, t)+\hat{\lambda}(t-t_0)}u_0, \nq
&B = \int_{t_0}^t (b(s)+f(s))e^{-J(s, t)+\hat{\lambda}(t-s)}ds, \nq
&C = \sigma\int_{t_0}^t e^{-J(s, t)+\hat{\lambda}(t-s)}dW(s).
\end{split}
\right.
\]

于是由It\^{o}积分的性质有
\begin{equation} \label{E u(t)^2}
\E\left[|u(t)|^2\right] = \E\left[|A|^2\right]+ \E\left[|B|^2\right] + \E\left[|C|^2\right] + 2\text{Re}\{\E[A^*B]\}.
\end{equation}

下面我们分别求出
$
\E\left[|A|^2\right], \E\left[|B|^2\right], \E\left[|C|^2\right], \E[AB].
$
首先由命题\ref{multiVariable gaussian exp 1}可知
\begin{proposition}  \label{multiVariable gaussian exp 3} 对于复值Gaussian随机变量$z$和$w$, 以及实值Gaussian随机变量$x$,
\[
\begin{split}
\E\left[zwe^{bx}\right] =& \left[\E[z]\E[w]+\Cov(z, w^*)+b(\E[z]\Cov(w, x))+\E[w]\Cov(z, x) \right. \nq
&\quad +\left.b^2\Cov(z, x)\Cov(w, x)\right]e^{b\E[x]+\frac{b^2}{2}\Var(x)}. \nq
\end{split}
\]
\end{proposition}

利用上述命题, 有
\[
\begin{split}
&\E\left[|A|^2\right] = \left(|\E[u_0]|^2 +\Var(u_0) - 4\text{Re}\left\{\E[u_0]^*\Cov(u_0, J(t_0, t)\right\} +4|\Cov(u_0, J(t_0, t)|^2\right) \times \nq
& \qquad\qquad e^{-2\hat{\gamma}(t-t_0)-2\E[J(t_0, t)]+2\Var(J(t_0, t))}, \nq
&\E\left[|B|^2\right] = \E\left[\left|\int_{t_0}^t (b(s)+f(s))e^{-J(s, t)+\hat{\lambda}(t-s)}ds\right|^2\right] \nq
&\qquad\qquad= \int_{t_0}^t ds\int_{t_0}^t dr \E\left[(b(s)+f(s))e^{-J(s, t)+\hat{\lambda}(t-s)}\left[(b(r)+f(r))e^{-J(r, t)+\hat{\lambda}(t-r)}\right]^*\right]
\end{split}
\]

在上式中, 被积的最后一项为
\[
\begin{split}
\E&\left[(b(s)+f(s))e^{-J(s, t)+\hat{\lambda}(t-s)}\left[(b(r)+f(r))e^{-J(r, t)+\hat{\lambda}(t-r)}\right]^*\right] \nw
% &= E\left[(b(s)+f(s))(b^*(r)+f^*(r))e^{-J(s, t)-J(r, t)+(-\hat{\gamma}+i\omega)(t-s)+(-\hat{\gamma}-i\omega)(t-r)}\right] \\
&= \E\left[(b(s)+f(s))(b^*(r)+f^*(r))e^{-J(s, t)-J(r, t)-\hat{\gamma}(2t-s-r)+\ii\omega(r-s)}\right] \nw
% &= e^{-J(s, t)-J(r, t)+\frac{1}{2}Var(J(s, t))+\frac{1}{2}Var(J(r, t))+Cov(J(s, t), J(r, t))-\hat{\gamma}(2t-s-r)+i\omega(r-s)} \times \\
% &\quad \{E\left[b(s)b^*(r)\right] + E[b(s)f^*(r)] + E[f(s)b^*(r)] + E[f(s)f^*(r)] \\
% &\quad+ Cov(b(s), b(r)) + Cov(b(s), f(r))+Cov(f(s), b(r))+Cov(f(s), f(r)) \\
% &\quad+ (E[b(s)]+E[f(s)])\times[Cov(b^*(r), -J(s, t))+Cov(b^*(r), -J(r, t)) \\
% &\quad\quad +Cov(f^*(r), -J(s, t))+Cov(f^*(r), -J(r, t))] \\
% &\quad+ (E[b^*(r)]+E[f^*(r)])\times[Cov(b(s), -J(s, t))+Cov(f(s), -J(s, t)) \\
% &\quad\quad +Cov(f(s), -J(r, t))+Cov(b(s), -J(r, t))] \\
% &\quad+ [Cov(b(s), -J(s, t))+Cov(b(s), -J(r, t))+Cov(f(s), -J(s, t))+Cov(f(s), -J(r,t))] \\
% &\quad\quad\times [Cov(b^*(r), -J(s, t))+Cov(b^*(r), -J(r, t))+Cov(f^*(r), -J(s,t))+Cov(f^*(r), -J(r, t))]\} \\
&= e^{-J(s, t)-J(r, t)+\frac{1}{2}\Var(J(s, t))+\frac{1}{2}\Var(J(r, t))+\Cov(J(s, t), J(r, t))-\hat{\gamma}(2t-s-r)+\ii\omega(r-s)} \times \nw
&\quad \{\E[b(s)b^*(r)]+\E[b(s)]\times[\Cov(b^*(r), -J(s, t))+\Cov(b^*(r), -J(r, t))] \nq
&\quad\quad+ \E[b^*(r)]\times[\Cov(b(s), -J(s, t))+\Cov(b(s), -J(r, t))]\nq
&\quad\quad+ [\Cov(b^*(r), J(s, t))+\Cov(b^*(r), J(r, t))]\times [\Cov(b(s), J(s, t))+\Cov(b(s), J(r, t))] \nq
&\quad\quad+ f^*(r)\times[\E[b(s)]-\Cov(b(s), J(s, t))-\Cov(b(s), J(r, t))] \nq
&\quad\quad+ f(s)\times[\E[b^*(r)]-\Cov(b(r), J^*(s, t))-\Cov(b(r), J^*(r, t))] \nq
&\quad\quad+f(s)f^*(r)\}.
\end{split}
\]

其中, 由It\^o公式和It\^o积分的性质即知
\[
\begin{split}
\E&[b(s)b(r)] = \E\left[\left(\hat{b}+(b_0-\hat{b})e^{\lambda_b(s-t_0)}+\sigma_b\int_{t_0}^s e^{\lambda_b(s-w)}dW_b(w)\right)\right.\times \nq
&\quad\quad\quad\quad\quad\quad\left.\left(\hat{b}+(b_0-\hat{b})e^{\lambda_b(r-t_0)}+\sigma_b\int_{t_0}^r e^{\lambda_b(r-w)}dW_b(w)\right) \right] \nw
&= \left(1-e^{\lambda_b(s-t_0)}-e^{\lambda_b(r-t_0)}+e^{\lambda_b(s+r-2t_0)}\right)\hat{b}^2 + \left(e^{\lambda_b(s-t_0)}+e^{\lambda_b(r-t_0)}-2e^{\lambda_b(s+r-2t_0)}\right)\hat{b}\E[b_0] \nq
&\quad + e^{\lambda_b(s+r-2t_0)}\left(\Var(b_0) + \E[b_0]^2\right) + \frac{\sigma_b^2}{2\gamma_b}\left(e^{-\gamma_b(s+r-2\min(s, r))}-e^{-\gamma_b(s+r-2t_0)}\right)e^{\ii\omega_b(s-r)}
\end{split}
\]

结合(\ref{E J(s, t)})和(\ref{Cov b0 J(s, t)})有
\[
\begin{split}
\Cov&(b(r), J(s, t)) = e^{\lambda_b(r-t_0)}\Cov(b_0, J(s, t)) + \sigma_b\Cov\left(\int_{t_0}^r e^{\lambda_b(r-w)}dW_b(w), J(s, t)\right) \nq
&= \frac{1}{d_\gamma}(e^{-d_\gamma(s-t_0)}-e^{-d_\gamma(t-t_0)})e^{\lambda_b(r-t_0)}\Cov(b_0, \gamma_0)
\end{split}
\]

当$t_0 \leqslant r \leqslant s \leqslant t$时, 我们这么计算$J(s, t)$和$J(r, t)$的协方差:
\[
\Cov(J(s, t), J(r, t)) = \Cov(J(s, t), J(r, s)+J(s, t)) = \Var(J(s, t)) + \Cov(J(s, t), J(r, s)).
\]

其中$\Var(J(s, t))$由(\ref{Var J(s, t) final})已求出. 由类似求$\Var(J(s, t))$的过程, 有
\[
\begin{split}
\Cov&(J(s, t), J(r, s)) = \Cov\left(\int_s^t(\gamma_0-\hat{\gamma})e^{-d_\gamma(s'-t_0)}ds' +\int_s^t \sigma_\gamma\int_{t_0}^{s'} e^{-d_\gamma(s'-x)}dW_\gamma(x)ds',\right. \nq
&\quad\quad\quad\left. \int_r^s(\gamma_0-\hat{\gamma})e^{-d_\gamma(s'-t_0)}ds' +\int_r^s \sigma_\gamma\int_{t_0}^{s'} e^{-d_\gamma(s'-x)}dW_\gamma(x)ds'\right) \nw
&= \frac{\Var(\gamma_0)}{d_\gamma^2}\left(e^{-d_\gamma(t-t_0)}-e^{-d_\gamma(s-t_0)})(e^{-d_\gamma(s-t_0)}e^{-d_\gamma(r-t_0)}\right)-\frac{\sigma_\gamma^2}{2d_\gamma^3}\left(e^{-d_\gamma(t-s)}-e^{-d_\gamma(t-r)}  \right.\nq
&\left.\quad+e^{-d_\gamma(t+s-2t_0)}-e^{-d_\gamma(t+r-2t_0)}-1+e^{-d_\gamma(s-r)}-e^{-2d_\gamma(s-t_0)}+e^{-d_\gamma(s+r-2t_0)}\right)
\end{split}
\]

当$t_0 \leqslant s \leqslant r \leqslant t$时, 只要考虑到
$
\Cov(J(s, t), J(r, t)) = \Cov(J(r, t), J(s, t)),
$
于是只要在结果中将$s$与$r$交换即可. 接下来,由It\^o等距有
\[
\begin{split}
\E&\left[|C|^2\right] = \sigma^2\E\left[\left(\int_{t_0}^t e^{-J(s, t)+\hat{\lambda}(t-s)}dW(s)\right)^2\right] \nq
&= \sigma^2\int_{t_0}^te^{-2\hat\gamma(t-s)}\E\left[e^{-2J(s, t)}\right]ds = \sigma^2\int_{t_0}^t e^{-2\hat\gamma(t-s)-2\E[J(s, t)]+2\Var(J(s, t))}ds \nw
\E&[A^*B] = \E\left[\left(e^{-J(t_0, t)+\hat{\lambda}(t-t_0)}u_0\right)^*\int_{t_0}^t(b(s)+f(s))e^{-J(s, t)+\hat{\lambda}(t-s)}ds\right]\nq
&= e^{-\hat\gamma(t-t_0)}\int_{t_0}^t \left(\E\left[u_0^*b_0e^{-J(t_0, t)-J(s, t)}\right]e^{(\lambda_b-\ii\omega)(s-t_0)}\right.\nq
&\quad+\left.\left(\hat{b}\left(1-e^{\lambda_b(s-t_0)}\right)+f(s)\right)\E\left[u_0e^{-J(t_0, t)-J(s, t)}\right]^*\right)ds
\end{split}
\]

由命题(\ref{multiVariable gaussian exp 3}), 以及类似$\E\left[B^2\right]$的计算过程, 类似的能得到
\[
\begin{split}
\E&\left[u_0^*b_0e^{-J(t_0, t)-J(s, t)}\right] = e^{-\E[J(s, t)]-\E[J(t_0, t)]+\frac{1}{2}\Var(J(s, t))+\frac{1}{2}\Var(J(t_0, t))+\Cov(J(s, t), J(t_0, t))}\times \nq
&\quad[\E[u_0^*]\E[b_0]+\Cov(u_0^*, b_0^*) - \E[u_0^*](\Cov(b_0, J(s, t))+\Cov(b_0, J(t_0, t))) \nq
&\quad-\E[b_0](\Cov(J(s, t), u_0^*)+\Cov(J(t_0, t), u_0^*))+(\Cov(u_0^*, J(s, t))+\Cov(u_0^*, J(t_0, t))) \nq
&\quad\times(\Cov(b_0, J(s, t))+\Cov(b_0, J(t_0, t)))], \nw
\E&\left[u_0e^{-J(t_0, t)-J(s, t)}\right] = e^{-\E[J(s, t)]-\E[J(t_0, t)]+\frac{1}{2}\Var(J(s, t))+\frac{1}{2}\Var(J(t_0, t))+\Cov(J(s, t), J(t_0, t))}\times \nq
&\quad(\E[u_0]-\Cov(u_0, J(s, t))-\Cov(u_0, J(t_0, t)))
\end{split}
\]
于是把上述的几个式子代回到(\ref{E u(t)^2})即得到了$\Var(u(t))$.

\subsection{协方差}
\subsubsection{$\Cov(u(t), u^*(t))$}
由定义知,
$
\Cov(u(t), u^*(t)) = \E\left[u(t)^2\right]-\E[u(t)]^2.
$
利用上一节中的记号, 由布朗运动$W(t)$的独立性,
\[
\E[u(t)^2] = \E[A^2]+\E[B^2]+2\E[AB].
\]

下面分别计算$\E[A^2]$, $\E[B^2]$和$\E[AB]$.
利用命题(\ref{multiVariable gaussian exp 3}), 和上一节类似的, 我们有
\[
\begin{split}
\E&[A^2] = \E[e^{-2J(t_0, t)+2\hat\lambda(t-t_0)}u_0^2] = e^{2\hat\lambda(t-t_0)-2\E[J(t_0, t)]+2\Var(J(t_0, t))}\times \nw
&\quad\left(\E[u_0]^2+\Cov(u_0, u_0^*)-4\E[u_0]\Cov(u_0, J(t_0, t))+4\Cov(u_0, J(t_0, t))^2\right), \nw
\E&\left[B^2\right] = \int_{t_0}^t ds \int_{t_0}^t dr \E\left[((b(s)+f(s))(b(r)+f(r))e^{-J(s, t)-J(r, t)+\hat\lambda(2t-s-r)}\right]
\end{split}
\]

上式中的被积项为
\[
\begin{split}
\E&\left[((b(s)+f(s))(b(r)+f(r))e^{-J(s, t)-J(r, t)+\hat\lambda(2t-s-r)}\right] \nw
&= e^{\hat\lambda(2t-s-r)-\E[J(s, t)]-\E[J(r, t)]+\frac{1}{2}\Var(J(s, t))+\frac{1}{2}\Var(r, t)+\Cov(J(s, t), J(r, t))} \times \nw
&\quad [\E[b(s)b(r)]-\E[b(s)]\times(\Cov(b(r), J(r, t))+\Cov(b(r), J(s, t))) \nq
&\quad- \E[b(r)]\times(\Cov(b(s), J(r, t))+\Cov(b(s), J(s, t))) \nq
&\quad + [\Cov(b(r), J(s, t))+\Cov(b(r), J(r, t))]\times[\Cov(b(s), J(s, t))+\Cov(b(s), J(r, t))]\nq
&\quad + f(r)\times[\E[b(s)]-\Cov(b(s), J(s, t))-\Cov(b(s), J(r, t))]\nq
&\quad +f(s)\times[\E[b(r)]-\Cov(b(r), J(r, t))-\Cov(b(r), J(s, t))]+f(s)f(r)],
\end{split}
\]

其中
\[
\begin{split}
\E&[b(s)b(r)] = \left(1-e^{\lambda_b(s-t_0)}-e^{\lambda_b(r-t_0)}+e^{\lambda_b(s+r-2t_0)}\right)\hat{b}^2 \nq
&+ \left(e^{\lambda_b(s-t_0)}+e^{\lambda_b(r-t_0)}-2e^{\lambda_b(s-t_0)(r-t_0)}\right)\hat{b}\E[b_0] + e^{\lambda_b(s+r-2t_0)}\left(\Var(b_0)+|\E[b_0]|^2\right), \nw
\E&[AB] = \E[e^{-J(t_0, t)+\hat{\lambda}(t-t_0)}u_0\int_{t_0}^t (b(s)+f(s))e^{-J(s, t)+\hat\lambda(t-s)}ds] \nq
&= e^{\hat\lambda(2t-s-t_0)}\int_{t_0}^t \left[e^{\lambda_b(s-t_0)}\E[u_0b_0e^{-J(t_0, t)-J(s, t)}]+(\hat{b}(1-e^{\lambda_b(s-t_0)})+f(s))\E[u_0e^{-J(t_0, t)-J(s, t)}]\right]ds.
\end{split}
\]

上式中的第二个期望在上一节中已经求出结果了, 而
\[
\begin{split}
\E&[u_0b_0e^{-J(t_0, t)-J(s, t)}] = e^{-\E[J(t_0, t)]-\E[J(s, t)]+\frac{1}{2}(\Var(J(t_0, t))+\Var(J(s, t)))+\Cov(J(t_0, t), J(s, t))}\times \nw
&\quad (\Cov(u_0, b_0^*)+\E[u_0]\E[b_0]-\E[u_0](\Cov(b_0, J(t_0, t))+\Cov(b_0, J(s, t)))\nq
&\quad -\E[b_0](\Cov(u_0, J(t_0, t))+\Cov(u_0, J(s, t)))\nq
&\quad +[\Cov(b_0, J(t_0, t))+\Cov(b_0, J(s, t))]\times[\Cov(u_0, J(s, t))+\Cov(u_0, J(t_0, t))]).
\end{split}
\]
\\

\subsubsection{$\Cov(u(t), \gamma(t))$}
% 下面考虑$Cov(u(t), \gamma(t))$.
由定义,
\[
\Cov(u(t), \gamma(t)) = \E[u(t)\gamma(t)] - \E[u(t)]\E[\gamma(t)] = \E[u(t)(\gamma(t)-\hat\gamma)] + \E[u(t)](\hat\gamma-\E[\gamma(t)]). 
\]

上式的第二项由(\ref{gamma_stat_1})和(\ref{E u(t)})可以直接计算得到. 下面我们计算上式的第一项. 由It\^o等距知,
\[
\begin{split}
\E&[u(t)(\gamma(t)-\hat\gamma)] = \E\left[\left(e^{-J(t_0, t)+\hat\lambda(t-t_0)}u_0+\int_{t_0}^t(b(s)+f(s))e^{-J(s, t)+\hat\lambda(s-t_0)}ds\right.\right. \nq
&\quad\quad\quad\quad\quad\quad\quad \left.\left.+\sigma\int_{t_0}^t e^{-J(s, t)+\hat\lambda(s-t_0)}dW(s)\right)\times(\gamma(t)-\hat\gamma)\right] \nq
&= e^{\hat\lambda(t-t_0)}\E\left[(\gamma(t)-\hat\gamma)u_0e^{\hat\lambda(t-t_0)}\right]+\int_{t_0}^t e^{-\hat\lambda(s-t_0)}\E\left[(b(s)+f(s))(\gamma(t)-\hat\gamma)e^{-J(s, t)}\right]ds
\end{split}
\]

由$J(t_0, t)$的定义(\ref{def J(s, t)})可知
$
\frac{\partial J(t_0, t)}{\partial t} = (\gamma(t)-\hat\gamma),
$
那么\\
% (\ref{E u(t) gamma})可以表示为
\[
\E[u(t)(\gamma(t)-\hat\gamma)] = -e^{\hat\lambda(t-t_0)}\frac{\partial}{\partial t}\E\left[u_0e^{-J(t_0, t)}\right]-\int_{t_0}^t e^{-\hat\lambda(s-t_0)}\frac{\partial}{\partial t}\E\left[(b(s)+f(s))e^{-J(s, t)}\right]ds,
\]

其中由(\ref{multiVariable gaussian exp 2})和协方差的线性性质,
\[
\begin{split}
\frac{\partial}{\partial t}&\E\left[u_0e^{-J(t_0, t)}\right] = \frac{\partial}{\partial t}\left[(\E[u_0]+\Cov(-J(t_0, t), u_0))e^{-\E[J(t_0, t)]+\frac{1}{2}\Var(J(t_0, t))}\right] \nq
&= \left(\frac{\partial}{\partial t}\Cov(-J(t_0, t), u_0)\right)e^{-\E[J(t_0, t)]+\frac{1}{2}\Var(J(t_0, t))} \nq
&\quad + (\E[u_0] + \Cov(u_0, -J(t_0, t)))\frac{\partial}{\partial t}e^{-\E[J(t_0, t)]+\frac{1}{2}\Var(J(t_0, t))} \nq
&= e^{-J(t_0, t)+\frac{1}{2}\Var(J(t_0, t))}\times\left[\Cov(-\gamma(t), u_0)+(\E[u_0]+\Cov(-J(t_0, t), u_0))\times\right. \nq
&\quad \left.(\hat\gamma-\E[\gamma(t)]+\frac{1}{2}\frac{\partial}{\partial t}\Var(J(t_0, t)))\right],
\end{split}
\]

\[
\begin{split}
\frac{\partial}{\partial t}&\E\left[(b(s)+f(s))e^{-J(s, t)}\right] = \frac{\partial}{\partial t}\left[(\E[b(s)]+f(s)+\Cov(b(s)+f(s), -J(s, t)))e^{-\E[J(s, t)]+\frac{1}{2}\Var(J(s, t))}\right] \nw
&= e^{-\E[J(s, t)]+\frac{1}{2}\Var(J(s, t))}\times\left[\Cov(-\gamma(t), b(s))+(\E[b(s)]+f(s)+\Cov(b(s), J(s, t)))\times \right. \nq
&\quad \left.\left(\hat\gamma - \E[\gamma(t)]+\frac{1}{2}\frac{\partial}{\partial t}\Var(J(s, t))\right)\right].
\end{split}
\]

且在上两式中,
\[
\begin{split}
\frac{\partial}{\partial t}&\Var(J(s, t)) = \frac{\partial}{\partial t}\left[\frac{1}{d_\gamma^2}\left(e^{-d_\gamma(s-t_0)}-e^{-d_\gamma(t-t_0)}\right)^2\Var(\gamma_0)\right. \nq
&\quad+ \left.\frac{\sigma_\gamma^2}{d_\gamma^3}\left(-1+d_\gamma(t-s)+e^{-d_\gamma(s+t-2t_0)}+e^{-d_\gamma(t-s)}-\frac{1}{2}e^{-2d_\gamma(t-t_0)}-\frac{1}{2}e^{-2d_\gamma(s-t_0)}\right)\right] \nq
&= \frac{\sigma_\gamma^2}{d_\gamma^2}\left(1-e^{-d_\gamma(t-s)}-e^{-d_\gamma(t+s-2t_0)}+e^{-2d_\gamma(t-t_0)}\right)+\frac{2}{d_\gamma}\Var(\gamma_0)\times\left(e^{-d_\gamma(t+s-2t_0)}-e^{-2d_\gamma(t-t_0)}\right).
\end{split}
\]
\\

\subsubsection{$\Cov(u(t), b(t))$}
% 下面考虑$Cov(u(t), b(t))$. 
由定义,
$
\Cov(u(t), b(t)) = \E[u(t)b^*(t)] - \E[u(t)]\E[b(t)]^*.
$
第二项由(\ref{E u(t)})和(\ref{b_stat_1})即可计算得到. 结合It\^o公式, 完全类似之前的计算过程, 第一项为
\[
\begin{split}
\E&[u(t)b^*(t)] = \E\left[u(t)\left(\hat{b}^*+(b_0^*-\hat{b}^*)e^{\lambda_b^*(t-t_0)}+\sigma_b\int_{t_0}^t e^{\lambda_b(t-s)}dW_b(s)\right)\right] \nq
&= \left(1-e^{\lambda_b^*(t-t_0)}\right)\hat{b}^*\E[u(t)] + \E\left[u(t)b_0^*e^{\lambda_b^*(t-t_0)}\right] \nq
&\quad + \E[\sigma\sigma_b\int_{t_0}^t\int_{t_0}^te^{-J(s, t)+\hat\lambda(s-t_0)+\lambda_b^*(t-\xi)}dW(s)dW_b(\xi)] \nq
&= \left(1-e^{\lambda_b^*(t-t_0)}\right)\hat{b}^*\E[u(t)] + \E\left[\left(e^{-J(t_0, t)+\hat\lambda(s-t_0)}+\int_{t_0}^t (b(s)+f(s))e^{-J(s, t)+\hat\lambda(s-t_0)}ds\right)b_0^*e^{\lambda_b^*(t-t_0)}\right] \nq
&\quad + \frac{\sigma\sigma_b}{2\gamma_b}\E\left[\int_{t_0}^t e^{-J(s, t)}e^{-\lambda(t-s)}\left(e^{\lambda_b^*(t-s)}-e^{-\ii\omega_b(t-s)-\gamma_b(s+t-2t_0)}\right)\right] \nq
&= \left(1-e^{\lambda_b^*(t-t_0)}\right)\hat{b}^*\E[u(t)] + e^{(\hat\lambda+\lambda_b^*)(t-t_0)}\E\left[u_0b_0^*e^{-J(t_0, t)}\right] + e^{\lambda_b^*(t-t_0)}\times \nq
&\quad \int_{t_0}^t e^{\hat\lambda(t-s)}\left(\E\left[b_0^*b(s)e^{-J(s, t)}\right]+f(s)\E\left[b_0e^{-J(s, t)}\right]^*ds\right) \nq
&\quad + \frac{\sigma\sigma_b}{2\gamma_b}\E\left[\int_{t_0}^t e^{-\E[J(s, t)]+\frac{1}{2}\Var(J(s, t))}e^{-\lambda(t-s)}\left(e^{\lambda_b^*(t-s)}-e^{-i\omega_b(t-s)-\gamma_b(s+t-2t_0)}\right)\right],
\end{split}
\]

利用命题(\ref{multiVariable gaussian exp 3}), 在上式中
\[
\begin{split}
\E&\left[b_0e^{-J(s, t)}\right] = e^{-\E[J(s, t)]+\frac{1}{2}\Var(J(s, t))}\times(\E[b_0]+\Cov(b_0, -J(s, t))), \nw
\E&\left[u_0b_0^*e^{-J(t_0, t)}\right] = e^{-\E[J(t_0, t)]+\frac{1}{2}\Var(J(t_0, t))}\times[\E[u_0]\E[b_0]^*+\Cov(u_0, b_0)+\E[u_0]\Cov(b_0, -J(t_0, t))^*\nq
&+\E[b_0]^*\Cov(u_0, -J(t_0, t)) + \Cov(u_0, -J(t_0, t))\Cov(b_0, -J(t_0, t))^*], \nw
\E&[b_0^*b(s)e^{-J(s, t)}] = e^{-J(s, t)+\frac{1}{2}\Var(J(s, t))}\times [\E[b_0]^*\E[b(s)]+\Cov(b_0, b(s))+\E[b_0]^*\Cov(b(s), -J(s, t))\nq
&+\E[b(s)]\Cov(b_0, -J(s, t))^*+\Cov(b_0, -J(s, t))^*\Cov(b(s), -J(s, t))].
\end{split}
\]
\\

\subsubsection{$\Cov(u(t), b^*(t))$}

% 最后我们考虑$Cov(u(t), b^*(t))$.
由定义,
$
\Cov(u(t), b^*(t)) = \E[u(t)b(t)] - \E[u(t)]\E[b(t)].
$

和前一部分类似的, 只要考虑上式的第一项.我们有
\[
\begin{split}
\E&[u(t)b(t)] = \E\left[u(t)\left(\hat{b}+(b_0-\hat{b})e^{\lambda_b(t-t_0)}+\sigma_b\int_{t_0}^t e^{\lambda_b(t-s)}dW_b(s)\right)\right] \nq
&= \left(1-e^{\lambda_b(t-t_0)}\right)\hat{b}\E[u(t)] + \E\left[u(t)b_0e^{\lambda_b(t-t_0)}\right] \nq
&\quad + \E[\sigma\sigma_b\int_{t_0}^t\int_{t_0}^te^{-J(s, t)+\hat\lambda(s-t_0)+\lambda_b(t-\xi)}dW(s)dW_b(\xi)] \nq
&= \left(1-e^{\lambda_b^*(t-t_0)}\right)\hat{b}\E[u(t)] + \E\left[\left(e^{-J(t_0, t)+\hat\lambda(s-t_0)}+\int_{t_0}^t (b(s)+f(s))e^{-J(s, t)+\hat\lambda(s-t_0)}ds\right)b_0e^{\lambda_b(t-t_0)}\right] \nq
&= \left(1-e^{\lambda_b(t-t_0)}\right)\hat{b}\E[u(t)] + e^{(\hat\lambda+\lambda_b)(t-t_0)}\E\left[u_0b_0e^{-J(t_0, t)}\right] + e^{\lambda_b(t-t_0)}\times \nq
&\quad \int_{t_0}^t e^{\hat\lambda(t-s)}\left(\E\left[b_0b(s)e^{-J(s, t)}\right]+f(s)\E\left[b_0e^{-J(s, t)}\right]ds\right) \nq
\end{split}
\]

在上式中, 由命题(\ref{multiVariable gaussian exp 3})知,
\[
\begin{split}
\E&\left[u_0b_0e^{-J(t_0, t)}\right] = e^{-\E[J(t_0, t)]+\frac{1}{2}\Var(J(t_0, t))}\times[\E[u_0]\E[b_0]+\Cov(u_0, b_0^*)+\E[b_0]\Cov(u_0, -J(t_0, t))\nq
&\quad + \E[u_0]\Cov(b_0, -J(t_0, t))+\Cov(u_0, -J(t_0, t))\Cov(b_0, -J(t_0, t))], \nw
\E&\left[b_0b(s)e^{-J(s, t)}\right] = e^{-J(s, t)+\frac{1}{2}\Var(J(s, t))}\times [\E[b_0]\E[b(s)]+\Cov(b_0, b(s)^*)+\E[b_0]\Cov(b(s), -J(s, t)) \nq
&\quad + \E[b(s)]\Cov(b_0, -J(s, t)) + \Cov(b_0, -J(s, t))\Cov(b(s), -J(s, t))].
\end{split}
\]

% \newcommand{\Skew}{\text{Skew}}
% \section{三阶矩}
% \subsection{偏度}
% 由定义,
% \[
% \Skew
% \]

\section{数值模拟和讨论}
% 考虑以下的这样一个简单例子. 在(\ref{E0})中, 我们不妨假设外部驱动力$f(t)$为一个周期驱动力,即
% $$
% f(t) = sin(t),
% $$
% 且方程组中的各个参数可以取为
% \begin{equation}
% \left\{
% \begin{aligned}
% &\omega = 2, \sigma = \frac{1}{2} \\
% &\omega_b = 5, \sigma_b = \frac{1}{2} \\
% &\gamma_b = 1, d_\gamma = 2 , \sigma_\gamma = \frac{1}{3} \\
% &\hat\gamma = 1, \hat{b} = 2+i \\
% \end{aligned}
% \right.
% \end{equation}

% 以及初始参数
% \begin{equation}
% \left\{
% \begin{aligned}
% &Re(u_0), Im(u_0), Re(b_0), Im(b_0) \sim \mathcal{N}(0, 1) \\
% &\gamma_0 \sim \mathcal{N}(0, 1) \\
% &Cov(u_0, \gamma_0) = \frac{1+i}{2} \\
% &Cov(u_0, b_0) = \frac{2+i}{3} \\
% &Cov(u_0, b_0^*) = \frac{1+2i}{3}
% \end{aligned}
% \right.
% \end{equation}

% 那么我们有
% \begin{equation}
% \begin{split}
% Cov(u_0, u_0^*) &= E\left[Re(u_0)^2+Im(u_0)^2\right] - \left(E[Re(u_0)]^2+E[Im(u_0)]^2\right) \\
% &= Var(Re(u_0))+Var(Im(u_0)) = 2
% \end{split}
% \end{equation}
\subsection{参数选择}
在实践中, 我们可以考虑以下的一个简单例子. 在(\ref{E0})中, 我们令外界驱动力
$$f(t) = \frac{3}{2}e^{0.1\ii t},$$方程组的参数可以取为
\begin{equation} \label{parameters}
\left\{
\begin{aligned}
&d = 1.5,  &d_\gamma = 0.01d\\
&\sigma = 0.1549,& \omega = 1.78 \\
&\sigma_\gamma = 5\sigma, &\gamma_b = 0.1d \\
&\sigma_b = 5\sigma, &\omega_b = \omega \\
&\hat{b} = 0, &\hat\gamma = 0
\end{aligned}
\right.
\end{equation}

而问题的初值条件可以做一个简单的假设, 即
\begin{equation}
\left\{
\begin{aligned}
&\text{Re}(u_0), ~\text{Im}(u_0) \sim \mathcal{N}(0, 1), ~\text{i.i.d.} \\
&\text{Re}(b_0), ~\text{Im}(b_0) \sim \mathcal{N}(0, 1), ~\text{i.i.d.} \\
&\gamma_0 \sim \mathcal{N}(0, 1)
\end{aligned}
\right.
\end{equation}

那么初始变量间的统计量为
\begin{equation}
\left\{
\begin{aligned}
&\Cov(u_0, u_0^*) = 0 \\
&\Cov(u_0, \gamma_0) = 0 \\
&\Cov(u_0, b_0) = 0 \\
&\Cov(u_0, b_0^*) = 0
\end{aligned}
\right.
\end{equation}

\subsection{数值模拟}
对布朗运动进行的It\^o积分可以由Euler-Maruyama方法进行模拟\cite{higham2000mean}\cite{higham2001algorithmic}:
\begin{equation}
X_j = X_{j-1}+f(X_{j-1})\Delta t+g(X_{t-1})(\dotW(\tau_j) - \dotW(\tau_{j-1}))
\end{equation}
其中
$$
\dotW(\tau_j) - \dotW(\tau_{j-1}) = \sum_{k = jR-R+1}^{jR} \mathrm{d}\dotW_k,
$$
在上式中$R$为E-M算法的步长, 且
$$
\mathrm{d}\dotW = \sqrt{\Delta t}\times\tt{randn()}.
$$
在对$u(t)$和$b(t)$进行数值模拟时, 我们需要分别模拟其实部和虚部. 为保证三个微分方程中白噪声的强度一致, 需要令
$$\dotW_{\text{Re}(u)}(t) = \frac{1}{\sqrt{2}}\dotW(t),$$
其余项和初值也做相同的处理.

\quad 首先, 对$R=1$进行$10^{6}$次模拟\cite{robert2004monte}, $u(t)$, $b(t)$的实部和虚部,$\gamma(t)$的期望, $u(t)$, $b(t)$和$\gamma(t)$的方差, 以及各项协方差的模拟结果如下图\cite{mckinney2012python}\cite{hunter2007matplotlib}.
% \begin{figure}[!ht]
% \centering
% \includegraphics[width = 12cm]{NumericalSimulation/res/EuRe.png}
% \caption{Simulation of E[Re(u)]}
% \label{Simulation of E[Re(u)]}
% \end{figure}

% \begin{figure}[!ht]
% \centering
% \includegraphics[width = 12cm]{NumericalSimulation/res/EuIm.png}
% \caption{Simulation of E[Im(u)]}
% \label{Simulation of E[Im(u)]}
% \end{figure}

% \begin{figure}[!ht]
% \centering
% \includegraphics[width = 12cm]{NumericalSimulation/res/EbRe.png}
% \caption{Simulation of E[Re(b)]}
% \label{Simulation of E[Re(b)]}
% \end{figure}

% \begin{figure}[!ht]
% \centering
% \includegraphics[width = 12cm]{NumericalSimulation/res/EbIm.png}
% \caption{Simulation of E[Im(b)]}
% \label{Simulation of E[Im(b)]}
% \end{figure}

% \begin{figure}[!ht]
% \centering
% \includegraphics[width = 12cm]{NumericalSimulation/res/Egamma.png}
% \caption{Simulation of E[$\gamma$]}
% \label{Simulation of E[gamma]}
% \end{figure}

% \begin{figure}[!ht]
% \centering
% \includegraphics[width = 12cm]{NumericalSimulation/res/Varu.png}
% \caption{Simulation of Var(u)}
% \label{Simulation of Var(u)}
% \end{figure}

% \begin{figure}[!ht]
% \centering
% \includegraphics[width = 12cm]{NumericalSimulation/res/Varb.png}
% \caption{Simulation of Var(b)}
% \label{Simulation of Var(b)}
% \end{figure}

% \begin{figure}[!ht]
% \centering
% \includegraphics[width = 12cm]{NumericalSimulation/res/Vargamma.png}
% \caption{Simulation of Var($\gamma$)}
% \label{Simulation of Var(gamma)}
% \end{figure}

\begin{figure}[!ht]
\centering
\subfloat[\text{E}(Re(u))]{
  \includegraphics[width = .2\textwidth]{NumericalSimulation/100k/EuRe.png}
  }\hfill
\subfloat[E(Im(u))]{
  \includegraphics[width = .2\textwidth]{NumericalSimulation/100k/EuIm.png}
  }\hfill
\subfloat[E(Re(b))]{
  \includegraphics[width = .2\textwidth]{NumericalSimulation/100k/EbRe.png}
  }\hfill
\subfloat[E(Im(b))]{
  \includegraphics[width = .2\textwidth]{NumericalSimulation/100k/EbIm.png}
  }\\
\subfloat[E($\gamma$)]{
  \includegraphics[width = .2\textwidth]{NumericalSimulation/100k/Egamma.png}
  }\hfill
\subfloat[Var(u)]{
  \includegraphics[width = .2\textwidth]{NumericalSimulation/100k/Varu.png}
  }\hfill
\subfloat[Var(b)]{
  \includegraphics[width = .2\textwidth]{NumericalSimulation/100k/Varb.png}
  }\hfill
\subfloat[Var($\gamma$)]{
  \includegraphics[width = .2\textwidth]{NumericalSimulation/100k/Vargamma.png}
  }
\caption{Simulation of Expectations and Variances, $n = 10^{6}$}
\label{Simulation of Expectations and Variances}
\end{figure}

\begin{figure}[!ht]
\centering
\subfloat[$\text{Re}(\Cov(u, u^*))$]{
  \includegraphics[width = .2\textwidth]{NumericalSimulation/100k/covuustarre.png}
}\hfill
\subfloat[$\text{Im}(\Cov(u, u^*))$]{
  \includegraphics[width = .2\textwidth]{NumericalSimulation/100k/covuustarim.png}
}\hfill
\subfloat[$\text{Re}(\Cov(u, \gamma))$]{
  \includegraphics[width = .2\textwidth]{NumericalSimulation/100k/covugammare.png}
}\hfill
\subfloat[$\text{Im}(\Cov(u, \gamma))$]{
  \includegraphics[width = .2\textwidth]{NumericalSimulation/100k/covugammaim.png}
}\\
\subfloat[$\text{Re}(\Cov(u, b))$]{
  \includegraphics[width = .2\textwidth]{NumericalSimulation/100k/covubre.png}
}\hfill
\subfloat[$\text{Im}(\Cov(u, b))$]{
  \includegraphics[width = .2\textwidth]{NumericalSimulation/100k/covubim.png}
}\hfill
\subfloat[$\text{Re}(\Cov(u, b^*))$]{
  \includegraphics[width = .2\textwidth]{NumericalSimulation/100k/covubstarre.png}
}\hfill
\subfloat[$\text{Im}(\Cov(u, b^*))$]{
  \includegraphics[width = .2\textwidth]{NumericalSimulation/100k/covubstarim.png}
}
\caption{Simulation of Covariances, $n = 10^{6}$}
\label{Simulation of Covariances}
\end{figure}

\newpage
下面我们测试了模拟次数为$n = 10^{7}$时的期望, 方差和协方差的模拟结果:

\begin{figure}[!ht]
\centering
\subfloat[\text{E}(Re(u))]{
  \includegraphics[width = .2\textwidth]{NumericalSimulation/1m/EuRe.png}
  }\hfill
\subfloat[E(Im(u))]{
  \includegraphics[width = .2\textwidth]{NumericalSimulation/1m/EuIm.png}
  }\hfill
\subfloat[E(Re(b))]{
  \includegraphics[width = .2\textwidth]{NumericalSimulation/1m/EbRe.png}
  }\hfill
\subfloat[E(Im(b))]{
  \includegraphics[width = .2\textwidth]{NumericalSimulation/1m/EbIm.png}
  }\\
\subfloat[E($\gamma$)]{
  \includegraphics[width = .2\textwidth]{NumericalSimulation/1m/Egamma.png}
  }\hfill
\subfloat[Var(u)]{
  \includegraphics[width = .2\textwidth]{NumericalSimulation/1m/Varu.png}
  }\hfill
\subfloat[Var(b)]{
  \includegraphics[width = .2\textwidth]{NumericalSimulation/1m/Varb.png}
  }\hfill
\subfloat[Var($\gamma$)]{
  \includegraphics[width = .2\textwidth]{NumericalSimulation/1m/Vargamma.png}
  }
\caption{Simulation of Expectations and Variances, $n = 10^{7}$}
\label{Simulation of Expectations and Variances}
\end{figure}


\begin{figure}[!ht]
\centering
\subfloat[$\text{Re}(\Cov(u, u^*))$]{
  \includegraphics[width = .2\textwidth]{NumericalSimulation/1m/CovuustarRe.png}
}\hfill
\subfloat[$\text{Im}(\Cov(u, u^*))$]{
  \includegraphics[width = .2\textwidth]{NumericalSimulation/1m/CovuustarIm.png}
}\hfill
\subfloat[$\text{Re}(\Cov(u, \gamma))$]{
  \includegraphics[width = .2\textwidth]{NumericalSimulation/1m/CovugammaRe.png}
}\hfill
\subfloat[$\text{Im}(\Cov(u, \gamma))$]{
  \includegraphics[width = .2\textwidth]{NumericalSimulation/1m/CovugammaIm.png}
}\\
\subfloat[$\text{Re}(\Cov(u, b))$]{
  \includegraphics[width = .2\textwidth]{NumericalSimulation/1m/CovubRe.png}
}\hfill
\subfloat[$\text{Im}(\Cov(u, b))$]{
  \includegraphics[width = .2\textwidth]{NumericalSimulation/1m/CovubIm.png}
}\hfill
\subfloat[$\text{Re}(\Cov(u, b^*))$]{
  \includegraphics[width = .2\textwidth]{NumericalSimulation/1m/CovubstarRe.png}
}\hfill
\subfloat[$\text{Im}(\Cov(u, b^*))$]{
  \includegraphics[width = .2\textwidth]{NumericalSimulation/1m/CovubstarIm.png}
}
\caption{Simulation of Covariances, $n = 10^{7}$}
\label{Simulation of Covariances}
\end{figure}




\subsection{讨论和展望}
从数值模拟结果我们可以看出, 当模拟次数$n = 10^{6}$时, 对期望的模拟结果和理论值基本重合, 方差的模拟结果和理论值有着$O(10^{-2})\sim O(10^{-3})$的误差, 而协方差的误差在$O(10^{-3})\sim O(10^{-4})$的数量级. 而当模拟次数$n = 10^{7}$时, 期望和方差的模拟结果和理论值几乎完全一致, $\Cov(u, u^*)$及$\Cov(u, \gamma)$的误差为$O(10^{-3})$, $\Cov(u, b)$和$\Cov(u, b^*)$的误差为$O(10^{-4})$. 由模拟结果可以知道, 这些统计量的误差随着模拟次数的增加而降低, 故我们有足够的理由认为当$n \to \infty$时其模拟结果和理论解会趋于一致. 这也说明了我们的理论解是符合实际的. 

事实上, 误差的来源在于理论结果中$u_0, ~b_0, ~\gamma_0$的统计量按照定值进行计算, 而在模拟中$u_0, ~b_0, ~\gamma_0$均为随机变量, 其统计量和理论结果有一定的误差. 根据大数定律\cite[Page 214-221]{shuyuanhe2006probability}此误差随模拟次数的增大而收敛至$0$.


% \begin{figure}[!ht]
% \centering
% \subfloat[$\text{Re}(\Cov(u, u^*))$]{
%   \includegraphics[width = .2\textwidth]{NumericalSimulation/res/CovuustarRe.png}
% }\hfill
% \subfloat[$\text{Im}(\Cov(u, u^*))$]{
%   \includegraphics[width = .2\textwidth]{NumericalSimulation/res/CovuustarIm.png}
% }\hfill
% \subfloat[$\text{Re}(\Cov(u, \gamma))$]{
%   \includegraphics[width = .2\textwidth]{NumericalSimulation/res/CovugammaRe.png}
% }\hfill
% \subfloat[$\text{Im}(\Cov(u, \gamma))$]{
%   \includegraphics[width = .2\textwidth]{NumericalSimulation/res/CovugammaIm.png}
% }\\
% \subfloat[$\text{Re}(\Cov(u, b))$]{
%   \includegraphics[width = .2\textwidth]{NumericalSimulation/res/CovubRe.png}
% }\hfill
% \subfloat[$\text{Im}(\Cov(u, b))$]{
%   \includegraphics[width = .2\textwidth]{NumericalSimulation/res/CovubIm.png}
% }\hfill
% \subfloat[$\text{Re}(\Cov(u, b^*))$]{
%   \includegraphics[width = .2\textwidth]{NumericalSimulation/res/CovubstarRe.png}
% }\hfill
% \subfloat[$\text{Im}(\Cov(u, b^*))$]{
%   \includegraphics[width = .2\textwidth]{NumericalSimulation/res/CovubstarIm.png}
% }
% \caption{Simulation of Covariances, Detailed}
% \label{Simulation of Covariances}
% \end{figure}

% 我们可以发现, 对$\Cov(u, b)$和$\Cov(u, b^*)$的模拟结果在模拟100,000次时有$O(10^{-4})$的误差, 而对$\Cov(u, u^*)$和$\Cov(u, \gamma)$的模拟误差达到了$O(10^{-2})$的数量级. 根据分析\cite{golub2012matrix}, 这是因为复数运算时的条件数为实数运算的二次方导致的.

此外, 根据相同的方法, 我们可以求出此随机微分方程组解的三节甚至四阶的统计量, 它们可以用初值和初始统计量的多重积分表示出来. 不过由于时间关系, 在这篇论文中没有将其详细的写出, 也没有对其进行数值上的模拟. 这些部分我们留待未来的研究中完成.
% \begin{figure}[!ht]s
% \centering
% \includegraphics[width = 12cm]{NumericalSimulation/res/CovuustarRe.png}
% \caption{Simulation of $Re(Cov(u, u^*))$}
% \label{Simulation of Re(Cov(u, u*))}
% \end{figure}

% \begin{figure}[!ht]
% \centering
% \includegraphics[width = 12cm]{NumericalSimulation/res/CovuustarIm.png}
% \caption{Simulation of $Im(Cov(u, u^*))$}
% \label{Simulation of Im(Cov(u, u*))}
% \end{figure}

% \begin{figure}[!ht]
% \centering
% \includegraphics[width = 12cm]{NumericalSimulation/res/CovugammaRe.png}
% \caption{Simulation of $Re(Cov(u, \gamma))$}
% \label{Simulation of Re(Cov(u, gamma))}
% \end{figure}

% \begin{figure}[!ht]
% \centering
% \includegraphics[width = 12cm]{NumericalSimulation/res/CovugammaIm.png}
% \caption{Simulation of $Im(Cov(u, \gamma))$}
% \label{Simulation of Im(Cov(u, gamma))}
% \end{figure}

% \begin{figure}[!ht]
% \centering
% \includegraphics[width = 12cm]{NumericalSimulation/res/CovubRe.png}
% \caption{Simulation of $Re(Cov(u, b)$}
% \label{Simulation of Re(Cov(u, b)}
% \end{figure}

% \begin{figure}[!ht]
% \centering
% \includegraphics[width = 12cm]{NumericalSimulation/res/CovubIm.png}
% \caption{Simulation of $Im(Cov(u, b))$}
% \label{Simulation of Im(Cov(u, b))}
% \end{figure}

% \begin{figure}[!ht]
% \centering
% \includegraphics[width = 12cm]{NumericalSimulation/res/CovubstarRe.png}
% \caption{Simulation of $Re(Cov(u, b^*))$}
% \label{Simulation of Re(Cov(u, b*))}
% \end{figure}

% \begin{figure}[!ht]
% \centering
% \includegraphics[width = 12cm]{NumericalSimulation/res/CovubstarIm.png}
% \caption{Simulation of $Im(Cov(u, b^*))$}
% \label{Simulation of Im(Cov(u, b*))}
% \end{figure}
\bibliographystyle{plain}
\bibliography{../../CONFIG/LaTeX-bib/Chuan}

\chapter*{\heiti 致谢}
\begin{flushleft}
感谢陆帅老师对本文研究方向的指导和帮助; \\
感谢唐博浩和宋杰承同学对本文的技术性建议; \\
感谢魏益民和赵冬华老师对我学习和研究的关心与支持; \\
感谢我的室友们, 大学和高中的同学们, 朋友和家人的陪伴; \\
最后, 感谢东区18号楼下的猫在最困难的的日子里给予我的慰藉.
\end{flushleft}

\end{document}{}