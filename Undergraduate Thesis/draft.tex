%!TEX program = pdfLaTeX 

%%%%%%%%%%%%%%%%%%%%%%%%%%%%%%%%%%%%%%%%%%%%%%%%%%%%%%%%%%%%%%%%%%%%%%%%%%%%%%%%%%%%5%%%%%%%%%%%%%%%
%  本文档可在安装了CTEX宏包, CTEX字体下的TEX系统运行,
%  访问http://www.ctex.org, 可以获得最新的宏包与字体安装包
%
%  请使用PDFLATEX对模板编译2次, 可得正确结果, 由于hyperref的设置中不支持DVI-PDF,
%  用LATEX编译时需要替换相应的命令, 详见相应注释.
%
% 文档是在原来李湛、何力同学的模板的基础上修改的, 主要包括以下几个地方:
%
%1.修正了原模板使用hyperref宏包中的设置, 使文档更加美观, 对设置作出了说明, 可以进一步修改
%2.修正了定理的样式, 原定理标题是黑体加粗, 现改为黑体, 原定理正文为倾斜楷体, 现改为楷体, 符合一般论文的格式
%3.对导言区的少部分命令修改, 删去了一些默认的重复的设置
%4.对模板的少部分正文进行充实
%5.对部分原来模板中的注释进行了修改, 删去了不必要的, 加入了一些中文的注释, 方便查阅
%
%  by 张越 Apr.12, 有问题请发送你的问题到:frank_melody@hotmail.com
%%%%%%%%%%%%%%%%%%%%%%%%%%%%%%%%%%%%%%%%%%%%%%%%%%%%%%%%%%%%%%%%%%%%
%%%%%%%%%%%%%%%%%%%%%%%%%%%%%%%%%%%%%%%%%%%%%%%%%%%%%%%%%%%%%%%%%%%%%%
% documentclass can be ctexart, ctexrep, ctexbook, 推荐使用模板中的CTEXREP
% cs4size - 默认的字体大. ∷
% punct - 对中文标点的位置(宽度)进行调整
% twoside - if you want to print on both side of the paper, or else you should omit this

\documentclass[notitlepage,cs4size,punct,oneside]{ctexrep}

% default paper settings, change it according to your word
\usepackage[a4paper,hmargin={2.54cm,2.54cm},vmargin={3.17cm,3.17cm}]{geometry}

\usepackage{amsmath,amssymb,amsthm}

% 公式编号的计数格式, 在章内计数
\numberwithin{equation}{section}

% set the abstract format, need abstract package

\usepackage[runin]{abstract}

%使用hyperref宏包, 对目录, 公式引用, 文献引用做超链接, 超链接方便电子版的阅读, 但不影响打印
% pdfborder对超链接的边框大小进行设置, 模板中默认边框大小为0
% colorlinks=true, 表示超链接对应的文字采用超链接边框的颜色, =false时保持原字体颜色
% linkcolor=blue, 设置超链接边框的颜色, 可以改为red,green等等.
% CJKbookmarks=true, 生成PDF中文书签,
% 非CTEX套装用户可能发现即便如此设置, 生成的PDF书签也是乱码, 需要用GBK2UNI.EXE解决
\usepackage[pdfborder={0 0 0},colorlinks=true,linkcolor=blue,CJKbookmarks=true]{hyperref}
%若要用LATEX编译, 请用下面的命令替代上述命令:
%\usepackage[dvipdfm,pdfborder={0 0 0},colorlinks=true,linkcolor=blue,CJKbookmarks=true]{hyperref}

\setlength{\absleftindent}{1.5cm} \setlength{\absrightindent}{1.5cm}
\setlength{\abstitleskip}{-\parindent}
\setlength{\absparindent}{0cm}

% Theorem style
\newtheoremstyle{mystyle}{3pt}{3pt}{\kaishu}{0cm}{\heiti}{}{1em}{}
\theoremstyle{mystyle}

\newtheorem{definition}{\hspace{2em}定义}[section]
% 如果没有章, 只有节, 把上面的[chapter]改成[section]
\newtheorem{theorem}[definition]{\hspace{2em}定理}
\newtheorem{axiom}[definition]{\hspace{2em}公理}
\newtheorem{lemma}[definition]{\hspace{2em}引理}
\newtheorem{proposition}[definition]{\hspace{2em}命题}
\newtheorem{corollary}[definition]{\hspace{2em}推论}
\newtheorem{remark}{\hspace{2em}注}[section]
%类似地定义其他“题头”. 这里“注”的编号与定义、定理等是分开的

\def\theequation{\arabic{section}.\arabic{equation}}
\setcounter{equation}{0}
\def\thedefinition{\arabic{section}.\arabic{definition}.}

% title - \zihao{1} for size requirement \heiti for font family requirement
\title{{\zihao{1}\heiti{} }}

\author{卢川,13300180056,信息与计算科学}

\date{}
%%%%%%%%%%%%%%%%%%%导言区设置完毕
%%%%%%%%%%%%%%%%%%%%%%%%%%%%%%%%%%%%%%%%%%%%%%%%%%%%%%%%%%%%%%%%%%%%%
\begin{document}
%Styles for chapters/section
%若要将章标题左对齐, 用下面这个语句替换相应的设置
%\CTEXsetup[nameformat={\raggedright\zihao{3}\bfseries},%
\CTEXsetup[nameformat={\zihao{3}\heiti},%
           titleformat={\zihao{3}},%
           beforeskip={0.8cm},afterskip={1.2cm}]{chapter}
\CTEXsetup[nameformat={\zihao{4}\bfseries},%
           titleformat={\zihao{4}},%
           name={第~,~节},number={\arabic{section}},%
           beforeskip={0.4cm},afterskip={0.4cm}]{section}
\CTEXsetup[format={\zihao{-4}\bfseries},%
           titleformat={\zihao{-4}},%
           number={\arabic{section}.\arabic{subsection}.},%
           beforeskip={0.4cm},afterskip={0.4cm}]{subsection}
\CTEXoptions[abstractname={摘要:}]
\CTEXoptions[bibname={\heiti 参考文献}]

\renewcommand{\thepage}{\roman{page}}
\setcounter{page}{1}
\tableofcontents\clearpage

\maketitle\renewcommand{\thepage}{\arabic{page}}
\thispagestyle{empty}\setcounter{page}{0}
%%%  论文的页码从正文开始计数, 摘要页不显示页码
% 撰写论文的摘要

\section{导言}
\subsection{问题简介}
考虑以下的一个随机微分方程组:
\begin{equation} \label{E0}
\begin{split}
& \frac{du(t)}{dt} = (-\gamma(t)+i\omega)u(t)+b(t)+f(t)+\sigma W(t), \\
% \end{equation}
% \begin{equation}
& \frac{db(t)}{dt} = (-\gamma_b+i\omega_b)(b(t)-\hat{b})+\sigma_b W_b(t), \\
% \end{equation}
% \begin{equation}
& \frac{d\gamma(t)}{dt} = -d_\gamma(\gamma(t)-\hat{\gamma})+\sigma_\gamma W_\gamma(t)
\end{split}
\end{equation}
由求解线性微分方程组的相关知识\cite{fulinjin1984ordinary}可以知道, (\ref{E0})的第二和第三项均为线性方程, 故有通解
\begin{equation} \label{Sb}
b(t) = \hat{b}+(b_0-\hat{b})e^{\lambda_b(t-t_0)}+\sigma_b\int_{t_0}^t e^{\lambda_b(t-s)}dW_b(s),
\end{equation}
\begin{equation} \label{Sg}
\gamma(t) = \hat{\gamma}+(\gamma_0-\hat{\gamma})e^{-d_{\gamma}(t-t_0)}+\sigma_{\gamma}\int_{t_0}^t e^{-d_{\gamma}(t-s)}dW_{\gamma}(s).
\end{equation}
其中$\lambda_b = -\gamma_b+i\omega_b$, $\hat{b}$与$\hat{\gamma}$分别是$b(t)$和$\gamma(t)$的固定偏差校正.
如果我们记
$$\hat{\lambda} = -\hat{\gamma}+i\omega, $$
$$J(s, t) = \int_s^t (\gamma(s')-\hat{\gamma})ds'$$
则(\ref{E0})的通解可以表示为
\begin{equation} \label{Su}
\begin{split}
u(t) &= e^{-J(t_0, t)+\hat{\lambda}(t-t_0)}u_0 + \int_{t_0}^t (b(s)+f(s))e{-J(s, t)+\hat{\lambda}(s-t_0)}ds \\
& + \sigma\int_{t_0}^{t}e{-J(s, t)+\hat{\lambda}(s-t_0)}dW(s).
\end{split}
\end{equation}

\subsection{一些前置引理}
在应用中, 往往将布朗运动置于一个随机微分方程(组)中, 来近似的模拟白噪声的性质\cite{hida1980brownian}. 在求解这样的随机微分方程时, 需要针对布朗运动做积分, 为此我们引入以下的It\^{o}积分\cite{oksendal2003stochastic}:
\begin{definition}(It\^{o}积分)	设$f\in\mathcal{V}(S, T)$. 则$f$的It\^{o}积分定义为
$$\int_S^T f(t, \omega)dB_t(\omega) = \lim_{n\to\infty}\int_S^T\phi_n(t, \omega)dB_t(\omega),$$
其中${\phi_n}$为基本函数序列, 且满足当$n \to \infty$时
$$E[\int_S^T (f(t, \omega)-\phi_n(t, \omega))^2dt] \to 0.$$
\end{definition}
由上述定义即可得到It\^{o}积分的一个重要性质:
\begin{theorem}(It\^{o}等距) 对于$\forall f\in\mathcal{V}(S, T)$有
$$E[(\int_S^T f(t, \omega)dB_t)^2] = E[\int_S^T f^2(t, \omega)dt].$$
\end{theorem}
为了便于计算, 我们还需要引入以下的一维It\^{o}公式.
\begin{theorem}(It\^{o}公式) 设$X_t$为一个如下的It\^{o}过程:
$$dX_t = udt+vdB_t,$$
$g(t, x) \in \mathit{C}^2([0, \infty) \times \mathbb{R}$, 则$Y_t = g(t, X_t)$也是一个It\^{o}过程, 且满足:
$$dY_t = \frac{\partial g}{\partial t}(t, X_t)dt + \frac{\partial g}{\partial x}(t, X_t)dX_t + \frac{1}{2}\frac{\partial^2 g}{\partial x^2}(t, X_t)\cdot(dX_t)^2,$$
其中
$$dt\cdot dt = dt\cdot dB_t = dB_t\cdot dt = 0, \quad dB_t\cdot dB_t = dt.$$
\end{theorem}
以下是上述It\^{o}公式的积分形式.
\begin{theorem}(分部积分) 设$f(s, \omega)$对几乎所有的$\omega$关于$s\in [0, t]$是连续的, 且为有界变差函数, 则有
$$\int_0^t f(s)dB_s = f(t)B_t - \int_0^t B_s df_s.$$
\end{theorem}
上述几个定理的证明请见\cite{oksendal2003stochastic}, 21-24页及36-39页.

\section{一阶统计量}
\

\bibliographystyle{plain}
\bibliography{../../CONFIG/LaTeX-bib/Chuan}
\end{document}
