%!TEX program = pdfLaTeX 

%%%%%%%%%%%%%%%%%%%%%%%%%%%%%%%%%%%%%%%%%%%%%%%%%%%%%%%%%%%%%%%%%%%%%%%%%%%%%%%%%%%%5%%%%%%%%%%%%%%%
%  本文档可在安装了CTEX宏包, CTEX字体下的TEX系统运行,
%  访问http://www.ctex.org, 可以获得最新的宏包与字体安装包
%
%  请使用PDFLATEX对模板编译2次, 可得正确结果, 由于hyperref的设置中不支持DVI-PDF,
%  用LATEX编译时需要替换相应的命令, 详见相应注释.
%
% 文档是在原来李湛、何力同学的模板的基础上修改的, 主要包括以下几个地方:
%
%1.修正了原模板使用hyperref宏包中的设置, 使文档更加美观, 对设置作出了说明, 可以进一步修改
%2.修正了定理的样式, 原定理标题是黑体加粗, 现改为黑体, 原定理正文为倾斜楷体, 现改为楷体, 符合一般论文的格式
%3.对导言区的少部分命令修改, 删去了一些默认的重复的设置
%4.对模板的少部分正文进行充实
%5.对部分原来模板中的注释进行了修改, 删去了不必要的, 加入了一些中文的注释, 方便查阅
%
%  by 张越 Apr.12, 有问题请发送你的问题到:frank_melody@hotmail.com
%%%%%%%%%%%%%%%%%%%%%%%%%%%%%%%%%%%%%%%%%%%%%%%%%%%%%%%%%%%%%%%%%%%%
%%%%%%%%%%%%%%%%%%%%%%%%%%%%%%%%%%%%%%%%%%%%%%%%%%%%%%%%%%%%%%%%%%%%%%
% documentclass can be ctexart, ctexrep, ctexbook, 推荐使用模板中的CTEXREP
% cs4size - 默认的字体大. ∷
% punct - 对中文标点的位置(宽度)进行调整
% twoside - if you want to print on both side of the paper, or else you should omit this

\documentclass[notitlepage,cs4size,punct,oneside]{ctexrep}

% default paper settings, change it according to your word
\usepackage[a4paper,hmargin={2.54cm,2.54cm},vmargin={3.17cm,3.17cm}]{geometry}

\usepackage{amsmath,amssymb,amsthm}

% 公式编号的计数格式, 在章内计数
\numberwithin{equation}{section}

% set the abstract format, need abstract package

\usepackage[runin]{abstract}

%使用hyperref宏包, 对目录, 公式引用, 文献引用做超链接, 超链接方便电子版的阅读, 但不影响打印
% pdfborder对超链接的边框大小进行设置, 模板中默认边框大小为0
% colorlinks=true, 表示超链接对应的文字采用超链接边框的颜色, =false时保持原字体颜色
% linkcolor=blue, 设置超链接边框的颜色, 可以改为red,green等等.
% CJKbookmarks=true, 生成PDF中文书签,
% 非CTEX套装用户可能发现即便如此设置, 生成的PDF书签也是乱码, 需要用GBK2UNI.EXE解决
\usepackage[pdfborder={0 0 0},colorlinks=true,linkcolor=blue,CJKbookmarks=true]{hyperref}
%若要用LATEX编译, 请用下面的命令替代上述命令:
%\usepackage[dvipdfm,pdfborder={0 0 0},colorlinks=true,linkcolor=blue,CJKbookmarks=true]{hyperref}

\setlength{\absleftindent}{1.5cm} \setlength{\absrightindent}{1.5cm}
\setlength{\abstitleskip}{-\parindent}
\setlength{\absparindent}{0cm}

% Theorem style
\newtheoremstyle{mystyle}{3pt}{3pt}{\kaishu}{0cm}{\heiti}{}{1em}{}
\theoremstyle{mystyle}

\newtheorem{definition}{\hspace{2em}定义}[section]
% 如果没有章, 只有节, 把上面的[chapter]改成[section]
\newtheorem{theorem}[definition]{\hspace{2em}定理}
\newtheorem{axiom}[definition]{\hspace{2em}公理}
\newtheorem{lemma}[definition]{\hspace{2em}引理}
\newtheorem{proposition}[definition]{\hspace{2em}命题}
\newtheorem{corollary}[definition]{\hspace{2em}推论}
\newtheorem{remark}{\hspace{2em}注}[section]
%类似地定义其他“题头”. 这里“注”的编号与定义、定理等是分开的

\def\theequation{\arabic{section}.\arabic{equation}}
\setcounter{equation}{1}
\def\thedefinition{\arabic{section}.\arabic{definition}}

% title - \zihao{1} for size requirement \heiti for font family requirement
\title{{\zihao{1}\heiti{} TEST}}

\author{卢川,13300180056,信息与计算科学}

\date{\today}
%%%%%%%%%%%%%%%%%%%导言区设置完毕
%%%%%%%%%%%%%%%%%%%%%%%%%%%%%%%%%%%%%%%%%%%%%%%%%%%%%%%%%%%%%%%%%%%%%
\begin{document}
%Styles for chapters/section
%若要将章标题左对齐, 用下面这个语句替换相应的设置
%\CTEXsetup[nameformat={\raggedright\zihao{3}\bfseries},%
\CTEXsetup[nameformat={\zihao{3}\heiti},%
           titleformat={\zihao{3}},%
           beforeskip={0.8cm},afterskip={1.2cm}]{chapter}
\CTEXsetup[nameformat={\zihao{4}\bfseries},%
           titleformat={\zihao{4}},%
           name={第~,~节},number={\arabic{section}},%
           beforeskip={0.4cm},afterskip={0.4cm}]{section}
\CTEXsetup[format={\zihao{-4}\bfseries},%
           titleformat={\zihao{-4}},%
           number={\arabic{section}.\arabic{subsection}},%
           beforeskip={0.4cm},afterskip={0.4cm}]{subsection}
\CTEXoptions[abstractname={摘要:}]
\CTEXoptions[bibname={\heiti 参考文献}]

\renewcommand{\thepage}{\roman{page}}
\setcounter{page}{1}
\tableofcontents\clearpage

\maketitle\renewcommand{\thepage}{\arabic{page}}
\thispagestyle{empty}\setcounter{page}{1}
%%%  论文的页码从正文开始计数, 摘要页不显示页码
% 撰写论文的摘要

\begin{abstract}
test test test \\
\noindent{\heiti 关键字:} test, test.
\end{abstract}

\section{导言}
\subsection{问题简介}
考虑以下的一个随机微分方程组, 其作用为改进对模型误差的滤波\cite{gershgorin2010test}:
\begin{equation} \label{E0}
\begin{split}
& \frac{du(t)}{dt} = (-\gamma(t)+i\omega)u(t)+b(t)+f(t)+\sigma W(t), \\
% \end{equation}
% \begin{equation}
& \frac{db(t)}{dt} = (-\gamma_b+i\omega_b)(b(t)-\hat{b})+\sigma_b W_b(t), \\
% \end{equation}
% \begin{equation}
& \frac{d\gamma(t)}{dt} = -d_\gamma(\gamma(t)-\hat{\gamma})+\sigma_\gamma W_\gamma(t)
\end{split}
\end{equation}
其中$\omega$为$u(t)$的振荡频率, $f(t)$为外部驱动力, $\sigma$表示白噪声$W(t)$的强度. 此外, 参数$\gamma_b$和$d_\gamma$表示振荡阻尼, $\sigma_b$和$\sigma_\gamma$分别表示加性和乘性修正((\ref{E0})中第2, 3式)中白噪声的强度. $\hat{b}$和$\hat{\gamma}$分别表示$b(t)$和$\gamma(t)$的固定平均偏差修正, $\omega_b$表示加性噪声的频率. 白噪声$W_\gamma(t)$是实值函数, 而$W(t)$及$W_b(t)$均为复值, 且其实部和虚部均为独立的白噪声. \\

我们一般认为方程组(\ref{E0})有初值
\begin{equation} \label{init_values}
\begin{split}
& u(t_0) = u_0, \\
& b(t_0) = b_0, \\
& \gamma(t_0) = \gamma_0,
\end{split}
\end{equation}
且$u_0$, $b_0$, $\gamma_0$均为独立的高斯随机变量, 其统计量均假设为已知(注意到$b(t)$和$u(t)$均为复数):
$E[u_0]$, $E[\gamma_0]$, $E[b_0]$, $Var(u_0)$, $Var(\gamma_0)$, $Var(b_0)$, $Cov(u_0, u_0^*)$, $Cov(u_0, \gamma_0)$, $Cov(u_0, b_0)$, $Cov(u_0, b_0^*)$.

由求解线性微分方程组的相关知识\cite{fulinjin1984ordinary}可以知道, (\ref{E0})的第二和第三项均为线性方程, 故有通解
\begin{equation} \label{Sb}
b(t) = \hat{b}+(b_0-\hat{b})e^{\lambda_b(t-t_0)}+\sigma_b\int_{t_0}^t e^{\lambda_b(t-s)}dW_b(s),
\end{equation}
\begin{equation} \label{Sg}
\gamma(t) = \hat{\gamma}+(\gamma_0-\hat{\gamma})e^{-d_{\gamma}(t-t_0)}+\sigma_{\gamma}\int_{t_0}^t e^{-d_{\gamma}(t-s)}dW_{\gamma}(s).
\end{equation}
其中$\lambda_b = -\gamma_b+i\omega_b$, $\hat{b}$与$\hat{\gamma}$分别是$b(t)$和$\gamma(t)$的固定偏差校正.
如果我们记
\begin{equation}
\hat{\lambda} = -\hat{\gamma}+i\omega,
\end{equation}
\begin{equation} \label{def J(s, t)}
J(s, t) = \int_s^t (\gamma(s')-\hat{\gamma})ds'
\end{equation}
则(\ref{E0})的通解可以表示为
\begin{equation} \label{Su}
\begin{split}
u(t) &= e^{-J(t_0, t)+\hat{\lambda}(t-t_0)}u_0 + \int_{t_0}^t (b(s)+f(s))e^{-J(s, t)+\hat{\lambda}(s-t_0)}ds \\
& + \sigma\int_{t_0}^{t}e^{-J(s, t)+\hat{\lambda}(s-t_0)}dW(s).
\end{split}
\end{equation}
以下将对方程(\ref{E0})的各个变量$u, b, \gamma$, 分别计算其统计量.

\section{一些前置引理}
\subsection{布朗运动的性质}
在应用中, 往往将布朗运动置于一个随机微分方程(组)中, 来近似的模拟白噪声的性质\cite{hida1980brownian}. 
为此, 我们需要引入布朗运动的一些基本性质.\cite{oksendal2003stochastic}\cite{nelson1967dynamical}
\begin{theorem} \label{brownian1}
布朗运动$B_t$是一个Gauss过程. 对于所有的$0 \leqslant t_1 \leqslant\cdots\leqslant t_k$, 随机变量$Z = (B_{t_1}, B_{t_2}\cdots ,B_{t_k})\in \mathbb{R}^{nk}$ 服从多重正态分布. 如果设$B_t$的初值为$x$, 则其期望为 
$$M = E[Z] = (x, x, \cdots, x) \in \mathbb{R}^{nk},$$
协方差矩阵为
$$C = \left(\begin{matrix}
			t_1I_n & t_1I_n & \cdots & t_1I_n \\
			t_1I_n & t_2I_n & \cdots & t_2I_n \\
			\vdots & \vdots & \ddots & \vdots \\
			t_1I_n & t_2I_n & \cdots & t_kI_n 
			\end{matrix}
			\right).$$
\end{theorem}
此定理的证明利用了特征函数的性质, 具体过程可见\cite{oksendal2003stochastic}的附录A. 以下的定理是(\ref{brownian1})的直接而显然的推论:
\begin{theorem}
假设布朗运动$B_t$满足定理(\ref{brownian1})中的条件, 那么对于$\forall t\geqslant 0$, $E[B_t] = x$, $E\left[(B_t - x)^2\right] = nt$, $E[(B_t - x)(B_s - x)] = n\cdot min(s, t)$. 而且如果$t \geqslant s$, 有$E\left[(B_t - B_s)^2\right] = n(t-s)$.
\end{theorem}
此外, 以下的定理也是布朗运动的一个重要性质, 其证明见\cite{oksendal2003stochastic}的(2.2.12)式.
\begin{theorem}
$B_t$的增量独立, 即对满足$\forall 0 \leqslant t_1 < t_2 \cdots < t_k$的$(t_1, t_2, \cdots t_k)$,
$$B_{t_1}, B_{t_2}-B{t_1}, \cdots, B_{t_k}-B_{t_{k-1}}$$相互独立.
\end{theorem}

\subsection{It\^{o}积分及其性质}
在求解类似(\ref{E0})的随机微分方程时, 需要针对布朗运动做积分, 为此我们引入以下的It\^{o}积分\cite{oksendal2003stochastic}\cite{lawler2006introduction}:
\begin{definition}(It\^{o}积分)	设$f\in\mathcal{V}(S, T)$. 则$f$的It\^{o}积分定义为
$$\int_S^T f(t, \omega)dB_t(\omega) = \lim_{n\to\infty}\int_S^T\phi_n(t, \omega)dB_t(\omega),$$
其中${\phi_n}$为基本函数序列, 且满足当$n \to \infty$时
$$E[\int_S^T (f(t, \omega)-\phi_n(t, \omega))^2dt] \to 0.$$
\end{definition}
由上述定义即可得到It\^{o}积分的一个重要性质:
\begin{theorem} \label{Ito isometric} (It\^{o}等距) 对于$\forall f\in\mathcal{V}(S, T)$有
$$E\left[(\int_S^T f(t, \omega)dB_t)^2\right] = E\left[\int_S^T f^2(t, \omega)dt\right].$$
\end{theorem}
这里$\mathcal{V}(S, T)$见\cite{oksendal2003stochastic}中的定义3.1.2至3.1.4. \\
利用上述方法, 即利用基本函数序列来逼近$\mathcal{V}(S, T)$中的函数, 我们能够得到It\^{o}积分的线性性质:
\begin{theorem} \label{Itoint_property}
设$f, g \in \mathcal{V}(0, T)$, $0 \leqslant S < U < T$. 那么
\begin{flalign*}
& (1)\quad \int_S^T fdB_t = \int_S^U fdB_t + \int_U^T fdB_t, \quad a.e.\\
& (2)\quad \int_S^T (cf+g)dB_t = c\cdot\int_S^T fdB_t + \int_S^T gdB_t, \quad a.e.\\
& (3)\quad E\left[\int_S^T fdB_t\right] = 0.
\end{flalign*}
\end{theorem}

为了便于计算, 我们还需要引入以下的一维It\^{o}公式\cite{jiangangying2016stochastic}.
\begin{theorem}(It\^{o}公式) \label{Ito formula} 设$X_t$为一个如下的It\^{o}过程:
$$dX_t = udt+vdB_t,$$
$g(t, x) \in C^2([0, \infty) \times \mathbb{R})$, 则$Y_t = g(t, X_t)$也是一个It\^{o}过程, 且满足:
$$dY_t = \frac{\partial g}{\partial t}(t, X_t)dt + \frac{\partial g}{\partial x}(t, X_t)dX_t + \frac{1}{2}\frac{\partial^2 g}{\partial x^2}(t, X_t)\cdot(dX_t)^2,$$
其中
$$dt\cdot dt = dt\cdot dB_t = dB_t\cdot dt = 0, \quad dB_t\cdot dB_t = dt.$$
\end{theorem}
以下是上述It\^{o}公式的积分形式.
\begin{theorem} \label{Itoint}(分部积分) 设$f(s, \omega)$对几乎所有的$\omega$关于$s\in [0, t]$是连续的, 且为有界变差函数, 则有
$$\int_0^t f(s)dB_s = f(t)B_t - \int_0^t B_s df_s.$$
\end{theorem}
上述几个定理的证明请见\cite{oksendal2003stochastic}, 21-24页及36-39页. \\

\subsection{高斯分布与特征函数}
在我们的假设中各个噪声信号都服从高斯分布, 为此我们需要知道多元高斯分布的概率密度\cite{shuyuanhe2006probability}.
\begin{proposition} \label{multiVariable gaussian pdf}
如果随机向量$\textbf{x}$服从多元正态分布
$$\textbf{x} \sim \mathcal{N}(\mu, \Sigma),$$
那么$\textbf{x}$的概率密度为
$$f_x(x_1, \cdots, x_k) = \frac{exp\left(-\frac{1}{2}(\textbf{x}-\mu)^\top\right)\Sigma^{-1}(\textbf{x}-\mu)}{\sqrt{(2\pi)^k|\Sigma|}}$$
\end{proposition}

除此之外, 我们还需要引入随机变量的特征函数这个工具, 用于计算较为复杂的随机变量的统计量.
\begin{definition}(特征函数) 如果$X$是实值随机变量, $E[sin(tX)], E[cos(tX)]$均存在, 那么称
$$\phi(t) = E\left[e^{itX}\right] = E[cos(tX)] + iE[sin(tX)], \quad t\in \mathbb{R}$$
为$X$的特征函数, 其中$i$为虚数单位.
\end{definition}
对于随机向量, 其特征函数定义为
\begin{definition} 设$\textbf{X} = (X_1, X_2, \cdots, X_n)$为随机向量, 则$\textbf{X}$的特征函数为
$$\phi(t) = E\left[exp(i\textbf{t}\textbf{X}^\top)\right], \quad \textbf{t} = (t_1, t_2, \cdots, t_n)\in \mathbb{R}^n.$$
\end{definition}
以下为高斯分布的特征函数.
\begin{proposition} \label{Characteristic function of Gaussian Distribution}
Gauss分布$N(\mu, \sigma^2)$的特征函数为
$$\phi(t) = e^{i\mu t - \frac{1}{2}\sigma^2t^2}$$
\end{proposition}
此命题的证明见\cite{shuyuanhe2006probability}第五章例2.4.
一个更一般的定义如下\cite{jiangangying2013probability}:
\begin{definition} \label{characteristic function definition 2} 设$\xi$为d-维随机变量, $F$为$\xi$的分布函数. 那么
$$\hat{F}(x) = \int_{\mathbb{R}^d} e^{ix\cdot y}dF(y), \quad x\in \mathbb{R}^d$$
(其中$x\cdot y$为$\mathbb{R}^d$空间中的内积) 称为分布函数$F$的特征函数.
\end{definition}
由此定义即可看出, 特征函数即为概率密度函数的Fourier变换:
\begin{equation} \label{characteristic function definition 3}
\hat{F}(x) = \int_{\mathbb{R}^d}e^{ix\cdot y}f(y)dy.
\end{equation}

以下是Fourier变换的一个基本性质, 在求$u(t)$的统计量时将会用到:
\begin{proposition}(Fourier变换的微分关系) \label{Fourier transform property 1}
如果
$$\lim_{|x|\to\infty} f(x) = 0,$$
且$f'(x)$的Fourier变换存在, 那么
$$\mathcal{F}[f'(x)] = i\omega\mathcal{F}[f(x)].$$
更一般的, 若
$f(\infty) = f'(\infty) = \cdots = f^{(k-1)}(\infty) = 0,$ 且$\mathcal{F}[f^{(k)}(x)]$存在, 则
$$\mathcal{F}[f^{(k)}(x)] = (i\omega)^k \mathcal{F}[f].$$
\end{proposition}

\section{$b(t)$与$\gamma(t)$的统计量}
\subsection{均值}
当t固定时, 由(\ref{Sb})式可知, $b(t)$的通解第一项为常数, 而第二项中只有$b_0$是一个随机变量. 那么由(\ref{init_values})可知, $b(t)$通解第二项的期望为$(E[b_0]-\hat{b})e^{\lambda_b(t-t_0)}$. 而由于
$$f = f(s) = e^{\lambda_b(t-s)}\in\mathcal{V}(S, T),$$
故由(\ref{Itoint_property})的(3)式可知,
$$E\left[\sigma_b\int_{t_0}^{t} e^{\lambda_b(t-s)dW_b(s)}\right] = 0.$$
从而
\begin{equation} \label{b_stat_1}
E(b(t)) = \hat{b} + (E[b_0] - \hat{b})e^{\lambda_b(t-t_0)}.
\end{equation}
对于$\gamma$亦有相同的结论
\begin{equation} \label{gamma_stat_1}
E(\gamma(t)) = \hat{\gamma} + (E[\gamma_0] - \hat{\gamma})e^{-d_{\gamma}(t-t_0)}.
\end{equation}

\subsection{方差}
当t固定时, 由\cite{shuyuanhe2006probability}的第四章相关知识, 考虑到$(b_0-\hat{b})e^{\lambda_b(t-t_0)}$与$\sigma_b\int_{t_0}^t e^{\lambda_b(t-s)}dW_b(s)$相互独立(由(\ref{init_values})可知),
\begin{equation} \label{b_stat_2_process}
\begin{split}
Var&(b(t)) = Var\left((b_0 - \hat{b})e^{\lambda_b(t-t_0)}\right) + Var\left(\sigma_b\int_{t_0}^{t}e^{\lambda_b(t-s)}dW_b(s)\right) \\
&= (e^{\lambda_b(t-t_0)})^2Var(b_0) + Var\left(\sigma_b\int_{t_0}^t e^{\lambda_b(t-s)}dW_b(s)\right) \\
&= Re\left\{e^{2(-\gamma_b+i\omega_b)(t-t_0)}Var(b_0)\right\} + E\left[\left(\sigma_b\int_{t_0}^t e^{\lambda_b(t-s)}dW_b(s)\right)^2\right] \\
&= e^{-2\gamma_b(t-t_0)}Var(b_0) + Re\left\{\sigma_b^2E\left[\int_{t_0}^t e^{2\lambda_b (t-s)}dt\right]\right\}
\end{split}
\end{equation}
上面的最后一式右边利用了It\^{o}等距(\ref{Ito isometric}).
计算右边即可得到
\begin{equation} \label{b_stat_2} 
Var(b(t)) = e^{-2\gamma_b(t-t_0)}Var(b_0) + \frac{\sigma_b^2}{2\gamma_b}(1-e^{-2\gamma_b(t-t_0)}).
\end{equation}
同样的, 我们也有
\begin{equation} \label{gamma_stat_2}
Var(\gamma(t)) = e^{-2d_\gamma(t-t_0)}Var(\gamma_0) + \frac{\sigma_\gamma^2}{2d_\gamma}(1-e^{-2d_\gamma(t-t_0)}).
\end{equation}

\subsection{协方差}
考虑到$b(t)$为复值函数, 由协方差的定义\cite{shuyuanhe2006probability}即可知
\begin{equation} \label{b_b*_Cov_process}
\begin{split}
Cov&(b(t), b(t)^*) = E\left[(b(t) - E[b(t)])(b(t)^* - E[b(t)^*])\right] \\
&= E\left[(b_0 - E[b_0]e^{\lambda_b (t-t_0)}+\sigma_b\int_{t_0}^{t} e^{\lambda_b (t-s)}dW_b(s))\right. \cdot \\
&\quad \left.((b_0^* - E[b_0^*])e^{\lambda_b(t-t_0)}+\sigma_b\int_{t_0}^t e^{\lambda_b (t-s)}dW_b(s))\right] \quad\cdots(\ref{b_stat_1}) \\
&= E\left[(b_0 - E[b_0])(b_0^* - E[b_0^*])e^{2\lambda_b (t-t_0)}\right] + \sigma_bE\left[(b_0 - E[b_0])\int_{t_0}^{t}e^{\lambda_b(t-s)}dW_b(s)\right]\\
&+ \sigma_bE\left[(b_0^*-E[b_0^*])e^{\lambda_b(t-t_0)}\int_{t_0}^{t}e^{\lambda_b(t-s)}dW_b(s)\right] + \sigma_b^2E\left[(\int_{t_0}^{t}e^{\lambda_b(t-s)}dW_b(s))^2\right]
\end{split}
\end{equation}
在上式的第二项中, 由于$b_0$与$\int_{t_0}^{t} e^{\lambda_b (t-s)}dW_b(s)$相互独立, 故由It\^{o}积分的性质(\ref{Itoint_property})可知
\begin{equation}
\sigma_bE\left[(b_0 - E[b_0])\int_{t_0}^{t}e^{\lambda_b(t-s)}dW_b(s)\right] = \sigma_bE\left[b_0 - E[b_0]\right]E\left[\int_{t_0}^{t}e^{\lambda_b(t-s)}dW_b(s)\right] = 0
\end{equation}
同样的, (\ref{b_b*_Cov_process})中第三项也为0, 而第四项由It\^{o}等距(\ref{Ito isometric})即知为0. 故
\begin{equation} \label{b_b*_Cov}
Cov(b(t), b(t)^*) = E[(b_0-E[b_0])(b_0^*-E[b_0^*])]e^{2\lambda_b(t-t_0)} = Cov(b_0, b_0^*)e^{2\lambda_b(t-t_0)}
\end{equation}
与之类似的,
\begin{equation} \label{b_gamma_Cov}
\begin{split}
Cov&(b(t), \gamma(t)) = E[(b(t) - E[b(t)])(\gamma(t)-E[\gamma(t)])] \\
&= E\left[\left(b_0 - E[b_0]e^{\lambda_b (t-t_0)}+\sigma_b\int_{t_0}^{t} e^{\lambda_b (t-s)}dW_b(s)\right)\right. \cdot \\
&\quad \left.\left((\gamma_0-E[\gamma_0])e^{-d_\gamma(t-t_0)}+\sigma_\gamma \int_{t_0}^{t} e^{-d_\gamma(t-s)}dW_\gamma(s)\right)\right] \\
&= E\left[(b_0-E[b_0])(\gamma_0-E[\gamma_0])\right]e^{(\lambda_b-d_\gamma)(t-t_0)} \\
&= Cov(b_0, \gamma_0)e^{(\lambda_b-d_\gamma)(t-t_0)}
\end{split}
\end{equation}
上述的第三个等式也用到了It\^{o}等距(\ref{Ito isometric}).
\section{$u(t)$的统计量}
\subsection{均值}
在$u(t)$的表达式(\ref{Su})中, $e^{\hat{\lambda}(t-t_0)}$, $f(t)$和$\sigma$均为常值. 那么由数学期望的线性性质(\cite{shuyuanhe2006probability}第四章定理2.2)和It\^{o}积分的性质(\ref{Itoint_property})可知
\begin{equation} \label{u_stat_1}
\begin{split}
E[u(t)] &= e^{\hat{\lambda}(t-t_0)}E\left[e^{-J_0(t_0, t)}u_0\right] + \int_{t_0}^t e^{\hat{\lambda}(t-s)}E\left[b(s)e^{-J(s, t)}\right]ds  \\
&+ \sigma\int_{t_0}^{t} e^{\hat{\lambda}(t-s)}f(s)E\left[e^{-J(s, t)}\right]ds.
\end{split}
\end{equation}
下面我们用高斯随机过程$J(s, t)$的特征函数\cite{gershgorin2008nonlinear}\cite{gershgorin2010filtering}来计算上式的右边. \\
首先我们有以下的命题:
\begin{proposition} \label{multiVariable gaussian exp 1}
$$E\left[ze^{ibx}\right] = (E[z]+ibCov(z, x))e^{ibE[x]-\frac{1}{2}b^2Var(x)}$$
其中z为复值高斯随机变量, x为实值高斯随机变量. 
\end{proposition}
\begin{proof}[证明]
不妨记
$$z = y + iw, \quad y, w\in \mathbb{R}. \label{z substitution}$$
那么我们只要计算出$E[ye^{ibx}]$和$E[we^{ibx}]$, 然后利用(\ref{z substitution})将其组合起来. 令
$\textbf{v} = (x, y, w)$, 那么由于$\textbf{v}$为一个满足多元Gauss分布的随机向量, 其特征函数由(\ref{Characteristic function of Gaussian Distribution})即可给出:
\begin{equation} \label{characteristic function of v}
\phi_\textbf{v}(\textbf{s}) = exp(i\textbf{s}^\top E[\textbf{v}]-\frac{1}{2}\textbf{s}^\top\Sigma\textbf{s}),
\end{equation}
其中$\Sigma$为协方差矩阵. 记$g(\textbf{v})$为$\textbf{v}$的概率密度函数, 那么由特征函数的定义(\ref{characteristic function definition 3}), 即特征函数为概率密度的Fourier变换即可知道,
\begin{equation}
\phi_\textbf{v}(\textbf{s}) = \frac{1}{(2\pi)^3}\int e^{i\textbf{s}^\top\textbf{v}}g(\textbf{v})d\textbf{v}
\end{equation}
利用Fourier变换的性质(\ref{Fourier transform property 1}), 我们不妨对$s_2$求偏导\cite{gershgorin2008nonlinear}, 有
\begin{equation} \label{partial over s2}
\frac{\partial \phi_\textbf{v}(\textbf{s})}{\partial s_2} = \frac{1}{(2\pi)^3}\int iy_0e^{i\textbf{s}^\top\textbf{v}}g(\textbf{v})d\textbf{v} = iE\left[y_0e^{i\textbf{s}^\top\textbf{v}}\right].
\end{equation}
令$\textbf{v} = (b, 0, 0)^\top$, 根据(\ref{partial over s2})即能得到
\begin{equation} \label{E over y0}
\left.E[y_0e^{ibx_0}] = -i\frac{\partial \phi_\textbf{v}(s)}{\partial s_2}\right|_{\textbf{s} = (b, 0, 0)^\top}
\end{equation}
同样的, 我们有
\begin{equation} \label{E over w0}
\left.E[w_0e^{ibx_0}] = -i\frac{\partial \phi_\textbf{v}(s)}{\partial s_3}\right|_{\textbf{s} = (b, 0, 0)^\top}
\end{equation}

由多元高斯分布的概率密度函数(\ref{multiVariable gaussian pdf})有
\begin{equation} \label{partial over s2-2}
\frac{\partial \phi_\textbf{v}(\textbf{s})}{\partial s_2} = (iE[y_0]-Var(y_0)s_2-Cov(x_0, y_0)s_1-Cov(y_0, w_0)s_3)\phi_{\textbf{v}}(s)
\end{equation}
\begin{equation} \label{partial over s3-2}
\frac{\partial \phi_\textbf{v}(\textbf{s})}{\partial s_3} = (iE[w_0]-Var(w_0)s_3-Cov(x_0, w_0)s_1-Cov(y_0, w_0)s_2)\phi_{\textbf{v}}(s)
\end{equation}
分别计算这两个偏导数(\ref{partial over s2-2})及(\ref{partial over s3-2})在$\textbf{s} = (b, 0, 0)^\top$处的值, 有
\begin{equation} \label{partial over s2 at s}
E\left[y_0e^{ibx_0}\right] = (E[y_0]+iCov(x_0, y_0)b)exp(ibE[x_0] - \frac{1}{2}Var(x_0)b^2)
\end{equation}
\begin{equation} \label{partial over s3 at s}
E\left[w_0e^{ibx_0}\right] = (E[w_0]+iCov(x_0, w_0)b)exp(ibE[x_0] - \frac{1}{2}Var(x_0)b^2)
\end{equation}
结合(\ref{partial over s2 at s}), (\ref{partial over s3 at s})即有
$$E\left[ze^{ibx}\right] = (E[z]+ibCov(z, x))e^{ibE[x]-\frac{1}{2}b^2Var(x)}.$$
\end{proof}
由此我们立刻可以得到
\begin{corollary} \label{multiVariable gaussian exp 2} 在命题\ref{multiVariable gaussian exp 1}的条件下,
$$E\left[ze^{bx}\right] = (E[z]+bCov(z, x))e^{bE[x]+\frac{1}{2}b^2Var(x)}.$$
\end{corollary}

利用推论\ref{multiVariable gaussian exp 2}即有
\begin{equation} \label{E u(t)}
\begin{split}
E&[u(t)] = e^{\hat{\lambda}(t-t_0)}(E[u_0]-Cov(u_0, J(t_0, t)))e^{-E[J(t_0, t)]+\frac{1}{2}Var(J(t_0, t))} \\
&+ \int_{t_0}^t e^{\hat{\lambda}(t-s)}(\hat{b}+e^{\lambda_b(s-t_0)}(E[b_0]-\hat{b}-Cov(b_0, J(s, t))))e^{-E[J(s, t)]+\frac{1}{2}Var(J(s, t))}ds \\
&+ \int_{t_0}^t e^{\hat{\lambda}(t-s)}f(s)e^{-E[J(s, t)]+\frac{1}{2}Var(J(s, t))}ds
\end{split}
\end{equation}
下面我们计算
$$Cov(u_0, J(s, t)), Cov(b_0, J(s, t)), E[J(s, t)], Var(J(s, t)).$$
先引入两个显然的结论, 其中$a, b, c, d$均为随机变量:
\begin{proposition} \label{Var 1}
$Var(a+b) = Var(a)+Var(b)+2Cov(a, b)$.
\end{proposition}
\begin{proposition} \label{Cov 1}
$Cov(a+b, c+d) = Cov(a, c)+Cov(a, d)+Cov(b, c)+Cov(b, d)$.
\end{proposition}
由于
\begin{equation}
\begin{split} \label{J(s, t) all}
J(s, t) &= \int_s^t \left[(\gamma_0-\hat{\gamma})e^{-d_\gamma(s'-t_0)}+\sigma_\gamma\int_{t_0}^{s'}e^{-d_\gamma(s'-x)}dW_\gamma(x)\right]ds' \\
& = \int_s^t(\gamma_0-\hat{\gamma})e^{-d_\gamma(s'-t_0)}ds' +\int_s^t \sigma_\gamma\int_{t_0}^{s'} e^{-d_\gamma(s'-x)}dW_\gamma(x)ds'
\end{split}
\end{equation}
由命题(\ref{Cov 1}),
\begin{equation} \label{Cov u0 J(s, t)}
\begin{split}
Cov&(u_0, J(s, t)) = Cov\left(u_0, \frac{1}{d_\gamma}(e^{-d_\gamma(s-t_0)}-e^{-d_\gamma(t-t_0)})(\gamma_0 - \hat{\gamma})\right) \\
&+ Cov\left(u_0, \int_s^t \sigma_\gamma \int_{t_0}^{s'} e^{-d_\gamma(s'-x)} dW_\gamma(x)ds'\right)\\
&= \frac{1}{d_\gamma}(e^{-d_\gamma(s-t_0)}-e^{-d_\gamma(t-t_0)})[Cov(u_0, \gamma_0)-Cov(u_0, \hat{\gamma})] \\
&= \frac{1}{d_\gamma}(e^{-d_\gamma(s-t_0)}-e^{-d_\gamma(t-t_0)})Cov(u_0, \gamma_0).
\end{split}
\end{equation}
这是因为$u_0$与$\hat{\gamma}$相互独立. 同理有
\begin{equation} \label{Cov b0 J(s, t)}
Cov(b_0, J(s, t)) = \frac{1}{d_\gamma}(e^{-d_\gamma(s-t_0)}-e^{-d_\gamma(t-t_0)})Cov(b_0, \gamma_0).
\end{equation}
为了计算$E[J(s, t)]$, 我们对(\ref{gamma_stat_1})求积分, 有
\begin{equation} \label{E J(s, t)}
\begin{split}
E[&J(s, t)] = E\left[\int_s^t (\gamma(s')-\hat{\gamma})ds'\right] = \int_s^t (E[\gamma(s')]-\hat{\gamma})ds' \\
&= \int_s^t (E[\gamma_0]-\hat{\gamma})e^{-d_\gamma(s'-t_0)}ds' \\
&= \frac{1}{d_\gamma}(e^{-d_\gamma(s-t_0)}-e^{-d_\gamma(t-t_0)})(E[\gamma_0]-\hat{\gamma}))
\end{split}
\end{equation}
结合(\ref{E J(s, t)})和(\ref{J(s, t) all})可知,
\begin{equation} \label{E J(s, t) 2}
E\left[\sigma_\gamma\int_s^t\int_{t_0}^{s'}e^{-d_\gamma(s'-x)}dW_\gamma(x)ds'\right] = 0.
\end{equation}
那么
\begin{equation} \label{Var J(s, t) process 1}
\begin{split}
Var&(J(s, t)) = E\left[J^2(s, t)\right] - E[J(s, t)]^2 \\
&= E\left[\left(\int_s^t (\gamma_0-\hat{\gamma})e^{-d_\gamma(s'-t_0)}ds'\right)^2\right] - E[J(s, t)]^2 \\
&+ 2E\left[\left(\int_s^t(\gamma_0-\hat{\gamma})e^{-d_\gamma(s'-t_0)}ds'\right)\left(\sigma_\gamma\int_s^t\int_{t_0}^{s'}e^{-d_\gamma(s'-x)}dW_\gamma(x)ds'\right)\right] \\
&+ \sigma_\gamma^2E\left[\left(\int_s^t\int_{t_0}^{s}e^{-d_\gamma(s'-s)}dW_\gamma(s)ds'\right)^2\right] \\
&= \frac{1}{d_\gamma^2}\left(e^{-d_\gamma(s-t_0)}-e^{-d_\gamma(t-t_0)}\right)^2(E[\gamma_0^2]-E[\gamma_0]^2) \\
&+ \sigma_\gamma^2E\left[\left(\int_{t_0}^t \frac{1}{d_\gamma}\left(1-e^{-d_\gamma}(t-x)\right)dW_\gamma(x)\right)^2\right]
\end{split}
\end{equation}
由It\^{o}等距(定理\ref{Ito isometric})和It\^{o}公式(定理\ref{Ito formula})知上式的第二项为
\begin{equation} \label{Var J(s, t) 2}
\begin{split}
\sigma_\gamma^2E&\left[\left(\int_{t_0}^t \frac{1}{d_\gamma}\left(1-e^{-d_\gamma}(t-x)\right)dW_\gamma(x)\right)^2\right] \\
&= \sigma_\gamma^2E\left[\left(\int_{t_0}^{s}\int_{t_0}^{t}e^{-d_\gamma(s'-x)}ds'dW_\gamma(x)+\int_s^t\int_x^t e^{-d_\gamma(s'-x)}ds'dW_\gamma(x)\right)^2\right] \\
% &= \frac{\sigma_\gamma^2}{d_\gamma^2}E\left[\left(\int_{t_0}^t \left(e^{-d_\gamma(s-x)}-e^{-d_\gamma(t-x)}dx\right)+\int_{t_0}^t \left(1-e^{-d_\gamma(t-x)\right)^2dx\right)\right]
&= \frac{\sigma_\gamma^2}{d_\gamma^2}E\left[\left(\int_{t_0}^{t} \left(e^{-d_\gamma(s-x)}-e^{-d_\gamma(t-x)}\right)dW_\gamma(x)+\int_{s}^{t} \left(1-e^{-d_\gamma(t-x)}\right)dW_\gamma(x)\right)^2\right] \\
&= \frac{\sigma_\gamma^2}{d_\gamma^2}\left\{\int_{t_0}^t \left(e^{-d_\gamma(s-x)}-e^{-d_\gamma(t-x)}\right)^2dx+\int_{s}^t\left(1-e^{-d_\gamma(t-x)}\right)^2dx\right. \\
&+\left.2E\left[\left(\int_{t_0}^t \left(e^{-d_\gamma(s-x)}-e^{-d_\gamma(t-x)}\right)dW_\gamma(x)\right)\left(\int_{s}^t\left(1-e^{-d_\gamma(t-x)}\right)dW_\gamma(x)\right)\right]\right\} \\
&= \frac{\sigma_\gamma^2}{d_\gamma^3}\left(-1+d_\gamma(t-s)+e^{-d_\gamma(s+t-2t_0)}+e^{-d_\gamma(t-s)}-\frac{1}{2}e^{-2d_\gamma(t-t_0)}-\frac{1}{2}e^{-2d_\gamma(s-t_0)}\right. \\
&+ \left.\frac{1}{2}e^{-2d_\gamma(s-t)}-\frac{1}{2}e^{-2d_\gamma(t-s)}\right) + 2E\left[\int_{\min\{s, t_0\}}^t\left(e^{-d_\gamma(s-x)}-e^{-d_\gamma(t-x)}\right)\left(1-e^{-d_\gamma(t-x)}\right)dt\right]
\end{split}
\end{equation}
这里我们将两个积分区域均扩展至$[\min\{s, t_0\}, t]$上, 并设积分区域较小的被积函数在延拓的区间上为0. 那么
\begin{equation} \label{Var J(s, t) final}
\begin{split}
Var&(J(s, t)) = \frac{1}{d_\gamma^2}\left(e^{-d_\gamma(s-t_0)}-e^{-d_\gamma(t-t_0)}\right)^2Var(\gamma_0) \\
&+ \frac{\sigma_\gamma^2}{d_\gamma^3}\left(-1+d_\gamma(t-s)+e^{-d_\gamma(s+t-2t_0)}+e^{-d_\gamma(t-s)}-\frac{1}{2}e^{-2d_\gamma(t-t_0)}-\frac{1}{2}e^{-2d_\gamma(s-t_0)}\right) 
\end{split}
\end{equation}
将(\ref{Var J(s, t) final}), (\ref{Cov u0 J(s, t)}), (\ref{Cov b0 J(s, t)})代入(\ref{E u(t)})式即得$u(t)$的均值.
\subsection{方差}
利用
\begin{equation}
Var(u(t)) = E\left[(|u(t)|^2)\right] - \left|E[u(t)]\right|^2.
\end{equation}
我们记
\begin{equation}
u(t) = A + B + C,
\end{equation}
其中
\begin{equation}
A = e^{-J(t_0, t)+\hat{\lambda}(t-t_0)}u_0,
\end{equation}
\begin{equation}
B = \int_{t_0}^t (b(s)+f(s))e^{-J(s, t)+\hat{\lambda}(t-s)}ds,
\end{equation}
\begin{equation}
C = \sigma\int_{t_0}^t e^{-J(s, t)+\hat{\lambda}(t-s)}dW(s).
\end{equation}

于是由It\^{o}积分的性质有
\begin{equation} \label{E u(t)^2}
E\left[|u(t)|^2\right] = E\left[|A|^2\right]+ E\left[|B|^2\right] + E\left[|C|^2\right] + 2Re{E[A^*B]}.
\end{equation}
下面我们分别求出
$$
E\left[|A|^2\right], E\left[|B|^2\right], E\left[|C|^2\right], E[AB].
$$
由命题\ref{multiVariable gaussian exp 1}可知
\begin{proposition}  \label{multiVariable gaussian exp 3} 对于复值Gaussian随机变量$z$和$w$, 以及实值Gaussian随机变量$x$,
\begin{equation}
\begin{split}
E&\left[zwe^{bx}\right] = \left[E[z]E[w]+Cov(z, w^*)+b(E[z]Cov(w, x))+E[w]Cov(z, x)+ \right. \\
& \left.b^2Cov(z, x)Cov(w, x)\right]e^{bE[x]+\frac{b^2}{2}Var(x)}. \\
\end{split}
\end{equation}
\end{proposition}

利用上述命题, 有
\begin{equation}
\begin{split}
E&\left[|A|^2\right] = \left(|E[u_0]|^2 +Var(u_0) - 4Re\left\{E[u_0]^*Cov(u_0, J(t_0, t)\right\} +4|Cov(u_0, J(t_0, t)|^2\right) \cdot \\
& e^{-2\hat{\gamma}(t-t_0)-2E[J(t_0, t)]+2Var(J(t_0, t))}.
\end{split}
\end{equation}
而
\begin{equation}
\begin{split}
E&\left[|B|^2\right] = E\left[\left|\int_{t_0}^t (b(s)+f(s))e^{-J(s, t)+\hat{\lambda}(t-s)}ds\right|^2\right] \\
&= \int_{t_0}^t ds\int_{t_0}^t dr E\left[(b(s)+f(s))e^{-J(s, t)+\hat{\lambda}(t-s)}\left[(b(r)+f(r))e^{-J(r, t)+\hat{\lambda}(t-r)}\right]^*\right]
\end{split}
\end{equation}


由命题(\ref{Var 1})和(\ref{Cov 1}), 并结合命题\ref{multiVariable gaussian exp 3}可知, 在上式中, 被积的最后一项为
\begin{equation} \label{E B^2}
\begin{split}
E&\left[(b(s)+f(s))e^{-J(s, t)+\hat{\lambda}(t-s)}\left[(b(r)+f(r))e^{-J(r, t)+\hat{\lambda}(t-r)}\right]^*\right] \\
&= E\left[(b(s)+f(s))(b^*(r)+f^*(r))e^{-J(s, t)-J(r, t)+(-\hat{\gamma}+i\omega)(t-s)+(-\hat{\gamma}-i\omega)(t-r)}\right] \\
&= E\left[(b(s)+f(s))(b^*(r)+f^*(r))e^{-J(s, t)-J(r, t)-\hat{\gamma}(2t-s-r)+i\omega(r-s)}\right] \\
&= e^{-J(s, t)-J(r, t)+\frac{1}{2}Var(J(s, t))+\frac{1}{2}Var(J(r, t))+Cov(J(s, t), J(r, t))-\hat{\gamma}(2t-s-r)+i\omega(r-s)} \times \\
&\quad \{E\left[b(s)b^*(r)\right] + E[b(s)f^*(r)] + E[f(s)b^*(r)] + E[f(s)f^*(r)] \\
&\quad+ Cov(b(s), b(r)) + Cov(b(s), f(r))+Cov(f(s), b(r))+Cov(f(s), f(r)) \\
&\quad+ (E[b(s)]+E[f(s)])\times[Cov(b^*(r), -J(s, t))+Cov(b^*(r), -J(r, t)) \\
&\quad\quad +Cov(f^*(r), -J(s, t))+Cov(f^*(r), -J(r, t))] \\
&\quad+ (E[b^*(r)]+E[f^*(r)])\times[Cov(b(s), -J(s, t))+Cov(f(s), -J(s, t)) \\
&\quad\quad +Cov(f(s), -J(r, t))+Cov(b(s), -J(r, t))] \\
&\quad+ [Cov(b(s), -J(s, t))+Cov(b(s), -J(r, t))+Cov(f(s), -J(s, t))+Cov(f(s), -J(r,t))] \\
&\quad\quad\times [Cov(b^*(r), -J(s, t))+Cov(b^*(r), -J(r, t))+Cov(f^*(r), -J(s,t))+Cov(f^*(r), -J(r, t))]\} \\
&= e^{-J(s, t)-J(r, t)+\frac{1}{2}Var(J(s, t))+\frac{1}{2}Var(J(r, t))+Cov(J(s, t), J(r, t))-\hat{\gamma}(2t-s-r)+i\omega(r-s)} \times \\
&\quad \{E[b(s)b^*(r)]+E[b(s)]\times[Cov(b^*(r), -J(s, t))+Cov(b^*(r), -J(r, t))] \\
&\quad\quad+ E[b^*(r)]\times[Cov(b(s), -J(s, t))+Cov(b(s), -J(r, t))]\\
&\quad\quad+ [Cov(b^*(r), J(s, t))+Cov(b^*(r), J(r, t))]\times [Cov(b(s), J(s, t))+Cov(b(s), J(r, t))] \\
&\quad\quad+ f^*(r)\times[E[b(s)]-Cov(b(s), J(s, t)-Cov(b(s), J(r, t)))] \\
&\quad\quad+ f(s)\times[E[b^*(r)]-Cov(b(r), J^*(s, t)-Cov(b(r), J^*(r, t)))] \\
&\quad\quad+f(s)f^*(r)\}. \\
\end{split}
\end{equation}
其中, 由It\^o公式和It\^o积分的性质即知
\begin{equation}
\begin{split}
E&[b(s)b(r)] = E\left[\left(\hat{b}+(b_0-\hat{b})e^{\lambda_b(s-t_0)}+\sigma_b\int_{t_0}^s e^{\lambda_b(s-w)}dW_b(w)\right)\right.\times \\
&\quad\quad\quad\quad\quad\quad\left.\left(\hat{b}+(b_0-\hat{b})e^{\lambda_b(r-t_0)}+\sigma_b\int_{t_0}^r e^{\lambda_b(r-w)}dW_b(w)\right) \right] \\
&= \left(1-e^{\lambda_b(s-t_0)}-e^{\lambda_b(r-t_0)}+e^{\lambda_b(s+r-2t_0)}\right)\hat{b}^2 + \left(e^{\lambda_b(s-t_0)}+e^{\lambda_b(r-t_0)}-2e^{\lambda_b(s+r-2t_0)}\right)\hat{b}E[b_0] \\
&\quad + e^{\lambda_b(s+r-2t_0)}\left(Var(b_0) + E[b_0]^2\right) + \frac{\sigma_b^2}{2\gamma_b}\left(e^{-\gamma_b(s+r-2min(s, r))}-e^{-\gamma_b(s+r-2t_0)}\right)e^{i\omega_b(s-r)}
\end{split}
\end{equation}
结合(\ref{E J(s, t)})和(\ref{Cov b0 J(s, t)})有
\begin{equation}
\begin{split}
Cov&(b(r), J(s, t)) = e^{\lambda_b(r-t_0)}Cov(b_0, J(s, t)) + \sigma_bCov\left(\int_{t_0}^r e^{\lambda_b(r-w)}dW_b(w), J(s, t)\right) \\
&= \frac{1}{d_\gamma}(e^{-d_\gamma(s-t_0)}-e^{-d_\gamma(t-t_0)})e^{\lambda_b(r-t_0)}Cov(b_0, \gamma_0)
\end{split}
\end{equation}
当$t_0 \leqslant r \leqslant s \leqslant t$时, 我们这么计算$J(s, t)$和$J(r, t)$的协方差:
\begin{equation}
Cov(J(s, t), J(r, t)) = Cov(J(s, t), J(r, s)+J(s, t)) = Var(J(s, t)) + Cov(J(s, t), J(r, s)).
\end{equation}
其中$Var(J(s, t))$由(\ref{Var J(s, t) final})已求出. 由类似(\ref{Var J(s, t) process 1})和(\ref{Var J(s, t) 2})的过程, 有
\begin{equation}
\begin{split}
Cov&(J(s, t), J(r, s)) = Cov\left(\int_s^t(\gamma_0-\hat{\gamma})e^{-d_\gamma(s'-t_0)}ds' +\int_s^t \sigma_\gamma\int_{t_0}^{s'} e^{-d_\gamma(s'-x)}dW_\gamma(x)ds'\right. \\
&\quad\quad\quad\left. \int_r^s(\gamma_0-\hat{\gamma})e^{-d_\gamma(s'-t_0)}ds' +\int_r^s \sigma_\gamma\int_{t_0}^{s'} e^{-d_\gamma(s'-x)}dW_\gamma(x)ds'\right) \\
&= \frac{Var(\gamma_0)}{d_\gamma^2}\left(e^{-d_\gamma(t-t_0)}-e^{-d_\gamma(s-t_0)})(e^{-d_\gamma(s-t_0)}e^{-d_\gamma(r-t_0)}\right)\\
&-\frac{\sigma_\gamma^2}{2d_\gamma^3}\left(e^{-d_\gamma(t-s)}-e^{-d_\gamma(t-r)}+e^{-d_\gamma(t+s-2t_0)}-e^{-d_\gamma(t+r-2t_0)}-1 \right.\\
&\left.+e^{-d_\gamma(s-r)}-e^{-2d_\gamma(s-t_0)}+e^{-d_\gamma(s+r-2t_0)}\right)
\end{split}
\end{equation}
当$t_0 \leqslant s \leqslant r \leqslant t$时, 只要考虑到
$$
Cov(J(s, t), J(r, t)) = Cov(J(r, t), J(s, t)),
$$
于是只要在结果中将$s$与$r$交换即可. 接下来,由It\^o等距有
\begin{equation}
\begin{split}
E&\left[|C|^2\right] = \sigma^2E\left[\left(\int_{t_0}^t e^{-J(s, t)+\hat{\lambda}(t-s)}dW(s)\right)^2\right] \\
&= \sigma^2\int_{t_0}^te^{-2\hat\gamma(t-s)}E\left[e^{-2J(s, t)}\right]ds = \sigma^2\int_{t_0}^t e^{-2\hat\gamma(t-s)-2E[J(s, t)]+2Var(J(s, t))}ds
\end{split}
\end{equation}

最后我们有
\begin{equation}
\begin{split}
E&[A^*B] = E\left[\left(e^{-J(t_0, t)+\hat{\lambda}(t-t_0)}u_0\right)^*\int_{t_0}^t(b(s)+f(s))e^{-J(s, t)+\hat{\lambda}(t-s)}ds\right]\\
&= e^{-\hat\gamma(t-t_0)}\int_{t_0}^t \left(E\left[u_0^*b_0e^{-J(t_0, t)-J(s, t)}\right]e^{(\lambda_b-i\omega)(s-t_0)}\right.\\
&\quad+\left.\left(\hat{b}\left(1-e^{\lambda_b(s-t_0)}\right)+f(s)\right)E\left[u_0e^{-J(t_0, t)-J(s, t)}\right]^*\right)ds
\end{split}
\end{equation}
由命题(\ref{multiVariable gaussian exp 3}), 以及上述的计算过程(\ref{E B^2}), 类似的能得到
\begin{equation}
\begin{split}
E&\left[u_0^*b_0e^{-J(t_0, t)-J(s, t)}\right] = e^{-E[J(s, t)]-E[J(t_0, t)]+\frac{1}{2}Var(J(s, t))+\frac{1}{2}Var(J(t_0, t))+Cov(J(s, t), J(t_0, t))}\times \\
&[E[u_0^*]E[b_0]+Cov(u_0^*, b_0^*) - E[u_0^*](Cov(b_0, J(s, t))\\
&+Cov(b_0, J(t_0, t)))-E[b_0](Cov(J(s, t), u_0^*)+Cov(J(t_0, t), u_0^*)) \\
&\quad +(Cov(u_0^*, J(s, t))+Cov(u_0^*, J(t_0, t)))\times(Cov(b_0, J(s, t))+Cov(b_0, J(t_0, t)))]
\end{split}
\end{equation}
以及
\begin{equation}
\begin{split}
E&\left[u_0e^{-J(t_0, t)-J(s, t)}\right] = e^{-E[J(s, t)]-E[J(t_0, t)]+\frac{1}{2}Var(J(s, t))+\frac{1}{2}Var(J(t_0, t))+Cov(J(s, t), J(t_0, t))}\times \\
&(E[u_0]-Cov(u_0, J(s, t))-Cov(u_0, J(t_0, t)))
\end{split}
\end{equation}
于是把上述的几个式子代回到(\ref{E u(t)^2})即得到了$Var(u(t))$.

\subsection{协方差}
首先,我们考虑$Cov(u(t), u^*(t))$.
由定义知,{}
$$
Cov(u(t), u^*(t)) = E\left[u(t)^2\right]-E[u(t)]^2.
$$
利用上一节中的记号, 由布朗运动$W(t)$的独立性,
\begin{equation}
E[u(t)^2] = E[A^2]+E[B^2]+2E[AB].
\end{equation}
下面分别计算$E[A^2]$, $E[B^2]$和$E[AB]$.
利用命题(\ref{multiVariable gaussian exp 3}),
\begin{equation}
\begin{split}
E&[A^2] = E[e^{-2J(t_0, t)+2\hat\lambda(t-t_0)}u_0^2] = e^{-2\hat\gamma(t-t_0)-2E[J(t_0, t)]+2Var(J(t_0, t))}\times \\
&\quad\left(E[u_0]^2+Cov(u_0, u_0^*)-4E[u_0]Cov(u_0, J(t_0, t))+4Cov(u_0, J(t_0, t))^2\right).
\end{split}
\end{equation}
和上一节类似的,
$$
E\left[B^2\right] = \int_{t_0}^t ds \int_{t_0}^t dr E\left[((b(s)+f(s))(b(r)+f(r))e^{-J(s, t)-J(r, t)+\hat\lambda(2t-s-r)}\right],
$$
且由命题(\ref{multiVariable gaussian exp 3})有
\begin{equation}
\begin{split}
E&\left[((b(s)+f(s))(b(r)+f(r))e^{-J(s, t)-J(r, t)+\hat\lambda(2t-s-r)}\right] \\
&= e^{-\hat\lambda(2t-s-r)-E[J(s, t)]-E[J(r, t)]+\frac{1}{2}Var(J(s, t))+\frac{1}{2}Var(r, t)+Cov(J(s, t), J(r, t))} \times \\
&\quad [E[b(s)b(r)]-E[b(s)]\times(Cov(b(r), J(r, t))+Cov(b(r), J(s, t))) \\
&\quad\quad- E[b(r)]\times(Cov(b(s), J(r, t))+Cov(b(s), J(s, t))) \\
&\quad\quad + [Cov(b(r), J(s, t))+Cov(b(r), J(r, t))]\times[Cov(b(s), J(s, t))+Cov(b(s), J(r, t))]\\
&\quad\quad + f(r)\times[E[b(s)]-Cov(b(s), J(s, t))-Cov(b(s), J(r, t))]\\
&\quad\quad +f(s)\times[E[b(r)]-Cov(b(r), J(r, t))-Cov(b(r), J(s, t))]+f(s)f(r)]
\end{split}
\end{equation}
其中
\begin{equation}
\begin{split}
E&[b(s)b(r)] = \left(1-e^{\lambda_b(s-t_0)}-e^{\lambda_b(r-t_0)}+e^{\lambda_b(s+r-2t_0)}\right)\hat{b}^2 \\
&+ \left(e^{\lambda_b(s-t_0)+e^{\lambda_b(r-t_0)}-2e^{\lambda_b(s-t_0)(r-t_0)}}\right)\hat{b}E[b_0] + e^{\lambda_b(s+r-2t_0)}\left(Var(b_0)+|E[b_0]|^2\right)
\end{split}
\end{equation}
同样的,
\begin{equation}
\begin{split}
E&[AB] = E[e^{-J(t_0, t)+\hat{\lambda}(t-t_0)}u_0\int_{t_0}^t (b(s)+f(s))e^{-J(s, t)+\hat\lambda(t-s)}ds] \\
&= e^{\hat\lambda(2t-s-t_0)}\int_{t_0}^t \left[e^{\lambda_b(s-t_0)}E[u_0b_0e^{-J(t_0, t)-J(s, t)}]+(\hat{b}(1-e^{\lambda_b(s-t_0)})+f(s))E[u_0e^{-J(t_0, t)-J(s, t)}]\right]ds.
\end{split}
\end{equation}
其中第二个期望在上一节中已经求出结果了, 而
\begin{equation}
\begin{split}
E&[u_0b_0e^{-J(t_0, t)-J(s, t)}] = e^{-E[J(t_0, t)]-E[J(s, t)]+\frac{1}{2}(Var(J(t_0, t))+Var(J(s, t)))+Cov(J(t_0, t), J(s, t))}\times \\
&\quad (Cov(u_0, b_0^*)+E[u_0]E[b_0]-E[u_0](Cov(b_0, J(t_0, t))+Cov(b_0, J(s, t)))\\
&\quad\quad -E[b_0](Cov(u_0, J(t_0, t))+Cov(u_0, J(s, t)))\\
&\quad\quad +[Cov(b_0, J(t_0, t))+Cov(b_0, J(s, t))]\times[Cov(u_0, J(s, t))+Cov(u_0, J(t_0, t))])
\end{split}
\end{equation}
代入即可. \\

下面考虑$Cov(u(t), \gamma(t))$.由定义,
\begin{equation}
Cov(u(t), \gamma(t)) = E[u(t)\gamma(t)] - E[u(t)]E[\gamma(t)] = E[u(t)(\gamma(t)-\hat\gamma)] + E[u(t)](\hat\gamma-E[\gamma(t)]). 
\end{equation}
上式的第二项由(\ref{gamma_stat_1})和(\ref{E u(t)})可以直接计算得到. 下面我们计算上式的第一项. 由It\^o等距知,
\begin{equation} \label{E u(t) gamma}
\begin{split}
E&[u(t)(\gamma(t)-\hat\gamma)] = E\left[\left(e^{-J(t_0, t)+\hat\lambda(t-t_0)}u_0+\int_{t_0}^t(b(s)+f(s))e^{-J(s, t)+\hat\lambda(s-t_0)}ds\right.\right. \\
&\quad \left.\left.+\sigma\int_{t_0}^t e^{-J(s, t)+\hat\lambda(s-t_0)}dW(s)\right)\times(\gamma(t)-\hat\gamma)\right] \\
&= e^{\hat\lambda(t-t_0)}E\left[(\gamma(t)-\hat\gamma)u_0e^{\hat\lambda(t-t_0)}\right]+\int_{t_0}^t e^{-\hat\lambda(s-t_0)}E\left[(b(s)+f(s))(\gamma(t)-\hat\gamma)e^{-J(s, t)}\right]ds
\end{split}
\end{equation}
由$J(t_0, t)$的定义(\ref{def J(s, t)})可知,
$$
\frac{\partial J(t_0, t)}{\partial t} = (\gamma(t)-\hat\gamma),
$$
那么(\ref{E u(t) gamma})可以表示为
\begin{equation}
-e^{\hat\lambda(t-t_0)}\frac{\partial}{\partial t}E\left[u_0e^{-J(t_0, t)}\right]-\int_{t_0}^t e^{\hat\lambda(s-t_0)}\frac{\partial}{\partial t}E\left[(b(s)+f(s))e^{-J(s, t)}\right]ds,
\end{equation}
其中由(\ref{multiVariable gaussian exp 2})和协方差的线性性质,
\begin{equation}
\begin{split}
\frac{\partial}{\partial t}&E\left[u_0e^{-J(t_0, t)}\right] = \frac{\partial}{\partial t}\left[(E[u_0]+Cov(-J(t_0, t), u_0))e^{-E[J(t_0, t)]+\frac{1}{2}Var(J(t_0, t))}\right] \\
&= \left(\frac{\partial}{\partial t}Cov(-J(t_0, t), u_0)\right)e^{-E[J(t_0, t)]+\frac{1}{2}Var(J(t_0, t))} \\
&\quad + (E[u_0] + Cov(u_0, -J(t_0, t)))\frac{\partial}{\partial t}e^{-E[J(t_0, t)]+\frac{1}{2}Var(J(t_0, t))} \\
&= e^{-J(t_0, t)+\frac{1}{2}Var(J(t_0, t))}\times\left[Cov(-\gamma(t), u_0)+(E[u_0]+Cov(-J(t_0, t), u_0))\times\right. \\
&\quad \left.(\hat\gamma-E[\gamma(t)]+\frac{1}{2}\frac{\partial}{\partial t}Var(J(t_0, t)))\right],
\end{split}
\end{equation}
以及
\begin{equation}
\begin{split}
\frac{\partial}{\partial t}&E\left[(b(s)+f(s))e^{-J(s, t)}\right] = \frac{\partial}{\partial t}\left[(E[b(s)]+f(s)+Cov(b(s)+f(s), -J(s, t)))e^{-E[J(s, t)]+\frac{1}{2}Var(J(s, t))}\right] \\
&= e^{-E[J(s, t)]+\frac{1}{2}Var(J(s, t))}\times\left[Cov(-\gamma(t), b(s))+(E[b(s)]+f(s)+Cov(b(s), J(s, t)))\times \right. \\
&\quad \left.\left(\hat\gamma - E[\gamma(t)]+\frac{1}{2}\frac{\partial}{\partial t}Var(J(s, t))\right)\right],
\end{split}
\end{equation}
且在上两式中,
\begin{equation}
\begin{split}
\frac{\partial}{\partial t}&Var(J(s, t)) = \frac{\partial}{\partial t}\left[\frac{1}{d_\gamma^2}\left(e^{-d_\gamma(s-t_0)}-e^{-d_\gamma(t-t_0)}\right)^2Var(\gamma_0)\right. \\
&\quad+ \left.\frac{\sigma_\gamma^2}{d_\gamma^3}\left(-1+d_\gamma(t-s)+e^{-d_\gamma(s+t-2t_0)}+e^{-d_\gamma(t-s)}-\frac{1}{2}e^{-2d_\gamma(t-t_0)}-\frac{1}{2}e^{-2d_\gamma(s-t_0)}\right)\right] \\
&= \frac{\sigma_\gamma^2}{d_\gamma^2}\left(1-e^{-d_\gamma(t-s)}-e^{-d_\gamma(t+s-2t_0)}+e^{-2d_\gamma(t-t_0)}\right)+\frac{2}{d_\gamma}Var(\gamma_0)\times\left(e^{-d_\gamma(t+s-2t_0)}-e^{-2d_\gamma(t-t_0)}\right).
\end{split}
\end{equation}
下面考虑$Cov(u(t), b(t))$. 由定义,
$$
Cov(u(t), b(t)) = E[u(t)b^*(t)] - E[u(t)]E[b(t)]^*.
$$
上式的第二项由(\ref{E u(t)})和(\ref{b_stat_1})即可计算得到. 结合It\^o公式, 完全类似之前的计算过程, 第一项为
\begin{equation}
\begin{split}
E&[u(t)b^*(t)] = E\left[u(t)\left(\hat{b}^*+(b_0^*-\hat{b}^*)e^{\lambda_b^*(t-t_0)}+\sigma_b\int_{t_0}^t e^{\lambda_b(t-s)}dW_b(s)\right)\right] \\
&= \left(1-e^{\lambda_b^*(t-t_0)}\right)\hat{b}^*E[u(t)] + E\left[u(t)b_0^*e^{\lambda_b^*(t-t_0)}\right] \\
&\quad + E[\sigma\sigma_b\int_{t_0}^t\int_{t_0}^te^{-J(s, t)+\hat\lambda(s-t_0)+\lambda_b^*(t-\xi)}dW(s)dW_b(\xi)] \\
&= \left(1-e^{\lambda_b^*(t-t_0)}\right)\hat{b}^*E[u(t)] + E\left[\left(e^{-J(t_0, t)+\hat\lambda(s-t_0)}+\int_{t_0}^t (b(s)+f(s))e^{-J(s, t)+\hat\lambda(s-t_0)}ds\right)b_0^*e^{\lambda_b^*(t-t_0)}\right] \\
&\quad + \frac{\sigma\sigma_b}{2\gamma_b}E\left[\int_{t_0}^t e^{-J(s, t)}e^{-\lambda(t-s)}\left(e^{\lambda_b^*(t-s)}-e^{-i\omega_b(t-s)-\gamma_b(s+t-2t_0)}\right)\right] \\
&= \left(1-e^{\lambda_b^*(t-t_0)}\right)\hat{b}^*E[u(t)] + e^{(\hat\lambda+\lambda_b^*)(t-t_0)}E\left[u_0b_0^*e^{-J(t_0, t)}\right] + e^{\lambda_b^*(t-t_0)}\times \\
&\quad \int_{t_0}^t e^{\hat\lambda(t-s)}\left(E\left[b_0^*b(s)e^{-J(s, t)}\right]+f(s)E\left[b_0e^{-J(s, t)}\right]^*ds\right) \\
&\quad + \frac{\sigma\sigma_b}{2\gamma_b}E\left[\int_{t_0}^t e^{-E[J(s, t)]+\frac{1}{2}Var(J(s, t))}e^{-\lambda(t-s)}\left(e^{\lambda_b^*(t-s)}-e^{-i\omega_b(t-s)-\gamma_b(s+t-2t_0)}\right)\right],
\end{split}
\end{equation}
并利用命题(\ref{multiVariable gaussian exp 3}), 在上式中
\begin{equation}
E\left[b_0e^{-J(s, t)}\right] = e^{-E[J(s, t)]+\frac{1}{2}Var(J(s, t))}\times(E[b_0]+Cov(b_0, -J(s, t))),
\end{equation}
\begin{equation}
\begin{split}
E&\left[u_0b_0^*e^{-J(t_0, t)}\right] = e^{-E[J(t_0, t)]+\frac{1}{2}Var(J(t_0, t))}\times[E[u_0]E[b_0]^*+Cov(u_0, b_0)+E[u_0]Cov(b_0, -J(t_0, t))^*\\
&+E[b_0]^*Cov(u_0, -J(t_0, t)) + Cov(u_0, -J(t_0, t))Cov(b_0, -J(t_0, t))^*],
\end{split}]
\end{equation}
\begin{equation}
\begin{split}
E&[b_0^*b(s)e^{-J(s, t)}] = e^{-J(s, t)+\frac{1}{2}Var(J(s, t))}\times [E[b_0]^*E[b(s)]+Cov(b_0, b(s))+E[b_0]^*Cov(b(s), -J(s, t))\\
&+E[b(s)]Cov(b_0, -J(s, t))^*+Cov(b_0, -J(s, t))^*Cov(b(s), -J(s, t))].
\end{split}
\end{equation}
最后我们考虑$Cov(u(t), b^*(t))$.由定义,
$$
Cov(u(t), b^*(t)) = E[u(t)b(t)] - E[u(t)]E[b(t)].
$$
和前一部分类似的, 只要考虑上式的第一项.我们有
\begin{equation}
\begin{split}
E&[u(t)b(t)] = E\left[u(t)\left(\hat{b}+(b_0-\hat{b})e^{\lambda_b(t-t_0)}+\sigma_b\int_{t_0}^t e^{\lambda_b(t-s)}dW_b(s)\right)\right] \\
&= \left(1-e^{\lambda_b(t-t_0)}\right)\hat{b}E[u(t)] + E\left[u(t)b_0e^{\lambda_b(t-t_0)}\right] \\
&\quad + E[\sigma\sigma_b\int_{t_0}^t\int_{t_0}^te^{-J(s, t)+\hat\lambda(s-t_0)+\lambda_b(t-\xi)}dW(s)dW_b(\xi)] \\
&= \left(1-e^{\lambda_b^*(t-t_0)}\right)\hat{b}E[u(t)] + E\left[\left(e^{-J(t_0, t)+\hat\lambda(s-t_0)}+\int_{t_0}^t (b(s)+f(s))e^{-J(s, t)+\hat\lambda(s-t_0)}ds\right)b_0e^{\lambda_b(t-t_0)}\right] \\
&= \left(1-e^{\lambda_b(t-t_0)}\right)\hat{b}E[u(t)] + e^{(\hat\lambda+\lambda_b)(t-t_0)}E\left[u_0b_0e^{-J(t_0, t)}\right] + e^{\lambda_b(t-t_0)}\times \\
&\quad \int_{t_0}^t e^{\hat\lambda(t-s)}\left(E\left[b_0b(s)e^{-J(s, t)}\right]+f(s)E\left[b_0e^{-J(s, t)}\right]ds\right) \\
\end{split}
\end{equation}
在上式中, 由命题(\ref{multiVariable gaussian exp 3})知,
\begin{equation}
\begin{split}
E&\left[u_0b_0e^{-J(t_0, t)}\right] = e^{-E[J(t_0, t)]+\frac{1}{2}Var(J(t_0, t))}\times[E[u_0]E[b_0]+Cov(u_0, b_0^*)+E[b_0]Cov(u_0, -J(t_0, t))\\
&\quad + E[u_0]Cov(b_0, -J(t_0, t))+Cov(u_0, -J(t_0, t))Cov(b_0, -J(t_0, t))],
\end{split}
\end{equation}
\begin{equation}
\begin{split}
E&\left[b_0b(s)e^{-J(s, t)}\right] = e^{-J(s, t)+\frac{1}{2}Var(J(s, t))}\times [E[b_0]E[b(s)]+Cov(b_0, b(s)^*)+E[b_0]Cov(b(s), -J(s, t)) \\
&\quad + E[b(s)]Cov(b_0, -J(s, t)) + Cov(b_0, -J(s, t))Cov(b(s), -J(s, t))]
\end{split}
\end{equation}

\section{数值模拟}
考虑以下的这样一个简单例子. 在(\ref{E0})中, 我们不妨取
\begin{equation}
\left\{
\begin{aligned}
&\omega = 1, \sigma = 0.5 \\
&\omega_b = 2.5, \sigma_b = 2 \\
&
\end{aligned}
\right
\end{equation}
\bibliographystyle{plain}
\bibliography{../../CONFIG/LaTeX-bib/Chuan}
\end{document}{}