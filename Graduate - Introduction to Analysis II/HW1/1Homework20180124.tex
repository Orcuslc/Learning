

\documentclass{article}%

\usepackage{amsmath}

\usepackage{graphicx}

\usepackage{amsfonts}%

\usepackage{amssymb}





\setlength{\topmargin}{-0.75in}

\setlength{\textheight}{9.25in}

\setlength{\oddsidemargin}{0.0in}

\setlength{\evensidemargin}{0.0in}

\setlength{\textwidth}{6.5in}

\def\labelenumi{\arabic{enumi}.}

\def\theenumi{\arabic{enumi}}

\def\labelenumii{(\alph{enumii})}

\def\theenumii{\alph{enumii}}

\def\p@enumii{\theenumi.}

\def\labelenumiii{\arabic{enumiii}.}

\def\theenumiii{\arabic{enumiii}}

\def\p@enumiii{(\theenumi)(\theenumii)}

\def\labelenumiv{\arabic{enumiv}.}

\def\theenumiv{\arabic{enumiv}}

\def\p@enumiv{\p@enumiii.\theenumiii}

\pagestyle{plain}

\setcounter{secnumdepth}{0}

\newtheorem{theorem}{Theorem}

\newtheorem{acknowledgement}[theorem]{Acknowledgement}

\newtheorem{algorithm}[theorem]{Algorithm}

\newtheorem{axiom}[theorem]{Axiom}

\newtheorem{case}[theorem]{Case}

\newtheorem{claim}[theorem]{Claim}

\newtheorem{conclusion}[theorem]{Conclusion}

\newtheorem{condition}[theorem]{Condition}

\newtheorem{conjecture}[theorem]{Conjecture}

\newtheorem{corollary}[theorem]{Corollary}

\newtheorem{criterion}[theorem]{Criterion}

\newtheorem{definition}[theorem]{Definition}

\newtheorem{example}[theorem]{Example}

\newtheorem{exercise}[theorem]{Exercise}

\newtheorem{lemma}[theorem]{Lemma}

\newtheorem{notation}[theorem]{Notation}

\newtheorem{problem}[theorem]{Problem}

\newtheorem{proposition}[theorem]{Proposition}

\newtheorem{remark}[theorem]{Remark}

\newtheorem{solution}[theorem]{Solution}

\newtheorem{summary}[theorem]{Summary}

\newenvironment{proof}[1][Proof]{\textbf{#1.} }{\ \rule{0.5em}{0.5em}}



\begin{document}



\begin{center}

\textbf{Homework 1}\bigskip

\end{center}



\noindent\textbf{Instructions}: 

\noindent In problems 3. - 5., references such as III.2.7 refer to Problem 7 in Section 2 of Chapter III in Conway's book.\smallskip



\noindent If you use results from books including Conway's, please be explicit about what results you are using.






\begin{center}

\emph{Homework 1 is due on Dropbox on Monday, February  5.}

\end{center} 

\medskip



\begin{enumerate}

\item Show that $\lim_{n\to \infty}{n^{\frac{1}{n}}} = 1$ without using logarithms.

\textbf{Proof.} Let $n^{1/n} = 1+y_n $. First we know $n^{1/n} > 1^{1/n} = 1 $, so $y_{n} > 0$. Then 
$$
n = (1+y_n)^n = 1+ny_n+\frac{n(n-1)}{2}y_n^2+\cdots+y_n^n > 1+\frac{n(n-1)}{2}y_n^2.
$$
Thus $\frac{n(n-1)}{2}y_n^2 < n-1 $, which means $y_n < \sqrt{2/n}. $ Hence $\lim\limits_{n\to\infty} y_n \le \lim\limits_{n\to\infty}\sqrt{2/n} = 0$, which means $y_n\to 0 $, and thus $n^{1/n}\to 1 $.


\item Given a power series, $\sum_{n=0}^{n=\infty} a_n(z - a)^n$, show that its radius of convergence $R$ satisfies the inequalities$$(\limsup{\vert \frac{a_{n+1}}{a_n} \vert})^{-1} \leq R \leq \limsup{\vert \frac{a_{n}}{a_{n+1}} \vert}.$$

\textbf{Proof.} We only proof the right inequality since it is just the same for the left one. If $R > r > \limsup|\frac{a_n}{a_{n+1}}| = \alpha $, then there is an $N > 0$ s.t. $r > |a_n/a_{n+1}| $ for all $n \ge N$. Let $B = |a_N|r^N $, then $|a_{N+1}|r^{N+1} = |a_{N+1}|rr^{N} > B $. Hence for all $n > N$ we have $|a_n|r^n > B $, which gives $|a_nz^n| \ge B |z|^n/|r|^n $ when $n > N$. But $|z|/|r| > 1$, which makes $|z|^n/|r|^n \to\infty $ when $n\to\infty$. Hence $\sum a_nz^n $ diverges, so $R \le \alpha$.


\item Problem III.1.6.

\textbf{(a).} By Theorem 1.3, 
$$
\limsup |a_n|^{1/n} = \limsup |a^n|^{1/n} = |a|,
$$
thus $R = \frac{1}{|a|}$.

\textbf{(b).} By Theorem 1.3,
$$
\limsup |a_n|^{1/n} = \limsup |a^{n^2}|^{1/n} = \limsup |a^n| = \left\{
\begin{aligned} 
&0, |a| < 1, \\
&1, |a| = 1, \\
&\infty, |a| > 1.
\end{aligned}
\right.
$$
Thus 
$$
R = \left\{
\begin{aligned}
&\infty, |a| < 1,\\
&1, |a| = 1,\\
&0, |a| > 1.
\end{aligned}
\right.
$$

\textbf{(c).} By Theorem 1.3,
$$
\limsup |a_n|^{1/n} = \limsup |k^n|^{1/n} = k,
$$
thus $R = \frac{1}{k}$.

\textbf{(d).} Since 
$$
\sum_{n=0}^{\infty}|z|^{n!} < \sum_{n=0}^{\infty}|z|,
$$
and the convergence radius of the latter series is $R' = 1$, we know $R\ge 1$. On the other hand, if $R > 1$, pick $1 < |z| = r < R$, then $|z|^{n!} = r^{n!}\to\infty $ when $n\to\infty$, hence the series diverges. Thus $R = 1$.

\item Problem III.1.7

\textbf{Proof.} On one hand, 
$$
\sum_{n=1}^{\infty} |a_n| |z^{n(n+1)}| \le \sum_{n=1}^{\infty}|a_n||z^n|,
$$
thus $R \ge R' = \lim|a_n/a_{n+1}| = \lim \frac{n+1}{n} = 1 $. On the other hand, if $R > 1$, pick $1 < r < R$, then $|a_n|z^{n(n+1)} = \frac{1}{n}r^{n(n+1)} = \frac{1}{n}(1+\delta)^{n(n+1)} > \frac{1}{n} (1+n\delta)^{n+1} > \frac{1}{n}(1+n(n+1)\delta) > (n+1)\delta. $ But the last term $\to\infty$ as $n\to\infty$, hence the series diverges. Thus $R = 1$.

When $z = 1$, the series becomes $\sum_{n=1}^{\infty}\frac{(-1)^n}{n}$, it is a Leibniz series, thus converges. When $z = -1$, since $n(n+1)$ is a even number, it is the same with $z = 1$, thus converges. When $z = i$, the series becomes
$$
\begin{aligned}
&\sum_{n=0}^{\infty} -\left(\frac{(-1)^{4n+1}}{(4n+1)}+\frac{(-1)^{4n+2}}{4n+2}\right)+\left(\frac{(-1)^{4n+3}}{(4n+3)}+\frac{(-1)^{4n+4}}{4n+4}\right) = \sum_{n=0}^{\infty}\frac{1}{4n+1}-\frac{1}{4n+2}-\frac{1}{4n+3}+\frac{1}{4n+4} \\
&= \sum_{n=0}^{\infty}(-1)^n (\frac{1}{2n+1}-\frac{1}{2n+2}) = \sum_{n=0}^{\infty}\frac{(-1)^n}{(2n+1)(2n+2)}.
\end{aligned}
$$
It is also a Leibniz series, so converges.


\item Problem III.2.6

\textbf{(i)} $z = x+iy$, then $e^z = e^x(\cos y+i\sin y) = i \to x = 0, y = 2k\pi + \frac{\pi}{2}$. Thus $z = i(2k\pi+\frac{\pi}{2}), k \in \mathbb{Z}$.

\textbf{(ii)} $e^x(\cos y+i\sin y) = -1 \to x = 0, y = 2k\pi+\pi $. Thus $z = i(2k\pi+\pi), k \in \mathbb{Z}$.

\textbf{(iii)} $e^x(\cos y+i\sin y) = -i \to x = 0, y = 2k\pi + \frac{3\pi}{2}$. Thus $z = i(2k\pi+\frac{3\pi}{2}), k \in \mathbb{Z}$.

\textbf{(iv)} $\frac{1}{2}(e^{iz}+e^{-iz}) = 0 \to e^{-y}(\cos x+i\sin x)+e^y(\cos x-i\sin x) = 0 \to \cos x = 0, e^{-y} = e^y \to y = 0, x = k\pi+\frac{\pi}{2}. $ Thus $z = k\pi+\frac{\pi}{2}, k\in\mathbb{Z} $.

\textbf{(v)} $\frac{1}{2i}(e^{iz}-e^{-iz}) = 0 \to e^{-y}(\cos x+i\sin x)-e^y(\cos x-i\sin x) = 0 \to -y = y, \sin x = 0 \to y = 0, x = k\pi $. Thus $z = k\pi, k\in\mathbb{Z}$.

\item Problem III.2.7
$$
\cos z\cos w = \frac{1}{4}(e^{iz}+e^{-iz})(e^{iw}+e^{-iw}) = \frac{1}{4}(e^{i(z+w)}+e^{-i(z+w)}+e^{i(z-w)}+e^{i(w-z)}),
$$
$$
\sin z\sin w = -\frac{1}{4}(e^{iz}-e^{-iz})(e^{iw}-e^{-iw}) = \frac{1}{4}(-e^{i(z+w)}-e^{-i(z+w)}+e^{i(z-w)}+e^{i(w-z)})
$$
$$
\cos z\sin w = \frac{1}{4i}(e^{iz}+e^{-iz})(e^{iw}-e^{-iw}) = \frac{1}{4i}(e^{i(z+w)}-e^{i(z-w)}+e^{i(w-z)}-e^{-i(z+w)})
$$
$$
\sin z\cos w = \frac{1}{4i}(e^{iz}-e^{-iz})(e^{iw}+e^{-iw}) = \frac{1}{4i}(e^{i(z+w)}+e^{i(z-w)}-e^{i(w-z)}-e^{i(z+w)})
$$
Hence, 
$$
\cos(z+w) = \frac{1}{2}(e^{i(z+w)}+e^{-i(z+w)}) = \cos z\cos w-\sin z\sin w.
$$
$$
\sin(z+w) = \frac{1}{2i}(e^{i(z+w)}-e^{-i(z+w)}) = \cos z\sin w+\sin z\cos w.
$$

\item Problem III.2.9.

\textbf{Proof.} 
$$
\begin{aligned}
|z_n-z| &= |r_ne^{i\theta_n}-re^{i\theta}| = |r_n\cos(\theta_n)-r\cos(\theta) + i(r_n\sin(\theta_n)-r\sin(\theta))| \\
&= \sqrt{(r_n\cos(\theta_n)-r\cos(\theta))^2+(r_n\sin(\theta_n)-r\sin(\theta))^2} \\
&= \sqrt{r_n^2+r^2-2r_nr\cos(\theta_n-\theta)}.
\end{aligned}
$$
If $\theta_n \nrightarrow \theta $, then there exists $\epsilon > 0$
$$
|z_n-z| \ge \sqrt{2r_nr(1-\cos(\theta_n-\theta))} > \epsilon,
$$
which contradicts with $z_n\to z $. Hence $\theta_n\to \theta $. So
$$
|z_n-z| \to |r_n-r|.
$$
Thus $r_n\to r $.

\item Problem III.2.13

$z = f(z)^n \to f(z) = z^{1/n} $, take a branch. Thus
$$
f(z) = z^{1/n} = e^{\frac{1}{n}\log z} = e^{\frac{1}{n} (\log |z| + iArg z+ i2k\pi)} = |z|^{1/n}e^{\frac{i}{n}(Argz+2k\pi)}.
$$
Pick any $k\in\mathbb{Z}$ and we can get a function satisfying the conditions.

\item Problem III.2.20

$$
\log(z_1\cdots z_n) = \log|z_1\cdots z_n|+ Arg(z_1\cdots z_n).
$$
We need to show that ($Arg$ is the principle branch.)
$$
Arg(z_1\cdots z_n) = \sum Arg(z_i).
$$
and
$$
\log|z_1\cdots z_n| = \sum\log|z_i|.
$$
The second term is trivial from logarithm of real numbers. For the first term, we use induction. Let $z_j = r_je^{i\theta_j} $, then from the condition we know $-\frac{\pi}{2} < \theta_j < \frac{\pi}{2}$ (*).

First, when $k = 2$, from the condition we know $2k\pi-\frac{\pi}{2} < \theta_1 + \theta_2 < 2k\pi+\frac{\pi}{2}$. But from (*) we know $-\pi < \theta_1 + \theta_2 <\pi $, thus $-\frac{\pi}{2} <\theta_1 + \theta_2 < \frac{\pi}{2}$. Hence $Arg(z_1z_2) = Arg(z_1)+Arg(z_2)$.

Suppose it holds when $\le n-1$. Then from induction we know $-\frac{\pi}{2} <\sum\limits_{j=1}^{n-1}\theta_j < \frac{\pi}{2}$, and using the same deduction with $k = 2$, we can know $-\frac{\pi}{2} <\sum\limits_{j=1}^{n}\theta_j < \frac{\pi}{2}$. Hence $Arg(\prod z_i) = \sum Arg(z_i)$.

If the restrictions are removed, the formula is not valid. For example, $z_1 = e^{i\pi}, z_2 = e^{i\frac{3\pi}{2}} $, then $\log(z_1z_2) = i\frac{\pi}{2}$, while $\log(z_1)+\log(z_2) = i\frac{5\pi}{2}$.

\end{enumerate}

\end{document}

