

\documentclass{article}%

\usepackage{amsmath}

\usepackage{graphicx}

\usepackage{amsfonts}%

\usepackage{amssymb}





\setlength{\topmargin}{-0.75in}

\setlength{\textheight}{9.25in}

\setlength{\oddsidemargin}{0.0in}

\setlength{\evensidemargin}{0.0in}

\setlength{\textwidth}{6.5in}

\def\labelenumi{\arabic{enumi}.}

\def\theenumi{\arabic{enumi}}

\def\labelenumii{(\alph{enumii})}

\def\theenumii{\alph{enumii}}

\def\p@enumii{\theenumi.}

\def\labelenumiii{\arabic{enumiii}.}

\def\theenumiii{\arabic{enumiii}}

\def\p@enumiii{(\theenumi)(\theenumii)}

\def\labelenumiv{\arabic{enumiv}.}

\def\theenumiv{\arabic{enumiv}}

\def\p@enumiv{\p@enumiii.\theenumiii}

\pagestyle{plain}

\setcounter{secnumdepth}{0}

\newtheorem{theorem}{Theorem}

\newtheorem{acknowledgement}[theorem]{Acknowledgement}

\newtheorem{algorithm}[theorem]{Algorithm}

\newtheorem{axiom}[theorem]{Axiom}

\newtheorem{case}[theorem]{Case}

\newtheorem{claim}[theorem]{Claim}

\newtheorem{conclusion}[theorem]{Conclusion}

\newtheorem{condition}[theorem]{Condition}

\newtheorem{conjecture}[theorem]{Conjecture}

\newtheorem{corollary}[theorem]{Corollary}

\newtheorem{criterion}[theorem]{Criterion}

\newtheorem{definition}[theorem]{Definition}

\newtheorem{example}[theorem]{Example}

\newtheorem{exercise}[theorem]{Exercise}

\newtheorem{lemma}[theorem]{Lemma}

\newtheorem{notation}[theorem]{Notation}

\newtheorem{problem}[theorem]{Problem}

\newtheorem{proposition}[theorem]{Proposition}

\newtheorem{remark}[theorem]{Remark}

\newtheorem{solution}[theorem]{Solution}

\newtheorem{summary}[theorem]{Summary}

\newenvironment{proof}[1][Proof]{\textbf{#1.} }{\ \rule{0.5em}{0.5em}}



\begin{document}



\begin{center}

\textbf{Homework 1}\bigskip

\end{center}



\noindent\textbf{Instructions}: 

\noindent In problems 3. - 5., references such as III.2.7 refer to Problem 7 in Section 2 of Chapter III in Conway's book.\smallskip



\noindent If you use results from books including Conway's, please be explicit about what results you are using.






\begin{center}

\emph{Homework 1 is due on Dropbox on Monday, February  5.}

\end{center} 

\medskip



\begin{enumerate}

\item Show that $\lim_{n\to \infty}{n^{\frac{1}{n}}} = 1$ without using logarithms.

\textbf{Proof.} Let $n^{1/n} = 1+y_n $. First we know $n^{1/n} > 1^{1/n} = 1 $, so $y_{n} > 0$. Then 
$$
n = (1+y_n)^n = 1+ny_n+\frac{n(n-1)}{2}y_n^2+\cdots+y_n^n > 1+\frac{n(n-1)}{2}y_n^2.
$$
Thus $\frac{n(n-1)}{2}y_n^2 < n-1 $, which means $y_n < \sqrt{2/n}. $ Hence $\lim\limits_{n\to\infty} y_n \le \lim\limits_{n\to\infty}\sqrt{2/n} = 0$, which means $y_n\to 0 $, and thus $n^{1/n}\to 1 $.


\item Given a power series, $\sum_{n=0}^{n=\infty} a_n(z - a)^n$, show that its radius of convergence $R$ satisfies the inequalities$$(\limsup{\vert \frac{a_{n+1}}{a_n} \vert})^{-1} \leq R \leq \limsup{\vert \frac{a_{n}}{a_{n+1}} \vert}.$$

\textbf{Proof.} We only proof the right inequality since it is just the same for the left one. If $R > r > \limsup|\frac{a_n}{a_{n+1}}| = \alpha $, then there is an $N > 0$ s.t. $r > |a_n/a_{n+1}| $ for all $n \ge N$. Let $B = |a_N|r^N $, then $|a_{N+1}|r^{N+1} = |a_{N+1}|rr^{N} > B $. Hence for all $n > N$ we have $|a_n|r^n > B $, which gives $|a_nz^n| \ge B |z|^n/|r|^n $ when $n > N$. But $|z|/|r| > 1$, which makes $|z|^n/|r|^n \to\infty $ when $n\to\infty$. Hence $\sum a_nz^n $ diverges, so $R \le \alpha$.


\item Problem III.1.6.

\textbf{(a).} By Theorem 1.3, 
$$
\limsup |a_n|^{1/n} = \limsup |a^n|^{1/n} = |a|,
$$
thus $R = \frac{1}{|a|}$.

\textbf{(b).} By Theorem 1.3,
$$
\limsup |a_n|^{1/n} = \limsup |a^{n^2}|^{1/n} = \limsup |a^n| = \left\{
\begin{aligned} 
&0, |a| < 1, \\
&1, |a| = 1, \\
&\infty, |a| > 1.
\end{aligned}
\right.
$$
Thus 
$$
R = \left\{
\begin{aligned}
&\infty, |a| < 1,\\
&1, |a| = 1,\\
&0, |a| > 1.
\end{aligned}
\right.
$$

\textbf{(c).} By Theorem 1.3,
$$
\limsup |a_n|^{1/n} = \limsup |k^n|^{1/n} = k,
$$
thus $R = \frac{1}{k}$.

\textbf{(d).} Since 
$$
\sum_{n=0}^{\infty}|z|^{n!} < \sum_{n=0}^{\infty}|z|,
$$
and the convergence radius of the latter series is $R' = 1$, we know $R\ge 1$. On the other hand, if $R > 1$, pick $1 < |z| = r < R$, then $|z|^{n!} = r^{n!}\to\infty $ when $n\to\infty$, hence the series diverges. Thus $R = 1$.

\item Problem III.1.7

\item Problem III.2.6

\item Problem III.2.7

\item Problem III.2.9.

\item Problem III.2.13

\item Problem III.2.20

\end{enumerate}

\end{document}

