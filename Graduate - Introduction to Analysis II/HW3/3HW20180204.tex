

\documentclass{article}%

\usepackage{amsmath}

\usepackage{graphicx}

\usepackage{amsfonts}%

\usepackage{amssymb}





\setlength{\topmargin}{-0.75in}

\setlength{\textheight}{9.25in}

\setlength{\oddsidemargin}{0.0in}

\setlength{\evensidemargin}{0.0in}

\setlength{\textwidth}{6.5in}

\def\labelenumi{\arabic{enumi}.}

\def\theenumi{\arabic{enumi}}

\def\labelenumii{(\alph{enumii})}

\def\theenumii{\alph{enumii}}

\def\p@enumii{\theenumi.}

\def\labelenumiii{\arabic{enumiii}.}

\def\theenumiii{\arabic{enumiii}}

\def\p@enumiii{(\theenumi)(\theenumii)}

\def\labelenumiv{\arabic{enumiv}.}

\def\theenumiv{\arabic{enumiv}}

\def\p@enumiv{\p@enumiii.\theenumiii}

\pagestyle{plain}

\setcounter{secnumdepth}{0}

\newtheorem{theorem}{Theorem}

\newtheorem{acknowledgement}[theorem]{Acknowledgement}

\newtheorem{algorithm}[theorem]{Algorithm}

\newtheorem{axiom}[theorem]{Axiom}

\newtheorem{case}[theorem]{Case}

\newtheorem{claim}[theorem]{Claim}

\newtheorem{conclusion}[theorem]{Conclusion}

\newtheorem{condition}[theorem]{Condition}

\newtheorem{conjecture}[theorem]{Conjecture}

\newtheorem{corollary}[theorem]{Corollary}

\newtheorem{criterion}[theorem]{Criterion}

\newtheorem{definition}[theorem]{Definition}

\newtheorem{example}[theorem]{Example}

\newtheorem{exercise}[theorem]{Exercise}

\newtheorem{lemma}[theorem]{Lemma}

\newtheorem{notation}[theorem]{Notation}

\newtheorem{problem}[theorem]{Problem}

\newtheorem{proposition}[theorem]{Proposition}

\newtheorem{remark}[theorem]{Remark}

\newtheorem{solution}[theorem]{Solution}

\newtheorem{summary}[theorem]{Summary}

\newenvironment{proof}[1][Proof]{\textbf{#1.} }{\ \rule{0.5em}{0.5em}}



\begin{document}



\begin{center}

\textbf{Homework 2}\bigskip

\end{center}



\noindent\textbf{Instructions}:References such as III.2.7 refer to Problem 7 in Section 2 of Chapter III in Conway's book.\smallskip



\noindent If you use results from books including Conway's, please be explicit about what results you are using.

\noindent \textbf{Reminder: Exam I will be from 6:30 to 10:30 on Thursday, February 15 in Room 40 of Schaeffer Hall.}






\begin{center}

\emph{Homework 3 is due in your Dropbox folder by 11:59, Sunday , February 19.}

\end{center} 

\medskip

\noindent Working on this homework will help you with Exam I, so please don't put it off until after the exam.



\begin{enumerate}

\item Problem IV.2.4

\textbf{(a)}
By Abel's transform, let $\{a_n\}, \{b_n\}$ be two sequences, and $B_k = \sum\limits_{i=1}^{k}b_i $. Then
$$
\sum_{k=1}^{n}a_kb_k = a_nB_n - \sum_{k=1}^{n-1}(a_{k+1}-a_k)B_k.
$$
Hence for each fixed $n$, denote $\sum\limits_{k=1}^{n}a_k $ by $A_n $
$$
C_n = \lim_{r\to1^-}\sum_{k=1}^{n}a_kr^k = r^nA_n - \sum_{k=1}^{n-1}r^k(r-1)A_k.
$$
Since $\sum a_n(z-a)^n $ have radius of convergence 1,
$$
\lim_{r\to 1^-} \sum_{n=1}^{\infty}a_nr^n < \infty,
$$
then we can change the order of limits:
$$
\lim_{r\to 1^-}\lim_{n\to\infty}\sum_{k=1}^{n}a_kr^k = \lim_{n\to\infty}\lim_{r\to 1^-}\sum_{k=1}^{n}a_kr^k = \lim_{n\to\infty} A_n
$$
since each $A_k $ is a finite number, which comes from $\sum a_n $ converges to $A$. Hence,
$$
\lim_{r\to 1^-}\sum_{n=1}^{\infty}a_nr^n = \lim_{n\to\infty} A_n = A.
$$

\textbf{(b)}
Consider $a_n = \frac{(-1)^{n+1}}{n} $, then by Proposition III.1.4,
$$
\lim_{n\to\infty}\left|\frac{a_n}{a_{n+1}}\right| = \lim\frac{n+1}{n} = 1.
$$
Hence the series $\sum a_n(z-a)^n $ have radius of convergence 1. Since the series $\sum a_n $ is a Leibniz series, then it converges to $A < \infty$.

Now consider the function $f(z) = \log z$, it is analytic on $|z-1| < 1$, and it has power series expansion
$$
f(z) = \sum_{n=0}^{\infty}b_n (z-1)^n,
$$
where $b_n = \frac{1}{n!}f^{(n)}(1) = \frac{(-1)^{n-1}}{n} (n \ge 1), $ and $b_0 = 0 $. By this we can find $a_i = b_i $ for each $i \ge 0$, thus
$$
\sum a_n = f(2) = \log 2
$$


\item Problem IV.2.6

\textbf{Sol.} In the region where $f(z) = \sqrt{z}$ is analytic, 
$$
f(z) = \sum_{n=0}^{\infty}a_n(z-1)^n,
$$
where
$$
a_n = \frac{1}{n!}f^{(n)}(1) = \frac{(-1)^{(n-1)}}{n!}\frac{(2n-3)!!}{2^{n}} (n \ge 1), ~a_0 = 1.
$$
and since
$$
\lim_{n\to\infty}\left|\frac{a_n}{a_{n+1}}\right| = \lim_{n\to\infty} \frac{2n+2}{2n-1} = 1,
$$
we know the radius of convergence is 1.

\item Problem IV.2.9

\textbf{(a)}
Let $f(z) = e^{z}-e^{-z} $, then by Corollary 2.13,
$$
f^{(n-1)}(0) = \frac{(n-1)!}{2\pi i}\int_{\gamma}\frac{e^z-e^{-z}}{z^n}dz = 1+(-1)^{n}.
$$
Thus
$$
\int_{\gamma}\frac{e^z-e^{-z}}{z^n}dz = \frac{2\pi i}{(n-1)!}(1+(-1)^n).
$$

\textbf{(b)}
Let $f(z) = 1$, then 
$$
f^{(n-1)}(\frac{1}{2}) = \frac{(n-1)!}{2\pi i}\int_{\gamma}\frac{1}{(z-\frac{1}{2})^n} dz = \left\{
\begin{aligned}
& 1, n = 1 \\
& 0, n \ge 2
\end{aligned}
\right.
$$
Thus
$$
\int_{\gamma}\frac{1}{(z-\frac{1}{2})^n} dz = \left\{
\begin{aligned}
& 2\pi i, n = 1 \\
& 0, n \ge 2
\end{aligned}
\right.
$$

\textbf{(c)}
First, we have
$$
\frac{1}{z^2+1} = \frac{1}{2i}(\frac{1}{z-i}-\frac{1}{z+i}).
$$
Then let $f(z) = 1 = g(z)$, then
$$
1 = f(i) = \frac{1}{2\pi i}\int_{\gamma}\frac{1}{z-i}dz = g(-i) = \frac{1}{2\pi i}\int_{\gamma}\frac{1}{z+i}dz.
$$
Hence 
$$
\int_{\gamma}\frac{1}{z^2+1}dz = \frac{1}{2i}(\int_\gamma\frac{dz}{z-i}-\int_\gamma\frac{dz}{z+i}) = 0.
$$

\textbf{(d)}
Let $f(z) = \sin z$, then $f$ is analytic on $\mathbb{C}$.
$$
f(0) = \frac{1}{2\pi i}\int_{\gamma}\frac{\sin z}{z}dz = 0.
$$
Hence
$$
\int_\gamma \frac{\sin z}{z}dz = 0.
$$

\textbf{(e)}
Let $f(z) = z^{1/m} $, then
$$
f^{(m-1)}(1) = \frac{(m-1)!}{2\pi i}\int_{\gamma}\frac{z^{1/m}}{(z-1)^m}dz = \prod_{i=0}^{m-1}(\frac{1}{m}-i).
$$
Hence
$$
\int_{\gamma}\frac{z^{1/m}}{(z-1)^m}dz  = \frac{2\pi i}{(m-1)!}\prod_{i=0}^{m-1}(\frac{1}{m}-i).
$$


\item Problem IV.2.11

First, 
$$
f(z) = \frac{1}{2i}\log(\frac{1+iz}{1-iz}) = \frac{1}{2i}(\log(1+iz)-\log(1-iz)).
$$
We know $\log(z)$ is analytic on $\mathbb{C}\setminus l$, where $l$ is a line starting from the origin. Thus $f$ is analytic on $\mathbb{C}\setminus l\{z_1, z_2\}$, where $z_1 = -i, ~z_2 = i $. Hence $l: Re(z) = 0 $.

On a branch of $f$, 
$$
\tan f(z) = \frac{1}{i}\frac{e^{if(z)}-e^{-if(z)}}{e^{if(z)}+e^{-if(z)}}.
$$
Since
$$
e^{if(z)} = e^{\frac{1}{2}\log(\frac{1+iz}{1-iz})} = (\frac{1+iz}{1-iz})^{1/2},
$$
we have
$$
\tan f(z) = \frac{1}{i}\frac{(\frac{1+iz}{1-iz})^{1/2}-(\frac{1+iz}{1-iz})^{-1/2}}{(\frac{1+iz}{1-iz})^{1/2}+(\frac{1+iz}{1-iz})^{-1/2}} = \frac{1}{i}\frac{2iz}{2} = z.
$$
By exercise III.3.19(d),
$$
f(z) = \frac{1}{2i}(\log\frac{1+iz}{1-iz}) = \frac{1}{2i}(\log\frac{z-i}{z+i}+\log i-\log (-i)) = -\frac{1}{2}\int_{-1}^{1}\frac{dt}{z-it} - \frac{\pi}{2}
$$


\item Problem IV.3.3

\item Problem IV.3.6

\item Problem IV.3.14

\item Problem IV.4. 2

\item Problem IV. 4.3

\item Problem IV.5.7

\item Problem IV.5.9


\end{enumerate}

\end{document}

