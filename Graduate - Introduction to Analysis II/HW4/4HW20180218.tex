

\documentclass{article}%

\usepackage{amsmath}

\usepackage{graphicx}

\usepackage{amsfonts}%

\usepackage{amssymb}





\setlength{\topmargin}{-0.75in}

\setlength{\textheight}{9.25in}

\setlength{\oddsidemargin}{0.0in}

\setlength{\evensidemargin}{0.0in}

\setlength{\textwidth}{6.5in}

\def\labelenumi{\arabic{enumi}.}

\def\theenumi{\arabic{enumi}}

\def\labelenumii{(\alph{enumii})}

\def\theenumii{\alph{enumii}}

\def\p@enumii{\theenumi.}

\def\labelenumiii{\arabic{enumiii}.}

\def\theenumiii{\arabic{enumiii}}

\def\p@enumiii{(\theenumi)(\theenumii)}

\def\labelenumiv{\arabic{enumiv}.}

\def\theenumiv{\arabic{enumiv}}

\def\p@enumiv{\p@enumiii.\theenumiii}

\pagestyle{plain}

\setcounter{secnumdepth}{0}

\newtheorem{theorem}{Theorem}

\newtheorem{acknowledgement}[theorem]{Acknowledgement}

\newtheorem{algorithm}[theorem]{Algorithm}

\newtheorem{axiom}[theorem]{Axiom}

\newtheorem{case}[theorem]{Case}

\newtheorem{claim}[theorem]{Claim}

\newtheorem{conclusion}[theorem]{Conclusion}

\newtheorem{condition}[theorem]{Condition}

\newtheorem{conjecture}[theorem]{Conjecture}

\newtheorem{corollary}[theorem]{Corollary}

\newtheorem{criterion}[theorem]{Criterion}

\newtheorem{definition}[theorem]{Definition}

\newtheorem{example}[theorem]{Example}

\newtheorem{exercise}[theorem]{Exercise}

\newtheorem{lemma}[theorem]{Lemma}

\newtheorem{notation}[theorem]{Notation}

\newtheorem{problem}[theorem]{Problem}

\newtheorem{proposition}[theorem]{Proposition}

\newtheorem{remark}[theorem]{Remark}

\newtheorem{solution}[theorem]{Solution}

\newtheorem{summary}[theorem]{Summary}

\newenvironment{proof}[1][Proof]{\textbf{#1.} }{\ \rule{0.5em}{0.5em}}



\begin{document}



\begin{center}

\textbf{Homework 4}\bigskip

\end{center}



\noindent\textbf{Instructions}:
\noindent In problems the problems below, references such as III.2.7 refer to Problem 7 in Section 2 of Chapter III in Conway's book.\smallskip



\noindent If you use results from books including Conway's, please be explicit about what results you are using.






\begin{center}

\emph{Homework 4 is due in class at Midnight March 9.}

\end{center} 

\medskip

Do the following problems:

\begin{enumerate}


\item IV.7.1

\textbf{Sol. (Discussed with a college classmate)} In fact, I don't think that this proposition is correct. For example, pick $G$ the unit disk $B(0, 1)$, and $\gamma = \gamma(t): [0, 1] \to B$, s.t. $\gamma(t) = t$ for $0\le t < 1$, and $\gamma(1) = 0$. Then $\gamma$ is closed, and by simple calculation we know $V(\gamma) = 2$, which shows $\gamma$ is rectifiable.

Let $f = \frac{1}{z-1}$, then $f$ is analytic in $B(0, 1)$. But when $t \to 1$, $f\circ \gamma(t)\to\infty$, hence it is not rectifiable.

\item IV.7.2

\textbf{(a)} Let $f(z) = z$, pick any $z_0 \in \{z\mid d(z,\partial G) < \frac{1}{2}r\} $, then since there is only one point $z = z_0 $ satisfies $f(z) = z_0 $, by Thm 7.2,
$$
n(\gamma; z_0) = \frac{1}{2\pi i}\int_{\gamma}\frac{1}{z-z_0}dz
$$
Since $\frac{1}{z-z_0}$ is analytic on $\{z\mid d(z, \partial G) < \frac{1}{2}r\}$, by Prop 2.15, we know the integral is $0$. Hence $\{z\mid d(z,\partial G) < \frac{1}{2}r\} \subset H$.																

\item V.1.1

\textbf{(a)} Around $z = 0$,
$$
\lim_{z\to 0}|zf(z)| = \lim_{z\to 0}|\sin(z)| =\frac{1}{2}\lim_{z\to 0}|e^{iz}-e^{-iz}| \le \lim_{z\to 0}|z| = 0.
$$
Hence by Thm 1.2, $z = 0$ is removable, and $f(0) = 1$ by power series expansion.

\textbf{(b)} At $z = 0$, $g(z) = \cos(z)$ is analytic, and $\cos(0) = 1$. Thus by Prop 1.4, $z = 0$ is a pole, and the singular part is $\frac{1}{z}$.

\textbf{(c)} At $z = 0$, $\lim_{z \to 0} zf(z) = \lim_{z\to 0}\cos z-1 = 0 $, then by Thm 1.2, $0$ is removable, and $f(0) = 0$ by power series expansion.

\textbf{(d)} At $z = 0$, 
$$
f(z) = \sum_{n=0}^{-\infty}\frac{1}{(-n)!}z^{n} ,
$$
hence $0$ is an essential singularity, and $f(0<|z|<\delta) = \{z\mid |z| > exp(\frac{1}{\delta})\} $.

\textbf{(e)} At $z = 0$,
$$
f(z) = \frac{1}{z^2}\sum_{n=1}^{\infty}(-1)^{n-1}\frac{z^n}{n} = \frac{1}{z}+\sum_{n=0}^{\infty}\frac{(-1)^{n+1}}{n+2}z^n.
$$
Hence $0$ is a pole, and the singularity part is $\frac{1}{z}$.

\textbf{(f)} At $z = 0$, 
$$
f(z) = z\sum_{n=0}^{\infty}(-1)^n\frac{z^{-2n}}{n!} = z+\sum_{n=-1}^{-\infty} (-1)^{-n}\frac{z^{2n+1}}{(-n)!}
$$
Hence $0$ is an essential singularity, and $f(0<|z|<\delta) = \mathbb{C}$.

\textbf{(g)} Around $z = 0$, notice $\frac{z^2+1}{z-1}$ is analytic, hence $0$ is a pole. Since $|z| < 1$,
$$
f(z) = 1-\frac{1}{z}+\frac{2}{z-1} = 1-\frac{1}{z}-2\sum_{n=0}^{\infty}z^n,
$$
we know the singular part is $-\frac{1}{z}$.

\textbf{(h)} For any $n > 0$, 
$$
\lim_{z\to 0}z^nf(z) = \lim_{z\to 0}z^n\frac{1}{\sum_{n=1}^{\infty}\frac{1}{n!}z^n} = \infty,
$$
hence $0$ is an essential singularity, and $f(0<|z|<\delta) = \{z\mid |z| > \frac{1}{1-e^{\delta}}\} $.

\textbf{(i)} 
$$
f(z) = z\sum_{n=0}^{\infty}(-1)^n\frac{z^{-(2n+1)}}{(2n+1)!} = 1+\sum_{n=-1}^{-\infty}(-1)^{-n}\frac{z^{2n-1}}{(-2n+1)!},
$$
hence $0$ is an essential singularity, and $f(0<|z|<\delta) = \{z\mid |z| < \delta\} $.

\textbf{(j)} Same with (i), 0 is an essential singularity, and $f(0<|z|<\delta) = \{z\mid |z| < \delta^n\}$.

\item V.1.4

\textbf{(a)} 
$$
f(z) = \frac{1}{z}(\frac{1}{1-z}-\frac{1}{2(1-z/2)}) = \frac{1}{z}(\sum_{n=0}^{\infty}z^n-\frac{1}{2}\sum_{n=0}^{\infty}(\frac{z}{2})^n) = \frac{1}{2z} + \sum_{n=0}^{\infty}(1-\frac{1}{2^{n+2}})z^n
$$

\textbf{(b)} 
% $$
% f(z) = \frac{1}{z-1}\frac{-1}{2}(\frac{1}{1-(z-1)}+\frac{1}{1+(z-1)}) = -\frac{1}{2}\frac{1}{z-1}(\sum_{n=0}^{\infty}(z-1)^n + \sum_{n=0}^\infty(-1)^n(z-1)^n) = -\frac{1}{z-1}-\sum_{n=0}^{\infty}(z-1)^{2n+1}.
% $$
$$
f(z) = \frac{1}{z}(\frac{1}{z-2}-\frac{1}{z-1}) = \frac{1}{z}(-\frac{1}{2}\frac{1}{1-\frac{z}{2}} - \frac{1}{z}\frac{1}{1-\frac{1}{z}}) = \frac{1}{z}(-\sum_{n=0}^{\infty}\frac{z^n}{2^{n+1}} -\sum_{n=0}^{\infty}z^{-n-1}) = -\sum_{n=-1}^{\infty}\frac{z^n}{2^{n+2}}-\sum_{n=-\infty}^{-2}z^n
$$

\textbf{(c)}
$$
f(z) = \frac{1}{z}(\frac{\frac{1}{z}}{1-\frac{2}{z}}-\frac{\frac{1}{z}}{1-\frac{1}{z}}) = \frac{1}{z}(\sum_{n=0}^{\infty}(\frac{2}{z})^n-\sum_{n=0}^{\infty}\frac{1}{z^n}) = \sum_{n=-\infty}^{-1}(2^{-(n+1)}-1)z^n.
$$

\item V.1.12

\textbf{Proof.}
By (1.11), since $f$ is analytic on $0 < |z| < \infty$,
$$
a_n = \frac{1}{2\pi i}\int_{\gamma}\frac{\exp(\frac{1}{2}\lambda(z+\frac{1}{z}) )}{z^{n+1}}dz
$$
pick $\gamma = \exp(it)$, the unit circle, then
$$
a_n = \frac{1}{2\pi }\int_{0}^{2\pi}e^{\lambda\cos t}e^{-int}dt = \frac{1}{2\pi}\int_{0}^{2\pi}e^{\lambda\cos t}(\cos nt -i\sin nt)dt,
$$
the real part is
$$
\frac{1}{2\pi}\int_{0}^{2\pi}e^{\lambda\cos t}\cos ntdt = \frac{1}{2\pi}\left(\int_{0}^{\pi}e^{\lambda\cos t}\cos ntdt -\int_{\pi}^{0}e^{\lambda \cos(2\pi -s)}\cos(2\pi-s)ds\right) = \frac{1}{\pi}\int_{0}^{\pi}e^{\lambda\cos t}\cos ntdt.
$$
and the imaginary part is 
$$
-\frac{i}{2\pi}\int_{0}^{2\pi}e^{\lambda\cos t}\sin ntdt = -\frac{i}{2\pi} \left(\int_{0}^{\pi}e^{\lambda\cos t}\sin ntdt + \int_{\pi}^{0}e^{\lambda\cos(2\pi -s)}\sin(2\pi-s)ds\right) = 0.
$$
Hence $a_n = \frac{1}{\pi}\int_{0}^{\pi}e^{\lambda\cos t}\cos ntdt $.


$$
b_n = \frac{1}{2\pi i}\int_{\gamma}\frac{\exp(\frac{1}{2}\lambda(z-\frac{1}{z}))}{z^{n+1}}dz
$$
pick $\gamma = \exp(it)$,
$$
b_n = \frac{1}{2\pi}\int_{0}^{2\pi}e^{\lambda i \sin t}e^{-int}dt = \frac{1}{2\pi}\int_{0}^{2\pi} \cos(nt-\lambda \sin t) -i\sin(nt - \lambda \sin t) dt.
$$
The real part is
$$
\begin{aligned}
\frac{1}{2\pi}\int_{0}^{2\pi} &\cos(nt-\lambda \sin t)dt = \frac{1}{2\pi}\left(\int_{0}^{\pi}\cos(nt-\lambda \sin t)dt - \int_{\pi}^{0}\cos(n(2\pi -s)-\lambda\sin(2\pi-s))ds\right) \\
&= \frac{1}{2\pi}\left(\int_{0}^{\pi}\cos(nt-\lambda \sin t)dt +\int_{0}^{\pi}\cos(-ns+\lambda \sin s)ds\right) = \frac{1}{\pi}\int_{0}^{\pi}\cos(nt -\lambda \sin t)dt.
\end{aligned}
$$
and the imaginary part is
$$
\begin{aligned}
-\frac{i}{2\pi}\int_{0}^{2\pi}&\sin(nt-\lambda \sin t)dt = -\frac{i}{2\pi} \left(\int_{0}^{\pi}\sin(nt-\lambda\sin t)dt -\int_{\pi}^{0}\sin(n(2\pi-s) -\lambda\sin(2\pi-s))ds\right) \\
&= -\frac{i}{2\pi} \left(\int_{0}^{\pi}\sin(nt-\lambda\sin t)dt-\int_{0}^{\pi}\sin(-ns+\lambda \sin s)ds\right) = 0.
\end{aligned}
$$
Hence $b_n = \frac{1}{\pi}\int_{0}^{\pi}\cos(nt -\lambda\sin t )dt $.

\item V.1.13

\textbf{(a)} Suppose $f$ is entire and has a removable singularity at $\infty$, then $g(z) = f(\frac{1}{z})$ has a removable singularity at $z = 0$, thus we can define $g(0) = a < \infty$, which means $f$ is bounded in the neighbourhood of $\infty$. Hence by Liouville Thm, $f$ is constant.

\textbf{(b)} By assumption, $g(z) = f(\frac{1}{z}) = \frac{1}{z^m}h(z)$, where $h(z)$ is analytic at $z = 0$. Since $f(z) = z^{m}h(\frac{1}{z})$ is entire, it means at $z = 0$, $h$ has a definition or has a removable singularity, and $h$ is entire. Hence, by (1) we know $h$ is a constant, which means $f$ is a polynomial of degree m.

\textbf{(c)} Let $f(z) = \frac{\prod (z-u_i)}{\prod (z-v_i)}$, then 
$$
g(z) = f(\frac{1}{z}) = \frac{\prod_{i=1}^{n} (\frac{1}{z}-u_i)}{\prod_{i=1}^{m} (\frac{1}{z}-v_i)}
$$
has a removable singularity at $z = 0$, which means $\lim_{z\to 0}g(z) $ exists and is not $\infty$. Notice
$$
\lim_{z \to 0}g(z) = \lim_{z\to 0}\frac{z^m\prod_{i=1}^{n} (\frac{1}{z}-u_i)}{\prod_{i=1}^{m} (1-v_iz)},
$$
if $m < n$, then $\lim_{z\to 0} g(z) = \infty$, which makes a contradiction. If $m \ge n$, then the limit is well-defined. Hence $f = \frac{p(z)}{q(z)}$, where $p, q$ are polynomials, and $\deg(p) \le \deg(q)$.

\textbf{(d)} Let $f(z) = \frac{\prod (z-u_i)}{\prod (z-v_i)}$, then 
$$
g(z) = f(\frac{1}{z}) = \frac{\prod_{i=1}^{n} (\frac{1}{z}-u_i)}{\prod_{i=1}^{k} (\frac{1}{z}-v_i)}
$$
has a pole of order m at $z = 0$, which means $g(z) = \frac{h(z)}{z^m}$, where $h(z)$ is analytic at $z = 0$. Then
$$
z^mg(z) = \frac{z^m\prod_{i=1}^{n} (\frac{1}{z}-u_i)}{\prod_{i=1}^{k} (\frac{1}{z}-v_i)} = \frac{z^{m+k}\prod_{i=1}^{n} (\frac{1}{z}-u_i)}{\prod_{i=1}^{m} (1-v_iz)},
$$
and we have $m+k = n$, otherwise the order of pole is not $m$. Hence $f = \frac{p}{q}$, and $\deg(p)-\deg(q) = m$.

\item V.1.17

\textbf{Proof.} If not, first suppose $f$ has a pole of order $m$ at $a$. Then $f(z) = \frac{g(z)}{(z-a)^m}$, and $g$ is analytic on $G$. 

i) If $g(a) = 0$, then $f(z) = 0$, and we can define $f(a) = 0$, thus $a$ is a removable singularity.

ii) If $g(a)\ne 0$, then according to the isolation of zeros, $\exists r > 0$, s.t. $g(z) \ne 0$ in $B(a, 2r)$. Consider $H = B(a, r)$, by max modulus theorem, $\min_{H}|g| = \min_{\partial H}|g|$, denote is as $c \ne 0$. Hence
$$
\begin{aligned}
\int\int_{H}|f(x+iy)|^2 dxdy = \int\int_{H}\frac{|g(x+iy)|^2}{|x+iy-a|^{2m}}dxdy \ge \int\int_{H}\frac{c^2}{|x+iy-a|^2}dxdy.
\end{aligned}
$$
Let $x = Re(a)+s\cos t, y = Im(a)+s\sin t $, then
$$
\int\int_{H}|f(x+iy)|^2 dxdy \ge \int_{0}^{r}\int_{0}^{2\pi}\frac{c^2}{s^{2m}}sdsdt = 2\pi c^2\int_{0}^{r}s^{1-2m}ds.
$$
If $m = 1$, then the integral becomes
$$
\left.\int_{0}^{r} s^{-1}ds = \ln(s)\right\arrowvert_{0}^{r} = \infty.
$$
If $m > 1$, then the integral is
$$
\left.\int_{0}^{r} s^{1-2m} ds = \frac{1}{2-2m}s^{2-2m}\right\arrowvert_{0}^r = \infty.
$$
Since the integrand is nonnegative,
$$
\int_{G} |f|^2 \ge \int_{H} |f|^2 = \infty,
$$
which makes a contradiction. With the same method, we know $a$ is not an essential singularity. Hence $a$ is a removable one.

By the deduction we know, if $0 < p < 2$, $a$ could be a removable singularity or a pole of order $m$, which satisfies $pm < 2$. If $p \ge 2$, then $a$ is a removable singularity.


\item V.2.1

\item V.2.2

\item V.2.3

\item V.2.4

\item V.2.5 






\end{enumerate}

\end{document}

