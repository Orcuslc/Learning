

\documentclass{article}%

\usepackage{amsmath}

\usepackage{graphicx}

\usepackage{amsfonts}%

\usepackage{amssymb}





\setlength{\topmargin}{-0.75in}

\setlength{\textheight}{9.25in}

\setlength{\oddsidemargin}{0.0in}

\setlength{\evensidemargin}{0.0in}

\setlength{\textwidth}{6.5in}

\def\labelenumi{\arabic{enumi}.}

\def\theenumi{\arabic{enumi}}

\def\labelenumii{(\alph{enumii})}

\def\theenumii{\alph{enumii}}

\def\p@enumii{\theenumi.}

\def\labelenumiii{\arabic{enumiii}.}

\def\theenumiii{\arabic{enumiii}}

\def\p@enumiii{(\theenumi)(\theenumii)}

\def\labelenumiv{\arabic{enumiv}.}

\def\theenumiv{\arabic{enumiv}}

\def\p@enumiv{\p@enumiii.\theenumiii}

\pagestyle{plain}

\setcounter{secnumdepth}{0}

\newtheorem{theorem}{Theorem}

\newtheorem{acknowledgement}[theorem]{Acknowledgement}

\newtheorem{algorithm}[theorem]{Algorithm}

\newtheorem{axiom}[theorem]{Axiom}

\newtheorem{case}[theorem]{Case}

\newtheorem{claim}[theorem]{Claim}

\newtheorem{conclusion}[theorem]{Conclusion}

\newtheorem{condition}[theorem]{Condition}

\newtheorem{conjecture}[theorem]{Conjecture}

\newtheorem{corollary}[theorem]{Corollary}

\newtheorem{criterion}[theorem]{Criterion}

\newtheorem{definition}[theorem]{Definition}

\newtheorem{example}[theorem]{Example}

\newtheorem{exercise}[theorem]{Exercise}

\newtheorem{lemma}[theorem]{Lemma}

\newtheorem{notation}[theorem]{Notation}

\newtheorem{problem}[theorem]{Problem}

\newtheorem{proposition}[theorem]{Proposition}

\newtheorem{remark}[theorem]{Remark}

\newtheorem{solution}[theorem]{Solution}

\newtheorem{summary}[theorem]{Summary}

\newenvironment{proof}[1][Proof]{\textbf{#1.} }{\ \rule{0.5em}{0.5em}}



\begin{document}



\begin{center}

\textbf{Homework 4}\bigskip

\end{center}



\noindent\textbf{Instructions}:
\noindent In problems the problems below, references such as III.2.7 refer to Problem 7 in Section 2 of Chapter III in Conway's book.\smallskip



\noindent If you use results from books including Conway's, please be explicit about what results you are using.






\begin{center}

\emph{Homework 4 is due in class at Midnight March 9.}

\end{center} 

\medskip

Do the following problems:

\begin{enumerate}


\item IV.7.1

\textbf{Sol. (Discussed with a college classmate)} In fact, I don't think that this proposition is correct. For example, pick $G$ the unit disk $B(0, 1)$, and $\gamma = \gamma(t): [0, 1] \to B$, s.t. $\gamma(t) = t$ for $0\le t < 1$, and $\gamma(1) = 0$. Then $\gamma$ is closed, and by simple calculation we know $V(\gamma) = 2$, which shows $\gamma$ is rectifiable.

Let $f = \frac{1}{z-1}$, then $f$ is analytic in $B(0, 1)$. But when $t \to 1$, $f\circ \gamma(t)\to\infty$, hence it is not rectifiable.

\item IV.7.2

\textbf{(a)} Let $f(z) = z$, pick any $z_0 \in \{z\mid d(z,\partial G) < \frac{1}{2}r\} $, then since there is only one point $z = z_0 $ satisfies $f(z) = z_0 $, by Thm 7.2,
$$
n(\gamma; z_0) = \frac{1}{2\pi i}\int_{\gamma}\frac{1}{z-z_0}dz
$$
Since $\frac{1}{z-z_0}$ is analytic on $\{z\mid d(z, \partial G) < \frac{1}{2}r\}$, by Prop 2.15, we know the integral is $0$. Hence $\{z\mid d(z,\partial G) < \frac{1}{2}r\} \subset H$.																

\item V.1.1

\textbf{(a)} Around $z = 0$,
$$
\lim_{z\to 0}|zf(z)| = \lim_{z\to 0}|\sin(z)| =\frac{1}{2}\lim_{z\to 0}|e^{iz}-e^{-iz}| \le \lim_{z\to 0}|z| = 0.
$$
Hence by Thm 1.2, $z = 0$ is removable, and $f(0) = 1$ by power series expansion.

\textbf{(b)} At $z = 0$, $g(z) = \cos(z)$ is analytic, and $\cos(0) = 1$. Thus by Prop 1.4, $z = 0$ is a pole, and the singular part is $\frac{1}{z}$.

\textbf{(c)} At $z = 0$, $\lim_{z \to 0} zf(z) = \lim_{z\to 0}\cos z-1 = 0 $, then by Thm 1.2, $0$ is removable, and $f(0) = 0$ by power series expansion.

\textbf{(d)} At $z = 0$, 
$$
f(z) = \sum_{n=0}^{-\infty}\frac{1}{(-n)!}z^{n} ,
$$
hence $0$ is an essential singularity, and $f(0<|z|<\delta) = \{z\mid |z| > exp(\frac{1}{\delta})\} $.

\textbf{(e)} At $z = 0$,
$$
f(z) = \frac{1}{z^2}\sum_{n=1}^{\infty}(-1)^{n-1}\frac{z^n}{n} = \frac{1}{z}+\sum_{n=0}^{\infty}\frac{(-1)^{n+1}}{n+2}z^n.
$$
Hence $0$ is a pole, and the singularity part is $\frac{1}{z}$.

\textbf{(f)} At $z = 0$, 
$$
f(z) = z\sum_{n=0}^{\infty}(-1)^n\frac{z^{-2n}}{n!} = z+\sum_{n=-1}^{-\infty} (-1)^{-n}\frac{z^{2n+1}}{(-n)!}
$$
Hence $0$ is an essential singularity, and $f(0<|z|<\delta) = \mathbb{C}$.

\textbf{(g)} Around $z = 0$, notice $\frac{z^2+1}{z-1}$ is analytic, hence $0$ is a pole. Since $|z| < 1$,
$$
f(z) = 1-\frac{1}{z}+\frac{2}{z-1} = 1-\frac{1}{z}-2\sum_{n=0}^{\infty}z^n,
$$
we know the singular part is $-\frac{1}{z}$.

\textbf{(h)} For any $n > 0$, 
$$
\lim_{z\to 0}z^nf(z) = \lim_{z\to 0}z^n\frac{1}{\sum_{n=1}^{\infty}\frac{1}{n!}z^n} = \infty,
$$
hence $0$ is an essential singularity, and $f(0<|z|<\delta) = \{z\mid |z| > \frac{1}{1-e^{\delta}}\} $.

\textbf{(i)} 
$$
f(z) = z\sum_{n=0}^{\infty}(-1)^n\frac{z^{-(2n+1)}}{(2n+1)!} = 1+\sum_{n=-1}^{-\infty}(-1)^{-n}\frac{z^{2n-1}}{(-2n+1)!},
$$
hence $0$ is an essential singularity, and $f(0<|z|<\delta) = \{z\mid |z| < \delta\} $.

\textbf{(j)} Same with (i), 0 is an essential singularity, and $f(0<|z|<\delta) = \{z\mid |z| < \delta^n\}$.

\item V.1.4



\item V.1.12

\item V.1.13

\item V.1.17

\item V.2.1

\item V.2.2

\item V.2.3

\item V.2.4

\item V.2.5 






\end{enumerate}

\end{document}

