

\documentclass{article}%

\usepackage{amsmath}

\usepackage{graphicx}

\usepackage{amsfonts}%

\usepackage{amssymb}





\setlength{\topmargin}{-0.75in}

\setlength{\textheight}{9.25in}

\setlength{\oddsidemargin}{0.0in}

\setlength{\evensidemargin}{0.0in}

\setlength{\textwidth}{6.5in}

\def\labelenumi{\arabic{enumi}.}

\def\theenumi{\arabic{enumi}}

\def\labelenumii{(\alph{enumii})}

\def\theenumii{\alph{enumii}}

\def\p@enumii{\theenumi.}

\def\labelenumiii{\arabic{enumiii}.}

\def\theenumiii{\arabic{enumiii}}

\def\p@enumiii{(\theenumi)(\theenumii)}

\def\labelenumiv{\arabic{enumiv}.}

\def\theenumiv{\arabic{enumiv}}

\def\p@enumiv{\p@enumiii.\theenumiii}

\pagestyle{plain}

\setcounter{secnumdepth}{0}

\newtheorem{theorem}{Theorem}

\newtheorem{acknowledgement}[theorem]{Acknowledgement}

\newtheorem{algorithm}[theorem]{Algorithm}

\newtheorem{axiom}[theorem]{Axiom}

\newtheorem{case}[theorem]{Case}

\newtheorem{claim}[theorem]{Claim}

\newtheorem{conclusion}[theorem]{Conclusion}

\newtheorem{condition}[theorem]{Condition}

\newtheorem{conjecture}[theorem]{Conjecture}

\newtheorem{corollary}[theorem]{Corollary}

\newtheorem{criterion}[theorem]{Criterion}

\newtheorem{definition}[theorem]{Definition}

\newtheorem{example}[theorem]{Example}

\newtheorem{exercise}[theorem]{Exercise}

\newtheorem{lemma}[theorem]{Lemma}

\newtheorem{notation}[theorem]{Notation}

\newtheorem{problem}[theorem]{Problem}

\newtheorem{proposition}[theorem]{Proposition}

\newtheorem{remark}[theorem]{Remark}

\newtheorem{solution}[theorem]{Solution}

\newtheorem{summary}[theorem]{Summary}

\newenvironment{proof}[1][Proof]{\textbf{#1.} }{\ \rule{0.5em}{0.5em}}



\begin{document}



\begin{center}

\textbf{Homework 6}\bigskip

\end{center}



\noindent\textbf{Instructions}:
\noindent In problems the problems below, references such as III.2.7 refer to Problem 7 in Section 2 of Chapter III in Conway's book.\smallskip



\noindent If you use results from books including Conway's, please be explicit about what results you are using.






\begin{center}

\emph{Homework 6 is due in class at Midnight Monday, April 23.}

\end{center} 

\medskip

Do the following problems:

\begin{enumerate}


\item VI.1.4  Notice that this is related to a problem on the second midterm.

\textbf{Proof.}
If not, suppose for each $\epsilon > 0$, there is a polynomial $p(z)$, s.t. $\sup\{|p(z) - z^{-1}| : z\in A\} < \epsilon$. We claim that $|p(z) - z^{-1}| \le \epsilon$ for all $z\in A$. Otherwise, by the continuity of $|p(z)|$, $\exists z_0\in O(z, r)\cap A $, s.t. $|p(z_0)-z^{-1}| > \epsilon$, which makes a contradiction.

Now pick $\epsilon = \frac{1}{2R}$, then in $A$, $|p(z)-\frac{1}{z}| \le \frac{1}{2R}$, which means
$$
|zp(z)-1|\le \frac{|z|}{|2R|} \le \frac{1}{2}
$$
on $A$, hence on $\{|z| = R\}\subset\partial A$. Hence, by Maximum Modulus Thm, since $g(z) = zp(z)-1$ is analytic in $O(0, R)$, we know for each $z\in O(0, R)$, $|g(z)| < \max|g(\partial A)| \le \frac{1}{2}$. However, $|g(0)| = 1 > \frac{1}{2}$, which makes a contradiction.


\item VI.1.5

\textbf{Proof.}
Let $\{z_k\}_{k=1}^n $ be zeros of $f$ in $B(0, \frac{1}{3}R)$, and 
$$
g(z) = \frac{\prod_{k=1}^{n}z_k}{\prod_{k=1}^{n}(z-z_k)}f(z),
$$
then $g(z)$ is analytic in $B(0, R)$, and $g(0) = f(0) = a$. Hence by Max Modulus Thm,
$$
a = |g(0)| \le \left| \frac{\prod_{k=1}^{n}z_k}{\prod_{k=1}^{n}(z-z_k)}f(z)\right|_{z\in\{|z| = R\}} \le \frac{(\frac{1}{3}R)^n}{(\frac{2}{3}R)^n}M = \frac{1}{2^n}M.
$$
Then $n \le \frac{\log (M/a)}{\log 2}$.


\item VI.1.6

Let $h(z) = \frac{f(z)}{g(z)}$, then since $f, g$ never vanish in $B(0, R)$, $h$ and $H = \frac{1}{h}$ are both analytic in $B(0, R)$. Then by M.M.T,
$$
|h(z)| \le \left|\frac{f(z)}{g(z)}\right|_{|z| = 1} = 1, 
$$
and
$$
|H(z)| = \left|\frac{1}{h(z)}\right| \le \left|\frac{g(z)}{f(z)}\right|_{|z| = 1} = 1,
$$
we know $|h(z)| \equiv 1$ in $B(0, R)$, and by M.M.T, $h(z)$ is a constant $\lambda$, and $|\lambda| = 1$.

\item VI.2.5

\textbf{(a)}
Consider 
$$
\varphi(z) = \varphi_{z_1}(z)\cdots\varphi_{z_n}(z),
$$
where $\varphi_{z_k}(z) $ is the Mobius transformation 
$$
\varphi_{z_k}(z) = \frac{z-z_k}{1-\bar{z_k}z}.
$$
Then 
$$
g(z) = \frac{f(z)}{\varphi(z)}
$$
is analytic in $D$. Now we only need to show that $|g| \le M$ in $D$. If not, assume there is a point $z_0\in D $ , s.t. $|g(z_0)| = M_0 > M $. Then for any $|z_0| < r < 1$, by max modulus thm, 
$$
\max_{|z| = r}|h(z)| \ge M_0.
$$
Hence there is some $z_r\in \{|z| = r\}$, s.t.
$$
M \ge |f(z_r)| \ge M_0\varphi(z_r).
$$
Let $r\to 1^- $, take limit on both side, we get
$$
M \ge M_0\lim_{r\to 1^-}\varphi(z_r) = M_0
$$
as $|\varphi| = 1$ on $\partial D$, which leads to a contradiction. Hence, $|f| \le M|\varphi|$.

\textbf{(b)}
With the same notation as (a), we notice
$$
h(0) = \frac{f(0)}{\varphi(0)} = (-1)^{n}M,
$$
which means $|h(0)| = M$. But we have shown $|h(z)| \le M$ on $D$, hence by M.M.T., 
$$
h(z) \equiv (-1)^nM, \forall z\in D.
$$
Hence
$$
f(z) = h(z)\varphi(z) = (-1)^nM\varphi(z).
$$

\item VI.2.8

\textbf{Sol.}
Let $g(z) = \varphi_{\frac{1}{2}}\circ f(z) $, then 
$$
g(0) = 0, ~g'(0) = \frac{-f(0)f'(0)+\frac{5}{4}f'(0)}{(1-\frac{1}{2}f(0))^2} = 1,
$$
and $g$ is analytic in $D$. Hence by Schwarz's Thm, there is some $|c| = 1$,
$$
g(z) = cz = \varphi_{\frac{1}{2}}\circ f(z),
$$ 
which means
$$
f(z) = \varphi_{-\frac{1}{2}}(cz),
$$
where $|c| = 1$. Thus the solution is not unique.


\item VII.2.1

\textbf{Proof.} 
If $f_n\to f $, then since $\gamma$ is compact, we know $f_n\to f $ uniformly on $\gamma$.

Conversely, for each $z\in G$, let $\gamma$ be a circle $C(z, r)\subset G$, then by Cauchy's theorem,
$$
f_n(z) = \frac{1}{2\pi i}\int_{\gamma}\frac{f_n(w)}{w-z}dw,
$$
hence
$$
|f(z)-f_n(z)| = \frac{1}{2\pi}\left|\int_{\gamma}\frac{f_n(w)-f(w)}{w-z}dw\right| \le 2\sup_{w\in\gamma}\{|f(w)-f_n(w)|\}
$$
For each $\epsilon > 0$, since $f_n\to f $ uniformly on $\gamma$, there is a $N > 0$, for each $n \ge N$ and each $w\in\gamma$, $|f_n(w) - f(w)| < \frac{1}{2}\epsilon$. Hence, 
$$
|f_n(z) - f(z)| < \epsilon.
$$
Thus $f_n \to f$ on every $z\in G$.

\item VII.2.4

\textbf{Proof.}
Since $\{f_n\}$ is locally bounded, we know by Montel's theorem that $\{f_n\}$ is normal. Then each subsequence of $\{f_n\}$ has a subsequence that converges to an analytic function, and these functions are equal in $A$ which has a limit point. Hence these limit functions are equal in $G$, which means $f_n\to f $.


\item VII.2.5

\textbf{(b) to (a)} In fact it is trivial since uniform boundness leads to local boundness.

\textbf{(a) to (b)} I don't think it is right. For example, let $f_n(z) = nz^n $ and $G$ be the unit disk $D$, then of course $\{f_n\}$ is locally bounded by Lemma 2.8. Hence by Montel's theorem, $\{f_n\}$ is normal. However, since 
$$
\lim_{z\to \partial D} |f_n(z)| = n,
$$
for each $\epsilon > 0$ and each $c > 0$, we can pick some $n$ large enough and $z$ sufficiently close to $\partial D$, and $|cf_n(z)| > \epsilon$.




\end{enumerate}

\end{document}

