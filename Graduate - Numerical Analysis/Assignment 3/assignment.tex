%!TEX program = xelate
%%%%%%%%%%%%%%%%%%%%%%%%%%%%%%%%%%%%%%%%%
% Modified By Orcuslc, 2016-9-21
% Modified for Assignments
% http://github.com/orcuslc
%
% Wilson Resume/CV
% Structure Specification File
% Version 1.0 (22/1/2015)
%
% This file has been downloaded from:
% http://www.LaTeXTemplates.com
%
% License:
% CC BY-NC-SA 3.0 (http://creativecommons.org/licenses/by-nc-sa/3.0/)
%
%%%%%%%%%%%%%%%%%%%%%%%%%%%%%%%%%%%%%%%%%

%----------------------------------------------------------------------------------------
%	PACKAGES AND OTHER DOCUMENT CONFIGURATIONS
%----------------------------------------------------------------------------------------
\documentclass[10pt]{article}

\usepackage{listings}
\usepackage{xcolor}
\usepackage{amsmath,amsthm,amssymb}
\usepackage{epstopdf}
\usepackage{graphicx}
\usepackage{clrscode3e}

\DeclareGraphicsExtensions{.eps,.ps,.jpg,.bmp}


\usepackage[a4paper, hmargin=25mm, vmargin=30mm, top=20mm]{geometry} % Use A4 paper and set margins

\usepackage{fancyhdr} % Customize the header and footer

\usepackage{lastpage} % Required for calculating the number of pages in the document

\usepackage{hyperref} % Colors for links, text and headings

\setcounter{secnumdepth}{0} % Suppress section numbering

%\usepackage[proportional,scaled=1.064]{erewhon} % Use the Erewhon font
%\usepackage[erewhon,vvarbb,bigdelims]{newtxmath} % Use the Erewhon font
\usepackage[utf8]{inputenc} % Required for inputting international characters
\usepackage[T1]{fontenc} % Output font encoding for international characters

\usepackage{fontspec} % Required for specification of custom fonts
\setmainfont[Path = ./fonts/,
Extension = .otf,
BoldFont = Erewhon-Bold,
ItalicFont = Erewhon-Italic,
BoldItalicFont = Erewhon-BoldItalic,
SmallCapsFeatures = {Letters = SmallCaps}
]{Erewhon-Regular}

\usepackage{color} % Required for custom colors
\definecolor{slateblue}{rgb}{0.17,0.22,0.34}

\usepackage{sectsty} % Allows customization of titles
\sectionfont{\color{slateblue}} % Color section titles

\fancypagestyle{plain}{\fancyhf{}\cfoot{\thepage\ of \pageref{LastPage}}} % Define a custom page style
\pagestyle{plain} % Use the custom page style through the document
\renewcommand{\headrulewidth}{0pt} % Disable the default header rule
\renewcommand{\footrulewidth}{0pt} % Disable the default footer rule

\setlength\parindent{0pt} % Stop paragraph indentation

% Non-indenting itemize
\newenvironment{itemize-noindent}
{\setlength{\leftmargini}{0em}\begin{itemize}}
{\end{itemize}}

% Text width for tabbing environments
\newlength{\smallertextwidth}
\setlength{\smallertextwidth}{\textwidth}
\addtolength{\smallertextwidth}{-2cm}

\newcommand{\sqbullet}{~\vrule height .8ex width .6ex depth -.05ex} % Custom square bullet point 


\newcommand{\tbf}[1]{\textbf{#1}}
\newcommand{\tit}[1]{\textit{#1}}
\newcommand{\mbb}[1]{\mathbb{#1}}
\newcommand{\blue}[1]{\color{blue}{#1}}
\newcommand{\red}[1]{\color{red}{#1}}
\newcommand{\sblue}[1]{\color{slateblue}{#1}}
\newcommand{\n}{\\[5pt]}
\newcommand{\tr}{^\top}
\newcommand{\vt}[1]{
\Vert #1 \Vert
}
\newcommand{\bra}[5]{
#1=\left\{
\begin{aligned}
#2 ,&\quad #4 \\
#3 ,&\quad #5
\end{aligned}
\right.
}

\renewcommand{\title}[2] {
{\Huge{\color{slateblue}\textbf{#1}}}
\hfill
\LARGE{\color{slateblue}\textbf{#2}} \\[10pt]
\large{\color{slateblue}\textbf{Chuan Lu, 13300180056, chuanlu13@fudan.edu.cn}} \\[1mm]
\rule{\textwidth}{0.5mm}
}

\newcommand{\problem}[2] {
\vspace{20pt}
\LARGE{\color{slateblue}\textbf{Problem #1.}}
\vspace{2mm}
#2 \\[10pt]
}

\renewcommand{\proof}[2] {
\large{\color{slateblue}\textit{\textbf{#1.}}}
#2 \qed \\[3mm]
}

\newcommand{\solution}[2] {
\large{\color{slateblue}\textit{\textbf{#1.}}}
#2 \\[3mm]
}


\newcommand{\algorithm}[2] {
\begin{codebox}
\Procname{$\proc{Algorithm #1}$}
#2
\end{codebox}
}

\newcommand{\refgroup}[1] {
\LARGE{\color{slateblue}\textbf{Reference}} 
\begin{tabbing}
\hspace{5mm} \= \kill
#1
\end{tabbing}
}

\newcommand{\reference}[1] {
\sqbullet \ \  \large{#1} \\
}
% \newcommand{\solution}[2] {
% \LARGE{\color{slateblue}\textit{#1}}
% \ #2 \qed
% }

% \newenvironment{problem}[2][Problem]{\begin{trivlist}
% \item[\hskip \labelsep {\bfseries #1}\hskip \labelsep {\bfseries #2.}]}{\end{trivlist}}
\usepackage{epstopdf}
\usepackage{graphics}
\usepackage{subfig}
\usepackage{listings}
\lstset{
  breaklines=true,
  xleftmargin=25pt,
  xrightmargin=25pt,
  aboveskip=0pt,
  belowskip=10pt,
  basicstyle=\ttfamily,
  showstringspaces=false,
  frame=ltrb,
  tabsize=4,
  numbers=left,
  numberstyle=\small,
  numbersep=8pt,
  morekeywords={*, factorial, sum, erlang},
  keywordstyle=\color{blue!70}, commentstyle=\color{red!50!green!50!blue!50},
}
\DeclareGraphicsExtensions{.eps,.ps,.jpg,.bmp}

\begin{document}

\title{Numerical Analysis \\ Assignment 3}
\date{\today}
\author{Chuan Lu}

\maketitle

\problem{1}{Problem 1.26}
\solution{(a)}{
According to Taylor series, 
$$
f(x) = \frac{e^x-e^{-x}}{2x} = \frac{1}{2x}\left(1+x+\frac{x^2}{2}+\frac{x^3}{3!}-1+x-\frac{x^2}{2}+\frac{x^3}{3!}+O(x^5) \right) = 1+\frac{x^3}{6}+O(x^4).
$$
Thus 
$$
\lim_{x\to 0}f(x) = 1.
$$
}
\solution{(b)}{
$$
\begin{aligned}
f(x) &= \frac{\log(1-x)+xe^{\frac{x}{2}}}{x^3} = \frac{1}{x^3}\left(-x-\frac{x^2}{2}-\frac{x^3}{3}-\frac{x^4}{4}+x\left(1+\frac{x}{2}+\frac{x^2}{8}+\frac{x^3}{48}\right)+O(x^5)\right) \\
&= \frac{1}{x^3}\left(-\frac{5}{24}x^3-\frac{11}{48}x^4+O(x^5)\right) \\
&= -\frac{5}{24}-\frac{11}{48}x+O(x^2).
\end{aligned}
$$
Thus 
$$
\lim_{x\to 0}f(x) = -\frac{5}{24}.
$$
}


\problem{2}{Problem 1.31}
\solution{Solution}{
The subroutine and the main program are shown as follows.
}
\begin{lstlisting}[language = MATLAB]
function res = smallest_to_largest(xx)
% Add an array from the smallest number to the largest number;
% `xx`: The input array, arranged from the largest to the smallest.
    res = single(0);
    for i = length(xx):-1:1
        res = res + xx(i);
    end
\end{lstlisting}
\begin{lstlisting}[language = MATLAB]
function res = largest_to_smallest(xx)
% Add an array from the largest number to the smallest number;
% `xx`: The input array, arranged from the largest to the smallest.
    res = single(0);
    for i = 1:length(xx)
        res = res + xx(i);
    end
\end{lstlisting}
\begin{lstlisting}[language = MATLAB]
function res = precise(xx)
% Add an array using double precision and chop/round the result to single precision;
% `xx`: The input array, arranged from the largest to the smallest.
    res = sum(xx);
    res = single(res);
\end{lstlisting}
\begin{lstlisting}[language = MATLAB]
% Main Script
n = 1e7;
xx = n:-1:1;
a = single(1./xx);
b = single(1./(xx.^2));
c = single(1./(xx.^3));
d = single(((-1).^xx)./xx);

a1 = smallest_to_largest(a);
a2 = largest_to_smallest(a);
a0 = precise(a);
disp([abs(a0-a1) abs(a0-a2)]);

b1 = smallest_to_largest(b);
b2 = largest_to_smallest(b);
b0 = precise(b);
disp([abs(b0-b1) abs(b0-b2)]);

c1 = smallest_to_largest(c);
c2 = largest_to_smallest(c);
c0 = precise(c);
disp([abs(c0-c1) abs(c0-c2)]);

d1 = smallest_to_largest(d);
d2 = largest_to_smallest(d);
d0 = precise(d);
disp([abs(d0-d1) abs(d0-d2)]);
\end{lstlisting}
\solution{Result}{
The output is as below.
}
\begin{lstlisting}[language = MATLAB]
>> main
   1.2897701e+00   7.4214935e-03

   2.0873547e-04   1.1920929e-07

   6.1988831e-06               0

   9.6559525e-06               0
\end{lstlisting}
% \hspace{1mm}

\problem{3}{Problem 1.32}
\solution{Solution}{
In this computer system, machine epsilon is $\epsilon = \beta^{-n}$.

Then we have
$$
\begin{aligned}
p_m &= a_0a_1\cdots a_m(1+w) \\
&= a_0a_1\cdots a_m(1+\epsilon_1)(1+\epsilon_2)\cdots (1+\epsilon_m) \\
\end{aligned}
$$
Since we can ignore those high-order terms in the error, we have
$$
|w| = \left|\sum_{i=1}^m\epsilon_i\right| \le n\delta,
$$
where $\delta$ is the upper bound of error terms, which means $\delta = \frac{\epsilon}{2}$ if the system uses rounding and $\delta = \epsilon$ if the system uses chopping.

When using rounding, we may suppose the error terms are independent variables satisfying a uniform distribution between $[-\delta, \delta]$, where $\delta = \frac{\epsilon}{2}$. Then
$$
E[w] = n\bar\epsilon, ~\bar\epsilon\sim \mathcal{N}\left(0, \frac{\delta^2}{3n}\right).
$$
When using chopping, we can suppose $-\sigma \le \epsilon_i \le 0$, where $\sigma = \epsilon$. In the same way,
$$
E[w] = n\bar\epsilon, ~\bar\epsilon\sim\mathcal{N}\left(-\frac{\sigma}{2}, \frac{\sigma^2}{3n}\right).
$$
}
\end{document}