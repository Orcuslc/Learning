%!TEX program = xelate
%%%%%%%%%%%%%%%%%%%%%%%%%%%%%%%%%%%%%%%%%
% Modified By Orcuslc, 2016-9-21
% Modified for Assignments
% http://github.com/orcuslc
%
% Wilson Resume/CV
% Structure Specification File
% Version 1.0 (22/1/2015)
%
% This file has been downloaded from:
% http://www.LaTeXTemplates.com
%
% License:
% CC BY-NC-SA 3.0 (http://creativecommons.org/licenses/by-nc-sa/3.0/)
%
%%%%%%%%%%%%%%%%%%%%%%%%%%%%%%%%%%%%%%%%%

%----------------------------------------------------------------------------------------
%	PACKAGES AND OTHER DOCUMENT CONFIGURATIONS
%----------------------------------------------------------------------------------------
\documentclass[10pt]{article}

\usepackage{listings}
\usepackage{xcolor}
\usepackage{amsmath,amsthm,amssymb}
\usepackage{epstopdf}
\usepackage{graphicx}
\usepackage{clrscode3e}

\DeclareGraphicsExtensions{.eps,.ps,.jpg,.bmp}


\usepackage[a4paper, hmargin=25mm, vmargin=30mm, top=20mm]{geometry} % Use A4 paper and set margins

\usepackage{fancyhdr} % Customize the header and footer

\usepackage{lastpage} % Required for calculating the number of pages in the document

\usepackage{hyperref} % Colors for links, text and headings

\setcounter{secnumdepth}{0} % Suppress section numbering

%\usepackage[proportional,scaled=1.064]{erewhon} % Use the Erewhon font
%\usepackage[erewhon,vvarbb,bigdelims]{newtxmath} % Use the Erewhon font
\usepackage[utf8]{inputenc} % Required for inputting international characters
\usepackage[T1]{fontenc} % Output font encoding for international characters

\usepackage{fontspec} % Required for specification of custom fonts
\setmainfont[Path = ./fonts/,
Extension = .otf,
BoldFont = Erewhon-Bold,
ItalicFont = Erewhon-Italic,
BoldItalicFont = Erewhon-BoldItalic,
SmallCapsFeatures = {Letters = SmallCaps}
]{Erewhon-Regular}

\usepackage{color} % Required for custom colors
\definecolor{slateblue}{rgb}{0.17,0.22,0.34}

\usepackage{sectsty} % Allows customization of titles
\sectionfont{\color{slateblue}} % Color section titles

\fancypagestyle{plain}{\fancyhf{}\cfoot{\thepage\ of \pageref{LastPage}}} % Define a custom page style
\pagestyle{plain} % Use the custom page style through the document
\renewcommand{\headrulewidth}{0pt} % Disable the default header rule
\renewcommand{\footrulewidth}{0pt} % Disable the default footer rule

\setlength\parindent{0pt} % Stop paragraph indentation

% Non-indenting itemize
\newenvironment{itemize-noindent}
{\setlength{\leftmargini}{0em}\begin{itemize}}
{\end{itemize}}

% Text width for tabbing environments
\newlength{\smallertextwidth}
\setlength{\smallertextwidth}{\textwidth}
\addtolength{\smallertextwidth}{-2cm}

\newcommand{\sqbullet}{~\vrule height .8ex width .6ex depth -.05ex} % Custom square bullet point 


\newcommand{\tbf}[1]{\textbf{#1}}
\newcommand{\tit}[1]{\textit{#1}}
\newcommand{\mbb}[1]{\mathbb{#1}}
\newcommand{\blue}[1]{\color{blue}{#1}}
\newcommand{\red}[1]{\color{red}{#1}}
\newcommand{\sblue}[1]{\color{slateblue}{#1}}
\newcommand{\n}{\\[5pt]}
\newcommand{\tr}{^\top}
\newcommand{\vt}[1]{
\Vert #1 \Vert
}
\newcommand{\bra}[5]{
#1=\left\{
\begin{aligned}
#2 ,&\quad #4 \\
#3 ,&\quad #5
\end{aligned}
\right.
}

\renewcommand{\title}[2] {
{\Huge{\color{slateblue}\textbf{#1}}}
\hfill
\LARGE{\color{slateblue}\textbf{#2}} \\[10pt]
\large{\color{slateblue}\textbf{Chuan Lu, 13300180056, chuanlu13@fudan.edu.cn}} \\[1mm]
\rule{\textwidth}{0.5mm}
}

\newcommand{\problem}[2] {
\vspace{20pt}
\LARGE{\color{slateblue}\textbf{Problem #1.}}
\vspace{2mm}
#2 \\[10pt]
}

\renewcommand{\proof}[2] {
\large{\color{slateblue}\textit{\textbf{#1.}}}
#2 \qed \\[3mm]
}

\newcommand{\solution}[2] {
\large{\color{slateblue}\textit{\textbf{#1.}}}
#2 \\[3mm]
}


\newcommand{\algorithm}[2] {
\begin{codebox}
\Procname{$\proc{Algorithm #1}$}
#2
\end{codebox}
}

\newcommand{\refgroup}[1] {
\LARGE{\color{slateblue}\textbf{Reference}} 
\begin{tabbing}
\hspace{5mm} \= \kill
#1
\end{tabbing}
}

\newcommand{\reference}[1] {
\sqbullet \ \  \large{#1} \\
}
% \newcommand{\solution}[2] {
% \LARGE{\color{slateblue}\textit{#1}}
% \ #2 \qed
% }

% \newenvironment{problem}[2][Problem]{\begin{trivlist}
% \item[\hskip \labelsep {\bfseries #1}\hskip \labelsep {\bfseries #2.}]}{\end{trivlist}}
\usepackage{epstopdf}
\usepackage{graphics}
\usepackage{subfig}
\usepackage{listings}
\lstset{
  breaklines=true,
  xleftmargin=25pt,
  xrightmargin=25pt,
  aboveskip=0pt,
  belowskip=10pt,
  basicstyle=\ttfamily,
  showstringspaces=false,
  frame=ltrb,
  tabsize=4,
  numbers=left,
  numberstyle=\small,
  numbersep=8pt,
  morekeywords={*, factorial, sum, erlang},
  keywordstyle=\color{blue!70}, commentstyle=\color{red!50!green!50!blue!50},
}
\DeclareGraphicsExtensions{.eps,.ps,.jpg,.bmp}

\begin{document}

\title{Numerical Analysis \\ Assignment 2}
\date{\today}
\author{Chuan}

\maketitle

\problem{1}{Problem 1.20}
\solution{Solution}{
The code of the first algorithm(the trivial one) is as follows.
}
\begin{lstlisting}[language = MATLAB]
function [res] = prob20(x, y)
% Problem 20;
% To compute \lim_{p\to\infty}(x^p+y^p)^{1/p};
    p = 2.^(1:20);
    res = ((x.^p+y.^p).^(1./p))';
\end{lstlisting}
\solution{Result}{
The result of computing this program is listed follows, from which we can see that the result either overflows or underflows when in extreme conditions.
}
\\
\begin{minipage}{0.5\textwidth}
\begin{lstlisting}[language = MATLAB]
>> format long
>> prob20(10^10, 10^10)

ans =

   1.0e+10 *

   1.414213562373095
   1.189207115002721
   1.090507732665258
   1.044273782427414
                 Inf
                 Inf
                 Inf
                 Inf
                 Inf
                 Inf
                 Inf
                 Inf
                 Inf
                 Inf
                 Inf
                 Inf
                 Inf
                 Inf
                 Inf
                 Inf
\end{lstlisting}
\end{minipage}
\hspace{10pt}
\begin{minipage}{0.5\textwidth}
\begin{lstlisting}[language = MATLAB]
>> format long
>> prob20(10^-10, 10^-10)

ans =

   1.0e-09 *

   0.141421356237310
   0.118920711500272
   0.109050773266526
   0.104427378242741
   0.102189679313363
                   0
                   0
                   0
                   0
                   0
                   0
                   0
                   0
                   0
                   0
                   0
                   0
                   0
                   0
                   0
\end{lstlisting}
\end{minipage}

\solution{Second part}{
When we use the idea in (1.3.8), the code is as follows.
}
\begin{lstlisting}[language = MATLAB]
function [res] = prob20revised(x, y)
% prob 20;
% revised the computation by 
% \lim_{p\to\infty} ((x/y)^p+1)^{1/p}*y
% where (x/y) < 1;
    p = 2.^(1:20);
    if(abs(x) < abs(y))
        res = ((x/y).^p+1).^(1./p)*y;
    else
        res = ((y/x).^p+1).^(1./p)*x;
    end
    res = res';
\end{lstlisting}
\solution{Result}{
And the result is listed as follows.
}
\\
\begin{minipage}{0.5\textwidth}
\begin{lstlisting}[language = MATLAB]
>> prob20revised(10^10, 10^10)

ans =

   1.0e+10 *

   1.414213562373095
   1.189207115002721
   1.090507732665258
   1.044273782427414
   1.021897148654117
   1.010889286051701
   1.005429901112803
   1.002711275050203
   1.001354719892108
   1.000677130693066
   1.000338508052682
   1.000169239705302
   1.000084616272694
   1.000042307241396
   1.000021153396965
   1.000010576642550
   1.000005288307292
   1.000002644150150
   1.000001322074201
   1.000000661036882
\end{lstlisting}
\end{minipage}
\hspace{10pt}
\begin{minipage}{0.5\textwidth}
\begin{lstlisting}[language = MATLAB]
>> prob20revised(10^-10, 10^-10)

ans =

   1.0e-09 *

   0.141421356237310
   0.118920711500272
   0.109050773266526
   0.104427378242741
   0.102189714865412
   0.101088928605170
   0.100542990111280
   0.100271127505020
   0.100135471989211
   0.100067713069307
   0.100033850805268
   0.100016923970530
   0.100008461627269
   0.100004230724140
   0.100002115339696
   0.100001057664255
   0.100000528830729
   0.100000264415015
   0.100000132207420
   0.100000066103688
\end{lstlisting}
\end{minipage}
\\
\solution{Another test}{
We can then try the case where $x$ and $y$ differs, and the computing result remains the same as the theoretical result that the limit converge to the larger number.
}
\begin{lstlisting}[language = MATLAB]
>> prob20revised(1, 2)

ans =

   2.236067977499790
   2.030543184868931
   2.000974897633078
   2.000001907334990
   2.000000000014552
   2.000000000000000
   2.000000000000000
   2.000000000000000
   2.000000000000000
   2.000000000000000
   2.000000000000000
   2.000000000000000
   2.000000000000000
   2.000000000000000
   2.000000000000000
   2.000000000000000
   2.000000000000000
   2.000000000000000
   2.000000000000000
\end{lstlisting}

\problem{2}{Computations of 
$$
\sqrt{1+x}-1, ~x\to 0.
$$}
\solution{Original}{
The code for original computation is as follows.
}
\begin{lstlisting}[language = MATLAB]
function res = prob2(x)
% problem 2;
% compute \sqrt{x+1}-1;
    res = ((x+1).^(1/2)-1)';
\end{lstlisting}
\solution{Modified}{
Theoretically, in order to avoid cancellation, we can modify the computation to 
$$
\frac{x}{\sqrt{1+x}+1}.
$$
and the code is as follows.
}
\begin{lstlisting}[language = MATLAB]
function res = prob2revised(x)
% prob 2;
% revised to x/(sqrt(x+1)+1);
    res = x./((x+1).^(1/2)+1);
    res = res';
\end{lstlisting}
\solution{Result}{
The results of the original function and revised function are listed as follows, in which the left is original result and the right is the revised result. 
}
\\
\begin{minipage}{0.5\textwidth}
\begin{lstlisting}[language = MATLAB]
>> x = 10.^(-1:-1:-16);
>> prob2(x)

ans =

   0.048808848170152
   0.004987562112089
   0.000499875062461
   0.000049998750062
   0.000004999987500
   0.000000499999875
   0.000000049999999
   0.000000005000000
   0.000000000500000
   0.000000000050000
   0.000000000005000
   0.000000000000500
   0.000000000000050
   0.000000000000005
   0.000000000000000
                   0
\end{lstlisting}
\end{minipage}
\hspace{10pt}
\begin{minipage}{0.5\textwidth}
\begin{lstlisting}[language = MATLAB]
>> x = 10.^(-1:-1:-16);
>> prob2revised(x)

ans =

   0.048808848170152
   0.004987562112089
   0.000499875062461
   0.000049998750062
   0.000004999987500
   0.000000499999875
   0.000000049999999
   0.000000005000000
   0.000000000500000
   0.000000000050000
   0.000000000005000
   0.000000000000500
   0.000000000000050
   0.000000000000005
   0.000000000000000
   0.000000000000000
\end{lstlisting}
\end{minipage}
\\
\solution{Discussion}{
From the result we can know when $x = 10^{-16}$, the original computation loses all significant numbers, while the modified computation still has a significant number.
}

\end{document}