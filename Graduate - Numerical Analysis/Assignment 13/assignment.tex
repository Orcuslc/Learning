%!TEX program = xelate
%%%%%%%%%%%%%%%%%%%%%%%%%%%%%%%%%%%%%%%%%
% Modified By Orcuslc, 2016-9-21
% Modified for Assignments
% http://github.com/orcuslc
%
% Wilson Resume/CV
% Structure Specification File
% Version 1.0 (22/1/2015)
%
% This file has been downloaded from:
% http://www.LaTeXTemplates.com
%
% License:
% CC BY-NC-SA 3.0 (http://creativecommons.org/licenses/by-nc-sa/3.0/)
%
%%%%%%%%%%%%%%%%%%%%%%%%%%%%%%%%%%%%%%%%%

%----------------------------------------------------------------------------------------
%	PACKAGES AND OTHER DOCUMENT CONFIGURATIONS
%----------------------------------------------------------------------------------------
\documentclass[10pt]{article}

\usepackage{listings}
\usepackage{xcolor}
\usepackage{amsmath,amsthm,amssymb}
\usepackage{epstopdf}
\usepackage{graphicx}
\usepackage{clrscode3e}

\DeclareGraphicsExtensions{.eps,.ps,.jpg,.bmp}


\usepackage[a4paper, hmargin=25mm, vmargin=30mm, top=20mm]{geometry} % Use A4 paper and set margins

\usepackage{fancyhdr} % Customize the header and footer

\usepackage{lastpage} % Required for calculating the number of pages in the document

\usepackage{hyperref} % Colors for links, text and headings

\setcounter{secnumdepth}{0} % Suppress section numbering

%\usepackage[proportional,scaled=1.064]{erewhon} % Use the Erewhon font
%\usepackage[erewhon,vvarbb,bigdelims]{newtxmath} % Use the Erewhon font
\usepackage[utf8]{inputenc} % Required for inputting international characters
\usepackage[T1]{fontenc} % Output font encoding for international characters

\usepackage{fontspec} % Required for specification of custom fonts
\setmainfont[Path = ./fonts/,
Extension = .otf,
BoldFont = Erewhon-Bold,
ItalicFont = Erewhon-Italic,
BoldItalicFont = Erewhon-BoldItalic,
SmallCapsFeatures = {Letters = SmallCaps}
]{Erewhon-Regular}

\usepackage{color} % Required for custom colors
\definecolor{slateblue}{rgb}{0.17,0.22,0.34}

\usepackage{sectsty} % Allows customization of titles
\sectionfont{\color{slateblue}} % Color section titles

\fancypagestyle{plain}{\fancyhf{}\cfoot{\thepage\ of \pageref{LastPage}}} % Define a custom page style
\pagestyle{plain} % Use the custom page style through the document
\renewcommand{\headrulewidth}{0pt} % Disable the default header rule
\renewcommand{\footrulewidth}{0pt} % Disable the default footer rule

\setlength\parindent{0pt} % Stop paragraph indentation

% Non-indenting itemize
\newenvironment{itemize-noindent}
{\setlength{\leftmargini}{0em}\begin{itemize}}
{\end{itemize}}

% Text width for tabbing environments
\newlength{\smallertextwidth}
\setlength{\smallertextwidth}{\textwidth}
\addtolength{\smallertextwidth}{-2cm}

\newcommand{\sqbullet}{~\vrule height .8ex width .6ex depth -.05ex} % Custom square bullet point 


\newcommand{\tbf}[1]{\textbf{#1}}
\newcommand{\tit}[1]{\textit{#1}}
\newcommand{\mbb}[1]{\mathbb{#1}}
\newcommand{\blue}[1]{\color{blue}{#1}}
\newcommand{\red}[1]{\color{red}{#1}}
\newcommand{\sblue}[1]{\color{slateblue}{#1}}
\newcommand{\n}{\\[5pt]}
\newcommand{\tr}{^\top}
\newcommand{\vt}[1]{
\Vert #1 \Vert
}
\newcommand{\bra}[5]{
#1=\left\{
\begin{aligned}
#2 ,&\quad #4 \\
#3 ,&\quad #5
\end{aligned}
\right.
}

\renewcommand{\title}[2] {
{\Huge{\color{slateblue}\textbf{#1}}}
\hfill
\LARGE{\color{slateblue}\textbf{#2}} \\[10pt]
\large{\color{slateblue}\textbf{Chuan Lu, 13300180056, chuanlu13@fudan.edu.cn}} \\[1mm]
\rule{\textwidth}{0.5mm}
}

\newcommand{\problem}[2] {
\vspace{20pt}
\LARGE{\color{slateblue}\textbf{Problem #1.}}
\vspace{2mm}
#2 \\[10pt]
}

\renewcommand{\proof}[2] {
\large{\color{slateblue}\textit{\textbf{#1.}}}
#2 \qed \\[3mm]
}

\newcommand{\solution}[2] {
\large{\color{slateblue}\textit{\textbf{#1.}}}
#2 \\[3mm]
}


\newcommand{\algorithm}[2] {
\begin{codebox}
\Procname{$\proc{Algorithm #1}$}
#2
\end{codebox}
}

\newcommand{\refgroup}[1] {
\LARGE{\color{slateblue}\textbf{Reference}} 
\begin{tabbing}
\hspace{5mm} \= \kill
#1
\end{tabbing}
}

\newcommand{\reference}[1] {
\sqbullet \ \  \large{#1} \\
}
% \newcommand{\solution}[2] {
% \LARGE{\color{slateblue}\textit{#1}}
% \ #2 \qed
% }

% \newenvironment{problem}[2][Problem]{\begin{trivlist}
% \item[\hskip \labelsep {\bfseries #1}\hskip \labelsep {\bfseries #2.}]}{\end{trivlist}}
\usepackage{epstopdf}
\usepackage{graphics}
\usepackage{subfig}
\usepackage{listings}
\lstset{
  breaklines=true,
  xleftmargin=25pt,
  xrightmargin=25pt,
  aboveskip=0pt,
  belowskip=10pt,
  basicstyle=\ttfamily,
  showstringspaces=false,
  frame=ltrb,
  tabsize=4,
  numbers=left,
  numberstyle=\small,
  numbersep=8pt,
  morekeywords={*, factorial, sum, erlang},
  keywordstyle=\color{blue!70}, commentstyle=\color{red!50!green!50!blue!50},
}
\DeclareGraphicsExtensions{.eps,.ps,.jpg,.bmp}

\begin{document}

\title{Numerical Analysis \\ Assignment 13}
\date{\today}
\author{Chuan Lu}

\maketitle

\problem{1}{Problem 5.14}
\solution{Sol}{
The result is as follows. We can see the result is better than the simple trapezoidal and simpson rule when $n = 2$ and $n = 3$.
}
\lstinputlisting{gauss_legendre.m}
\lstinputlisting{prob1.m}
\lstinputlisting{prob1_res.m}

\problem{2}{Problem 5.15}
\solution{Sol}{
In fact, this problem is just (3) in Problem 1, the result of which is the third column listed above. When compared to Newton-Cotes, we can see that Newton-Cotes formula does not converge when $n$ gets larger, but Gauss-Lengendre quadrature seems to converge when $n$ becomes larger. 
}

\problem{3}{Problem 5.17}
\solution{Sol}{
From Problem 4.20 we know, with weight function $w(x) = -\ln(x)$,
$$
\varphi_0(x) = 1, ~\varphi_1(x) = \frac{12}{\sqrt{7}}(x-\frac{1}{4}), ~\varphi_2(x) = \frac{\sqrt{647}}{180\sqrt{7}}(x^2-\frac{5}{7}(x-\frac{1}{4})-\frac{1}{9}) = \frac{\sqrt{647}}{180\sqrt{7}}(x^2-\frac{5}{7}x+\frac{17}{252})
$$
Thus the roots of $\varphi_2(x)$ are
$$
x_{1, 2} = \frac{15\pm\sqrt{106}}{42}
$$
and by (5.3.7),
$$
w_1+w_2 = 1,
$$
$$
w_1x_1+w_2x_2 = \frac{1}{2}
$$
we get
$$
w_1 = \frac{1}{2}+\frac{3}{\sqrt{106}}, ~w_2 = \frac{1}{2}-\frac{3}{\sqrt{106}}.
$$
Hence, 
$$
I(f) = (\frac{1}{2}+\frac{3}{\sqrt{106}})f(\frac{15+\sqrt{106}}{42})+(\frac{1}{2}-\frac{3}{\sqrt{106}})f(\frac{15-\sqrt{106}}{42}).
$$
}

\problem{4}{Problem 5.19}
\solution{Sol}{
From Problem 4.24 we know, 
$$
S_n(x) = \frac{1}{n+1}T_{n+1}'(x) = \frac{1}{\sqrt{1-x^2}}\sin((n+1)\cos^{-1}x).
$$
Then the roots of $S_{n}(x)$ are
$$
x_{n, j} = \cos(\frac{j\pi}{n+1}), ~j = 1, 2, \cdots, n.
$$
From the recursion relation we know $a_n = 2$, $c_n = \frac{\gamma_{n}}{\gamma_{n-1}} = 1$. Since 
$$
\gamma_0 = (S_0, S_0) = \int_{-1}^{1}\sqrt{1-x^2}dx = \frac{\pi}{2},
$$
we know $\gamma_n  = \frac{\pi}{2}$. Thus
$$
w_j = \frac{-a_n\gamma_n}{S_n'(x_j)S_{n+1}(x_j)} = -\frac{\pi\sin^2\frac{j\pi}{n+1}}{(n+1)\cos(j\pi)} = \left\{
\begin{aligned}
&-\frac{\pi}{n+1}\sin^2\frac{j\pi}{n+1}, ~n = 2k \\
&\frac{\pi}{n+1}\sin^2\frac{j\pi}{n+1}, ~n=2k+1
\end{aligned}
\right.
$$
From (5.3.10) we also know the error
$$
E_n = \frac{\gamma_n}{A_n^2(2n)!}f^{(2n)}(\eta) = \frac{\pi}{2^{n+1}(2n)!}f^{(2n)}(\eta),
$$
for some $\eta\in [-1, 1]$.
}

\end{document}