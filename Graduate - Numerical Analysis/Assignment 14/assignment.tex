%!TEX program = xelate
%%%%%%%%%%%%%%%%%%%%%%%%%%%%%%%%%%%%%%%%%
% Modified By Orcuslc, 2016-9-21
% Modified for Assignments
% http://github.com/orcuslc
%
% Wilson Resume/CV
% Structure Specification File
% Version 1.0 (22/1/2015)
%
% This file has been downloaded from:
% http://www.LaTeXTemplates.com
%
% License:
% CC BY-NC-SA 3.0 (http://creativecommons.org/licenses/by-nc-sa/3.0/)
%
%%%%%%%%%%%%%%%%%%%%%%%%%%%%%%%%%%%%%%%%%

%----------------------------------------------------------------------------------------
%	PACKAGES AND OTHER DOCUMENT CONFIGURATIONS
%----------------------------------------------------------------------------------------
\documentclass[10pt]{article}

\usepackage{listings}
\usepackage{xcolor}
\usepackage{amsmath,amsthm,amssymb}
\usepackage{epstopdf}
\usepackage{graphicx}
\usepackage{clrscode3e}

\DeclareGraphicsExtensions{.eps,.ps,.jpg,.bmp}


\usepackage[a4paper, hmargin=25mm, vmargin=30mm, top=20mm]{geometry} % Use A4 paper and set margins

\usepackage{fancyhdr} % Customize the header and footer

\usepackage{lastpage} % Required for calculating the number of pages in the document

\usepackage{hyperref} % Colors for links, text and headings

\setcounter{secnumdepth}{0} % Suppress section numbering

%\usepackage[proportional,scaled=1.064]{erewhon} % Use the Erewhon font
%\usepackage[erewhon,vvarbb,bigdelims]{newtxmath} % Use the Erewhon font
\usepackage[utf8]{inputenc} % Required for inputting international characters
\usepackage[T1]{fontenc} % Output font encoding for international characters

\usepackage{fontspec} % Required for specification of custom fonts
\setmainfont[Path = ./fonts/,
Extension = .otf,
BoldFont = Erewhon-Bold,
ItalicFont = Erewhon-Italic,
BoldItalicFont = Erewhon-BoldItalic,
SmallCapsFeatures = {Letters = SmallCaps}
]{Erewhon-Regular}

\usepackage{color} % Required for custom colors
\definecolor{slateblue}{rgb}{0.17,0.22,0.34}

\usepackage{sectsty} % Allows customization of titles
\sectionfont{\color{slateblue}} % Color section titles

\fancypagestyle{plain}{\fancyhf{}\cfoot{\thepage\ of \pageref{LastPage}}} % Define a custom page style
\pagestyle{plain} % Use the custom page style through the document
\renewcommand{\headrulewidth}{0pt} % Disable the default header rule
\renewcommand{\footrulewidth}{0pt} % Disable the default footer rule

\setlength\parindent{0pt} % Stop paragraph indentation

% Non-indenting itemize
\newenvironment{itemize-noindent}
{\setlength{\leftmargini}{0em}\begin{itemize}}
{\end{itemize}}

% Text width for tabbing environments
\newlength{\smallertextwidth}
\setlength{\smallertextwidth}{\textwidth}
\addtolength{\smallertextwidth}{-2cm}

\newcommand{\sqbullet}{~\vrule height .8ex width .6ex depth -.05ex} % Custom square bullet point 


\newcommand{\tbf}[1]{\textbf{#1}}
\newcommand{\tit}[1]{\textit{#1}}
\newcommand{\mbb}[1]{\mathbb{#1}}
\newcommand{\blue}[1]{\color{blue}{#1}}
\newcommand{\red}[1]{\color{red}{#1}}
\newcommand{\sblue}[1]{\color{slateblue}{#1}}
\newcommand{\n}{\\[5pt]}
\newcommand{\tr}{^\top}
\newcommand{\vt}[1]{
\Vert #1 \Vert
}
\newcommand{\bra}[5]{
#1=\left\{
\begin{aligned}
#2 ,&\quad #4 \\
#3 ,&\quad #5
\end{aligned}
\right.
}

\renewcommand{\title}[2] {
{\Huge{\color{slateblue}\textbf{#1}}}
\hfill
\LARGE{\color{slateblue}\textbf{#2}} \\[10pt]
\large{\color{slateblue}\textbf{Chuan Lu, 13300180056, chuanlu13@fudan.edu.cn}} \\[1mm]
\rule{\textwidth}{0.5mm}
}

\newcommand{\problem}[2] {
\vspace{20pt}
\LARGE{\color{slateblue}\textbf{Problem #1.}}
\vspace{2mm}
#2 \\[10pt]
}

\renewcommand{\proof}[2] {
\large{\color{slateblue}\textit{\textbf{#1.}}}
#2 \qed \\[3mm]
}

\newcommand{\solution}[2] {
\large{\color{slateblue}\textit{\textbf{#1.}}}
#2 \\[3mm]
}


\newcommand{\algorithm}[2] {
\begin{codebox}
\Procname{$\proc{Algorithm #1}$}
#2
\end{codebox}
}

\newcommand{\refgroup}[1] {
\LARGE{\color{slateblue}\textbf{Reference}} 
\begin{tabbing}
\hspace{5mm} \= \kill
#1
\end{tabbing}
}

\newcommand{\reference}[1] {
\sqbullet \ \  \large{#1} \\
}
% \newcommand{\solution}[2] {
% \LARGE{\color{slateblue}\textit{#1}}
% \ #2 \qed
% }

% \newenvironment{problem}[2][Problem]{\begin{trivlist}
% \item[\hskip \labelsep {\bfseries #1}\hskip \labelsep {\bfseries #2.}]}{\end{trivlist}}
\usepackage{epstopdf}
\usepackage{graphics}
\usepackage{subfig}
\usepackage{listings}
\lstset{
  breaklines=true,
  xleftmargin=25pt,
  xrightmargin=25pt,
  aboveskip=0pt,
  belowskip=10pt,
  basicstyle=\ttfamily,
  showstringspaces=false,
  frame=ltrb,
  tabsize=4,
  numbers=left,
  numberstyle=\small,
  numbersep=8pt,
  morekeywords={*, factorial, sum, erlang},
  keywordstyle=\color{blue!70}, commentstyle=\color{red!50!green!50!blue!50},
}
\DeclareGraphicsExtensions{.eps,.ps,.jpg,.bmp}

\begin{document}

\title{Numerical Analysis \\ Assignment 13}
\date{\today}
\author{Chuan Lu}

\maketitle

\problem{1}{Problem 5.23}
\solution{(a)}{
Integrate (5.4.2) with respect to x on $[0, 1]$, we get from the convergence of the function series,
$$
\int_{0}^{1}(\frac{t}{e^t-1}e^{tx}-\frac{t}{e^t-1})dx = \sum_{j = 1}^{\infty}\frac{t^j}{j!}\int_{0}^{1}B_j(x)dx,
$$
thus
$$
1-\frac{t}{e^t-1} = \sum_{j=1}^{\infty}\frac{t^j}{j!}\int_{0}^{1}B_j(x)dx,
$$
with Taylor expansion of the left term of (5.4.5), we have $B_0 = 1$, thus
$$
\sum_{j=1}^{\infty}(B_j + \int_{0}^{1}B_j(x)dx)\frac{t^j}{j!} = 0.
$$
The left term can be regarded as a Taylor series of a function $f$ at $t = 0$ which satisfies $f(0) = 0$. By the uniqueness of Taylor expansion of a analytic function, we know $f\equiv 0$. Then for all $j > 0$,
$$
B_j + \int_{0}^{1}B_j(x)dx = 0.
$$
Now we know $B_0 = 1, ~B_1 = -\frac{1}{2}$, thus from definition (5.4.5),
$$
\frac{t}{e^t-1}+\frac{1}{2}t = 1+\sum_{j=2}^{\infty}B_j\frac{t^j}{j!} = \frac{t}{2}\frac{e^t+1}{e^t-1} \equiv g(t).
$$
We have
$$
g(-t) = -\frac{t}{2}\frac{e^{-t}+1}{e^{-t}-1} = -\frac{t}{2}\frac{1+e^t}{1-e^t} = g(t),
$$
hence $g(t)$ is an even function. Substitute $t = -t$ into last term,
$$
1+\sum_{j=2}^{\infty}(-1)^{j}B_j\frac{t^j}{j!} = g(-t) = g(t) = 1+\sum_{j=2}^{\infty}B_j\frac{t^j}{j!}.
$$
By simplification,
$$
\sum_{j = 1}^{\infty}B_{2j+1}\frac{t^{2j+1}}{(2j+1)!} = 0.
$$
With the same arguments and the uniqueness of Taylor expansion, we have
$$
B_{2j+1} = 0, ~\text{for all}~ j \ge 1.
$$
}
\solution{(b)}{
Take derivatives respective to $x$ on both sides of (5.4.2), 
$$
\frac{t^2}{e^t-1}e^{tx} = \sum_{j=1}^{\infty}B_j'(x)\frac{t^j}{j!}.
$$
Hence,
$$
\sum_{j=1}^{\infty}B_j'(x)\frac{t^j}{j!} = t\frac{t(e^{tx}-1)}{e^t-1} + \frac{t^2}{e^t-1} = \sum_{j=1}^{\infty}B_j(x)\frac{t^{j+1}}{j!} + \sum_{j=0}^{\infty}B_j\frac{t^{j+1}}{j!} = \sum_{j=2}^{\infty} jB_{j-1}(x)\frac{t^j}{j!} + \sum_{j=1}^{\infty}jB_{j-1}\frac{t^j}{j!}
$$
we have
$$
tB_1'(x) + \sum_{j=2}^{\infty}B_j'(x)\frac{t^j}{j!} = B_0t+\sum_{j=2}^{\infty}j(B_{j-1}(x)+B_{j-1})\frac{t^j}{j!}.
$$
Since $tB_1'(x) = B_0t$, with the uniqueness of Taylor expansion,
$$
B_j'(x) = j(B_{j-1}'(x)+B_{j-1}), ~\text{for all}~ j \ge 2.
$$
As proved in (a), $B_{2j+1} = 0 $ for all $j \ge 1$, the conclusion holds.
}

\problem{2}{Problem 5.31}

\problem{3}{Problem 5.37}

\problem{4}{Problem 5.40}

\end{document}