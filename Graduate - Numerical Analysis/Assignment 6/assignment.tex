%!TEX program = xelate
%%%%%%%%%%%%%%%%%%%%%%%%%%%%%%%%%%%%%%%%%
% Modified By Orcuslc, 2016-9-21
% Modified for Assignments
% http://github.com/orcuslc
%
% Wilson Resume/CV
% Structure Specification File
% Version 1.0 (22/1/2015)
%
% This file has been downloaded from:
% http://www.LaTeXTemplates.com
%
% License:
% CC BY-NC-SA 3.0 (http://creativecommons.org/licenses/by-nc-sa/3.0/)
%
%%%%%%%%%%%%%%%%%%%%%%%%%%%%%%%%%%%%%%%%%

%----------------------------------------------------------------------------------------
%	PACKAGES AND OTHER DOCUMENT CONFIGURATIONS
%----------------------------------------------------------------------------------------
\documentclass[10pt]{article}

\usepackage{listings}
\usepackage{xcolor}
\usepackage{amsmath,amsthm,amssymb}
\usepackage{epstopdf}
\usepackage{graphicx}
\usepackage{clrscode3e}

\DeclareGraphicsExtensions{.eps,.ps,.jpg,.bmp}


\usepackage[a4paper, hmargin=25mm, vmargin=30mm, top=20mm]{geometry} % Use A4 paper and set margins

\usepackage{fancyhdr} % Customize the header and footer

\usepackage{lastpage} % Required for calculating the number of pages in the document

\usepackage{hyperref} % Colors for links, text and headings

\setcounter{secnumdepth}{0} % Suppress section numbering

%\usepackage[proportional,scaled=1.064]{erewhon} % Use the Erewhon font
%\usepackage[erewhon,vvarbb,bigdelims]{newtxmath} % Use the Erewhon font
\usepackage[utf8]{inputenc} % Required for inputting international characters
\usepackage[T1]{fontenc} % Output font encoding for international characters

\usepackage{fontspec} % Required for specification of custom fonts
\setmainfont[Path = ./fonts/,
Extension = .otf,
BoldFont = Erewhon-Bold,
ItalicFont = Erewhon-Italic,
BoldItalicFont = Erewhon-BoldItalic,
SmallCapsFeatures = {Letters = SmallCaps}
]{Erewhon-Regular}

\usepackage{color} % Required for custom colors
\definecolor{slateblue}{rgb}{0.17,0.22,0.34}

\usepackage{sectsty} % Allows customization of titles
\sectionfont{\color{slateblue}} % Color section titles

\fancypagestyle{plain}{\fancyhf{}\cfoot{\thepage\ of \pageref{LastPage}}} % Define a custom page style
\pagestyle{plain} % Use the custom page style through the document
\renewcommand{\headrulewidth}{0pt} % Disable the default header rule
\renewcommand{\footrulewidth}{0pt} % Disable the default footer rule

\setlength\parindent{0pt} % Stop paragraph indentation

% Non-indenting itemize
\newenvironment{itemize-noindent}
{\setlength{\leftmargini}{0em}\begin{itemize}}
{\end{itemize}}

% Text width for tabbing environments
\newlength{\smallertextwidth}
\setlength{\smallertextwidth}{\textwidth}
\addtolength{\smallertextwidth}{-2cm}

\newcommand{\sqbullet}{~\vrule height .8ex width .6ex depth -.05ex} % Custom square bullet point 


\newcommand{\tbf}[1]{\textbf{#1}}
\newcommand{\tit}[1]{\textit{#1}}
\newcommand{\mbb}[1]{\mathbb{#1}}
\newcommand{\blue}[1]{\color{blue}{#1}}
\newcommand{\red}[1]{\color{red}{#1}}
\newcommand{\sblue}[1]{\color{slateblue}{#1}}
\newcommand{\n}{\\[5pt]}
\newcommand{\tr}{^\top}
\newcommand{\vt}[1]{
\Vert #1 \Vert
}
\newcommand{\bra}[5]{
#1=\left\{
\begin{aligned}
#2 ,&\quad #4 \\
#3 ,&\quad #5
\end{aligned}
\right.
}

\renewcommand{\title}[2] {
{\Huge{\color{slateblue}\textbf{#1}}}
\hfill
\LARGE{\color{slateblue}\textbf{#2}} \\[10pt]
\large{\color{slateblue}\textbf{Chuan Lu, 13300180056, chuanlu13@fudan.edu.cn}} \\[1mm]
\rule{\textwidth}{0.5mm}
}

\newcommand{\problem}[2] {
\vspace{20pt}
\LARGE{\color{slateblue}\textbf{Problem #1.}}
\vspace{2mm}
#2 \\[10pt]
}

\renewcommand{\proof}[2] {
\large{\color{slateblue}\textit{\textbf{#1.}}}
#2 \qed \\[3mm]
}

\newcommand{\solution}[2] {
\large{\color{slateblue}\textit{\textbf{#1.}}}
#2 \\[3mm]
}


\newcommand{\algorithm}[2] {
\begin{codebox}
\Procname{$\proc{Algorithm #1}$}
#2
\end{codebox}
}

\newcommand{\refgroup}[1] {
\LARGE{\color{slateblue}\textbf{Reference}} 
\begin{tabbing}
\hspace{5mm} \= \kill
#1
\end{tabbing}
}

\newcommand{\reference}[1] {
\sqbullet \ \  \large{#1} \\
}
% \newcommand{\solution}[2] {
% \LARGE{\color{slateblue}\textit{#1}}
% \ #2 \qed
% }

% \newenvironment{problem}[2][Problem]{\begin{trivlist}
% \item[\hskip \labelsep {\bfseries #1}\hskip \labelsep {\bfseries #2.}]}{\end{trivlist}}
\usepackage{epstopdf}
\usepackage{graphics}
\usepackage{subfig}
\usepackage{listings}
\lstset{
  breaklines=true,
  xleftmargin=25pt,
  xrightmargin=25pt,
  aboveskip=0pt,
  belowskip=10pt,
  basicstyle=\ttfamily,
  showstringspaces=false,
  frame=ltrb,
  tabsize=4,
  numbers=left,
  numberstyle=\small,
  numbersep=8pt,
  morekeywords={*, factorial, sum, erlang},
  keywordstyle=\color{blue!70}, commentstyle=\color{red!50!green!50!blue!50},
}
\DeclareGraphicsExtensions{.eps,.ps,.jpg,.bmp}

\begin{document}

\title{Numerical Analysis \\ Assignment 6}
\date{\today}
\author{Chuan Lu}

\maketitle

\problem{1}{Problem 3.1, Page 185}
\solution{(a)}{
If we expand $V_{n}(x)$ by the last row, then we can know
$$
V_n(x) = A_{n+1,n+1}x^n+A_{n+1, n}x^{n-1}+\cdots + A_{n, 1},
$$
where $A_{i, j}$ is the cofactors of $V$. Then $V_n(x)$ is a polynomial of degree $n$, thus it has $n$ roots. By replacing $x$ with $x_{i}, i = 0, 1, \cdots, n-1$, we can find $V_{n}(x_i) = 0$, thus $x = x_{i}$ are exactly roots of $V_{n}$. On the other hand, the coefficient of $x^{n}$ is just $V_{n-1}(x_{n-1}) $, thus
$$
V_n(x) = V_{n-1}(x_{n-1})\prod_{i=0}^{n-1}(x-x_i).
$$
}
\solution{(b)}{
With the definition of $X$ we know
$$
\begin{aligned}
\det(X) &= V_n(x_n) = V_{n-1}(x_{n-1})\prod_{i=0}^{n-1}(x_n-x_i) \\
&= V_{n-2}(x_{n-2})\prod_{i=0}^{n-1}(x_n-x_i)\prod_{i=0}^{n-2}(x_{n-1}-x_i)\\
&= \cdots \\ 
&= V_0(x_0)\prod_{k=0}^{n-1}\prod_{i=0}^{k-1}(x_{k+1}-x_i) = \prod_{0\le j<i\le n}(x_i-x_j).
\end{aligned}
$$
}

\problem{2}{Problem 3.6, Page 186}
\solution{Solution}{
We know from linear interpolation,
$$
|E(x)| = \frac{(x-x_0)(x_1-x)}{2}\cdot \sin(\xi), ~x_0 \le \xi \le x_1.
$$
Since $|f''(t)| = |sin(t)| \le 1$, we should take $h$, s.t. $\frac{h^2}{8} \le 1e^{-6}$. Thus we may take $h = 0.002$.

And we need a table entries of $7$ significant digits so as not to let the rounding error dominate the interpolation error.
}

\problem{3}{Problem 3.8, Page 187}
\solution{Solution}{
The Lagrange quadratic interpolation of $f$ at $x_i$ is
$$
L(x) = \frac{(x-x_1)(x-x_2)}{(x_0-x_1)(x_0-x_2)}f_0 + \frac{(x-x_0)(x-x_2)}{(x_1-x_0)(x_1-x_2)}f_1 + \frac{(x-x_0)(x-x_1)}{(x_2-x_0)(x_2-x_1)}f_2.
$$
Then the rounding error of quadratic interpolation is
$$
\begin{aligned}
R(x) &= \frac{(x-x_1)(x-x_2)}{(x_0-x_1)(x_0-x_2)}\epsilon_0 + \frac{(x-x_0)(x-x_2)}{(x_1-x_0)(x_1-x_2)}\epsilon_1 + \frac{(x-x_0)(x-x_1)}{(x_2-x_0)(x_2-x_1)}\epsilon_2 \\
&= \frac{1}{2h^2}((x-x_1)(x-x_2)\epsilon_0 - 2(x-x_0)(x-x_2)\epsilon_1 + (x-x_0)(x-x_1)\epsilon_2)
\end{aligned}
$$
Since the maximum of a quadratic function is at either endpoints or vertex, 
$$
\max|R(x)| \le \max\left(|\epsilon_0|, ~|\epsilon_2|, \frac{1}{2h^2}\left(\frac{h^2}{4}+2h^2 +\frac{h^2}{4} \right)|\epsilon|~\right) = 1.25|\epsilon|.
$$
}

\problem{4}{Problem 3.21, Page 189}
\solution{Solution}{
First notice $p(x)-1$ has three roots. So we may assume
$$
p(x) = q(x)x(x-1)(x+1)+1, ~q(x) = ax^2+bx+c.
$$
Then we replace $x$ with $-2, 2, 3$, we have
$$
\left\{
\begin{aligned}
&q(-2) = 1 = 4a-2b+c \\
&q(2) = 1 = 4a+2b+c \\
&q(3) = 1 = 9a+3b+c
\end{aligned}
\right.
$$
Then $q(x) = 1$ has at least three roots, which means $q(x) = 1$. Thus the degree of $p(x)$ is 3.
}

\problem{5}{Problem 3.24, Page 189}
\solution{Proof}{
With (3.2.12) we know
$$
|\Psi(x)| = \left|\frac{f^{(n)}(\xi)}{n!}\prod_{i=0}^n(x-x_i)\right| < \frac{1}{n!}h^{n+1}\times n! = h^{n+1}. 
$$
Then $\max|e^x-p_n(x)|\to 0$, when $n\to\infty$.
}
\end{document}