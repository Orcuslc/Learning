%!TEX program = xelate
%%%%%%%%%%%%%%%%%%%%%%%%%%%%%%%%%%%%%%%%%
% Modified By Orcuslc, 2016-9-21
% Modified for Assignments
% http://github.com/orcuslc
%
% Wilson Resume/CV
% Structure Specification File
% Version 1.0 (22/1/2015)
%
% This file has been downloaded from:
% http://www.LaTeXTemplates.com
%
% License:
% CC BY-NC-SA 3.0 (http://creativecommons.org/licenses/by-nc-sa/3.0/)
%
%%%%%%%%%%%%%%%%%%%%%%%%%%%%%%%%%%%%%%%%%

%----------------------------------------------------------------------------------------
%	PACKAGES AND OTHER DOCUMENT CONFIGURATIONS
%----------------------------------------------------------------------------------------
\documentclass[10pt]{article}

\usepackage{listings}
\usepackage{xcolor}
\usepackage{amsmath,amsthm,amssymb}
\usepackage{epstopdf}
\usepackage{graphicx}
\usepackage{clrscode3e}

\DeclareGraphicsExtensions{.eps,.ps,.jpg,.bmp}


\usepackage[a4paper, hmargin=25mm, vmargin=30mm, top=20mm]{geometry} % Use A4 paper and set margins

\usepackage{fancyhdr} % Customize the header and footer

\usepackage{lastpage} % Required for calculating the number of pages in the document

\usepackage{hyperref} % Colors for links, text and headings

\setcounter{secnumdepth}{0} % Suppress section numbering

%\usepackage[proportional,scaled=1.064]{erewhon} % Use the Erewhon font
%\usepackage[erewhon,vvarbb,bigdelims]{newtxmath} % Use the Erewhon font
\usepackage[utf8]{inputenc} % Required for inputting international characters
\usepackage[T1]{fontenc} % Output font encoding for international characters

\usepackage{fontspec} % Required for specification of custom fonts
\setmainfont[Path = ./fonts/,
Extension = .otf,
BoldFont = Erewhon-Bold,
ItalicFont = Erewhon-Italic,
BoldItalicFont = Erewhon-BoldItalic,
SmallCapsFeatures = {Letters = SmallCaps}
]{Erewhon-Regular}

\usepackage{color} % Required for custom colors
\definecolor{slateblue}{rgb}{0.17,0.22,0.34}

\usepackage{sectsty} % Allows customization of titles
\sectionfont{\color{slateblue}} % Color section titles

\fancypagestyle{plain}{\fancyhf{}\cfoot{\thepage\ of \pageref{LastPage}}} % Define a custom page style
\pagestyle{plain} % Use the custom page style through the document
\renewcommand{\headrulewidth}{0pt} % Disable the default header rule
\renewcommand{\footrulewidth}{0pt} % Disable the default footer rule

\setlength\parindent{0pt} % Stop paragraph indentation

% Non-indenting itemize
\newenvironment{itemize-noindent}
{\setlength{\leftmargini}{0em}\begin{itemize}}
{\end{itemize}}

% Text width for tabbing environments
\newlength{\smallertextwidth}
\setlength{\smallertextwidth}{\textwidth}
\addtolength{\smallertextwidth}{-2cm}

\newcommand{\sqbullet}{~\vrule height .8ex width .6ex depth -.05ex} % Custom square bullet point 


\newcommand{\tbf}[1]{\textbf{#1}}
\newcommand{\tit}[1]{\textit{#1}}
\newcommand{\mbb}[1]{\mathbb{#1}}
\newcommand{\blue}[1]{\color{blue}{#1}}
\newcommand{\red}[1]{\color{red}{#1}}
\newcommand{\sblue}[1]{\color{slateblue}{#1}}
\newcommand{\n}{\\[5pt]}
\newcommand{\tr}{^\top}
\newcommand{\vt}[1]{
\Vert #1 \Vert
}
\newcommand{\bra}[5]{
#1=\left\{
\begin{aligned}
#2 ,&\quad #4 \\
#3 ,&\quad #5
\end{aligned}
\right.
}

\renewcommand{\title}[2] {
{\Huge{\color{slateblue}\textbf{#1}}}
\hfill
\LARGE{\color{slateblue}\textbf{#2}} \\[10pt]
\large{\color{slateblue}\textbf{Chuan Lu, 13300180056, chuanlu13@fudan.edu.cn}} \\[1mm]
\rule{\textwidth}{0.5mm}
}

\newcommand{\problem}[2] {
\vspace{20pt}
\LARGE{\color{slateblue}\textbf{Problem #1.}}
\vspace{2mm}
#2 \\[10pt]
}

\renewcommand{\proof}[2] {
\large{\color{slateblue}\textit{\textbf{#1.}}}
#2 \qed \\[3mm]
}

\newcommand{\solution}[2] {
\large{\color{slateblue}\textit{\textbf{#1.}}}
#2 \\[3mm]
}


\newcommand{\algorithm}[2] {
\begin{codebox}
\Procname{$\proc{Algorithm #1}$}
#2
\end{codebox}
}

\newcommand{\refgroup}[1] {
\LARGE{\color{slateblue}\textbf{Reference}} 
\begin{tabbing}
\hspace{5mm} \= \kill
#1
\end{tabbing}
}

\newcommand{\reference}[1] {
\sqbullet \ \  \large{#1} \\
}
% \newcommand{\solution}[2] {
% \LARGE{\color{slateblue}\textit{#1}}
% \ #2 \qed
% }

% \newenvironment{problem}[2][Problem]{\begin{trivlist}
% \item[\hskip \labelsep {\bfseries #1}\hskip \labelsep {\bfseries #2.}]}{\end{trivlist}}
\usepackage{epstopdf}
\usepackage{graphics}
\usepackage{subfig}
\usepackage{listings}
\lstset{
  breaklines=true,
  xleftmargin=25pt,
  xrightmargin=25pt,
  aboveskip=0pt,
  belowskip=10pt,
  basicstyle=\ttfamily,
  showstringspaces=false,
  frame=ltrb,
  tabsize=4,
  numbers=left,
  numberstyle=\small,
  numbersep=8pt,
  morekeywords={*, factorial, sum, erlang},
  keywordstyle=\color{blue!70}, commentstyle=\color{red!50!green!50!blue!50},
}
\DeclareGraphicsExtensions{.eps,.ps,.jpg,.bmp}

\begin{document}

\title{Numerical Analysis \\ Assignment 10}
\date{\today}
\author{Chuan Lu}

\maketitle

\problem{1}{Problem 4.16, Page 242}
\solution{Solution}{
For each $m < n$, by integration by parts, 
$$
\begin{aligned}
\int_{0}^{\infty}e^{-x} x^m \varphi_n(x)dx &= \frac{(-1)^n}{n!}\int_{0}^{\infty} x^m\frac{d^n}{dx^n}(x^ne^{-x})dx \\
&= \frac{(-1)^n}{n!}\left.\left(x^m\frac{d^{n-1}}{dx^{n-1}}(x^ne^{-x})\right|_{0}^{\infty}-\int_{0}^{\infty}mx^{m-1}\frac{d^{n-1}}{dx^{n-1}}(x^ne^{-x})dx \right)
\end{aligned}
$$
Since 
$$
x^m\frac{d^{n-1}}{dx^{n-1}}(x^ne^{-x}) = e^{-x}N(x),
$$
where $N(x)$ is a polynomial of degree $n-1+m$, by L'Hospital's Rule we know the first term in the integration is $0$. Then by induction we know
$$
\begin{aligned}
\int_{0}^{\infty}e^{-x} x^m \varphi_n(x)dx &= \frac{(-1)^{n+1}m}{n!}\int_{0}^{\infty}x^{m-1}\frac{d^{n-1}}{dx^{n-1}}(x^n e^{-x})dx \\ 
&= \frac{(-1)^{n+m}m!}{n!}\int_{0}^{\infty}\frac{d^{n-m}}{dx^{n-m}}(x^n e^{-x})dx \\
&= \frac{(-1)^{n+m}m!}{n!}\frac{d^{n-m}}{dx^{n-m}}\int_{0}^{\infty}x^n e^{-x}dx \\
&= \frac{(-1)^{n+m}m!}{n!}\frac{d^{n-m}}{dx^{n-m}} (n!) = 0.
\end{aligned}
$$
In the deduction we used the property that $f(x) = x^n e^{-x}$ is absolutely continuous. Since $\varphi_m(x)$ is a polynomial of degree $m < n$, we know 
$$
(\varphi_n(x), \varphi_m(x)) = 0, ~\text{and}~ (\varphi_n(x), \varphi_n(x)) \ne 0.
$$
Hence $\{\varphi_n(x)\}$ is a family of orthogonal polynomials.
}

\problem{2}{Problem 4.18, Page 242}
\solution{Solution}{
First, we derive $c_n$. Multiply both sides of (4.4.21) by $w(x)\varphi_{n-1}(x)$, and then integrate, we get
$$
\int w\varphi_{n+1}\varphi_{n-1}dx = \int a_n wx\varphi_n\varphi_{n-1} + \int b_nw\varphi_n\varphi_{n-1} - c_n\int w\varphi_{n-1}^2.
$$
Using the orthogonality of $\varphi_n$, the left side is 0, and the second term of right side is 0. Then 
$$
a_n\int wx\varphi_n\varphi_{n-1} = c_n\int w\varphi_{n-1}^2.
$$
Since
$$
a_n\int wx\varphi_n\varphi_{n-1} = a_n\int  w\varphi_n(A_{n-1}x^n+B_{n-1}x^{n-1}+\cdots) = a_n\int w\varphi_n A_{n-1}x^n = a_n \frac{A_{n-1}}{A_n}\int w\varphi_n^2,
$$
we have
$$
c_n = \frac{a_n A_{n-1}\gamma_n}{A_n\gamma_{n-1}} = \frac{A_{n+1}A_{n-1}\gamma_n}{A_n^2\gamma_{n-1}}.
$$
Now we consider $b_n$. Multiply both sides of (4.4.21) by $w(x)\varphi_n(x)$, then integrate both sides, we get
$$
\int w\varphi_{n+1}\varphi_n = \int a_n wx\varphi_n^2+ \int b_n w\varphi_n^2 - \int c_n w\varphi_{n-1}\varphi_n.
$$
Using orthogonality, we get
$$
\int a_n wx\varphi_n^2 + \int b_n w\varphi_n^2 = 0.
$$
The first term can be wrote as 
$$
\begin{aligned}
\int a_n wx\varphi_n^2 &= a_n\int w(A_n x^{n+1}+B_n x+\cdots)\varphi_n = a_n \int w (\frac{A_n}{A_{n+1}}\varphi_{n+1}- \frac{A_n B_{n+1}-A_{n+1}B_n}{A_{n+1}}x^n+\cdots)\varphi_n \\
&= a_n\int w(B_n -\frac{A_n}{A_{n+1}}B_{n+1})x^n\varphi_n = a_n\int w\frac{1}{A_n}(B_n -\frac{A_n}{A_{n+1}}B_{n+1})\varphi_n^2.
\end{aligned}
$$
Thus
$$
a_n (\frac{B_n}{A_n}-\frac{B_{n+1}}{A_{n+1}}) \gamma_n + b_n\gamma_n = 0,
$$
we know
$$
b_n = a_n (\frac{B_{n+1}}{A_{n+1}} - \frac{B_n}{A_n}).
$$
}


\problem{3}{Problem 4.21, Page 243}
\solution{Proof}{
Denote $\varphi_n(x) = A_n x^n+B_n x^{n-1}+\cdots $, and $A_n > 0$. Then by (4.4.21), 
$$
\varphi_{n+1}(x) = (a_n x+b_n)\varphi_n(x)-c_n\varphi_{n-1}(x).
$$
We add a $\varphi_0(x)$ to this series, and $\varphi_{0}(x) = A_0 > 0$.
First, when $n = 1$, since
$$
\int_{a}^{b} w(x)\varphi_1(x)\varphi_0(x)dx = \int_{a}^{b}A_0w(x)\varphi_1(x)dx = 0 = A_0\varphi_1(\xi)\int_{a}^{b}w(x) dx,
$$
we know $\varphi_1(\xi) = 0, ~\xi\in (a, b)$. Then we show $\varphi_2(x)$ has two different roots in $(a, b)$. First,
$$
\varphi_2(\xi) = (a_1\xi+b_1)\varphi_1(\xi)-c_1\varphi_0(\xi) = -c_1\varphi_0(\xi) < 0.
$$
Suppose $\varphi_2(x)$ does not change sign in $(a, b)$, then
$$
\int w(x)\varphi_2(x)\varphi_0(x) = A_0\int w(x) \varphi_2(x) < 0.
$$
It makes a contradiction with orthogonality. Then there $\exists x_1\in(a, b) $, s.t. $\varphi_2(x_1) = 0$. Since $A_2 > 0$, $x_1$ cannot be a double root. If $\varphi_2(x)$ has only one root in $(a, b)$, then 
$$
\varphi_2(x)(x-x_1) = q(x)(x-x_1)^2,
$$
integrate by $w(x)$, then since $(x-x_1)$ is of degree 1, left side is 0. But we know $q(x)$ has no root in $(a, b)$, it does not change sign in $(a, b)$, thus the integration is not 0. It makes a contradiction. Thus $\varphi_2(x)$ has two different roots in $(a, b)$, and with $\varphi_2(\xi) < 0$ and $A_2 > 0$, we know the two roots are in $(a, \xi)$ and $(\xi, b)$ seperately. Using the same method, we know that $\varphi_n(x)$ has $n$ different roots in $(a, b)$.

Now we assume this proposition holds for $\varphi_m(x)$, $m \le n$. Denote roots of $\varphi_{n}(x)$ to be $x_i$, then since
$$
\varphi_{n+1}(x_i) = (a_n x_i+b_n)\varphi_n(x_i)-c_n\varphi_{n-1}(x_i) = -c_n\varphi_{n-1}(x_i),
$$
$$
\varphi_{n+1}(x_{i+1}) = (a_n x_{i+1}+b_n)\varphi_n(x_{i+1})-c_n\varphi_{n-1}(x_{i+1}) = -c_n\varphi_{n-1}(x_{i+1}),
$$
from the assumption we know $\varphi_{n-1}(x_i) $ and $\varphi_{n-1}(x_{i+1}) $ has different signs, which means $\varphi_n(x) $ has a root in each of this intervals. Then from $\varphi_n $ has $n$ different roots in $(a, b)$, from induction we get the result.
}

\problem{4}{Problem 4.23, Page 243}
\solution{Solution}{
We know the orthogonal functions on $[-1, 1]$ with weight function $w(x) = \frac{1}{\sqrt{1-x^2}}$ are Chebyshev polynomials (normalized):
$$
T_0(x) = \frac{1}{\sqrt{\pi}}, ~T_1(x) = \sqrt{\frac{2}{\pi}}x, ~T_2(x) = \sqrt{\frac{2}{\pi}}(2x^2-1).
$$
For any $p(x) = a_0+a_1x+a_2x^2 $ of degree 2 that minimizes the distance, we have
$$
a_j = (f, T_j), ~j = 0, 1, 2.
$$
Thus
$$
a_0 = (f, T_0) = \frac{1}{\sqrt{\pi}}\int_{-1}^{1} \frac{1}{\sqrt{1-x^2}}\cos^{-1}xdx = \frac{1}{\sqrt{\pi}}\int_{\pi}^{0}-ydy = \frac{(\pi)^{\frac{3}{2}}}{2},
$$
$$
a_1 = (f, T_1) = \sqrt{\frac{2}{\pi}}\int_{-1}^{1}\frac{x}{\sqrt{1-x^2}}\cos^{-1}xdx = \left.\sqrt{\frac{2}{\pi}}\int_{0}^{\pi} y\cos ydy = \sqrt{\frac{2}{\pi}}(y\sin y\right|_{0}^{\pi} - \int_{0}^{\pi} \sin ydy) = -\sqrt{\frac{8}{\pi}},
$$
$$
a_2 = (f, T_2) = \sqrt{\frac{2}{\pi}}\int_{-1}^{1}\frac{2x^2-1}{\sqrt{1-x^2}}\cos^{-1}xdx = \sqrt{\frac{2}{\pi}}\left.\int_{0}^{\pi}y\cos 2ydy = \sqrt{\frac{2}{\pi}}(\frac{y\sin 2y}{2}\right|_{0}^{\pi} - \frac{1}{2}\int_{0}^{\pi}\sin 2ydy) = 0,
$$
Thus
$$
p_2(x) = -\frac{4}{\pi}x+\frac{\pi}{2}.
$$
}
\end{document}