%!TEX program = xelate
%%%%%%%%%%%%%%%%%%%%%%%%%%%%%%%%%%%%%%%%%
% Modified By Orcuslc, 2016-9-21
% Modified for Assignments
% http://github.com/orcuslc
%
% Wilson Resume/CV
% Structure Specification File
% Version 1.0 (22/1/2015)
%
% This file has been downloaded from:
% http://www.LaTeXTemplates.com
%
% License:
% CC BY-NC-SA 3.0 (http://creativecommons.org/licenses/by-nc-sa/3.0/)
%
%%%%%%%%%%%%%%%%%%%%%%%%%%%%%%%%%%%%%%%%%

%----------------------------------------------------------------------------------------
%	PACKAGES AND OTHER DOCUMENT CONFIGURATIONS
%----------------------------------------------------------------------------------------
\documentclass[10pt]{article}

\usepackage{listings}
\usepackage{xcolor}
\usepackage{amsmath,amsthm,amssymb}
\usepackage{epstopdf}
\usepackage{graphicx}
\usepackage{clrscode3e}

\DeclareGraphicsExtensions{.eps,.ps,.jpg,.bmp}


\usepackage[a4paper, hmargin=25mm, vmargin=30mm, top=20mm]{geometry} % Use A4 paper and set margins

\usepackage{fancyhdr} % Customize the header and footer

\usepackage{lastpage} % Required for calculating the number of pages in the document

\usepackage{hyperref} % Colors for links, text and headings

\setcounter{secnumdepth}{0} % Suppress section numbering

%\usepackage[proportional,scaled=1.064]{erewhon} % Use the Erewhon font
%\usepackage[erewhon,vvarbb,bigdelims]{newtxmath} % Use the Erewhon font
\usepackage[utf8]{inputenc} % Required for inputting international characters
\usepackage[T1]{fontenc} % Output font encoding for international characters

\usepackage{fontspec} % Required for specification of custom fonts
\setmainfont[Path = ./fonts/,
Extension = .otf,
BoldFont = Erewhon-Bold,
ItalicFont = Erewhon-Italic,
BoldItalicFont = Erewhon-BoldItalic,
SmallCapsFeatures = {Letters = SmallCaps}
]{Erewhon-Regular}

\usepackage{color} % Required for custom colors
\definecolor{slateblue}{rgb}{0.17,0.22,0.34}

\usepackage{sectsty} % Allows customization of titles
\sectionfont{\color{slateblue}} % Color section titles

\fancypagestyle{plain}{\fancyhf{}\cfoot{\thepage\ of \pageref{LastPage}}} % Define a custom page style
\pagestyle{plain} % Use the custom page style through the document
\renewcommand{\headrulewidth}{0pt} % Disable the default header rule
\renewcommand{\footrulewidth}{0pt} % Disable the default footer rule

\setlength\parindent{0pt} % Stop paragraph indentation

% Non-indenting itemize
\newenvironment{itemize-noindent}
{\setlength{\leftmargini}{0em}\begin{itemize}}
{\end{itemize}}

% Text width for tabbing environments
\newlength{\smallertextwidth}
\setlength{\smallertextwidth}{\textwidth}
\addtolength{\smallertextwidth}{-2cm}

\newcommand{\sqbullet}{~\vrule height .8ex width .6ex depth -.05ex} % Custom square bullet point 


\newcommand{\tbf}[1]{\textbf{#1}}
\newcommand{\tit}[1]{\textit{#1}}
\newcommand{\mbb}[1]{\mathbb{#1}}
\newcommand{\blue}[1]{\color{blue}{#1}}
\newcommand{\red}[1]{\color{red}{#1}}
\newcommand{\sblue}[1]{\color{slateblue}{#1}}
\newcommand{\n}{\\[5pt]}
\newcommand{\tr}{^\top}
\newcommand{\vt}[1]{
\Vert #1 \Vert
}
\newcommand{\bra}[5]{
#1=\left\{
\begin{aligned}
#2 ,&\quad #4 \\
#3 ,&\quad #5
\end{aligned}
\right.
}

\renewcommand{\title}[2] {
{\Huge{\color{slateblue}\textbf{#1}}}
\hfill
\LARGE{\color{slateblue}\textbf{#2}} \\[10pt]
\large{\color{slateblue}\textbf{Chuan Lu, 13300180056, chuanlu13@fudan.edu.cn}} \\[1mm]
\rule{\textwidth}{0.5mm}
}

\newcommand{\problem}[2] {
\vspace{20pt}
\LARGE{\color{slateblue}\textbf{Problem #1.}}
\vspace{2mm}
#2 \\[10pt]
}

\renewcommand{\proof}[2] {
\large{\color{slateblue}\textit{\textbf{#1.}}}
#2 \qed \\[3mm]
}

\newcommand{\solution}[2] {
\large{\color{slateblue}\textit{\textbf{#1.}}}
#2 \\[3mm]
}


\newcommand{\algorithm}[2] {
\begin{codebox}
\Procname{$\proc{Algorithm #1}$}
#2
\end{codebox}
}

\newcommand{\refgroup}[1] {
\LARGE{\color{slateblue}\textbf{Reference}} 
\begin{tabbing}
\hspace{5mm} \= \kill
#1
\end{tabbing}
}

\newcommand{\reference}[1] {
\sqbullet \ \  \large{#1} \\
}
% \newcommand{\solution}[2] {
% \LARGE{\color{slateblue}\textit{#1}}
% \ #2 \qed
% }

% \newenvironment{problem}[2][Problem]{\begin{trivlist}
% \item[\hskip \labelsep {\bfseries #1}\hskip \labelsep {\bfseries #2.}]}{\end{trivlist}}
\usepackage{epstopdf}
\usepackage{graphics}
\usepackage{subfig}
\usepackage{listings}
\lstset{
  breaklines=true,
  xleftmargin=25pt,
  xrightmargin=25pt,
  aboveskip=0pt,
  belowskip=10pt,
  basicstyle=\ttfamily,
  showstringspaces=false,
  frame=ltrb,
  tabsize=4,
  numbers=left,
  numberstyle=\small,
  numbersep=8pt,
  morekeywords={*, factorial, sum, erlang},
}
\DeclareGraphicsExtensions{.eps,.ps,.jpg,.bmp}

\begin{document}

\title{Numerical Analysis \\ Assignment 5}
\date{\today}
\author{Chuan Lu}

\maketitle

\problem{1}{Problem 2.21, Page 121}
\solution{Solution}{
According to Theorem 2.7, we should choose $c$ so as to have $|g'(\alpha)| < 1$. Thus
$$
|g'(\alpha)| = |1+cf'(\alpha)| < 1.
$$
So $c$ satisfies $-2 < cf'(\alpha) < 0$. For a good rate of convergence, pick $c$ s.t. $cf'(\alpha) \sim 0$.
}

\problem{2}{Problem 2.24, Page 121}
\solution{Solution}{
\textbf{(a)} 
$$
g(x) = -16+6x+\frac{12}{x}.
$$
Thus $g'(2) = 3 > 1.$ So the iteration does not converge. 

\textbf{(b)}
$$
g(x) = \frac{2}{3}x+\frac{1}{x^2}.
$$
Thus $g'(3^{\frac{1}{3}}) = \frac{2}{3}-2\alpha^{-3} = 0$. So the iteration converges. Since
$$
\begin{aligned}
\lim_{n\to\infty}\frac{\alpha-x_{n+1}}{(\alpha-x_n)^2} &= \lim_{n\to\infty}\frac{-2x_n^3+3\alpha x_n^2-3}{3x_n^2(\alpha-x_n)^2} = \lim_{x_n\to\alpha}\frac{-2x_n^3+3\alpha x_n^2-3}{3x_n^2(\alpha-x_n)^2}\\
&= \lim_{x_n\to\alpha}\frac{-6x_n^2+6\alpha x_n}{6x_n(\alpha-x_n)^2-6x_n^2(\alpha-x_n)} = \lim_{x_n\to\alpha}\frac{1}{\alpha-2x_n} \\
&= -\frac{1}{\alpha}.
\end{aligned}
$$
We know the iteration is second-order convergent.

\textbf{(c)}
$$
g(x) = \frac{12}{1+x}.
$$
Thus $g'(\alpha) = -\frac{3}{4}$, so the iteration converges. Since
$$
\lim_{n\to\infty}\frac{\alpha-x_{n+1}}{\alpha-x_n} = \frac{(1+x_n)\alpha-12}{(1+x_n)(\alpha-x_n)} = \frac{\alpha}{\alpha-2x_n-1} = -\frac{3}{4},
$$
We know the iteration is linear convergence, and the rate is $\frac{3}{4}$.
}

\problem{3}{Problem 2.28, Page 122}
\solution{Proof}{
Assume $f(x) = (x-\alpha)h(x), ~h(\alpha) \ne 0$. Then
$$
\begin{aligned}
g(x) &= x - \frac{f(x)}{D(x)} = x-\frac{(x-\alpha)^2h^2(x)}{(x-\alpha)(h(x)+1)h((x-\alpha)h(x)+x)-(x-\alpha)h(x)} \\
&= x - \frac{(x-\alpha)h^2(x)}{(h(x)+1)h((x-\alpha)h(x)+x)-h(x)}.
\end{aligned}
$$
Thus 
$$
\begin{aligned}
g'(\alpha) &= \left.1-\frac{(h^2(x)+2(x-\alpha)h(x)h'(x))((h(x)+1)h((x-\alpha)h(x)+x)-h(x)) - (x-\alpha)M}{((h(x)+1)h((x-\alpha)h(x)+x)-h(x))^2}\right|_{x=\alpha} \\
&= 1-\frac{h^4(\alpha)}{h^4(\alpha)} = 0.
\end{aligned}
$$
And we can know that $g''(\alpha)\ne 0$. Accoding to Theorem 2.8, this iteration is a second-order method.
}

\problem{4}{Problem 2.48, Page 126}
\solution{Proof}{
According to the continuity of $\lVert\cdot\rVert_{\infty}$, we can find an $\epsilon$ and a closed set $B(\alpha, \delta)$, s.t. $\forall x\in B(\alpha, \delta)$, $\lVert G(x)\rVert_{\infty} \le 1-\epsilon$. Then $\forall x_0 \in B = B(\alpha, \delta)$,
$$
\lVert \alpha-x_{n+1}\rVert = \lVert g(\alpha)-g(x_n)\rVert = \lVert G(\xi)\rVert \cdot \lVert \alpha-x_n\rVert \le (1-\epsilon) \lVert \alpha-x_n\rVert \le (1-\epsilon)^n \lVert \alpha-x_0\rVert.
$$
Thus $\lVert \alpha-x_{n+1}\rVert \to 0, ~t\to\infty$. So the iteration converges to $\alpha$. And since $g(B) \subset B$ according to that $\lVert\alpha-g(x)\rVert < \lVert\alpha-x\rVert$, and $\max \lVert G(x)\rVert < 1$, we know that it satisfies the condition of Theorem 2.9.
}

\problem{5}{Problem 2.50, Page 127}
\solution{Solution}{

}
\end{document}