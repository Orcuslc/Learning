%!TEX program = xelatex

% \documentclass[10pt]{article}
%%%%%%%%%%%%%%%%%%%%%%%%%%%%%%%%%%%%%%%%%
% Modified By Orcuslc, 2016-9-12
% http://github.com/orcuslc
%
% Wilson Resume/CV
% Structure Specification File
% Version 1.0 (22/1/2015)
%
% This file has been downloaded from:
% http://www.LaTeXTemplates.com
%
% License:
% CC BY-NC-SA 3.0 (http://creativecommons.org/licenses/by-nc-sa/3.0/)
%
%%%%%%%%%%%%%%%%%%%%%%%%%%%%%%%%%%%%%%%%%

%----------------------------------------------------------------------------------------
%	PACKAGES AND OTHER DOCUMENT CONFIGURATIONS
%----------------------------------------------------------------------------------------

\usepackage[a4paper, hmargin=25mm, vmargin=30mm, top=20mm]{geometry} % Use A4 paper and set margins

\usepackage{fancyhdr} % Customize the header and footer

\usepackage{lastpage} % Required for calculating the number of pages in the document

\usepackage{hyperref} % Colors for links, text and headings

\setcounter{secnumdepth}{0} % Suppress section numbering

%\usepackage[proportional,scaled=1.064]{erewhon} % Use the Erewhon font
%\usepackage[erewhon,vvarbb,bigdelims]{newtxmath} % Use the Erewhon font
\usepackage[utf8]{inputenc} % Required for inputting international characters
\usepackage[T1]{fontenc} % Output font encoding for international characters

\usepackage{fontspec} % Required for specification of custom fonts
\setmainfont[Path = ./fonts/,
Extension = .otf,
BoldFont = Erewhon-Bold,
ItalicFont = Erewhon-Italic,
BoldItalicFont = Erewhon-BoldItalic,
SmallCapsFeatures = {Letters = SmallCaps}
]{Erewhon-Regular}

\usepackage{color} % Required for custom colors
\definecolor{slateblue}{rgb}{0.17,0.22,0.34}

\usepackage{sectsty} % Allows customization of titles
\sectionfont{\color{slateblue}} % Color section titles

\fancypagestyle{plain}{\fancyhf{}\cfoot{\thepage\ of \pageref{LastPage}}} % Define a custom page style
\pagestyle{plain} % Use the custom page style through the document
\renewcommand{\headrulewidth}{0pt} % Disable the default header rule
\renewcommand{\footrulewidth}{0pt} % Disable the default footer rule

\setlength\parindent{0pt} % Stop paragraph indentation

% Non-indenting itemize
\newenvironment{itemize-noindent}
{\setlength{\leftmargini}{0em}\begin{itemize}}
{\end{itemize}}

% Text width for tabbing environments
\newlength{\smallertextwidth}
\setlength{\smallertextwidth}{\textwidth}
\addtolength{\smallertextwidth}{-2cm}

\newcommand{\sqbullet}{~\vrule height 1ex width .8ex depth -.2ex} % Custom square bullet point definition

%----------------------------------------------------------------------------------------
%	MAIN HEADER COMMAND
%----------------------------------------------------------------------------------------

\renewcommand{\title}[1]{
{\huge{\color{slateblue}\textbf{#1}}}\\ % Header section name and color
\rule{\textwidth}{0.5mm}\\ % Rule under the header
}

%----------------------------------------------------------------------------------------
%	JOB COMMAND: Modified by Orcuslc 2016-09-12
%----------------------------------------------------------------------------------------

\newcommand{\job}[6]{
\begin{tabbing}
\hspace{2cm} \= \kill
\textbf{#1} \> \href{#4}{\textbf{#3}} \\
\textbf{#2} \>\+ \textit{#5} \\
\begin{minipage}{\smallertextwidth}
\vspace{2mm}
#6
\end{minipage}
\end{tabbing}
\vspace{2mm}
}

%----------------------------------------------------------------------------------------
%	SKILL GROUP COMMAND - Modified by Orcuslc 2016-09-12
%----------------------------------------------------------------------------------------

\newcommand{\skillgroup}[2]{
% \begin{tabbing}
% \hspace{5mm} \= \kill
% \sqbullet \> \textbf{#1}
% \end{tabbing}
% \begin{tabbing}
% \vspace{-2pt}
% \hspace{3cm} \= \kill
% #2
% \end{tabbing}
% }
\begin{tabbing}
\hspace{5mm} \= \kill
\sqbullet \>\+ \textbf{#1} \\

\begin{minipage}{\smallertextwidth}
\vspace{2mm}
#2
\end{minipage}
\end{tabbing}
}

\newcommand{\skill}[2]{
\begin{tabbing}
\hspace{1cm} \> \hspace{3cm} \= \kill
\>\textbf{#1:} \> #2 \\
\end{tabbing}
% \hspace
}

%----------------------------------------------------------------------------------------
%	INTERESTS GROUP COMMAND
%-----------------------------------------------------------------------------------------

\newcommand{\interestsgroup}[1]{
\begin{tabbing}
\hspace{5mm} \= \kill
#1
\end{tabbing}
\vspace{-10mm}
}

\newcommand{\interest}[1]{\sqbullet \> \textbf{#1}\\[3pt]} % Define a custom command for individual interests


%---------------------------------------------------------------
%   AWARDS GROUP COMMAND: Created by Orcuslc 2016-09-12
%---------------------------------------------------------------
\newcommand{\awardgroup}[1]{
\begin{tabbing}
\hspace{5mm} \= \kill
#1
\end{tabbing}
\vspace{2mm}
}

\newcommand{\award}[2]{
% \begin{tabbing}
% \hspace{0} \> \hspace{2cm} \= \kill
\hspace{2cm} \> \hspace{2cm} \= \kill
\sqbullet \> \textbf{#1} \> #2 \\[3pt]
% \hspace{2cm} \= \kill
% \end{tabbing}

}

%----------------------------------------------------------------------------------------
%	TABBED BLOCK COMMAND: Modified by Orcuslc 2016-09-12
%----------------------------------------------------------------------------------------

\newcommand{\tabbedblock}[1]{
\begin{tabbing}
\hspace{2cm} \= \hspace{4cm} \= \hspace{4cm} \= \hspace{4cm} \= \kill
#1
\end{tabbing}
}

%-----------------------------------------------------------------
%  TABBED BLOCK 2 COMMAND: Created by Orcuslc 2016-09-12
%-----------------------------------------------------------------
\newcommand{\tabblock}[5]{
\begin{tabbing}
% \hspace{2cm} \= \kill
\hspace{2cm} \= \hspace{4cm} \= \hspace{4cm} \= \hspace{4cm} \= \kill

\textbf{#1} \> \textbf{#3} \\
\textbf{#2} \> \textbf{#4} \\[5pt]
\>\+
#5
\end{tabbing}
}


%-----------------------------------------------------------------
%  RESEARCH COMMAND: Created by Orcuslc 2016-09-12
%-----------------------------------------------------------------
\newcommand{\research}[7]{
\begin{tabbing}
\hspace{2cm} \= \kill
\textbf{#1} \> \textbf{#3}, under the supervision of \href{#5}{\textbf{#4}} \\
% \href{#4}{#3} \\
\textbf{#2} \>\+ \textit{#6} \\
\begin{minipage}{\smallertextwidth}
\vspace{0.5mm}
#7
\end{minipage}
\end{tabbing}
\vspace{2mm}
}

%-------------------------------------------------------------------------
%      Programming Projects Command - Created by Orcuslc 2016-9-14
%-------------------------------------------------------------------------
% \newcommand{\projectgroup}[1]{
% \begin{tabbing}
% % \hspace{5mm} \= \kill
% \vspace{5mm}
% #1
% \end{tabbing}
% }

\newcommand{\projectgroup}[1]{
% \begin{tabbing}
\hspace{5mm} \= \kill
#1
% \end{tabbing}
\vspace{2mm}
}


\newcommand{\project}[3]{
% \begin{tabbing}
\hspace{2cm} \> \hspace{2cm} \= \kill
\sqbullet \> \textbf{#1} \> #2 \\
\> \textbf{code:} \> \href{#3}{#3} \\[5pt]
% \end{tabbing}
}
\usepackage{epstopdf}
\usepackage{graphics}
\usepackage{subfig}
\usepackage{listings}
\usepackage{xcolor}
\usepackage{courier}
\renewcommand{\sfdefault}{phv}

\DeclareGraphicsExtensions{.eps,.ps,.jpg,.bmp}

\begin{document}

\title{Assignment 2}{16.10.2}
\tiny{Code of this assignment are in the attachment, and they can also be found in GitHub: https://github.com/Orcuslc/Learning/tree/master/Numerical\%20Algorithm\%20and\%20Case\%20Studies/Homework2/code}

% \parbox{0.3\textwidth}{
% Chuan Lu}
% \hfill
% \parbox{0.3\textwidth}{
% 13300180056}
% \hfill
% \parbox{0.3\textwidth}{
% chuanlu13@fudan.edu.cn}

\problem{1 \& 2}{}
\solution{Abstract}{
\\[5pt]
Details of the implementation of LU decomposition in MATLAB.
}
\solution{Introduction}{
\\[5pt]
In most cases pivoting is necessary. If we randomly choose a matrix $M\in \mathbb{R}^{n\times n}$, the probability that $M(1, 1) = max_{i}M(i, 1)$ is $\frac{1}{n}$. So the efficiency of pivoting is a great concern. \\[5pt]
In the experiment below, we generate the matrix $M$ by using MATLAB function \texttt{randn(n, n)}; and we use seperately
$$\texttt{[L, U] = lu(A)}$$
$$\texttt{[L, U, P] = lu(A)}$$
for the two problems.
}
\solution{Results \& Discussion}{
After trying for several times with different $n$, the dimension of the random matrix, we found that in most cases, calling \texttt{[L, U] = lu(A)} leads to $L$ not being a lower triangular matrix, but $U$ is always an upper triangular matrix. \\[5pt]
When we count the number of non-zero elements of each row in $L$, we can find out the list of the numbers is a permutation of \texttt{[1, 2, ..., n]}. \\[5pt]
Thus I tried to call \texttt{[L1, U1, P] = lu(A)}, and the $P$ is exactly the permutation matrix: $$PL1 = L$$.
So it is obvious how MATLAB express the $P$: it simply multiplied it with the L; In order to represent this, I guess that MATLAB records the number of lines that need interchanging in each step; when the algorithm ends, it conbines the records to gain a permutation matrix. \\[5pt]
In order to realize $P*b$, we should use the pernutation record: in each of the $n$ steps, change the elements of two lines, and it will only need $O(n)$ time, and no extra space.
}

\problem{3}{}
\solution{Abstract \& Introduction}{
As is known to all, using Cholesky decomposition for a symmetric positive definite matrix will need half of the flops comparing LU decomposition; In the mean time, since there is no need of pivoting for a symmetric positive definite matrix, in each step $n-1$ times of comparing and $2n$ times of exchanging will be saved on average.
}
\solution{Result}{
We generate the symmetric matrix A by
$$\texttt{A = randn(n, n); A = A + A';}$$
and we did the experiment with $n = 4, 6, 8, 10$. The result of calling 
$$\texttt{L = chol(A)}$$
is:
$$\texttt{Error using \textbf{chol}: Matrix must be positive definite}.$$
}
\solution{Analysis}{
If $A\in \mathbb{R}^{n\times n}$ is not positive definite, then according to the properties of symmetric elementary transformations, $PAP^\top$ is still not positive definite, where $P$ is a permutation matrix. \\[5pt]
Thus, we can assume $A(1:2, 1:2)$ is not positive definite. According to the algorithm, we have
$$l_{11} = \sqrt{a_{11}},$$
$$l_{21} = \frac{a_{21}}{l_{11}},$$
$$l_{22} = (a_{22} - l_{21}^2)^{1/2}.$$
That means $l_{22} = a_{22} - \frac{a_{21}^2}{a_{11}} = \frac{a_{22}a_{11} - a_{21}^2}{a_{11}}$. Since A is symmetric and not positive definite, we have $a_{22}a_{11}-a_{21}^2 = a_{22}a_{11}-a{21}a_{12}<0$. So $l_{22}$ does not exist.
}

\problem{4}{}
\solution{Introduction}{
The $LDL^\top$ decomposition is used for Hermitian indefinite matrices.
}
\solution{Discussion}{
I found an introduction to $LDL^\top$ decomposition from Golub's \textit{Matrix Computations}, Page 186. It said the method was developed by Bunch and Parlett. I searched their paper in Google Scholar, but it seemed that our school has no access to siam.org, so I cannot read the paper.
}


\refgroup{
\reference{Gene H. Golub \textit{et al}, \textit{Matrix Computations}, Page 186-187.}
}


\end{document}