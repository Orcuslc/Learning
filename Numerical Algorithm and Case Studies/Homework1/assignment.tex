%!TEX program = xelatex

% \documentclass[10pt]{article}
%%%%%%%%%%%%%%%%%%%%%%%%%%%%%%%%%%%%%%%%%
% Modified By Orcuslc, 2016-9-21
% Modified for Assignments
% http://github.com/orcuslc
%
% Wilson Resume/CV
% Structure Specification File
% Version 1.0 (22/1/2015)
%
% This file has been downloaded from:
% http://www.LaTeXTemplates.com
%
% License:
% CC BY-NC-SA 3.0 (http://creativecommons.org/licenses/by-nc-sa/3.0/)
%
%%%%%%%%%%%%%%%%%%%%%%%%%%%%%%%%%%%%%%%%%

%----------------------------------------------------------------------------------------
%	PACKAGES AND OTHER DOCUMENT CONFIGURATIONS
%----------------------------------------------------------------------------------------
\documentclass[10pt]{article}

\usepackage{listings}
\usepackage{xcolor}
\usepackage{amsmath,amsthm,amssymb}
\usepackage{epstopdf}
\usepackage{graphicx}
\usepackage{clrscode3e}

\DeclareGraphicsExtensions{.eps,.ps,.jpg,.bmp}


\usepackage[a4paper, hmargin=25mm, vmargin=30mm, top=20mm]{geometry} % Use A4 paper and set margins

\usepackage{fancyhdr} % Customize the header and footer

\usepackage{lastpage} % Required for calculating the number of pages in the document

\usepackage{hyperref} % Colors for links, text and headings

\setcounter{secnumdepth}{0} % Suppress section numbering

%\usepackage[proportional,scaled=1.064]{erewhon} % Use the Erewhon font
%\usepackage[erewhon,vvarbb,bigdelims]{newtxmath} % Use the Erewhon font
\usepackage[utf8]{inputenc} % Required for inputting international characters
\usepackage[T1]{fontenc} % Output font encoding for international characters

\usepackage{fontspec} % Required for specification of custom fonts
\setmainfont[Path = ./fonts/,
Extension = .otf,
BoldFont = Erewhon-Bold,
ItalicFont = Erewhon-Italic,
BoldItalicFont = Erewhon-BoldItalic,
SmallCapsFeatures = {Letters = SmallCaps}
]{Erewhon-Regular}

\usepackage{color} % Required for custom colors
\definecolor{slateblue}{rgb}{0.17,0.22,0.34}

\usepackage{sectsty} % Allows customization of titles
\sectionfont{\color{slateblue}} % Color section titles

\fancypagestyle{plain}{\fancyhf{}\cfoot{\thepage\ of \pageref{LastPage}}} % Define a custom page style
\pagestyle{plain} % Use the custom page style through the document
\renewcommand{\headrulewidth}{0pt} % Disable the default header rule
\renewcommand{\footrulewidth}{0pt} % Disable the default footer rule

\setlength\parindent{0pt} % Stop paragraph indentation

% Non-indenting itemize
\newenvironment{itemize-noindent}
{\setlength{\leftmargini}{0em}\begin{itemize}}
{\end{itemize}}

% Text width for tabbing environments
\newlength{\smallertextwidth}
\setlength{\smallertextwidth}{\textwidth}
\addtolength{\smallertextwidth}{-2cm}

\newcommand{\sqbullet}{~\vrule height .8ex width .6ex depth -.05ex} % Custom square bullet point 


\newcommand{\tbf}[1]{\textbf{#1}}
\newcommand{\tit}[1]{\textit{#1}}
\newcommand{\mbb}[1]{\mathbb{#1}}
\newcommand{\blue}[1]{\color{blue}{#1}}
\newcommand{\red}[1]{\color{red}{#1}}
\newcommand{\sblue}[1]{\color{slateblue}{#1}}
\newcommand{\n}{\\[5pt]}
\newcommand{\tr}{^\top}
\newcommand{\vt}[1]{
\Vert #1 \Vert
}
\newcommand{\bra}[5]{
#1=\left\{
\begin{aligned}
#2 ,&\quad #4 \\
#3 ,&\quad #5
\end{aligned}
\right.
}

\renewcommand{\title}[2] {
{\Huge{\color{slateblue}\textbf{#1}}}
\hfill
\LARGE{\color{slateblue}\textbf{#2}} \\[10pt]
\large{\color{slateblue}\textbf{Chuan Lu, 13300180056, chuanlu13@fudan.edu.cn}} \\[1mm]
\rule{\textwidth}{0.5mm}
}

\newcommand{\problem}[2] {
\vspace{20pt}
\LARGE{\color{slateblue}\textbf{Problem #1.}}
\vspace{2mm}
#2 \\[10pt]
}

\renewcommand{\proof}[2] {
\large{\color{slateblue}\textit{\textbf{#1.}}}
#2 \qed \\[3mm]
}

\newcommand{\solution}[2] {
\large{\color{slateblue}\textit{\textbf{#1.}}}
#2 \\[3mm]
}


\newcommand{\algorithm}[2] {
\begin{codebox}
\Procname{$\proc{Algorithm #1}$}
#2
\end{codebox}
}

\newcommand{\refgroup}[1] {
\LARGE{\color{slateblue}\textbf{Reference}} 
\begin{tabbing}
\hspace{5mm} \= \kill
#1
\end{tabbing}
}

\newcommand{\reference}[1] {
\sqbullet \ \  \large{#1} \\
}
% \newcommand{\solution}[2] {
% \LARGE{\color{slateblue}\textit{#1}}
% \ #2 \qed
% }

% \newenvironment{problem}[2][Problem]{\begin{trivlist}
% \item[\hskip \labelsep {\bfseries #1}\hskip \labelsep {\bfseries #2.}]}{\end{trivlist}}
\usepackage{epstopdf}
\usepackage{graphics}
\usepackage{subfig}
\usepackage{listings}
\lstset{
  numbers=left,
  frame=shadowbox,
  keywordstyle=\bf\color{blue!70},
  identifierstyle=\bf,
  numberstyle=\color[RGB]{0,192,192},
  commentstyle=\it\color[RGB]{0,96,96},
  stringstyle=\rmfamily\slshape\color[RGB]{128,0,0},
  showstringspaces=false,
  extendedchars=false
    }
\DeclareGraphicsExtensions{.eps,.ps,.jpg,.bmp}

\begin{document}

\title{Homework 1}{16.9.22}

% \parbox{0.3\textwidth}{
% Chuan Lu}
% \hfill
% \parbox{0.3\textwidth}{
% 13300180056}
% \hfill
% \parbox{0.3\textwidth}{
% chuanlu13@fudan.edu.cn}

\problem{1}{}
\solution{Abstract}{
\\[5pt]
About the computation of roots and its sensitivity on coefficients of a quadratic polynomial.
}

\solution{Introduction and Algorithm}{
\\[5pt]
In middle school we learned that the roots of a quadratic polynomial $ax^2+bx+c=0$ can be expressed as $x_1 = \frac{-b+\sqrt{b^2-4ac}}{2a}, x_2 = \frac{-b-\sqrt{b^2-4ac}}{2a}$. \\[5pt] 
However in computation, when $4ac\ll b^2$, the so-called catastrophic cancellation would happen: if $b\ge 0$, then $\sqrt{b^2-4ac}\approx b$, which would cause a decreas in the number of significant digits of $-b+\sqrt{b^2-4ac}$. This statement also holds when $b\le 0$. \\[10pt] 
In order to avoid this, we use a different but equivalent expression for the roots. \\[10pt]
When $b\ge 0$, $\left\{
\begin{aligned}
x_1 & = & \frac{-b-\sqrt{b^2-4ac}}{2a} \\
x_2 & = & \frac{-2c}{b+\sqrt{b^2-4ac}}
\end{aligned}
\right.$; 
when $b < 0$, $\left\{
\begin{aligned}
x_1 & = & \frac{-b+\sqrt{b^2-4ac}}{2a} \\
x_2 & = & \frac{2c}{-b+\sqrt{b^2-4ac}} 
\end{aligned}
\right.$ \\[10pt]
The correctness can be easily proved by using Vieta theorem.
}

\solution{Code}{
\\[5pt]
The code can be found in the attachment(\texttt{extract\_quadratic.m}); We use the name \texttt{\textbf{MyFunc}} to refer this function in the following sections.
}

\solution{Results}{
\\[5pt]
In the following sections, we always use $x_1$ to refer to the larger root, and $x_2$ to the smaller root. \\[5pt]
For $(b=-56, c=1)$, the result is in Table 1 below: \\
\begin{table}[htbp] \large \centering
\begin{tabular}{|c|c|c|}
\hline
& $x_1$ & $x_2$ \\
\hline
\texttt{MyFunc()} & 55.9821 & 0.0179 \\
\hline
\texttt{root()} & 55.9821 & 0.0179 \\
\hline
\end{tabular}
\caption{Result of $(b=-56, c=1)$}
\end{table}\\
For $(b=-10^8, c=1)$, the result is in Table 2 below:\\
\begin{table}[htbp] \large \centering
\begin{tabular}{|c|c|c|}
\hline
& $x_1$ & $x_2$ \\
\hline
\texttt{MyFunc()} & 100000000 & 1.0000e-08 \\
\hline
\texttt{root()} & $1.0000*10^8$  & $0.0000*10^8$ \\
\hline
\end{tabular}
\caption{Result of $(b=-10^8,c=1)$}
\end{table} \\
When $4ac\approx b^2$, the results are the same; but when $4ac\ll b^2$, the significant number of the smaller root using \texttt{root} is eliminated.
}

\solution{Sensitivity}{
\\[5pt]
According to Taylor series, $$x(t+\delta)\approx x(t)+\delta\frac{dx}{dt}(t).$$
Therefore the sensitivity of x on t can be defined as $s_x(t) = dx/dt$. \\[5pt]
In this problem, $F(x, b, c) = x^2+bx+c$. According to implicit function theorem, $$\frac{\partial x}{\partial b} = -\frac{F_x}{F_b} = -\frac{2x+b}{x} = -2+\frac{b}{x}, $$
$$\frac{\partial x}{\partial c} = -\frac{F_x}{F_c} = -(2x+b).$$ \\[5pt]
In consequence, when $\lvert x\rvert\ll\lvert b\rvert$, $x$ is very sensitive to a little change in $b$. Similarly, when $\lvert2x+b\rvert\gg 1$, $x$ is very sensitive to a little change in $c$.
}

\solution{Discussion and Conclusion}{
With the conclusion in Sensitivity part, we can make some explanation for the difference between results of \texttt{MyFunc} and \texttt{root}. \\[5pt]
1. For the first case, when $b=-56, c=1$, and the result $x_1\approx56, x_2\approx0.018$. \\[5pt]
In this case $\frac{\partial x_1}{\partial b} \approx -3, \frac{\partial x_1}{\partial c}\approx -56,$ and $\frac{\partial x_2}{\partial b} \approx -3113, \frac{\partial x_1}{\partial c}\approx 56$. Since the sensitivities are not large enough to have strong effects on a \texttt{\textbf{format short}} computation, the result of \texttt{MyFunc} and \texttt{root} seems the same. \\[5pt]
2. For the second case, when $b=-10^8, c=1$, and the result $x_1\approx 10^8, x_2\approx 10^{-8}.$ \\[5pt]
In this case $\frac{\partial x_1}{\partial b} \approx -3, \frac{\partial x_1}{\partial c}\approx -10^8,$ and $\frac{\partial x_2}{\partial b} \approx 10^16, \frac{\partial x_2}{\partial c} \approx 10^8$. \\[5pt]
Apparently, sensitivity of $x_2$ on both $b$ and $c$ are large enough to make a obvious change in significant numbers. It leads to that the result of $x_2$ being $0$ when using \texttt{root}.
In this scenario we may regard this numerial problem an ill-conditioned one; though the numerical algorithm by MATLAB may be a stable one, the results can be far from true ones.
}

\problem{2}{}
\solution{Abstract}{
\\[5pt]
(a) Program row and column oriented algorithms and compare the time cost between them \\
(b) Program algorithms for update a matrix using different methods and compare the time cost.
}

\solution{Introduction and Algrithm}{
\\[5pt]
For problem (a), we will use back substitution algorithm in row-oriented and column-oriented ways for solving an upper triangular system. 
\algorithm{1.1, Row-oriented Back Substitution}{
\zi Input: An upper triangular matrix $U\in \mathbb{R}^{n\times n}$, and a vector $b\in \mathbb{R}^{n}$.
\zi Output: Updated $b\in \mathbb{R}^{n}$.
\li $b(n) = b(n)/U(n, n)$
\li \For $i \gets n-1$ \To $1$ \By $-1$
\li   \Do
          $b(i) = (b(i)-U(i, i+1:n)\cdot b(i+1:n))/U(i, i)$
      \End
\li \Return $b$
}
\algorithm{1.2, Column-oriented Back Substitution}{
\zi Input: An upper triangular matrix $U\in \mathbb{R}^{n\times n}$, and a vector $b\in \mathbb{R}^{n}$.
\zi Output: Updated $b\in \mathbb{R}^{n}$.
\li \For $j \gets n$ \To $2$ \By $-1$
\li   \Do
          $b(j) = b(j)/U(j, j)$
\li       $b(1:j-1)=b(1:j-1)-b(j)\cdot U(1:j-1, j)$
      \End
\li $b(1) = b(1)/U(1, 1)$
}
These two algorithms do not need extra space. \\

For problem (b), the five algorithms are shown below.

\algorithm{2.1, Loop-update}{
\zi Input: Three matrices $A, B, C\in\mathbb{R}^{n\times n}$.
\zi Output: The updated $C\in\mathbb{R}^{n\times n}$.
\li \For $i \gets 1$ \To n
\li   \Do
      \For $j \gets 1$ \To n
\li     \Do
        \For $k \gets 1$ \To n
\li         \Do
              $C(i, j) = C(i, j)+A(i, k)\cdot B(k, j)$
            \End
        \End
      \End
\li \Return $C$      
}
\algorithm{2.2, Dot-update}{
\zi Input: Three matrices $A, B, C\in\mathbb{R}^{n\times n}$.
\zi Output: The updated $C\in\mathbb{R}^{n\times n}$.
\li \For $i \gets 1$ \To $n$
\li   \Do
        \For $j \gets 1$ \To $n$
\li       \Do
            $C(i, j) = C(i, j)+A(i, :)\cdot B(:, j)$
          \End
      \End
\li \Return $C$
}
\algorithm{2.3, Daxpy-update}{
\zi Input: Three matrices $A, B, C\in\mathbb{R}^{n\times n}$.
\zi Output: The updated $C\in\mathbb{R}^{n\times n}$.
\li \For $i \gets 1$ \To $n$
\li   \Do
        \For $j \gets 1$ \To $n$
\li       \Do
            $C(:, i) = C(:, i)+A(:, j)\cdot B(j, i)$
          \End
      \End
\li \Return $C$        
}
\algorithm{2.4, MatVec-update}{
\zi Input: Three matrices $A, B, C\in\mathbb{R}^{n\times n}$.
\zi Output: The updated $C\in\mathbb{R}^{n\times n}$.
\li \For $i \gets 1$ \To $n$
\li   \Do
        $C(:, i) = C(:, i)+A\cdot B(:, i)$
      \End
\li \Return $C$
}

\algorithm{2.5, OuterDot-update}{
\zi Input: Three matrices $A, B, C\in\mathbb{R}^{n\times n}$.
\zi Output: The updated $C\in\mathbb{R}^{n\times n}$.
\li \For $i \gets 1$ \To $n$
\li   \Do
        $C = C+A(:, i)\cdot \B(i, :)$
      \End
\li \Return $C$
}
I am not quite sure about the MatVec update.
}

\solution{Code}{
The code can be found in the attachments(\texttt{row\_back\_substitute.m},\ \texttt{col\_back\_substitute.m},\ \texttt{mmloop.m},\ \texttt{mmdot.m},\ \texttt{mmdaxpy.m},\ \texttt{mmmatvec.m},\ \texttt{mmouterdot.m}). And the test script is \texttt{test.m}.\\[5pt]
Code for checking errors is ignored here, since I think in this problem it is of less importance.
}
\solution{Result}{
For problem (a), the time cost is in Table 3 below. (Of course I know that MATLAB is using some precompile skills to accelerate the computing speed, so I run the same test several times after the time cost converge to a stable status.) \\[5pt]
\begin{table}[htbp] \large \centering
\begin{tabular}{|c|c|c|c|c|}
\hline
n & 128 & 256 & 512 & 1024 \\
\hline
row-oriented & 0.000477 &  0.000957 & 0.002858 & 0.009390\\
\hline
col-oriented & 0.000436 & 0.000912 & 0.002631 & 0.006705\\
\hline
\end{tabular}
\caption{Time cost of solving a upper triangle system}
\end{table}
However, after test my program several times, I found that when $n$ is small, which, to be exact, when $n<50$, difference between solution of row-oriented and col-oriented algorithms $\lVert x_r - x_c \rVert$ is quite small. In this case $\lVert x_r - x_c \rVert \le 10^{-4} \Vert U\Vert$. However, when $n$ is larger, the difference becomes larger. 
Table 4 shows the relative norm of difference vectors between result of the two als and result of directly using MATLAB's inner functions(i represents MATLAB's inner function). $$\Vert\Delta_{error}\Vert = \frac{\Vert x_1 - x_2\Vert}{\Vert U\Vert}$$
\begin{table}[htbp] \large \centering
\begin{tabular}{|c|c|c|c|c|}
\hline
n & 128 & 256 & 512 & 1024 \\
\hline
r-c & $5.948e^{-6}$ & $1.504e^{27}$ & $6.904e^{85}$ & $4.179e^{168}$\\
\hline
r-i & $5.680e^{-6}$ & $1.521e^{27}$ & $2.952e^{85}$ & $6.968e^{167}$\\
\hline
c-i & $1.091e^{-6}$ & $1.693e^{25}$ & $2.261e^{85}$ & $4.875e^{168}$ \\
\hline
\end{tabular}
\caption{Relative norm of difference vector}
\end{table}

For Problem (b), the time cost for als is shown in Table 5 below.
\begin{table}[htbp] \large \centering
\begin{tabular}{|c|c|c|c|c|}
\hline
n & 128 & 256 & 512 & 1024 \\
\hline
loop & 0.050033 & 0.425775 & 3.196842 & 39.002625 \\
\hline
dot & 0.026959 & 0.134890 & 0.787452 & 13.141148 \\
\hline
daxpy & 0.003710 & 0.023131 & 0.158915 & 1.182440 \\
\hline
matvec & 0.001167 & 0.005713 & 0.026144 & 0.375322 \\
\hline
outerdot & 0.007415 & 0.030481 & 0.426991 & 4.375097 \\
\hline
\end{tabular}
\caption{Time cost of updating a matrix}
\end{table}

The norm of relative error (with MATLAB inner *) is shown in Table 6 below.
$$\Vert\Delta_{error}\Vert = \frac{\Vert\hat{C} - C\Vert}{\Vert C\Vert}$$
\begin{table}[htbp] \large \centering
\begin{tabular}{|c|c|c|c|c|}
\hline
n & 128 & 256 & 512 & 1024 \\
\hline
loop & $1.841e^{-15}$ & $2.614e^{-15}$ & $1.408e^{-14}$ & $2.660e^{-14}$ \\
\hline
dot & $3.515e^{-15}$ & $6.756e^{-15}$ & $5.502e^{-15}$ & $6.474e^{-15}$ \\
\hline
daxpy & $1.144e^{-15}$ & $2.614e^{-15}$ & $1.408e^{-14}$ & $2.660e^{-14}$ \\
\hline
matvec & $1.178e^{-15}$ & $1.643e^{-15}$ & $1.385e^{-14}$ & $2.628e^{-14}$ \\
\hline
outerdot & $1.195e^{-13}$ & $2.614e^{-15}$ & $1.408e^{-14}$ & $2.660e^{-14}$ \\
\hline
\end{tabular}
\caption{Norm of error in updating a matrix}
\end{table}
}
\solution{Discussion}{
From result of problem (a), we found that column-oriented algorithm is slightly quicker that row-oriented algorithm in solving upper triangular system. \\[5pt]
From the algorithm's point of view, in $i^{th}$ loop, col-oriented al need to do (i-1) times' multiplication, (i-1) times' subtraction, and 1 division, while row-oriented al need to do (n-i) times multiplication, (n-i) times addition, and 1 division. \\[5pt]
In consequence, C-O al need to do $\frac{(n-1)(n-2)}{2}$ times' multiplication and substraction, and (n-1) times' division; while R-O al need to do 1 more division than C-O. So I am curious why C-O cost only 2/3 time comparing with R-O when $U \in\mathbb{R}^{1024\times 1024}$. \\[5pt]
On the other hand, since the matrices are generated by random numbers, it is of much possibility that those matrices are nearly singular. It is not difficult to understand why the result of $x$ has such a huge error. \\[5pt]

For problem (b), result of the five algorithms are of good precision, which can be found with norm of errors. \\[5pt]
Let's analysis the flops of the als. Loop update needs $n^3$ times of loops, each contains a multiplication and an addition. Dot update needs $n^2$ times of loops, each contains an addition and a Vectorized Dot. Daxpy update needs $n^2$ times of loops, each contains an Vectorized Addition and a Vectorized Scalar multiplication with a vector. MatVec update needs $n$ times of loops, each contains a Vectorized Addition and a Vectorized Scalar multiplication with a matrix. OuterDot update needs $n$ times of loops, each contains a Vectorized Addtion with a matrix and an outer dot of two vectors.\\[5pt]
Though the total flops of the five als are the same, we all know that in MATLAB, \texttt{for} loops are much slower that vectorized computations. So the result of experiment is trivial.
}



\refgroup{
\reference{Jixiu Chen, \textit{Mathematical Analysis II}, Page 174-175.}
\reference{D. P. O’Leary, \textit{Scientific Computing with case studies}, Page 23-24.}
\reference{Gene H. Golub, \textit{Matrix Computations}, Page 10-11.}
\reference{Shufang Xu, \textit{Numerical Linear Algebra}, Page 11-13.}
}


\end{document}