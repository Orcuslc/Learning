%!TEX program = xelate
%%%%%%%%%%%%%%%%%%%%%%%%%%%%%%%%%%%%%%%%%
% Modified By Orcuslc, 2016-9-21
% Modified for Assignments
% http://github.com/orcuslc
%
% Wilson Resume/CV
% Structure Specification File
% Version 1.0 (22/1/2015)
%
% This file has been downloaded from:
% http://www.LaTeXTemplates.com
%
% License:
% CC BY-NC-SA 3.0 (http://creativecommons.org/licenses/by-nc-sa/3.0/)
%
%%%%%%%%%%%%%%%%%%%%%%%%%%%%%%%%%%%%%%%%%

%----------------------------------------------------------------------------------------
%	PACKAGES AND OTHER DOCUMENT CONFIGURATIONS
%----------------------------------------------------------------------------------------
\documentclass[10pt]{article}

\usepackage{listings}
\usepackage{xcolor}
\usepackage{amsmath,amsthm,amssymb}
\usepackage{epstopdf}
\usepackage{graphicx}
\usepackage{clrscode3e}

\DeclareGraphicsExtensions{.eps,.ps,.jpg,.bmp}


\usepackage[a4paper, hmargin=25mm, vmargin=30mm, top=20mm]{geometry} % Use A4 paper and set margins

\usepackage{fancyhdr} % Customize the header and footer

\usepackage{lastpage} % Required for calculating the number of pages in the document

\usepackage{hyperref} % Colors for links, text and headings

\setcounter{secnumdepth}{0} % Suppress section numbering

%\usepackage[proportional,scaled=1.064]{erewhon} % Use the Erewhon font
%\usepackage[erewhon,vvarbb,bigdelims]{newtxmath} % Use the Erewhon font
\usepackage[utf8]{inputenc} % Required for inputting international characters
\usepackage[T1]{fontenc} % Output font encoding for international characters

\usepackage{fontspec} % Required for specification of custom fonts
\setmainfont[Path = ./fonts/,
Extension = .otf,
BoldFont = Erewhon-Bold,
ItalicFont = Erewhon-Italic,
BoldItalicFont = Erewhon-BoldItalic,
SmallCapsFeatures = {Letters = SmallCaps}
]{Erewhon-Regular}

\usepackage{color} % Required for custom colors
\definecolor{slateblue}{rgb}{0.17,0.22,0.34}

\usepackage{sectsty} % Allows customization of titles
\sectionfont{\color{slateblue}} % Color section titles

\fancypagestyle{plain}{\fancyhf{}\cfoot{\thepage\ of \pageref{LastPage}}} % Define a custom page style
\pagestyle{plain} % Use the custom page style through the document
\renewcommand{\headrulewidth}{0pt} % Disable the default header rule
\renewcommand{\footrulewidth}{0pt} % Disable the default footer rule

\setlength\parindent{0pt} % Stop paragraph indentation

% Non-indenting itemize
\newenvironment{itemize-noindent}
{\setlength{\leftmargini}{0em}\begin{itemize}}
{\end{itemize}}

% Text width for tabbing environments
\newlength{\smallertextwidth}
\setlength{\smallertextwidth}{\textwidth}
\addtolength{\smallertextwidth}{-2cm}

\newcommand{\sqbullet}{~\vrule height .8ex width .6ex depth -.05ex} % Custom square bullet point 


\newcommand{\tbf}[1]{\textbf{#1}}
\newcommand{\tit}[1]{\textit{#1}}
\newcommand{\mbb}[1]{\mathbb{#1}}
\newcommand{\blue}[1]{\color{blue}{#1}}
\newcommand{\red}[1]{\color{red}{#1}}
\newcommand{\sblue}[1]{\color{slateblue}{#1}}
\newcommand{\n}{\\[5pt]}
\newcommand{\tr}{^\top}
\newcommand{\vt}[1]{
\Vert #1 \Vert
}
\newcommand{\bra}[5]{
#1=\left\{
\begin{aligned}
#2 ,&\quad #4 \\
#3 ,&\quad #5
\end{aligned}
\right.
}

\renewcommand{\title}[2] {
{\Huge{\color{slateblue}\textbf{#1}}}
\hfill
\LARGE{\color{slateblue}\textbf{#2}} \\[10pt]
\large{\color{slateblue}\textbf{Chuan Lu, 13300180056, chuanlu13@fudan.edu.cn}} \\[1mm]
\rule{\textwidth}{0.5mm}
}

\newcommand{\problem}[2] {
\vspace{20pt}
\LARGE{\color{slateblue}\textbf{Problem #1.}}
\vspace{2mm}
#2 \\[10pt]
}

\renewcommand{\proof}[2] {
\large{\color{slateblue}\textit{\textbf{#1.}}}
#2 \qed \\[3mm]
}

\newcommand{\solution}[2] {
\large{\color{slateblue}\textit{\textbf{#1.}}}
#2 \\[3mm]
}


\newcommand{\algorithm}[2] {
\begin{codebox}
\Procname{$\proc{Algorithm #1}$}
#2
\end{codebox}
}

\newcommand{\refgroup}[1] {
\LARGE{\color{slateblue}\textbf{Reference}} 
\begin{tabbing}
\hspace{5mm} \= \kill
#1
\end{tabbing}
}

\newcommand{\reference}[1] {
\sqbullet \ \  \large{#1} \\
}
% \newcommand{\solution}[2] {
% \LARGE{\color{slateblue}\textit{#1}}
% \ #2 \qed
% }

% \newenvironment{problem}[2][Problem]{\begin{trivlist}
% \item[\hskip \labelsep {\bfseries #1}\hskip \labelsep {\bfseries #2.}]}{\end{trivlist}}
\usepackage{epstopdf}
\usepackage{graphics}
\usepackage{subfig}
\usepackage{listings}
\lstset{
  numbers=left,
    framexleftmargin=10mm,
    frame=none,
    backgroundcolor=\color[RGB]{245,245,244},
  keywordstyle=\bf\color{blue},
  identifierstyle=\bf,
  numberstyle=\color[RGB]{0,192,192},
  commentstyle=\it\color[RGB]{0,96,96},
  stringstyle=\rmfamily\slshape\color[RGB]{128,0,0},
  showstringspaces=false,
  extendedchars=false
    }
\DeclareGraphicsExtensions{.eps,.ps,.jpg,.bmp}

\begin{document}
\title{Assignment 9}{}

\problem{1}{Problem 2, Page 151}
\proof{Proof}{
First, one notices that $e_0 = 0$, thus
\[
e_i = e_0 + h\sum_{j = 0}^{i-1} \delta_x^+ = h\sum_{j=0}^{i-1}\delta_x^+e_j.
\]
\qquad\qquad Using Schwarz inequality, one finds
\[
\begin{aligned}
\vt{\mathbf{e}}_{\mathnormal{l}^2}^2 &= \vt{\left(h\sum_{j=0}^{i-1}\delta_x^+e_j\right)_i}_{l^2}^2 = \sum_{i=1}^{N}\left(h\sum_{j=0}^{i-1}\delta_x^+e_j\right)^2 \leqslant \sum_{i=1}^{N}ih^2\sum_{j=0}^{i-1}\left(\delta_x^+e_j\right)^2 \n
&\leqslant N^2h^2\sum_{i=1}^{N-1}(\delta_x^+e_j)^2 = \vt{\delta_x^+\mathbf{e}}_{l^2}.
\end{aligned}
\]

\qquad\qquad So we have proved (3.1.54). Next we use this inequality to proof the convergence. \n
\qquad\qquad The truncation error of the differenced system is
\[
R_i = -\delta_x^2u(x_i) - f(x_i) = -\delta_x^2u(x_i) + \frac{d^2u}{dx^2} = 	-\frac{h^2}{12}+O(h^4).
\]
\qquad\qquad Define $e_i = u(x_i) - u_i$, then $e_i$ satisfies
\[
-\delta_x^2 e_i = R_i.
\]
\qq\qq So we have
\[
\vt{\mathbf{e}} \leqslant \vt{\sigma_x^+e} \leqslant \vt{R} \sim O(h^2).
\]
\qq\qq It means the function value is of order-2 convergent. When it comes to the derivatives,
\[
\frac{du}{dx}(x_i) = \frac{u(x_{i+1})-u(x_{i-1})}{2h} + \frac{h^2}{12}u^{(3)}(x_i) + O(h^3).
\]
\qq\qq Using the same methods we can get that the values of derivatives are of order-2 convergent.
}


\problem{2}{Problem 3, Page 151}
\proof{Proof}{
\[
-\frac{d}{dx}\left(a(x)\frac{du}{dx}\right) = -(a'(x)u'(x) + a(x)u''(x)) = f
\]
\qq\qq Thus the three-point-difference scheme is
\[
-a_i'\delta_xu_i - a_i\delta_x^2u_i = f_i.
\]
\qq\qq When $f_i \geqslant 0$, 
\[
\begin{aligned}
u_{i+1} &= 2u_i - u_{i-1} - \frac{h^2}{a_i}f_i - \frac{h_ia_i'}{a_i}\delta_xu_i \n
& \leqslant (2-\frac{h_ia_i'}{a_i})u_i - (1-\frac{h_ia_i'}{a_i})u_{i-1},
\end{aligned}
\]
\qq\qq where we used the backward scheme in the last term. According to the same process as Lemma 3.1.12, we finds that either $u_i$ being a constant, or the minimum is reached on the boarder.
}
\solution{Max-module estimation}{
We construct a problem:
\[
\left\{
\begin{aligned}
&v_0 = v_n = 0, \n
&-\delta_x(a_i\delta_xu_i) = \vt{\mathbf{R}}_{l^\infty},
\end{aligned}
\right.
\]
\qq\qq Then $v_i = \frac{1}{N}\sum_{j=1}^{i}\frac{\frac{i}{N}\vt{\mathbf{R}}_{l^\infty} + \frac{\sum_{j=1}^{N}\frac{i\vt{\mathbf{R}}_{l^\infty}}{a_i}}{\sum_{j=1}^{N}\frac{1}{a_i}}}{a_i}$.
Thus
\[
\vt{e_i} \leqslant v_i \leqslant C\vt{\mathbf{R}}_{l^\infty} \leqslant C_2h^2.
\]
}
% \refgroup{
% \reference{a a a a  a a}
% }


\end{document}