\documentclass{article}
\usepackage{listings}
\usepackage{xcolor}
\usepackage[margin=1in]{geometry}
\usepackage{amsmath,amsthm,amssymb}
\usepackage{graphicx}
\usepackage{epstopdf}
\DeclareGraphicsExtensions{.eps,.ps,.jpg,.bmp}
\lstset{
	numbers=left,
    framexleftmargin=10mm,
    frame=none,
    backgroundcolor=\color[RGB]{245,245,244},
	keywordstyle=\bf\color{blue},
	identifierstyle=\bf,
	numberstyle=\color[RGB]{0,192,192},
	commentstyle=\it\color[RGB]{0,96,96},
	stringstyle=\rmfamily\slshape\color[RGB]{128,0,0},
	showstringspaces=false
    }

\newcommand{\N}{\mathbb{N}}
\newcommand{\R}{\mathbb{R}}
\newcommand{\Z}{\mathbb{Z}}
\newcommand{\Q}{\mathbb{Q}}

\newenvironment{theorem}[2][Theorem]{\begin{trivlist}
\item[\hskip \labelsep {\bfseries #1}\hskip \labelsep {\bfseries #2.}]}{\end{trivlist}}
\newenvironment{lemma}[2][Lemma]{\begin{trivlist}
\item[\hskip \labelsep {\bfseries #1}\hskip \labelsep {\bfseries #2.}]}{\end{trivlist}}
\newenvironment{exercise}[2][Exercise]{\begin{trivlist}
\item[\hskip \labelsep {\bfseries #1}\hskip \labelsep {\bfseries #2.}]}{\end{trivlist}}
\newenvironment{problem}[2][Problem]{\begin{trivlist}
\item[\hskip \labelsep {\bfseries #1}\hskip \labelsep {\bfseries #2.}]}{\end{trivlist}}
\newenvironment{question}[2][Question]{\begin{trivlist}
\item[\hskip \labelsep {\bfseries #1}\hskip \labelsep {\bfseries #2.}]}{\end{trivlist}}
\newenvironment{corollary}[2][Corollary]{\begin{trivlist}
\item[\hskip \labelsep {\bfseries #1}\hskip \labelsep {\bfseries #2.}]}{\end{trivlist}}

\begin{document}

\title{Homework 2016-03-11}
\author{Chuan Lu\\
13300180056}

\maketitle

\begin{problem}{1}
\text{ }\\
Display the error of Gauss-Legendre Formula.
\end{problem}

\begin{proof}
\subsection{The code is shown as follows.}
\begin{lstlisting}[language={MATLAB}]
function [ point, weight ] = gauss_coefficient_legendre( order )
% Calculate the coefficients of gauss integral formula with legendre
% polynomial.

b = 0.5 ./ sqrt(1 - (2 * (1:order)) .^ (-2));
[v, lambda] = eig(diag(b, 1) + diag(b, -1));
[point, i] = sort(diag(lambda));
weight = 2 * v(1, i) .^ 2;

end
\end{lstlisting}

\begin{lstlisting}[language={MATLAB}]
function [ value ] = gauss_integral( func, order )
% GAUSS_INTEGRAL Calculate the intergral of func in [0, 1] with
% Gauss-Legendre Formula.
if order == 1
    value = feval(func, 0.5);
    return
end
[point, weight] = gauss_coefficient_legendre(order - 1);
transfered_point = point * 0.5 + 0.5;
value = weight * feval(func, transfered_point) * 0.5;

end
\end{lstlisting}

\begin{lstlisting}[language = {MATLAB}]
% get function
function func = get_func( n )
    func = @(x)(x .^ n);
\end{lstlisting}

\begin{lstlisting}[language = {MATLAB}]
% homework 4
clear all;
result_matrix = zeros(7, 6);
for n = 1:7
    func = get_func(n);
    result_matrix(n, 1) = integral(func, 0, 1);
    for order = 1:5
        func = get_func(n);
        result_matrix(n, order + 1) = gauss_integral(func, order);
    end
end
nlist = 1:5;
figure(1);
for i = 1 : 7
    plot(nlist, result_matrix(i, 2:6) - result_matrix(i, 1));
    hold on
end
title('The error of Gauss-Legendre formula');
legend('x^1', 'x^2', 'x^3', 'x^4', 'x^5', 'x^6', 'x^7', 'Location', 'Best');
\end{lstlisting}

\subsection{The result is shown as follows.}
\begin{figure}[htbp]
\centering
\includegraphics[width = 15cm]{pic1.eps}
\caption{The relationship of error with order}
\label{Error}
\end{figure}
\end{proof}

% \begin{problem}{2}
% \text{ }\\
% Calculate the integral of $x^{n}$ with Newton-Cotes Formula.
% \end{problem}
% \begin{proof}
% \subsection{The code is shown as follows.}

% \begin{lstlisting}[language = {MATLAB}]
% function [ coeff ] = cotes_coefficient( order )
% %Calculate the cotes coefficient with the order given
% %Use the equation given at Page179, exercise3
% n = order
% matrix = zeros(n, n);
% for i = 1 : n
%     matrix( i , : ) = (1:n) .^ i;
% end
% c = (n .^ (1:n)) ./ (2:n+1);
% c = transpose(c);
% coeff = matrix \ c;
% coeff = [coeff(n); coeff];
% end
% \end{lstlisting}

% \begin{lstlisting}[language = {MATLAB}]
% function[ coeff ] = get_coeff( order )
% % get_coeff of newton_cotes formula of interval
% % order means the order of the formula
% switch order
%     case 1
%         coeff = [1, 1];
%     case 2
%         coeff = [1, 4, 1];
%     case 3
%         coeff  = [1,3,3,1];
%     case 4
%         coeff = [7,32, 12, 32, 7];
%     case 5
%         coeff = [19, 75, 50, 50, 75, 19];
%     case 6
%         coeff = [41, 216, 27, 272, 27, 216, 41];
%     case 7
%         coeff = [751, 3577, 1323, 2989, 2989, 1323, 3577, 751];
%     case 8
%         coeff = [989, 5888, -928, 10496, -4540, 10496, -928, 5888, 989];
%     otherwise
%         error('order should be between 0 and 8 --get_cotes_coeff');
% end
%     coeff = coeff / sum(coeff);
% \end{lstlisting}

% \begin{lstlisting}[language = {MATLAB}]
% function [ coeff ] = new_cotes_coefficient( order )
% %Calculate the cotes coefficient with the order given
% %Use the equation given at Page179, exercise3
% n = order;
% matrix = zeros(n, n);
% for i = 1 : n
%     matrix( i , : ) = (((1:n) ./ n) .^ i);
% end
% c = 1 ./ ( 2: n + 1);
% c = transpose(c);
% coeff = matrix \ c;
% coeff = [coeff(n); coeff];
% end
% \end{lstlisting}

% \begin{lstlisting}[language = {MATLAB}]
% function [ result ] = newton_cotes( func, inteval, order )
% %NEWTON_COTES Use Newton_Cotes Formula to Calculate the intergral.

% if nargin < 2
%     error(' Too few inputted arguments, please check if function and inteval for integrating have been inputted or not  -- Newton_Cotes');
% elseif nargin == 2
%     order = 2;
% end

% if order < 0 || order > 8
%     error(' Order Error: The order should be between 0 and 8. --Newton_Cotes ');
% end

% low = inteval( 1 );
% high = inteval( 2 );

% if order == 0
%     result = feval(func, (low + high) / 2) * (high - low);
% else
%     coeff = get_coeff(order);
%     eval_points = low : (high - low) / order : high;
%     eval_points = transpose( eval_points );
%     result = sum( coeff * feval(func, eval_points)) * (high - low);
% end
% \end{lstlisting}

% \begin{lstlisting}[language = {MATLAB}]
% function func = get_func( n )
%     func = @(x)(x .^ n);
% \end{lstlisting}

% \begin{lstlisting}[language = {MATLAB}]
% % Homework1.3.1
% inteval = [0, 1];
% result_matrix = zeros(7, 6);
% for n = 1:7
%     func = get_func(n);
%     result_matrix(n, 1) = integral(func, 0, 1);
%     for order = 0:4
%         func = get_func(n);
%         result_matrix(n, order + 2) = newton_cotes(func, inteval, order);
%     end
% end
% disp(result_matrix)
% \end{lstlisting}

% \subsection{The result is shown as the table above.}

% \begin{table}
% \begin{tabular}{c|c|c|c|c|c|c}
%     \hline
%     & Exact Value & medium formula & trapezium formula & Simpson formula & 3/8 formula & Cotes formula \\
%     \hline
%     $x^{1}$ & 0.5000 & 0.5000 & 0.5000 & 0.5000 & 0.5000 & 0.5000 \\
%     \hline
%     $x^{2}$   & 0.3333  &  0.2500  &  0.5000 &  0.3333  &  0.3333  &  0.3333 \\
%     \hline
%     $x^{3}$ & 0.2500  &  0.1250 &   0.5000 & 0.2500  &  0.2500    &    0.2500 \\
%     \hline
%     $x^{4}$  & 0.2000 &   0.0625  &  0.5000  & 0.2083 &   0.2037    &   0.2000 \\
%     \hline
%     $x^{5}$   &  0.1667  &  0.0313 &  1 0.5000 &   0.1875 &  0.1759  &  0.1667 \\
%     \hline
%     $x^{6}$  &   0.1429  &  0.0156  &  0.5000  &  0.1771  &  0.1584  &  0.1432 \\
%     \hline
%     $x^{7}$   &  0.1250  &  0.0078  &  0.5000  &  0.1719   & 0.1471  &  0.1263 \\
%     \hline
%     \end{tabular}
%     \caption{Integral of $x^{n}$ using Newton-Cotes Formula}
%     \end{table}
% \end{proof}
% \end{document}
% \begin{lstlisting}[language={MATLAB}]
% % make matrix
% % author: chuanlu
% % 2016-03-02
% function [A] = make_matrix(op, n)

%     if nargin < 1
%         error('More args needed --make matrix');
%     elseif nargin == 1
%         op = 'A1';
%     end

%     if op == 'A1'
%         c1 = zeros(1, n - 1);
%         c1(2 : n - 1) = -3;
%         c2 = zeros(1, n - 2) + 2;
%         A = eye(n) + diag(c1, -1) + diag(c2, -2);
%     elseif op == 'A2'
%         c1 = zeros(1, n - 1);
%         c1(2 : n - 1) = -3;
%         c2 = zeros(1, n - 2) + 2;
%         A = eye(n) + diag(c1, -1) + diag(c2, -2);
%         A(1, n) = -1;
%     else
%         error('Operation Failed to Match A1 or A2');
%     end
% \end{lstlisting}

% \begin{lstlisting}[language={MATLAB}]
% % homework2.m
% % author: chuanlu
% % 2016-03-02

% op1 = 'A1';
% op2 = 'A2';
% N = 100;
% cond1 = zeros(1, N);
% cond2 = zeros(1, N);
% for n = 1 : N
%     A1 = make_matrix(op1, n);
%     A2 = make_matrix(op2, n);
%     cond1(n) = cond(A1);
%     cond2(n) = cond(A2);
% end
% n = [1:N];
% figure(1);
% semilogy(n, cond1, '*-');
% figure(2);
% plot(n, cond2, '*-');
% %     disp('cond1:');
% %     disp(cond1);
% %     disp('cond2:');
% %     disp(cond2);
% \end{lstlisting}
% The result is shown as follows.
% \end{proof}

% \begin{problem}{3}
% \text{ }\\
% Given $\Vert x_{n+2} - x_{n+1} \Vert$  $\leqslant$ $\alpha\Vert x_{n+1} - x_{n} \Vert$, then $\Vert x^{\star} - x_{n} \Vert$ $\leqslant$ $\frac{\alpha_{n}}{1-\alpha}\Vert x_{1} - x_{0}\Vert$.
% \end{problem}

% \begin{proof}
% $\Vert x_{n} - x_{n - 1} \Vert\leqslant\alpha\Vert x_{n - 1} - x_{n - 2} \Vert\leqslant\dots\leqslant\alpha^{n-1}\Vert x_{1} - x_{0}\Vert$ \\
% \par Hence, $\Vert x_{\star} - x_{n} \Vert\leqslant\Vert x_{n} - x_{n+1} \Vert+\Vert x_{n+1} - x_{n+2}\Vert + \dots\leqslant(\alpha^{n}+\alpha^{n+1}+\dots)\Vert x_{1} - x_{0}\Vert$ \\
% \par$= \frac{\alpha^{n}}{1-\alpha}\Vert x_{1} - x_{0}\Vert$
% \end{proof}

\end{document}
% % \begin{problem}{4}
% % \text{ }\\
% % Suppose that $[a]$ is a unit in $\Z_{n}$ and $[b]$ is an element of $\Z_{n}$. Prove that the equation $[a]x = b$ has exactly one solution in $\Z_{n}$
% % \end{problem}

% % \begin{proof}

% % \end{proof}

% % \begin{problem}{5}
% % \text{ }\\
% % Suppose that $[a]$ and $[b]$ are both units in $\Z_{n}$.  Show that the product $[a] \cdot [b]$ is also a unit in $\Z_{n}.$ (Note that this confirms closure under multiplication in the group $U_{n})$.
% % \end{problem}

% % \begin{proof}

% % \end{proof}

% % \begin{problem}{6}
% % \text{ }\\
% % Which of the following are Groups? Which of the following are not groups, and why?\\

% % \indent (1)  $G = \{{2, 4, 6, 8}\}$ in $\Z_{10}$.  Where $a \star b = ab$\\
% % \indent (2)  $G = \Q^{\ast}$, where $a \star b = \frac{a}{b}$\\
% % \indent (3) $G = \Z$, where $a \star b = a - b$\\
% % \indent (4) $G = \{ {2^{x}\mid x \in \Q} \}$, where $a \star b = ab$\\

% % \end{problem}

% % \begin{proof}

% % \end{proof}

% % \begin{problem}{7}
% % \text{ }\\
% % Consider the set $Q =$ \{ $\pm$1, $\pm$i, $\pm$j, $\pm$k\} of the complex matrices as follows:\\
% % \[
% % 1=
% %   \begin{bmatrix}
% %     1 & 0\\
% %     0 & 1\\
% %   \end{bmatrix}
% % \]

% % \[
% % i=
% %   \begin{bmatrix}
% %     i & 0\\
% %     0 & $-$i\\
% %   \end{bmatrix}
% % \]

% % \[
% % j=
% %   \begin{bmatrix}
% %     0 & 1\\
% %     $-$1 & 0\\
% %   \end{bmatrix}
% % \]

% % \[
% % k=
% %   \begin{bmatrix}
% %     0 & i\\
% %     i & 0\\
% %   \end{bmatrix}
% % \]
% % Show that $Q$ is a group under matrix multiplication by writing out its multiplicaiton table. (Note: $Q$ is called the quartenion group).
% % \end{problem}
% % \begin{proof}

% % \end{proof}

% % \end{document}
