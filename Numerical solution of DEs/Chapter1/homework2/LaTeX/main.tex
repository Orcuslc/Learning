\documentclass[12pt]{article}
\usepackage{CJK}
\usepackage{listings}
\usepackage{xcolor}
\usepackage[margin=1in]{geometry}
\usepackage{amsmath,amsthm,amssymb}
\usepackage{graphicx}
\usepackage{epstopdf}
\DeclareGraphicsExtensions{.eps,.ps,.jpg,.bmp}
\lstset{
	numbers=left,
    framexleftmargin=10mm,
    frame=none,
    backgroundcolor=\color[RGB]{245,245,244},
	keywordstyle=\bf\color{blue},
	identifierstyle=\bf,
	numberstyle=\color[RGB]{0,192,192},
	commentstyle=\it\color[RGB]{0,96,96},
	stringstyle=\rmfamily\slshape\color[RGB]{128,0,0},
	showstringspaces=false
    }
% * <orcuslc@hotmail.com> 2016-03-02T15:13:09.253Z:
%
% ^.

\newcommand{\N}{\mathbb{N}}
\newcommand{\R}{\mathbb{R}}
\newcommand{\Z}{\mathbb{Z}}
\newcommand{\Q}{\mathbb{Q}}

\newenvironment{theorem}[2][Theorem]{\begin{trivlist}
\item[\hskip \labelsep {\bfseries #1}\hskip \labelsep {\bfseries #2.}]}{\end{trivlist}}
\newenvironment{lemma}[2][Lemma]{\begin{trivlist}
\item[\hskip \labelsep {\bfseries #1}\hskip \labelsep {\bfseries #2.}]}{\end{trivlist}}
\newenvironment{exercise}[2][Exercise]{\begin{trivlist}
\item[\hskip \labelsep {\bfseries #1}\hskip \labelsep {\bfseries #2.}]}{\end{trivlist}}
\newenvironment{problem}[2][Problem]{\begin{trivlist}
\item[\hskip \labelsep {\bfseries #1}\hskip \labelsep {\bfseries #2.}]}{\end{trivlist}}
\newenvironment{question}[2][Question]{\begin{trivlist}
\item[\hskip \labelsep {\bfseries #1}\hskip \labelsep {\bfseries #2.}]}{\end{trivlist}}
\newenvironment{corollary}[2][Corollary]{\begin{trivlist}
\item[\hskip \labelsep {\bfseries #1}\hskip \labelsep {\bfseries #2.}]}{\end{trivlist}}

\begin{document}

\title{Homework 2016-03-04}
\author{Chuan Lu\\
13300180056}

\maketitle

\begin{problem}{1}
\text{ }\\
$$Given\quad G(x) = e^{-x} \quad and \quad F(x) = x - e^{-x},$$ compare Fixed-point iteration and Newton-Raphson iteration.
\end{problem}

\begin{proof}
\subsection{The code is shown as follows.}
\begin{lstlisting}[language={MATLAB}]
function[root, root_list] = fixed_point_iter(func, x0, tol, order)
% FIXED_POINT_ITER Extract root ROOT of function FUNC with init
% point X0 given, using fixed point iteration
% author: chuanlu
% 2016-03-04

if nargin < 3
    error('More arguments are needed --newton_raphson_iter');
elseif nargin == 3
    tol = 1e-6;
    order = 100;
end

count = 0;
root_list = zeros(1, 1);
while 1
    count = count + 1;
    root_list(count) = x0;
    x1 = feval(func, x0);
    if abs(x1 - x0) < tol || count >= order
        root = x1;
        return
    elseif count >= 100
        warning('Count over 100, may not be convergence');
        root = x1;
        return
    end
    x0 = x1;
end
\end{lstlisting}

\begin{lstlisting}[language={MATLAB}]
function[root, root_list] = newton_raphson_iter(func1, func2, x0, tol, order)
% Extract root ROOT of function FUNC1 with its derivative FUNC2
% and init point X0 given

if nargin < 3
    error('More arguments are needed --newton_raphson_iter');
elseif nargin == 3
    tol = 1e-6;
    order = 100;
elseif nargin == 4
    order = 100;
end

count = 0;
root_list = zeros(1, 1);
while 1
    count = count + 1;
    root_list(count) = x0;
    f1 = feval(func1, x0);
    f2 = feval(func2, x0);
    x1 = x0 - f1 / f2;
    if abs(x1 - x0) < tol
        disp(count);
        root = x1;
        return
    elseif count >= order
        root = x1;
        return
    elseif count > 100
        warning('Count over 100, may not be convergence');
        root = x1;
        return
    end
    x0 =x1;
end
\end{lstlisting}

\begin{lstlisting}[language = {MATLAB}]
% homework 1.2.1
format long

f = @(x)(x - exp(-x));
f2 = @(x)(1 + exp(-x));
g = @(x)(exp(-x));

x0 = 1;
tol = 1e-16;
n1 = 4;
n2 = 24;
[root1, root_list1] = fixed_point_iter(g, x0, tol, n2);
[root2, root_list2] = newton_raphson_iter(f, f2, x0, tol, n1);
disp('root1')
disp(root1)
disp('root_list1')
disp(root_list1)
disp('root2')
disp(root2)
disp('root_list2')
disp(root_list2)
\end{lstlisting}
\subsection{The result is shown as follows.}
 \begin{tabular}{c|c}
   \hline
   Fixed-point iter & Newton-Raphson iter\\
   \hline
   % after \\: \hline or \cline{col1-col2} \cline{col3-col4} ...
   1.000000000000000 &  1.000000000000000\\
  \hline
   0.367879441171442 &  0.537882842739990\\
   \hline
   0.692200627555346 &  0.566986991405413\\
   \hline
   0.500473500563637 &  0.567143285989123\\
   \hline
   0.606243535085597 &  \\
   \hline
   0.545395785975027 &  \\
   \hline
   0.579612335503379 &  \\
   \hline
   0.560115461361089 &  \\
   \hline
   0.571143115080177 &  \\
   \hline
   0.564879347391050 &  \\
   \hline
   0.568428725029061 &  \\
   \hline
   0.566414733146883 &  \\
   \hline
   0.567556637328283 &  \\
   \hline
   0.566908911921495 &  \\
   \hline
   0.567276232175570 &  \\
   \hline
   0.567067898390788 &  \\
   \hline
   0.567186050099357 &  \\
   \hline
   0.567119040057215 &  \\
   \hline
   0.567157044001298 &  \\
   \hline
   0.567135490206278 &  \\
   \hline
   0.567147714260119 &  \\
   \hline
   0.567140781458298 &  \\
   \hline
   0.567144713346570 &  \\
   \hline
   0.567142483401307 &  \\
   \hline
   Root &  Root\\
   \hline
   0.567143748099411 & 0.567143290409784\\
   \hline
 \end{tabular}
\subsection{Analysis}
When using Newton-Raphson iteration to extract the root of an equation, according to (1.2.22), the error is of second order, while the error of fixed-point iteration is of first-order. Hence, the speed of convergence is much quicker for Newton-Raphson iteration. 

\end{proof}

\begin{problem}{2}
\text{ }\\
For a n-dimension vector \textbf{x}, when $$1\leqslant p\leqslant q,$$ prove  $$\Vert\textbf{x}\Vert_q \leqslant\Vert\textbf{x}\Vert_p\leqslant n^{\frac{1}{p}-\frac{1}{q}}\Vert\textbf{x}\Vert_q$$
\end{problem}
\begin{proof}
On one hand, assume $\Vert\textbf{x}\Vert_q = 1$, then $$\Vert x_i\Vert\leqslant 1,\quad for\quad 1\leqslant i \leqslant n$$\\
Because $1\leqslant p\leqslant q,$ then $\Vert x_{i}\Vert^{q}\leqslant\Vert x_{i}\Vert^{p}.$ Hence, $(\sum_{i=1}^{n}\Vert x_{i}\Vert^p)^\frac{1}{p}\geqslant(\sum_{i=1}^{n}\Vert x_{i}\Vert^q)^\frac{1}{q},$ which means $\Vert x\Vert_q\leqslant\Vert x\Vert_p.$\\
On the other hand, according to $H\ddot{o}lder$ inequation, $$\frac{(\sum_{i=1}^{n}\Vert x_{i}\Vert^{p})^\frac{1}{p}}{n^\frac{1}{p}} \leqslant \frac{(\sum_{i=1}^{n}\Vert x_{i}\Vert^{q})^\frac{1}{q}}{n^\frac{1}{q}},$$ hence $$\Vert\textbf{x}\Vert_p\leqslant n^{\frac{1}{p}-\frac{1}{q}}\Vert\textbf{x}\Vert_q$$
\end{proof}
\end{document}
% \begin{lstlisting}[language={MATLAB}]
% % make matrix
% % author: chuanlu
% % 2016-03-02
% function [A] = make_matrix(op, n)

%     if nargin < 1
%         error('More args needed --make matrix');
%     elseif nargin == 1
%         op = 'A1';
%     end

%     if op == 'A1'
%         c1 = zeros(1, n - 1);
%         c1(2 : n - 1) = -3;
%         c2 = zeros(1, n - 2) + 2;
%         A = eye(n) + diag(c1, -1) + diag(c2, -2);
%     elseif op == 'A2'
%         c1 = zeros(1, n - 1);
%         c1(2 : n - 1) = -3;
%         c2 = zeros(1, n - 2) + 2;
%         A = eye(n) + diag(c1, -1) + diag(c2, -2);
%         A(1, n) = -1;
%     else
%         error('Operation Failed to Match A1 or A2');
%     end
% \end{lstlisting}

% \begin{lstlisting}[language={MATLAB}]
% % homework2.m
% % author: chuanlu
% % 2016-03-02

% op1 = 'A1';
% op2 = 'A2';
% N = 100;
% cond1 = zeros(1, N);
% cond2 = zeros(1, N);
% for n = 1 : N
%     A1 = make_matrix(op1, n);
%     A2 = make_matrix(op2, n);
%     cond1(n) = cond(A1);
%     cond2(n) = cond(A2);
% end
% n = [1:N];
% figure(1);
% semilogy(n, cond1, '*-');
% figure(2);
% plot(n, cond2, '*-');
% %     disp('cond1:');
% %     disp(cond1);
% %     disp('cond2:');
% %     disp(cond2);
% \end{lstlisting}
% The result is shown as follows.
% \end{proof}

% \begin{problem}{3}
% \text{ }\\
% Given $\Vert x_{n+2} - x_{n+1} \Vert$  $\leqslant$ $\alpha\Vert x_{n+1} - x_{n} \Vert$, then $\Vert x^{\star} - x_{n} \Vert$ $\leqslant$ $\frac{\alpha_{n}}{1-\alpha}\Vert x_{1} - x_{0}\Vert$.
% \end{problem}

% \begin{proof}
% $\Vert x_{n} - x_{n - 1} \Vert\leqslant\alpha\Vert x_{n - 1} - x_{n - 2} \Vert\leqslant\dots\leqslant\alpha^{n-1}\Vert x_{1} - x_{0}\Vert$ \\
% \par Hence, $\Vert x_{\star} - x_{n} \Vert\leqslant\Vert x_{n} - x_{n+1} \Vert+\Vert x_{n+1} - x_{n+2}\Vert + \dots\leqslant(\alpha^{n}+\alpha^{n+1}+\dots)\Vert x_{1} - x_{0}\Vert$ \\
% \par$= \frac{\alpha^{n}}{1-\alpha}\Vert x_{1} - x_{0}\Vert$
% \end{proof}

% \end{document}
% % \begin{problem}{4}
% % \text{ }\\
% % Suppose that $[a]$ is a unit in $\Z_{n}$ and $[b]$ is an element of $\Z_{n}$. Prove that the equation $[a]x = b$ has exactly one solution in $\Z_{n}$
% % \end{problem}

% % \begin{proof}

% % \end{proof}

% % \begin{problem}{5}
% % \text{ }\\
% % Suppose that $[a]$ and $[b]$ are both units in $\Z_{n}$.  Show that the product $[a] \cdot [b]$ is also a unit in $\Z_{n}.$ (Note that this confirms closure under multiplication in the group $U_{n})$.
% % \end{problem}

% % \begin{proof}

% % \end{proof}

% % \begin{problem}{6}
% % \text{ }\\
% % Which of the following are Groups? Which of the following are not groups, and why?\\

% % \indent (1)  $G = \{{2, 4, 6, 8}\}$ in $\Z_{10}$.  Where $a \star b = ab$\\
% % \indent (2)  $G = \Q^{\ast}$, where $a \star b = \frac{a}{b}$\\
% % \indent (3) $G = \Z$, where $a \star b = a - b$\\
% % \indent (4) $G = \{ {2^{x}\mid x \in \Q} \}$, where $a \star b = ab$\\

% % \end{problem}

% % \begin{proof}

% % \end{proof}

% % \begin{problem}{7}
% % \text{ }\\
% % Consider the set $Q =$ \{ $\pm$1, $\pm$i, $\pm$j, $\pm$k\} of the complex matrices as follows:\\
% % \[
% % 1=
% %   \begin{bmatrix}
% %     1 & 0\\
% %     0 & 1\\
% %   \end{bmatrix}
% % \]

% % \[
% % i=
% %   \begin{bmatrix}
% %     i & 0\\
% %     0 & $-$i\\
% %   \end{bmatrix}
% % \]

% % \[
% % j=
% %   \begin{bmatrix}
% %     0 & 1\\
% %     $-$1 & 0\\
% %   \end{bmatrix}
% % \]

% % \[
% % k=
% %   \begin{bmatrix}
% %     0 & i\\
% %     i & 0\\
% %   \end{bmatrix}
% % \]
% % Show that $Q$ is a group under matrix multiplication by writing out its multiplicaiton table. (Note: $Q$ is called the quartenion group).
% % \end{problem}
% % \begin{proof}

% % \end{proof}

% % \end{document}
