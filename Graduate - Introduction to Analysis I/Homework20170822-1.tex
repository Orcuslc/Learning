
\documentclass{article}%
\usepackage{amsmath}
\usepackage{graphicx}
\usepackage{amsfonts}%
\usepackage{amssymb}


\setlength{\topmargin}{-0.75in}
\setlength{\textheight}{9.25in}
\setlength{\oddsidemargin}{0.0in}
\setlength{\evensidemargin}{0.0in}
\setlength{\textwidth}{6.5in}
\def\labelenumi{\arabic{enumi}.}
\def\theenumi{\arabic{enumi}}
\def\labelenumii{(\alph{enumii})}
\def\theenumii{\alph{enumii}}
\def\p@enumii{\theenumi.}
\def\labelenumiii{\arabic{enumiii}.}
\def\theenumiii{\arabic{enumiii}}
\def\p@enumiii{(\theenumi)(\theenumii)}
\def\labelenumiv{\arabic{enumiv}.}
\def\theenumiv{\arabic{enumiv}}
\def\p@enumiv{\p@enumiii.\theenumiii}
\pagestyle{plain}
\setcounter{secnumdepth}{0}
\newtheorem{theorem}{Theorem}
\newtheorem{acknowledgement}[theorem]{Acknowledgement}
\newtheorem{algorithm}[theorem]{Algorithm}
\newtheorem{axiom}[theorem]{Axiom}
\newtheorem{case}[theorem]{Case}
\newtheorem{claim}[theorem]{Claim}
\newtheorem{conclusion}[theorem]{Conclusion}
\newtheorem{condition}[theorem]{Condition}
\newtheorem{conjecture}[theorem]{Conjecture}
\newtheorem{corollary}[theorem]{Corollary}
\newtheorem{criterion}[theorem]{Criterion}
\newtheorem{definition}[theorem]{Definition}
\newtheorem{example}[theorem]{Example}
\newtheorem{exercise}[theorem]{Exercise}
\newtheorem{lemma}[theorem]{Lemma}
\newtheorem{notation}[theorem]{Notation}
\newtheorem{problem}[theorem]{Problem}
\newtheorem{proposition}[theorem]{Proposition}
\newtheorem{remark}[theorem]{Remark}
\newtheorem{solution}[theorem]{Solution}
\newtheorem{summary}[theorem]{Summary}
\newenvironment{proof}[1][Proof]{\textbf{#1.} }{\ \rule{0.5em}{0.5em}}

\begin{document}

\begin{center}
\textbf{Introduction to Analysis $I$\\Homework 1\\Tuesday, August 22, 2017}\bigskip
\end{center}

\noindent\textbf{Instructions}: You may submit this homework ``the old fashion'' way, i.e., using paper and pencil (or pen), but if you do so, please use at least one sheet of ($8\frac{1}{2}%
\times11$) paper per problem. Write your name at the top of each sheet you
use. Please write neatly. Staple the sheets together or use a paper clip.

However, I encourage you to do at least some of the problems using LaTeX.  As of the third assignment, you will have to submit your homework in LaTeX.

\noindent If you use results from books, Royden or others, please be explicit about what results you are using.



\begin{center}
\emph{Homework 1 is due by the start of class on Wednesday, September 6.}
\end{center} 
\medskip

\begin{enumerate}
\item \textbf{Problem 34, page 20}.  Show that the assertion of the Heine-Borel Theorem is equivalent to the Completeness Axiom for the real numbers.  Show that the assertion of the Nested Set Theorem is equivalent to the Completeness Axiom for the real numbers.

\bigskip
\textbf{Collaborators:} 
\smallskip
 
\textbf{Solution:}
\begin{proof}
\textbf{Completeness Axiom $\to$ Heine-Borel theorem} is shown in the proof of Heine-Borel theorem in Page 18.
\\[5pt]
\textbf{Heine-Borel theorem $\to$ Completeness Axiom:} \\
Let $E$ be a nonempty and upper-bounded set of real numbers, and let $b$ be one of it's upper boundaries. Pick $a\in E$, then $[a, b]$ is a closed set. Let $U$ be the set of upper boundaries of $E$, and $T = E\bigcap [a, b]$. If the supremum does not exist, then $\forall x\in[a, b]$, either $\exists a_1\in E$, s.t. $a_1 > x$, or $\exists b_1\in T$, s.t. $b_1 < x$. If it is the former case, we define an open set $A_1 = (a-1, a_1)$; else we define an open set $B_1 = (b_1, b+1)$. Since we can define such an open set to cover every $x\in [a, b]$, there exists a open cover $\{A_i\}\bigcup\{B_i\}$. According to Heine-Borel theorem, there exists a finite cover $\{A\}_{i = 1}^{n_1}\bigcup\{B\}_{i = 1}^{n_2}$. If we denote $A_0 = \bigcup_{i=1}^{n_1}\{A_i\}, B_0 = \bigcup_{i=1}^{n_2}\{B_i\}$, then since $A_0$ consists of elements in $T$, while $B_0$ consists of elements of $E's$ upper bound, we have $A_0\bigcap B_0 = \emptyset$. Then there exists a number $\epsilon\in T$, s.t. $\epsilon\notin A_0\bigcup B_0$. It is contradictory to the assumption that $A_0\bigcup B_0$ is a cover of $[a, b]$. So $E$ has a supremum. \\[5pt]
\textbf{Completeness Axiom $\to$ Nested Set theorem:} \\
Let $\{[a_i, b_i\}$ being a descending closed set series. Denote $A = \{a_i\}, B = \{b_i\}$, then $A$ has an upper bound $b_0$, and $B$ has an lower bound $a_0$. According to the Completeness Axiom, $A$ has a supremum, denoted by $x$, and B has a infimum, denoted by $y$. If $x > y$, then according to the definition of supremum and infimum, there exists $j_1, j_2$, s.t. $y \le b_{j_2} < \frac{x+y}{2} < a_{j_1} \le x$. It contradicts with that $a_i < b_j, \forall i, j$. so $x \le y$, which means $\bigcap \{[a_i, b_i]\} \ne \emptyset$. \\
For the general case, it is the same using the similar skills.\\[5pt]
\textbf{Nested Set theorem $\to$ Completeness Axiom theorem:}\\
Let $E$ be a nonempty and upper-bounded set of real numbers, and let $T$ is the set consists of all upper bounders of $E$. Pick $a_1\notin T$, and $b_1 \notin T$, then $a_1 < b_1$.
If $\frac{a_1+b_1}{2} \in T$, then $[a_2, b_2] = \left[a_1, \frac{a_1+b_1}{2}\right]$; else $[a_2, b_2] = \left[\frac{a_1+b_1}{2}, b_1\right]$. 
Construct $[a_3, b_3], \cdots$ as above, we can get a nested set series $\{[a_n, b_n]\}$, satisfying $a_n\notin T, b_n\in T$. With Nested Set theorem, there exist a real number $x$ belongs to all these closed sets, and $\lim_{n\to\infty}a_n = \lim_{n\to\infty}b_n = x$.\\
If $x\notin T$, then exists $y\in E$, s.t. $x < y$. So $b_n < x$ when $n$ is large enough. This makes a contradiction with $b_n\in T$, so $x\in T$. On the othe hand, if there exists $z\in T$, s.t. $z < x$, then $a_n > z$ when n is large enough. This makes a contradiction with $a_n\in E$. So $x$ is the supremum of $E$.
\end{proof}

\bigskip

\item \textbf{Problem 44, page 24}.  Let $p$ be a natural number greater than $1$, and $x$ a real number, $0<x<1$.  Show that there is a sequence $\{a_n\}$ of integers with $0\leq a_n < p$ for each $n$ such that \[
x = \sum_{n=1}^{\infty} \frac{a_n}{p^n} \]and that this sequence is unique except when $x$ is of the form $q/p^n$, in which case there are exactly two such sequences.  Show that, conversely, if $\{a_n\}$ is any sequence of integers with $0\leq a_n < p$, then the sequence \[\sum_{n=1}^{\infty} \frac{a_n}{p^n} \]converges to a real number $x$ with $0\leq x \leq 1$. (If $p = 10$, this series is called the \emph{decimal} expansion of $x$.  For $p = 2$ it is called the \emph{binary} expansion of $x$, and for $p = 3$, the \emph{tenary} expansion.) 


\bigskip
\textbf{Collaborators:}\\
\smallskip
 
\textbf{Solution:}
\begin{proof}
\textbf{Existance:} \\
Let $a_1 = \lfloor px\rfloor$, $a_2 = \lfloor p^2x-a_1p\rfloor$, $\cdots$, $a_n = \lfloor p^nx - \sum_{i=1}^{n-1}a_ip^{n-i}\rfloor=-$. From this inductive construction we know that the subsum $S_n = \sum_{i=1}^n\frac{a_i}{p^i}$ forms a increasing sequence, while it is upper bounded by $x$. According to the Monotone Convergence Criterion, this sequence has a limitation. Moreover, $|x - S_n| \le \frac{1}{p^{n-1}}$ for each n, so $\lim_{n\to\infty} |x-S_n|= 0$, that means $x$ is a limitation of $\{S_n\}$. By the same criterion, we can know that $x$ is the only limitation of $\{S_n\}$, that means $x = \sum_{n=1}^{\infty}\frac{a_n}{p^n}$. \\[5pt]
\textbf{Uniqueness:} \\
First, we need to prove that $0$ can be represented in this way by two sequences if we may let $a_0 = -1$. The two sequences are trivial: the first one is $a_i = 0$, and the second one is $a_0 = -1, a_i = p-1$. If there exist a third sequence ${b_i}$, then if $b_0 \ge 0, b_j > 0$, then $\sum_{b_i} \ge \frac{1}{p^j} > 0$. If $b_0 \le -1, b_j < p-1$, the $\sum_{b_i} \le -\frac{1}{p^j} < 0$. So there only exists two sequences for $0$. When $x = \frac{q}{p^n}$, it is just the same as $x = 0$ since we only need to shuffle the point n times to the right. So there also exists two sequences for $x$. \\
When $x \ne \frac{q}{p^n}$, assume there exists two sequences for representing $x$, denoted by $\{a_i\}$ and $\{b_i\}$. According to this assumption, there must exist a $j$, s.t. $a_j \ne q-1$. Let $k$ be the first position of difference between these two sequences, then $|\sum{a_i}-\sum{b_i}| >= \frac{1}{p^k}-\sum_{i=k+1}^{\infty}\frac{|a_i-b_i|}{p^i} > \frac{1}{p^j}$. So this contradicts with our assumption, which means $x$ only has one sequence in this form.
\end{proof}
\bigskip



\item \textbf{Problem 46, page 25}. Show that the assertion of the Bolzano-Weierstrass Theorem is equivalent to the Completeness Axiom of the real numbers.  Show that the assertion of the Monotone Convergence Theorem is equivalent to the Completeness Axiom for the real numbers.


\bigskip
\textbf{Collaborators:}\\
\smallskip
 
\textbf{Solution:} 
\begin{proof}
\textbf{Completeness Axiom $\to$ Monotone Convergence Criterion is shown is the proof in Page 21.} \\
\textbf{Monotone Convergence Criterion $\to$ Completeness Axiom:} \\
Let $E$ be a upper bounded set, and $T$ be the set of upper bound of $E$. Pick $a_1 \in E, b_1 \in T$. If $\frac{a_1+b_1}{2}\in E$, we choose $a_2 = \frac{a_1+b_1}{2}, b_2 = b_1$; else we choose $a_2 = a_1, b_2 = \frac{a_1+b_1}{2}$. Similarly, we get two sequences $\{a_n\}, \{b_n\}$, and $\{a_n\}$ is a upper bounded monotone increasing sequence, $\{b_n\}$ is a lower bounded monotone decreasing sequence, and $b_n-a_n\to 0$. According to Monotone Convergence Criterion, there exists $\alpha$, s.t. $\alpha = \lim_{n\to\infty}a_n$. Next we prove $\alpha = \sup E$. \\
First, $\forall \epsilon > 0, \exists N_\epsilon$, s.t. when $n > N_\epsilon$, $\alpha-\epsilon < a_n$. Since $a_n$ is not an upper bound of $E$, there exists $x_\epsilon \in E$, s.t. $\alpha-\epsilon < a_n < x_\epsilon$. On the other hand, if there exists $x_0\in E$, s.t. $x_0 > \alpha$, then $\lim_{n\to\infty}b_n = \lim_{n\to\infty}a_n+\lim_{n\to\infty}\frac{b_1-a_1}{2^n} = \alpha$. Thus $\exists n_0 > 0$, s.t. $x_0 > b_{n_0}$. It contradicts with the assumption that $b_i$ are upper bounds of $E$. So $\alpha = \sup E$. \\[5pt]
Since \textbf{Nested Set theorem $\to$ Bolzano-Weierstarss theorem} has been proved in Page 21, and \textbf{Bolzano-Weierstarss theorem $\to$ Cauchy Convergence Criterion} has been proved in Page 22, and we have proved the Nested Set theorem is equavilent to Completeness Axiom, we only need to prove \textbf{Cauchy Convergence Criterion $\to$ Nested Set theorem}. \\
Assume $\{[a_n, b_n\}$ is a sequence of closed sets, and $\lim_{n\to\infty}(b_n-a_n) = 0$. Then pick $m > n$, we have $0 \le a_m-a_n < b_n-a_n\to 0$. So $\{a_n\}$ is Cauchy, which means $\lim_{n\to\infty}a_n = \alpha$, and $\lim_{n\to\infty}b_n = \alpha$. Since $\{a_n\}$ is monotone increasing while $\{b_n\}$ is monotone decreasing, we know $\alpha$ is the only number belongs to every closed set.
\end{proof}
\bigskip



\item Let $\mathbb{F}$ be an ordered field with the property that if $[a,b]$ is any closed interval and if $f:[a,b]\rightarrow \mathbb{F}$ is any continuous function such that $f(a)>0$ and $f(b)<0$, then there is an $x$, $a < x < b$, such that $f(x)=0$.  Show that such a field $\mathbb{F}$ has the least upper bound property.


\bigskip
\textbf{Collaborators:}\\
\smallskip
 
\textbf{Solution:}
\begin{proof}
Let $E$ be a upper-bounded set, and $T$ be the set of all upper bounds of $E$. Construct a function f like this: 
$$
f = \left\{
\begin{aligned} 
1, x\in E \\
-1, x \in T
\end{aligned}
\right.
$$
We assume that $E$ has no supremum. That means $\forall x\in E, \forall \epsilon > 0$(We may assume $x$ is a inner point of $E$), we can find $x_1 > x \in E$, so $\forall x_2 \in (x-\delta, x_1), |f(x_1)-f(x)| = 0 < \epsilon$. So $f$ is continuous on $x$. With the arbitrariness of $x$, $f$ is continuous on $E$. Similarly, $f$ is continuous on $T$. Now we prove that $f$ is continuous on $E\bigcup T$. If not, assume $f$ is not continuous on $x_0$. If $x_0 \in E$, then $f$ should be discontinuous on the right side, which means $\exists \epsilon > 0, \forall \eta > 0, \exists x$, s.t. $|x_0 - x| < \eta, |f(x_0)-f(x)|>\epsilon$. That just means $x_0$ is the supremum of $E$, which makes a contradiction. If $x_0 \in T$, similarly $x_0$ is the supremum of $E$ according to definition. So $f$ is continuous on $E\bigcup T$. Using the property in the problem, for $a \in E, b\in T$, there must exist $x$, s.t. $f(x) = 0$. It contradicts with the construction of $f$. So E has a supremum, which means $\mathcal{F}$ has the least upper bound property.
\end{proof}
\bigskip


\item Show that if $\mathbb{F}$ is an ordered field such that
\begin{enumerate}
\item the order for $\mathbb{F}$ satisfies the Archimedean property, and
\item every Cauchy sequence is convergent, 
\end{enumerate}
then $\mathbb{F}$ satisfies the Completeness Axiom. (There are examples of ordered fields that \emph{don't} have the Archimedean property and \emph{do} have the property that every Cauchy sequence is convergent.  Can you come up with such an example?)


\bigskip
\textbf{Collaborators:}\\
\smallskip
 
\textbf{Solution:}
\begin{proof}
Let $E$ be any upper bounded set in $\mathcal{F}$. Let $T$ be the set of all upper bounds of $E$. Select $x_1 \in E$, $y_1 \in T$, and we construct two sequences like this:
$$
x_2 = \left\{
\begin{aligned}
&x_1, ~\frac{x_1+y_1}{2}\in T, \\
&\frac{x_1+y_1}{2}, ~\frac{x_1+y_1}{2}\in E
\end{aligned}
\right.
$$
$$
y_2 = \left\{
\begin{aligned}
&\frac{x_1+y_1}{2}, ~\frac{x_1+y_1}{2}\in T \\
&y_1, ~\frac{x_1+y_1}{2}\in E
\end{aligned}
\right.
$$
Let $\{z_n\} = \{x_1, y_1, y_2, x_2,\cdots\}$, let $\xi = y_1-x_1$, then according to Archimedean property, for $\forall \epsilon > 0$, there exists a $N > 0$, $\forall m, n > N$, $|z_m-z_n| < \frac{\xi}{2^N} < \epsilon$. Thus $\{z_n\}$ is a Cauchy sequence, so it convergent to $z$. \\
Now we prove that $z$ is the supremum of $E$. First, we have $\{x_i\} \le z \le \{y_i\}$, or if there exists $n$, s.t. $x_n > z$, then with the monotoness of $\{x_i\}$, for $\forall m > n$, $|x_m - z| \le |x_n-z|$, which contradicts with that $\{x_n\}\subset\{z_n\}\to z$. In fact this already contains that $z$ is the minimal element of $T$, whichi means $z$ is the supremum of $E$. Thus we proved the Completeness Axiom.
\end{proof}
\bigskip


\item Suppose that for $i = 1,2$, $\mathbb{F}_i$ is an ordered field satisfying the Completeness Axiom. Show that there is a unique isomorphism $\alpha$ from $\mathbb{F}_1$ onto $\mathbb{F}_2$. That is, $\alpha$ is a field isomorphism that preserves the order ($x \leq_1 y$ implies $\alpha(x) \leq_2 \alpha(y)$, where $\leq_i$ denotes ``less than or equal to'' in the field $\mathbb{F}_i$). Thus, there is at most \emph{one} field of ``real numbers''. 


\bigskip
\textbf{Collaborators:}\\
\smallskip

\textbf{Solution:} 
\begin{proof}
\textbf{Uniqueness:} \\
Assume there exists two field isomorphisms $f_1, f_2$ from $\mathcal{F}_1$ onto $\mathcal{F}_2$. Let $E$ is an upper bounded set in $\mathcal{F}_1$, and according to the Completeness Axiom, it's supremum is $e$. Denote $E_1 = f_1(E)\in \mathcal{F}_2, E_2 = f_2(E)\in\mathcal{F}_2$. Then according to properties of isomorphism, $E_1, E_2$ are both upper bounded sets, so they both have supremums, denoted by $e_1, e_2$. Then since the isomorphism preserves the order, we have $f_1(e) = f_2(e) = e_1 = e_2$. Since for each $x\in \mathcal{F}_1$, we can construct a set with $x$ being it's supremum (for example, a sequence converge to $x$), we have $f_1(x) = f_2(x)$. Thus $f_1 = f_2$. \\[5pt]
\textbf{Existence:} \\
We construct the function like this: \\
According to the properties of fields, there exists $0_1, 1_1\in \mathcal{F}_1,~ 0_2, 1_2\in \mathcal{F}_2$, 
Since $\mathcal{F}_1, \mathcal{F}_2$ are fields, they close to multiplications. Let $I_1$ be the identity of $\mathcal{F}_1$, and $I_2$ be the identity of $\mathcal{F}_2$. Then since each non-zero element have an inverse, we can define rational numbers in $F_i$:
$$
q = (c_1I_i)\times (c_2I_i)^{-1},
$$
where $x^{-1}$ means the inverse in it's field. \\
Now we construct the map like this: \\
$$
\begin{aligned}
f: ~\mathcal{F}_1&\to\mathcal{F}_2 \\
(c_1I_1)\times(c_2I_1)^{-1} &\mapsto (c_1I_2)\times(c_2I_2)^{-1}.
\end{aligned}
$$
For $x \in \mathcal{F}_1$, if $x$ is not a rational number, we call it an irrational number. 
If $x$ is an irrational number, if we can construct two monotone sequences of rational numbers ${a_i}, {b_i}$, s.t. $a_{i+1}>a_{i}$, $b_{i+1} < b_{i}$, and ${a_i}, {b_i} \to x$, then since the map preserves the order for rational numbers, we can define 
$$
f(x) = \lim_{i\to\infty}f(a_i) = \lim_{i\to\infty}f(b_i).
$$
for irrational numbers. Then using the linear property of limit, it is trivial to prove that this function is a field isomorphism.

First we prove that $\mathcal{F}$ has the Archimedean property. If not, suppose for $a, b\in \mathcal{F}$, there do not exist a natural number n, $n\cdot a > b$. Suppose $E$ is a bounded set, and $b \in E$, while $a \notin E$. According to the Completeness Axiom, $E$ has a supremum, denoted by $x_0$. 


suppose $E$ is a bounded set in $\mathcal{F}$, and with the completeness property, $E$ has a supremum, denoted by $x_0$. 
According to properties of ordered field, there exists $a_0$, 

\end{proof}
\bigskip


\end{enumerate}
\end{document}
