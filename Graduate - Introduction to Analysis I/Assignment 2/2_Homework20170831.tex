
\documentclass{article}%
\usepackage{amsmath}
\usepackage{graphicx}
\usepackage{amsfonts}%
\usepackage{amssymb}


\setlength{\topmargin}{-0.75in}
\setlength{\textheight}{9.25in}
\setlength{\oddsidemargin}{0.0in}
\setlength{\evensidemargin}{0.0in}
\setlength{\textwidth}{6.5in}
\def\labelenumi{\arabic{enumi}.}
\def\theenumi{\arabic{enumi}}
\def\labelenumii{(\alph{enumii})}
\def\theenumii{\alph{enumii}}
\def\p@enumii{\theenumi.}
\def\labelenumiii{\arabic{enumiii}.}
\def\theenumiii{\arabic{enumiii}}
\def\p@enumiii{(\theenumi)(\theenumii)}
\def\labelenumiv{\arabic{enumiv}.}
\def\theenumiv{\arabic{enumiv}}
\def\p@enumiv{\p@enumiii.\theenumiii}
\pagestyle{plain}
\setcounter{secnumdepth}{0}
\newtheorem{theorem}{Theorem}
\newtheorem{acknowledgement}[theorem]{Acknowledgement}
\newtheorem{algorithm}[theorem]{Algorithm}
\newtheorem{axiom}[theorem]{Axiom}
\newtheorem{case}[theorem]{Case}
\newtheorem{claim}[theorem]{Claim}
\newtheorem{conclusion}[theorem]{Conclusion}
\newtheorem{condition}[theorem]{Condition}
\newtheorem{conjecture}[theorem]{Conjecture}
\newtheorem{corollary}[theorem]{Corollary}
\newtheorem{criterion}[theorem]{Criterion}
\newtheorem{definition}[theorem]{Definition}
\newtheorem{example}[theorem]{Example}
\newtheorem{exercise}[theorem]{Exercise}
\newtheorem{lemma}[theorem]{Lemma}
\newtheorem{notation}[theorem]{Notation}
\newtheorem{problem}[theorem]{Problem}
\newtheorem{proposition}[theorem]{Proposition}
\newtheorem{remark}[theorem]{Remark}
\newtheorem{solution}[theorem]{Solution}
\newtheorem{summary}[theorem]{Summary}
\newenvironment{proof}[1][Proof]{\textbf{#1.} }{\ \rule{0.5em}{0.5em}}

\begin{document}

\begin{center}
\textbf{Introduction to Analysis $I$\\Homework 2\\Thursday, August 31, 2017}\bigskip
\end{center}

\noindent\textbf{Instructions}: You may submit this homework ``the old fashion'' way, i.e., using paper and pencil (or pen), but if you do so, please use at least one sheet of ($8\frac{1}{2}%
\times11$) paper per problem. Write your name at the top of each sheet you
use. Please write neatly. Staple the sheets together or use a paper clip.

However, I encourage you to do at least some of the problems using LaTeX.  As of the third assignment, you will have to submit your homework in LaTeX.

\noindent If you use results from books, Royden or others, please be explicit about what results you are using.



\begin{center}
\emph{Homework 2 is due by the start of class on Wednesday, September 13.}
\end{center} 
\medskip

\begin{enumerate}
\item Recall the unconventional terminology I introduced in class: If $J$ is an infinite set and $f:J\to \mathbb{R}$ is a real-valued function, then the symbol $\sum_{j\in J} f(j)$ is the series determined by $J$, with terms $f(j)$.  We say $\sum_{j\in J} f(j)$ \emph{converges to $S$ unconditionally} if for every $\epsilon > 0$ there is a finite subset $F_0 \subset J$ such that if $F$ is any finite subset of $J$ that contains $F_0$, then $|\sum_{j\in F} f(j) - S|< \epsilon$. Show that if $\sum_{j\in J} f(j)$ converges to some $S$ unconditionally, then $\{j\in J \mid f(j)\neq 0\}$ is countable.

\bigskip
\textbf{Collaborators:}\\
\smallskip

\textbf{Solution:}
\begin{proof} 
According to the assumption, $\forall \epsilon > 0$, there is a finite subset $F_0$, $\forall$ finite $F\supset F_0$, $-\epsilon-(\sum_{j\in F_0}f(j)-S) < \sum_{j\in F\setminus F_0}f(j) < \epsilon-(\sum_{j\in F_0}f(j)-S)$. Denote $\alpha = \epsilon-(\sum_{j\in F_0}f(j)-S)$. \\[2pt]
Suppose that $E$ = $\{j\in J|f(j)\ne 0\}$ is uncountable, denote $E_1 = \{j\in J|f(j) > 0\}$, $E_2 = \{j\in J|f(j) < 0\}$, then $E = E_1+E_2$. Then either $E_1$ or $E_2$ is uncountable, otherwise $E$ is countable. We may assume that $E_1$ is uncountable. We can construct a sequence of subsets $\{F_n\}$ like this: \\[2pt]
First, $F_1 = \{j\in J|f(j) > \alpha\}$ is countable, otherwise $F_1\setminus F_0$ is uncountable, which means we can find a finite set from $F_1\setminus F_0$, s.t. $\sum_{j\in F_1\setminus F_0}f(j) > \alpha$. Similarly, $F_{n+1} = \{j\in J|f(j) > \frac{\alpha}{2^n}\}$ are all countable, which means $G_n = E_1 - F_{n+1}$ is uncountable. However, since the countable sum of countable sequences is also countable, so $\lim_{n\to\infty}G_n = \{j\in J|f(j)\le 0\}$ is uncountable, and it contradicts to the assumption. So there must exist a $k > 0$, $F_k$ is uncountable, and thus $F_k \setminus F_0$ is uncountable. Thus we can select $2^k$ elements from $F_k$, and their sum is larger then $\alpha$, and it also contradicts with our assumption.
\end{proof}
\bigskip

\item Assume now that $J$ is countably infinite and show that the following assertions about a series $\sum_{j\in J} f(j)$ are equivalent:
\begin{enumerate}
\item The series $\sum_{j\in J} f(j)$ converges unconditionally.
\item For each bijection $\phi:\mathbb{N}\to J$, the series $\sum_{k = 1}^{\infty} f\circ(\phi(k))$ converges in the sense discussed in the text, and that the sum is the sum of $\sum_{j\in J} f(j)$.
\item The series of absolute values $\sum_{j\in J} |f(j)|$ converges.
\end{enumerate}

\bigskip
\textbf{Collaborators:}\\
\smallskip
 
\textbf{Solution:}
\begin{proof}
Since $J$ is countably infinite, we may assume $J = \{j_i\}_{i=0}^{\infty}$. \\[2pt]
\textbf{(a) $\to$ (b):}
Since $\sum_{j\in J}f(j)$ converges unconditionally, $\forall \epsilon > 0$, $\exists$ a finite set $F_0$, $\forall F\supset F_0$,
$$
\left|\sum_{j\in F}f(j)-\sum_{i=1}^{\infty}f(j_i)\right|<\epsilon.
$$
Since $\phi$ is a bijection, there $\exists N > 0$, $\phi(\{1, 2, \cdots, N\})\supset F_0$. Denote $\phi(\{1, 2, \cdots, N\}) = F$, then
$$
\left|\sum_{i=1}^{N}f(\phi(i))-\sum_{i=1}^{\infty}f(j_i)\right| = \left|\sum_{i=1}^{N}f(\phi(i))-\sum_{i=1}^{\infty}f(\phi(j_i))\right| < \epsilon.
$$
Thus the series converges, and the sum is $\sum_{j\in J}f(j)$. \\[2pt]
\textbf{(b) $\to$ (c):}
According to (b) we have $\sum_{k=1}^{\infty}f(j_k)$ converges. Suppose $\sum_{j\in J}|f(j)|$ does not converge, we first prove that $\sum_{k=1}^{\infty}f(j_k)^+$ and $\sum_{k=1}^{\infty}f(j_k)^-$ do not converge, where $f(j_k)^+, f(j_k)^-$ means the absolute value of positive and negative terms in $f(j)$, seperately. \\[2pt]
Suppose $\sum_{k=1}^{\infty}f(j_k)+$ converges, then since
$$
\sum_{k=1}^{\infty}f(j_k)^- = \sum_{k=1}^{\infty}f(j_k)^+-\sum_{k=1}^{\infty}f(j_k),
$$
we know that $\sum_{k=1}^{\infty}f(j_k)^-$ converges, which means
$$
\sum_{k=1}^{\infty}|f(k_j)| = \sum_{k=1}^{\infty}f(k_j)^++\sum_{k=1}^{\infty}f(k_j)^-
$$
also converges, which lead to a contradiction. \\[2pt]
Thus we can rearrange the order of $\sum_{j\in J}f(j)$ like this: for each $f(a_i) > 0$, put enough $f(b_i) < 0$ after it, making $f(a_i) + \sum_{i}f(b_i) < 1$. The possibility of this map comes from that $\sum_{j}f(j)^-\to\infty$. Then we constructed a bijection from $\mathbb{N}\to J$. However, $\sum_{j}\phi(f(j)) < \sum_{j_i}-1\to -\infty$, which makes a contradiction. Thus $\sum_{j}|f(j)|$ converges. \\[2pt]
\textbf{(c) $\to$ (a):} Since the series of absolute values $\sum_{j\in J}|f(j)|$ converges, we have the original series $\sum_{j\in J}|f(j)|$ converges, and mark that sum as $S$. Then according to the definition, for $\forall \epsilon > 0$, there $\exists N > 0$, $\sum_{i=N}^{\infty}|f(j_i)| < \epsilon$. If we take $F_0 = \{j_1, j_2, \cdots, j_{N-1}\}$, then $\forall$ finite $F\supset F_0$, $\sum_{j\in F\setminus F_0} f(j) \le \sum_{i=N}^{\infty}|f(j_i)|< \epsilon$. That means (a) holds.
\end{proof}
\bigskip



\item \begin{enumerate}
\item Show that the series \[
\sum_{n\geq 1, n\neq m} \frac{1}{m^2 - n^2}\]is convergent and has sum equal to $-\frac{3}{4m^2}$ (decompose the rational fraction $1/(m^2 - x^2)$).
\item Let $u_{mn} = \frac{1}{m^2 - n^2}$ if $m\neq n$, and let $u_{nn} = 0$. Show that \[
\sum_{m=0}^{\infty}\left(\sum_{n=0}^{\infty} u_{mn}\right) = -\sum_{n=0}^{\infty}\left(\sum_{m=0}^{\infty} u_{mn}\right) \neq 0.\]
\item Explain what, if anything, the computations of parts a) and b) of this problem have to do with Problem 2.
\end{enumerate}


\bigskip
\textbf{Collaborators:}\\
\smallskip
 
\textbf{Solution:}
\begin{proof}
\textbf{(a)} \\[2pt]
$$
\begin{aligned}
\sum_{n\ne m}\frac{1}{m^2-n^2} &= -\frac{1}{2m}\sum_{n\ne m}\left(\frac{1}{n-m}-\frac{1}{n+m} \right)\\
&=-\frac{1}{2m}(-\frac{1}{m-1}-\frac{1}{m+1}-\cdots-1-\frac{1}{2m-1}\\
&\quad + 1-\frac{1}{2m+1}+\cdots+\frac{1}{m}-\frac{1}{3m}\\
&\quad + \frac{1}{m+1}-\frac{1}{3m+1}+\cdots+\frac{1}{2m}-\frac{1}{4m}+\cdots) \\
&= -\frac{1}{2m}\left(\lim_{n\to\infty}(-\sum_{i=-m}^{m}\frac{1}{n+i}) + \frac{1}{m} + \frac{1}{2m}\right) \\
&= -\frac{3}{4m}.
\end{aligned}
$$
\\
\textbf{(b)} 
According to (a),
$$
\sum_{n=0}^{\infty}u_{mn} = -\frac{3}{4m^2}+\frac{1}{m^2} = \frac{1}{4m^2}, ~\sum_{m=0}^{\infty}u_{mn} = -\frac{1}{4n^2}.
$$
Thus
$$
\sum_{m=0}^{\infty}\left(\sum_{n=0}^{\infty} u_{mn}\right) = -\sum_{n=0}^{\infty}\left(\sum_{m=0}^{\infty} u_{mn}\right) = \sum_{n=1}^{\infty}\frac{1}{4n^2}\ne 0.
$$
\\
\textbf{(c)}
\end{proof}
\bigskip



\item (Problems 17, 18, and 19, page 43) These problems tie well together, so I thought I would make them one big problem.

\begin{enumerate}
\item Show that a set $E$ is measurable if and only if for each $\epsilon > 0$, there is a closed set $F$ and an open set $\mathcal{O}$ for which $F\subseteq E \subseteq \mathcal{O}$ and $m^*(\mathcal{O} \sim F)< \epsilon.$
\item Suppose $E$ is a set with finite outer measure. Show that there is a $G_{\delta}$ set $G$ such that $
 E \subseteq G \text{ and } m^*(E) = m(G).$  Show that $E$ is measurable if and only if there is an $F_{\sigma}$ set $F$ contained in $E$ such that $m(F) = m^*(E)$.
\item Suppose $E$ is a set with finite outer measure. Show that if $E$ is \emph{not} measurable then there is an open set $\mathcal{O}$ containing $E$ that has finite outer measure for which \[
m^*(\mathcal{O}\sim E) > m^*(\mathcal{O}) - m^*(E).                                                                                                                                                                                                
                                                                                                                                                                                               \]

  
\end{enumerate}


\bigskip
\textbf{Collaborators:}\\
\smallskip
 
\textbf{Solution:}
\begin{proof}
\textbf{(a):}\\[2pt]
\textbf{Sufficiency:}
For $\forall A\subset \mathbb{R}$, first we have $m^*(A) \le m^*(A\bigcap E) + m^*(A\bigcap E^c)$. So we only need to show that $m^*(A) \ge m^*(A\bigcap E) + m^*(A\bigcap E^c)$. According to assumption, $m^*(A\bigcap E) + m^*(A\bigcap E^c) \le m^*(A\bigcap O)+m^*(A\bigcap F^c) < m^*(A\bigcap O)+m^*(A\bigcap O^c)+\epsilon$. Since the open set $O$ is measurable, we have $m^*(A\bigcap )+m^*(A\bigcap O^c) = m^*(O)$. Thus by the arbitrariness of $\epsilon$, we know that $m^*(A) = m^*(A\bigcap E)+m^*(A\bigcap E^c)$. \\[2pt]
\textbf{Necessity:} P24
We first prove that 
\end{proof}
\bigskip


\item (Problem 28, Page 47) Show that continuity of measure together  with finite additivity of measure implies countable additivity of measure.

\bigskip
\textbf{Collaborators:}\\
\smallskip
 
\textbf{Solution:}
\bigskip


\end{enumerate}
\end{document}
