
\documentclass{article}%
\usepackage{amsmath}
\usepackage{graphicx}
\usepackage{amsfonts}%
\usepackage{amssymb}


\setlength{\topmargin}{-0.75in}
\setlength{\textheight}{9.25in}
\setlength{\oddsidemargin}{0.0in}
\setlength{\evensidemargin}{0.0in}
\setlength{\textwidth}{6.5in}
\def\labelenumi{\arabic{enumi}.}
\def\theenumi{\arabic{enumi}}
\def\labelenumii{(\alph{enumii})}
\def\theenumii{\alph{enumii}}
\def\p@enumii{\theenumi.}
\def\labelenumiii{\arabic{enumiii}.}
\def\theenumiii{\arabic{enumiii}}
\def\p@enumiii{(\theenumi)(\theenumii)}
\def\labelenumiv{\arabic{enumiv}.}
\def\theenumiv{\arabic{enumiv}}
\def\p@enumiv{\p@enumiii.\theenumiii}
\pagestyle{plain}
\setcounter{secnumdepth}{0}
\newtheorem{theorem}{Theorem}
\newtheorem{acknowledgement}[theorem]{Acknowledgement}
\newtheorem{algorithm}[theorem]{Algorithm}
\newtheorem{axiom}[theorem]{Axiom}
\newtheorem{case}[theorem]{Case}
\newtheorem{claim}[theorem]{Claim}
\newtheorem{conclusion}[theorem]{Conclusion}
\newtheorem{condition}[theorem]{Condition}
\newtheorem{conjecture}[theorem]{Conjecture}
\newtheorem{corollary}[theorem]{Corollary}
\newtheorem{criterion}[theorem]{Criterion}
\newtheorem{definition}[theorem]{Definition}
\newtheorem{example}[theorem]{Example}
\newtheorem{exercise}[theorem]{Exercise}
\newtheorem{lemma}[theorem]{Lemma}
\newtheorem{notation}[theorem]{Notation}
\newtheorem{problem}[theorem]{Problem}
\newtheorem{proposition}[theorem]{Proposition}
\newtheorem{remark}[theorem]{Remark}
\newtheorem{solution}[theorem]{Solution}
\newtheorem{summary}[theorem]{Summary}
\newenvironment{proof}[1][Proof]{\textbf{#1.} }{\ \rule{0.5em}{0.5em}}

\begin{document}

\begin{center}
\textbf{Introduction to Analysis $I$\\Homework 4\\Wednesday, September 27, 2017}\bigskip
\end{center}

\noindent\textbf{Instructions}: This and all subsequent homeworks must be submitted written in \LaTeX.

\noindent If you use results from books, Royden or others, please be explicit about what results you are using.



\begin{center}
\emph{Homework 4 is due by midnight, Saturday, October 7.}
\end{center} 
\medskip

\begin{enumerate}
\item  (Problem 18, Page 63) Let $I$ be a closed bounded interval and let $f$ be a bounded measurable function defined on $I$. Let $\epsilon > 0$.  Show that there is a step function $h$ on $I$ and a measurable subset $F$ of $I$ for which \[
|h-f|< \epsilon \text{ on } F \text{ and } m(I\sim F)< \epsilon .\]


\bigskip
\textbf{Collaborators:} None
\smallskip
 
\textbf{Solution:}
For $\epsilon > 0$, according to Lusin's Theorem, there exists a closed subset $F$ of $I$, s.t. $m(I\setminus F) < \epsilon$, and $f$ is continuous on $F$. 
Then for $x_0\in F$, for this $\epsilon > 0$, 
$$
\exists \sigma > 0, ~\forall x\in (x_0-\delta, ~x_0+\delta)\cap F, ~|f(x)-f(x_0)| < \epsilon.
$$
Since
$$
\bigcup_{x_0\in F} (x_0-\delta_i, ~x_0+\delta_i)
$$
is an open cover of the closed set $F$, it has a finite subcover, denoted by
$$
\bigcup_{i=1}^n (x_i-\delta_i, ~x_i+\delta_i).
$$
If we define
$$
h = \left\{
\begin{aligned}
&f(x_i), ~\text{for} ~x\in (x_i-\delta_i, ~x_i+\delta_i), \\
&0, ~\text{others}
\end{aligned}
\right.
$$
then $h$ takes constant values in finite intervals, and $h = 0$ in other parts, we know $h$ is a step function. And from the construction above, on each interval $(x_i-\delta_i, x_i+\delta_i)\cap F$, 
$$
|h-f|< \epsilon ~\text{on}~ (x_i-\delta_i, x_i+\delta_i)\cap F.
$$
Thus $h$ is the step function needed.

\bigskip

\item (Problem 22, Page 64) (Dini's Theorem)  Let $\{f_n\}$ be an increasing sequence of continuous functions on $[a,b]$ which converges pointwise on $[a,b]$ to the continuous function $f$ on $[a,b]$.  Show that the convergence is uniform on $[a,b]$.


\bigskip
\textbf{Collaborators:} None
\smallskip
 
\textbf{Solution:}
If the convergence is not uniform, then there $\exists \epsilon > 0, \forall N > 0, ~\exists n > N, ~\exists x\in [a, b], ~|f(x)-f_n(x)| > \epsilon $.

We can construct a sequence $\{x_n\}$ like this:

$$
N = 1, ~\exists n_1 > 1, ~\exists x_1\in[a, b], ~|f_{n_1}(x_1)-f(x_1)| \geqslant \epsilon.
$$
$$
N = n_1, ~\exists n_2 > n_1, ~\exists x_2\in[a, b], ~|f_{n_2}(x_2)-f(x_2)| \geqslant \epsilon.
$$
$$
\cdots
$$
$$
N = n_{k-1}, ~\exists n_k > n_{k-1}, ~\exists x_k\in[a, b], ~|f_{n_k}(x_k)-f(x_k)| \geqslant \epsilon.
$$
$$
\cdots
$$
Then since $x_i\in [a, b]$, according to Bolzano-Weierstrass Theorem, there is a convergent subsequence in $\{x_i\}$, and we may denote it by $\{y_i\}$ for convinence. Assume $y_i\to y\in [a, b]$, then since
$$
\lim_{n\to\infty}f_n(y) = f(y),
$$
for this $\epsilon > 0, ~\exists N$, s.t.
$$
|f_N(y)-f(y)|<\epsilon.
$$
since $f_N$ is continuous, with $y_k\to y$, there $\exists K > 0$, 
$$
|f_N(y_k)-f(y_k)| < \epsilon
$$
holds for all $k > K$. Notice that $\{f_n\}$ is an increasing sequence, when $n > N ~\text{and}~ k > K$, 
$$
|f_n(y_k)-f(y_k)| \leqslant |f_N(y_k)-f(y_k)|<\epsilon.
$$
Since $n_k\to\infty$ when $k\to\infty$, when $k$ is sufficiently large we have $k > K, ~n_k > N$. Thus 
$$
|f_{n_k}(x_k)-f(x_k)| < \epsilon,
$$ 
which makes a contradiction with the assumption. Thus the convergence is uniform.
\bigskip



\item  (Problem 5, Page 364) Show that an extended real-valued function $f$ on $X$ is measurable if and only if for each rational number $c$, $\{x\in X \mid f(x)< c\}$ is a  measurable set.


\bigskip
\textbf{Collaborators:} None
\smallskip
 
\textbf{Solution:}
Notice that for each $r\in \mathbb{R}\supset\mathbb{Q}$, we have
$$
\{x\in X\mid f(x) < r\} = \bigcup_{\substack{c < r \\ c \in \mathbb{Q}}}\{x\in X\mid f(x) < c\}.
$$
Then since Lebesgue measurable sets make a $\sigma-$Algebra, we know $\{x\in X\mid f(x) < r\}$ is measurable. On the other hand, since each rational number is a real number, the other direction stands.
\bigskip

\item (Problem 13, Page 365) Let $\{f_n\}$ be a sequence of real-valued functions on $X$ such that for each natural number $n$, $\mu \{x\in X \mid |f_n (x) - f_{n+1}(x)|> 1/2^n\}<1/2^n$. Show that $\{f_n\}$ is pointwise convergent a.e. on $X$.

\bigskip
\textbf{Collaborators:}\\
\smallskip
 
\textbf{Solution:}
\bigskip

\item  (Problem 15, Page 365)  A sequence $\{f_n\}$ of measurable real-valued functions on $X$ is said to \emph{converge in measure} to a measurable function $f$  provided that for each $\eta > 0$, \[
\lim_{n\to \infty} \mu\{x\in X \mid |f_n(x) - f(x)|>\eta \} = 0.\]A sequence $\{f_n\}$ of measurable functions is said to be \emph{Cauchy in measure} provided that for each $\epsilon > 0$ and $\eta > 0$, there is an index $N$ such that for each $m,n\geq N$, \[
\mu\{x\in X \mid |f_n(x) - f_m(x)| > \eta \} < \epsilon.\]
\begin{enumerate}
\item Show that if $\mu(X)< \infty$ and if $\{f_n\}$ converges pointwise a.e. on $X$ to a measurable function $f$, then $\{f_n\}$ converges to $f$ in measure.
\item Show that if $\{f_n\}$ converges to $f$ in measure, then there is a subsequence of $\{f_n\}$ that converges pointwise a.e. to $f$.
\item Show that if $\{f_n\}$ is Cauchy in measure, then there is a measurable function $f$ to which $\{f_n\}$ converges in measure.\end{enumerate}

\item (Problem 16, Page 365) Assume $\mu (X) < \infty$.  Show that $\{f_n\}$ converges to $f$ in measure if and only if each  subsequence of $\{f_n\}$ has a further subsequence that converges pointwise a.e. on $X$ to $f$.  Use this to show that for two sequences that converge in measure, the product sequence also converges in measure to the product of the limits. 

\item Show that if $f$ is an lower semicontinuous (resp. upper semicontinuous) function on an interval $[a,b]$, then there is a family $\{f_{\alpha}\}$ of continuous functions on the interval $[a,b]$ such that $f(x) = \sup\{f_{\alpha}(x) \mid \alpha \in A \}$ (resp. $f(x) = \inf\{f_{\alpha}(x) \mid \alpha \in A \}$) for all $x \in [a,b]$.  


\bigskip
\textbf{Collaborators:}\\
\smallskip
 
\textbf{Solution:}
\bigskip






\end{enumerate}
\end{document}
