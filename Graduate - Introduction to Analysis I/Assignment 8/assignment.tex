%!TEX program = xelate
%%%%%%%%%%%%%%%%%%%%%%%%%%%%%%%%%%%%%%%%%
% Modified By Orcuslc, 2016-9-21
% Modified for Assignments
% http://github.com/orcuslc
%
% Wilson Resume/CV
% Structure Specification File
% Version 1.0 (22/1/2015)
%
% This file has been downloaded from:
% http://www.LaTeXTemplates.com
%
% License:
% CC BY-NC-SA 3.0 (http://creativecommons.org/licenses/by-nc-sa/3.0/)
%
%%%%%%%%%%%%%%%%%%%%%%%%%%%%%%%%%%%%%%%%%

%----------------------------------------------------------------------------------------
%	PACKAGES AND OTHER DOCUMENT CONFIGURATIONS
%----------------------------------------------------------------------------------------
\documentclass[10pt]{article}

\usepackage{listings}
\usepackage{xcolor}
\usepackage{amsmath,amsthm,amssymb}
\usepackage{epstopdf}
\usepackage{graphicx}
\usepackage{clrscode3e}

\DeclareGraphicsExtensions{.eps,.ps,.jpg,.bmp}


\usepackage[a4paper, hmargin=25mm, vmargin=30mm, top=20mm]{geometry} % Use A4 paper and set margins

\usepackage{fancyhdr} % Customize the header and footer

\usepackage{lastpage} % Required for calculating the number of pages in the document

\usepackage{hyperref} % Colors for links, text and headings

\setcounter{secnumdepth}{0} % Suppress section numbering

%\usepackage[proportional,scaled=1.064]{erewhon} % Use the Erewhon font
%\usepackage[erewhon,vvarbb,bigdelims]{newtxmath} % Use the Erewhon font
\usepackage[utf8]{inputenc} % Required for inputting international characters
\usepackage[T1]{fontenc} % Output font encoding for international characters

\usepackage{fontspec} % Required for specification of custom fonts
\setmainfont[Path = ./fonts/,
Extension = .otf,
BoldFont = Erewhon-Bold,
ItalicFont = Erewhon-Italic,
BoldItalicFont = Erewhon-BoldItalic,
SmallCapsFeatures = {Letters = SmallCaps}
]{Erewhon-Regular}

\usepackage{color} % Required for custom colors
\definecolor{slateblue}{rgb}{0.17,0.22,0.34}

\usepackage{sectsty} % Allows customization of titles
\sectionfont{\color{slateblue}} % Color section titles

\fancypagestyle{plain}{\fancyhf{}\cfoot{\thepage\ of \pageref{LastPage}}} % Define a custom page style
\pagestyle{plain} % Use the custom page style through the document
\renewcommand{\headrulewidth}{0pt} % Disable the default header rule
\renewcommand{\footrulewidth}{0pt} % Disable the default footer rule

\setlength\parindent{0pt} % Stop paragraph indentation

% Non-indenting itemize
\newenvironment{itemize-noindent}
{\setlength{\leftmargini}{0em}\begin{itemize}}
{\end{itemize}}

% Text width for tabbing environments
\newlength{\smallertextwidth}
\setlength{\smallertextwidth}{\textwidth}
\addtolength{\smallertextwidth}{-2cm}

\newcommand{\sqbullet}{~\vrule height .8ex width .6ex depth -.05ex} % Custom square bullet point 


\newcommand{\tbf}[1]{\textbf{#1}}
\newcommand{\tit}[1]{\textit{#1}}
\newcommand{\mbb}[1]{\mathbb{#1}}
\newcommand{\blue}[1]{\color{blue}{#1}}
\newcommand{\red}[1]{\color{red}{#1}}
\newcommand{\sblue}[1]{\color{slateblue}{#1}}
\newcommand{\n}{\\[5pt]}
\newcommand{\tr}{^\top}
\newcommand{\vt}[1]{
\Vert #1 \Vert
}
\newcommand{\bra}[5]{
#1=\left\{
\begin{aligned}
#2 ,&\quad #4 \\
#3 ,&\quad #5
\end{aligned}
\right.
}

\renewcommand{\title}[2] {
{\Huge{\color{slateblue}\textbf{#1}}}
\hfill
\LARGE{\color{slateblue}\textbf{#2}} \\[10pt]
\large{\color{slateblue}\textbf{Chuan Lu, 13300180056, chuanlu13@fudan.edu.cn}} \\[1mm]
\rule{\textwidth}{0.5mm}
}

\newcommand{\problem}[2] {
\vspace{20pt}
\LARGE{\color{slateblue}\textbf{Problem #1.}}
\vspace{2mm}
#2 \\[10pt]
}

\renewcommand{\proof}[2] {
\large{\color{slateblue}\textit{\textbf{#1.}}}
#2 \qed \\[3mm]
}

\newcommand{\solution}[2] {
\large{\color{slateblue}\textit{\textbf{#1.}}}
#2 \\[3mm]
}


\newcommand{\algorithm}[2] {
\begin{codebox}
\Procname{$\proc{Algorithm #1}$}
#2
\end{codebox}
}

\newcommand{\refgroup}[1] {
\LARGE{\color{slateblue}\textbf{Reference}} 
\begin{tabbing}
\hspace{5mm} \= \kill
#1
\end{tabbing}
}

\newcommand{\reference}[1] {
\sqbullet \ \  \large{#1} \\
}
% \newcommand{\solution}[2] {
% \LARGE{\color{slateblue}\textit{#1}}
% \ #2 \qed
% }

% \newenvironment{problem}[2][Problem]{\begin{trivlist}
% \item[\hskip \labelsep {\bfseries #1}\hskip \labelsep {\bfseries #2.}]}{\end{trivlist}}
\usepackage{epstopdf}
\usepackage{graphics}
\usepackage{subfig}
\usepackage{listings}
\lstset{
  breaklines=true,
  xleftmargin=25pt,
  xrightmargin=25pt,
  aboveskip=0pt,
  belowskip=10pt,
  basicstyle=\ttfamily,
  showstringspaces=false,
  frame=ltrb,
  tabsize=4,
  numbers=left,
  numberstyle=\small,
  numbersep=8pt,
  morekeywords={*, factorial, sum, erlang},
  keywordstyle=\color{blue!70}, commentstyle=\color{red!50!green!50!blue!50},
}
\DeclareGraphicsExtensions{.eps,.ps,.jpg,.bmp}

\begin{document}

\title{Introduction to Analysis \\ Assignment 8}
\date{\today}
\author{Chuan Lu}

\maketitle

\problem{1}{Problem 37, Page 123}
\solution{Sol}{
(i) Let
$$
f(x) = \left\{
\begin{aligned}
&x\sin(\frac{1}{x}), ~0 < x \le 1, \\
&0, ~x = 0
\end{aligned}
\right.
$$
Then $f$ is continuous on $(0, 1]$. Since $\lim\limits_{x\to 0^+}f(x) = 0 = f(0)$, $f$ is continuous on $[0, 1]$. For $x_1, x_2 \in [\epsilon, 1]$ where $\epsilon > 0$, 
$$
\begin{aligned}
|f(x_1)-f(x_2)| &= |x_1\sin\frac{1}{x_1}-x_2\sin\frac{1}{x_2}| = |(x_1-x_2)\sin\frac{1}{x_1}+x_2(\sin\frac{1}{x_1}-\sin\frac{1}{x_2})| \\
&\le |x_1-x_2|\sin\frac{1}{x_1} + 2x_2|\cos\frac{1}{2}(\frac{1}{x_1}+\frac{1}{x_2})\sin\frac{1}{2}(\frac{1}{x_1}-\frac{1}{x_2})| \\
&\le |x_1-x_2|\sin\frac{1}{x_1} + 2x_2|\sin\frac{x_2-x_1}{2x_1x_2}| \le |x_1-x_2|\sin\frac{1}{x_1}+2x_2|\frac{x_2-x_1}{2x_1x_2}| \\
&= |x_1-x_2|(\sin\frac{1}{x_1}+\frac{1}{x_1}) \le |x_1-x_2|(1+\frac{1}{\epsilon}).
\end{aligned}
$$ 
Thus $f$ is Lipschitz, with Proposition 7 we know $f$ is absolutely continuous. 

However, with Problem 35 we know $f$ is not of bounded variation on $[0, 1]$, and with Remark on Page 122 we know $f$ is not absolutely continuous on $[0, 1]$.

(ii) If not, then there is a $\epsilon > 0$, s.t. for each $\delta > 0$, there is a finite disjoint collection $\{(a_k, b_k)\}$ of open intervals satisfying $\sum\limits_{k=1}^{n}(b_k-a_k) < \delta $, s.t. $\sum\limits_{k=1}^{n}|f(b_k)-f(a_k)| \ge \epsilon $. As suggested in conditions, for each $c > 0$, $f$ is absolutely continuous on $[c, 1]$. Then these open intervals must lie in $[0, c]$ for every $c$. With the continuity of $f$ on $[0, 1]$, there exists $c > 0$, s.t. $0 < f(c) - f(0) <\epsilon$. If we take $\delta = c$, because $f$ is increasing, $\sum\limits_{k=1}^{n}(b_k-a_k) < f(c) - f(0) <\epsilon$. It contradicts with our assumption. Hence $f$ is absolutely continuous on $[0, 1]$.

(iii) First we show $f$ is absolutely continuous by showing that it satisfies the condition in (ii). For each $c > 0$, since on $[c, 1]$ we have 
$$ 
\sqrt{x_1}-\sqrt{x_2}| = \frac{|x_1-x_2|}{\sqrt{x_1}+\sqrt{x_2}} \le \frac{|x_1-x_2|}{2\sqrt{c}},
$$
we know for each $\epsilon > 0$, pick $\delta = 2\sqrt{c}\epsilon$, then for each collection $\{(a_k, b_k)\}$ satisfying $\sum\limits_{k=0}^{n}|b_k-a_k| < \delta $,
$$
\sum_{k=1}^{n}|f(b_k)-f(a_k)| < \sum_{k=1}^{n}\frac{|x_1-x_2|}{2\sqrt{c}} < \epsilon.
$$
Hence $f$ is absolutely continuous on $[c, 1]$. Since $f$ is increasing, we know $f$ is absolutely continuous on $[0, 1]$.

On the other hand, if there exists $\lambda > 0$, s.t. 
$$
|f(x_1)-f(x_2)| < \lambda|x_1-x_2|
$$
for each $x_1, x_2 $, then if we pick $\max(x_1, x_2) < \frac{1}{4\lambda^2}$, from the argument above we know
$$
|f(x_1)-f(x_2)| = \frac{|x_1-x_2|}{\sqrt{x_1}+\sqrt{x_2}} > \lambda|x_1-x_2|.
$$
Hence $f$ is not Lipschitz.
}

\problem{2}{Problem 39, Page 123}
\solution{Sol}{
Suppose $E$ is a measurable set, then for each $\epsilon' > 0$, there exists an open set $O\supset E$, and $m(O\setminus E) < \epsilon'$. Let $O = \bigcup\limits_{k=1}^{\infty} (a_k, b_k)$ be the open decomposition of $O$, then $(a_i, b_i)$ are pairwise disjoint.

If $f$ is absolutely continuous, by Problem 38, for each $\epsilon > 0$, there is a $\delta > 0$, for each $\{(a_k, b_k)\}$ satisfying $\sum\limits_{k=1}^{\infty}(b_k-a_k) < \delta$, $\sum\limits_{k=1}^{\infty}|f(b_k)-f(a_k)| < \epsilon $. In fact, if we take $\delta_1 < \delta - \epsilon'$, then since $f$ is increasing, $f(E) \in f(\cup\{(a_k, b_k)\})$ when $E \in \cup\{(a_k, b_k)\}$, thus 
$$
m^*(f(E)) < m(f(\cup\{(a_k, b_k)\})) < \delta.
$$

On the other hand, since open sets are measurable, with Problem 39 we know the reverse holds.
}

\problem{3}{Problem 41, Page 123}
\solution{Sol}{
(i) With the continuity of $f$ and the compactness of $[a, b]$, we know the maximum and minimum of $f$ on $[a, b]$ exists, thus $f$ maps $[a, b]$ to a closed set. Thus $f$ maps a $F_\sigma $ set to a $F_\sigma$ set.

(ii) Problem 40 tells $f$ maps a set of measure zero to a set of measure zero. Since each measurable set could be represented as a union of a measure-zero set and a $F_\sigma $ set, we know $f$ maps a measurable set to a measurable set.
}

\problem{4}{Problem 49, Page 128}

\problem{5}{Problem 56, Page 129}

\problem{6}{Problem 59, Page 129}

\end{document}