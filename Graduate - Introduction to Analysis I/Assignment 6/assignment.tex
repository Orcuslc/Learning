%!TEX program = xelate
%%%%%%%%%%%%%%%%%%%%%%%%%%%%%%%%%%%%%%%%%
% Modified By Orcuslc, 2016-9-21
% Modified for Assignments
% http://github.com/orcuslc
%
% Wilson Resume/CV
% Structure Specification File
% Version 1.0 (22/1/2015)
%
% This file has been downloaded from:
% http://www.LaTeXTemplates.com
%
% License:
% CC BY-NC-SA 3.0 (http://creativecommons.org/licenses/by-nc-sa/3.0/)
%
%%%%%%%%%%%%%%%%%%%%%%%%%%%%%%%%%%%%%%%%%

%----------------------------------------------------------------------------------------
%	PACKAGES AND OTHER DOCUMENT CONFIGURATIONS
%----------------------------------------------------------------------------------------
\documentclass[10pt]{article}

\usepackage{listings}
\usepackage{xcolor}
\usepackage{amsmath,amsthm,amssymb}
\usepackage{epstopdf}
\usepackage{graphicx}
\usepackage{clrscode3e}

\DeclareGraphicsExtensions{.eps,.ps,.jpg,.bmp}


\usepackage[a4paper, hmargin=25mm, vmargin=30mm, top=20mm]{geometry} % Use A4 paper and set margins

\usepackage{fancyhdr} % Customize the header and footer

\usepackage{lastpage} % Required for calculating the number of pages in the document

\usepackage{hyperref} % Colors for links, text and headings

\setcounter{secnumdepth}{0} % Suppress section numbering

%\usepackage[proportional,scaled=1.064]{erewhon} % Use the Erewhon font
%\usepackage[erewhon,vvarbb,bigdelims]{newtxmath} % Use the Erewhon font
\usepackage[utf8]{inputenc} % Required for inputting international characters
\usepackage[T1]{fontenc} % Output font encoding for international characters

\usepackage{fontspec} % Required for specification of custom fonts
\setmainfont[Path = ./fonts/,
Extension = .otf,
BoldFont = Erewhon-Bold,
ItalicFont = Erewhon-Italic,
BoldItalicFont = Erewhon-BoldItalic,
SmallCapsFeatures = {Letters = SmallCaps}
]{Erewhon-Regular}

\usepackage{color} % Required for custom colors
\definecolor{slateblue}{rgb}{0.17,0.22,0.34}

\usepackage{sectsty} % Allows customization of titles
\sectionfont{\color{slateblue}} % Color section titles

\fancypagestyle{plain}{\fancyhf{}\cfoot{\thepage\ of \pageref{LastPage}}} % Define a custom page style
\pagestyle{plain} % Use the custom page style through the document
\renewcommand{\headrulewidth}{0pt} % Disable the default header rule
\renewcommand{\footrulewidth}{0pt} % Disable the default footer rule

\setlength\parindent{0pt} % Stop paragraph indentation

% Non-indenting itemize
\newenvironment{itemize-noindent}
{\setlength{\leftmargini}{0em}\begin{itemize}}
{\end{itemize}}

% Text width for tabbing environments
\newlength{\smallertextwidth}
\setlength{\smallertextwidth}{\textwidth}
\addtolength{\smallertextwidth}{-2cm}

\newcommand{\sqbullet}{~\vrule height .8ex width .6ex depth -.05ex} % Custom square bullet point 


\newcommand{\tbf}[1]{\textbf{#1}}
\newcommand{\tit}[1]{\textit{#1}}
\newcommand{\mbb}[1]{\mathbb{#1}}
\newcommand{\blue}[1]{\color{blue}{#1}}
\newcommand{\red}[1]{\color{red}{#1}}
\newcommand{\sblue}[1]{\color{slateblue}{#1}}
\newcommand{\n}{\\[5pt]}
\newcommand{\tr}{^\top}
\newcommand{\vt}[1]{
\Vert #1 \Vert
}
\newcommand{\bra}[5]{
#1=\left\{
\begin{aligned}
#2 ,&\quad #4 \\
#3 ,&\quad #5
\end{aligned}
\right.
}

\renewcommand{\title}[2] {
{\Huge{\color{slateblue}\textbf{#1}}}
\hfill
\LARGE{\color{slateblue}\textbf{#2}} \\[10pt]
\large{\color{slateblue}\textbf{Chuan Lu, 13300180056, chuanlu13@fudan.edu.cn}} \\[1mm]
\rule{\textwidth}{0.5mm}
}

\newcommand{\problem}[2] {
\vspace{20pt}
\LARGE{\color{slateblue}\textbf{Problem #1.}}
\vspace{2mm}
#2 \\[10pt]
}

\renewcommand{\proof}[2] {
\large{\color{slateblue}\textit{\textbf{#1.}}}
#2 \qed \\[3mm]
}

\newcommand{\solution}[2] {
\large{\color{slateblue}\textit{\textbf{#1.}}}
#2 \\[3mm]
}


\newcommand{\algorithm}[2] {
\begin{codebox}
\Procname{$\proc{Algorithm #1}$}
#2
\end{codebox}
}

\newcommand{\refgroup}[1] {
\LARGE{\color{slateblue}\textbf{Reference}} 
\begin{tabbing}
\hspace{5mm} \= \kill
#1
\end{tabbing}
}

\newcommand{\reference}[1] {
\sqbullet \ \  \large{#1} \\
}
% \newcommand{\solution}[2] {
% \LARGE{\color{slateblue}\textit{#1}}
% \ #2 \qed
% }

% \newenvironment{problem}[2][Problem]{\begin{trivlist}
% \item[\hskip \labelsep {\bfseries #1}\hskip \labelsep {\bfseries #2.}]}{\end{trivlist}}
\usepackage{epstopdf}
\usepackage{graphics}
\usepackage{subfig}
\usepackage{listings}
\lstset{
  breaklines=true,
  xleftmargin=25pt,
  xrightmargin=25pt,
  aboveskip=0pt,
  belowskip=10pt,
  basicstyle=\ttfamily,
  showstringspaces=false,
  frame=ltrb,
  tabsize=4,
  numbers=left,
  numberstyle=\small,
  numbersep=8pt,
  morekeywords={*, factorial, sum, erlang},
  keywordstyle=\color{blue!70}, commentstyle=\color{red!50!green!50!blue!50},
}
\DeclareGraphicsExtensions{.eps,.ps,.jpg,.bmp}

\begin{document}

\title{Introduction to Analysis \\ Assignment 6}
\date{\today}
\author{Chuan Lu}

\maketitle

\problem{1}{Problem 29, Section 4.4, Page 89}
\solution{Sol}{
Both are not true. We can construct a $f$ like this:
$$
f(x) = \left\{
\begin{aligned}
1+\frac{1}{n^2}, ~n \le x < n+\frac{1}{2}, ~\forall n\in\mathbb{N} \\
-1, ~n+\frac{1}{2} \le x < n+1, ~\forall n\in\mathbb{N}
\end{aligned}
\right.
$$
Then $f$ is measurable, and $f$ is bounded on any bounded set, and
$$
a_n = \int_{n}^{n+1}f = \frac{1}{2n^2}.
$$
Clearly the series $\sum\limits_{n=1}^{\infty}a_n = \sum\limits_{n=1}^{\infty}\frac{1}{2n^2} $ converges absolutely, but 
$$
\int_{1}^{\infty}|f| = \sum_{n=1}^{\infty}(1+\frac{1}{2n^2}) = \infty,
$$
which means $f$ is not integrable on $[1, \infty)$.
}

\problem{2}{Problem 33, Section 4.4, Page 90}
\solution{Proof}{
First,
$$
|f_n-f| \le |f| + |f_n|, ~\forall n.
$$
Then since $f$ is integrable on $E$, if $\lim\limits_{n\to\infty}\int_E |f_n| = \int_{E}|f| $, we know $|f_n|+|f|$ converges pointwise a.e. to $2|f|$, and
$$
\lim_{n\to\infty}\int_E(|f_n|+|f|) = 2\int_E |f| < \infty,
$$
with General Lebesgue Dominated Convergence Theorem, notice $|f_n-f| $ converges pointwise a.e. to 0,
$$
\lim_{n\to\infty}\int_E|f_n-f| = \int_E 0 = 0.
$$
On the other hand, notice
$$
|f_n|-|f| \le |f_n-f|, ~\forall n.
$$
with the same method, since $\int_E|f-f_n|\to 0 $, and $|f-f_n| $ converges pointwise to 0,
$$
\lim_{n\to\infty}\int_E |f_n-f| = \int_E 0 = 0,
$$
we know from $|f_n|-|f| $ converges pointwise a.e. to 0,
$$
\lim_{n\to\infty}\int_E |f_n|-|f| = \int_E 0 = 0.
$$
Hence
$$
\lim_{n\to\infty}\int_E|f_n| = \int_E |f|.
$$
}

\problem{3}{Problem 35, Section 4.4, Page 90}
\solution{Proof}{
Denote $f_n(x) = f(x, a_n) $, in which $\{a_n\}$ is any series which converges to 0. Then from the condition we know $f_n(x) $ converges pointwise to $f(x)$, and $|f_n(x)| \le g(x)$. Then using Lebesgue Dominated Convergence Theorem, since $g$ is integrable on $[0, 1]$, we have
$$
\lim_{n\to\infty} \int_{0}^{1}f_n(x)dx = \int_{0}^{1} f(x)dx.
$$
It shows that 
$$
\limsup_{y\to 0}\int_{0}^{1}f(x, y)dx = \liminf_{y\to 0}\int_{0}^{1}f(x, y)dx = \int_{0}^{1}f(x)dx, 
$$
whic means 
$$
\lim_{y\to 0}\int_{0}^{1}f(x, y) dx = \int_{0}^{1}f(x)dx.
$$
For the continuity of $h$, we need to show that $\forall y_0 \in [0, 1]$, $\forall \epsilon > 0$, $\exists \delta > 0$, when $|y-y_0| <\delta $, we have $h(y)-h(y_0) = |\int_{0}^{1}f(x, y)dx-\int_{0}^{1}f(x, y_0)|dx < \epsilon$. Since $f(x, y)$ is continuous in $y$ for each $x$, then for each fixed $x$, $\exists \delta_1$, when $|y-y_0| <\delta_1 $, $|f(x, y)-f(x, y_0)| < \epsilon$. Then 
$$
|h(y)-h(y_0)| = \left|\int_{0}^{1} f(x, y)dx - \int_{0}^{1} f(x, y_0)dx\right| \le \int_{0}^{1} |f(x, y)-f(x, y_0)|dx < \epsilon.
$$
We know the continuity of $h$ since we can pick $\delta = \delta_1 $.
}

\problem{4}{Problem 36, Section 4.4, Page 90}
\solution{Proof}{
For any fixed $y\in [0, 1]$, suppose $\{h_n\}$ is a sequence with $h_n\to 0 $. Let
$$
f_n(x) = \frac{f(x, y+h_n)-f(x, y)}{h_n}
$$
Since $\partial f/\partial y$ exists, 
$$
\lim_{n\to\infty}f_n(x) = \lim_{h\to 0}\frac{f(x, y+h)-f(x, y)}{h} = \frac{\partial f}{\partial y}(x, y).
$$
It means $f_n(x)$ converges pointwise to $\frac{\partial f}{\partial y}(x, y)$.
Thus
$$
\exists N > 0, ~\forall n > N, \left|f_n(x) - \frac{\partial f}{\partial y}(x, y)\right| < 1.
$$
Since 
$$
\left|\frac{\partial f}{\partial y}(x, y)\right| \le g(x)
$$
we have
$$
|f_n(x)| \le g(x)+1,
$$
and $g(x)+1$ is integrable on $[0, 1]$. By Lebesgue Dominated Convergence Theorem,
$$
\int_{0}^{1}f_n(x)dx \to \int_{0}^{1}\frac{\partial f}{\partial y}(x, y)dx.
$$
Since $\{h_n\}$ is arbitrary, and $f_n$ is integrable, we know
$$
\limsup_{n\to\infty}\int_{0}^{1}f_n(x)dx = \liminf_{n\to\infty}\int_{0}^{1}f_n(x) dx = \lim_{h\to 0}\frac{1}{h}\left(\int_0^1 f(x, y+h)dx-\int_{0}^{1}f(x, y)dx\right) = \frac{d}{dy}\int_{0}^{1}f(x, y)dx.
$$
}

\problem{5}{Problem 38, Section 4.5, Page 91}
\solution{(i)}{
$$
\lim_{n\to\infty}\int_{0}^{n} fdx = \lim_{n\to\infty}\sum_{m = 1}^{n}\frac{(-1)^m}{m} = -\ln2,
$$
But
$$
\int_{1}^{\infty}f^+ = \sum_{m=1}^{\infty}\frac{1}{2m} = \infty,
$$
So $f$ is not integrable.
}
\solution{(ii)}{
$$
\lim_{n\to\infty}\int_{1}^{n}f = \int_{1}^{\infty}\frac{\sin x}{x}dx,
$$
with Dirichlet's Criterion, we know this integral converges. But
$$
\int_{1}^{\infty}|f| \ge \int_{1}^{\infty}\frac{1}{2x}dx - \int_{1}^{\infty}\frac{\cos 2x}{x}dx,
$$
and the second term converges with Dirichlet's Criterion, but the first term $\to\infty$, we know this integral diverges to $\infty$. Thus $f$ is not integrable.

This two counterexamples do not contradict to the continuity: $f$ is not integrable over the whole set $E = [1, \infty)$.
}

\problem{6}{Problem 39, Section 4.5, Page 91}
\solution{Proof (i)}{
Denote 
$$
F_1 = E_1, ~F_n = E_n \setminus \bigcup_{m=1}^{n-1}E_m, ~n \ge 2.
$$
Then $\{F_i\}$ is a sequence of disjoint measurable subsets of $E$. Then using Theorem 20,
$$
\int_{\cup_{n=1}^{\infty}E_n} f = \sum_{n=1}^{\infty}\int_{F_n}f = \lim_{n\to\infty}\sum_{m=1}^{n}\int_{F_m}f = \lim_{n\to\infty}\int_{E_n}f.
$$
}
\solution{Proof of (ii)}{
Using the same method with (i), only changing $F$ to 
$$
F_1 = E_1, ~F_n = E_1 \setminus\bigcup_{m=1}^{n-1}E_m, ~n\ge 2.
$$
The other parts of proof is just the same.
}

\problem{7}{Problem 44, Section 4.6, Page 95}
\solution{(i)}{
First, when $f$ is nonnegative, from the definition of integrable functions we know for any $\epsilon > 0$, there is a bounded measurable function with finite support $0 \le h(x) \le f(x)$, s.t. 
$$
\int_{\mathbb{R}} (f-h) = \int_{\mathbb{R}}f - \int_{\mathbb{R}}h \le \frac{1}{2}\epsilon.
$$
Let the support set of $h$ be $M$ with $m(M) < \infty$. Then using Simple Approximation Theorem, there is a simple function $\eta$ on M, s.t. $0 \le h - \eta \le \frac{\epsilon}{2m(M)}$. Let $\eta = 0$ on $\mathbb{R}\setminus M$, then $\eta$ has finite support, and 
$$
\int_{\mathbb{R}}|f-\eta| = \int_{\mathbb{R}}(f-h+h-\eta) = \int_{\mathbb{R}}f-h + \int_{M} h-\eta \le \frac{1}{2}\epsilon + m(M)\frac{\epsilon}{2m(M)} = \epsilon.
$$
When $f$ is an arbitrary integrable function, $f^+, ~f^- $ are nonnegative functions. Let $E_+ = \{x\mid f^+ > 0\}, ~E_- = \{x\mid f^- > 0\} $, then there exists nonnegative simple functions $\eta^+, ~\eta^- $, s.t. $\eta^+ = 0 $ on $\mathbb{R}\setminus E_+ $, and $\eta^+ $ has finite support on $E_+ $, $\eta^- = 0 $ on $\mathbb{R}\setminus E_- $, and $\eta^- $ has finite support on $E_- $, and they satisfies 
$$
\int_{\mathbb{R}} |f^+-\eta^+| < \epsilon, ~\int_{\mathbb{R}} |f^--\eta^-| < \epsilon.
$$
Let
$$
\eta = \left\{
\begin{aligned}
\eta^+, ~x\in E_+ \\ 
\eta^-, ~x\in E_-
\end{aligned},
\right.
$$
then $\eta = \eta^+-\eta^- $ is a simple function with finite support, and
$$
\int_{\mathbb{R}}|f-\eta| = \int_{\mathbb{R}} |f^+-f^--(\eta^+-\eta^-)| \le \int_{\mathbb{R}}|f^+-\eta^+| + \int_{\mathbb{R}}|f^--\eta^-| < 2\epsilon.
$$
}
\solution{(ii)}{
From (i) we know there is a simple function $\eta$ which has finite support (denoted as $E$) and $\int_{\mathbb{R}}|f-\eta| < \epsilon $. Since $E$ is a measurable set of finite measure, with Lemma 22, for any $\delta_1 > 0$, there is a $n > 0$,
$$
m(E\cap(\mathbb{R}\setminus [-n, n])) < \delta_1.
$$
Let $I = [-n, n]$, using the result of Problem 3.18, for any $\delta_2 > 0$, there is a step function $s$ on $I$, and a close set $F\subset I$, s.t. $|\eta - s| < \delta_2 $ on $F$, and $m(I\setminus F) < \delta_2 $. Set $s(x) = 0$ for $x$ outside $I$.

With Proposition 23, for each $\epsilon > 0$, $\exists \delta > 0 $, if $m(A) < \delta$, then $\int_{A}|f| < \epsilon $. Let $\delta_1 = \delta$, $\delta_2 = \min(\delta, \frac{\epsilon}{2n}) $, then
$$
\begin{aligned}
\int_{\mathbb{R}}|f-s| &\le \int_{\mathbb{R}}|f-\eta|+\int_{\mathbb{R}}|\eta-s| < \epsilon + \int_{F} |\eta-s|+ \int_{I\setminus F}|\eta-s| + \int_{\mathbb{R}\setminus I}|\eta-s| \\
&< \epsilon + 2n\frac{\epsilon}{2n} + \epsilon + \epsilon = 4\epsilon.
\end{aligned} 
$$
}
\solution{(iii)}{
Use Lusin's Theorem, the proof is the same with (ii), and we only need to change the step function to continuous function.
}

\problem{8}{Problem 25, Section 18.2, Page}
\solution{Solution}{
With $\eta$ being the counting measure, suppose nonnegative $f$ on $\mathbb{N}$, and let nonnegative $f_n(x)$ be
$$
f_n(k) = \left\{
\begin{aligned}
&f(k), ~0 \le k \le n \\
&0, ~k > n
\end{aligned}
\right.
$$
then $f_n \to f$ pointwise on $\mathbb{N}$, and $\{f_n\}$ is increasing. So by Monotone Convergence Thm, 
$$
\lim_{n\to\infty}\int_{\mathbb{N}}f_n = \int_{\mathbb{N}}f.
$$
Since
$$
\int_{\mathbb{N}}f_n = \sum_{i = 1}^{n}\int_{\{i\}}f_n + \int_{\{i \ge n\}} f_n = \sum_{i=1}^{n}\int_{\{i\}}f_n = \sum_{i=1}^{n}f_n(i),
$$
we have
$$
\int_{\mathbb{N}}f = \sum_{n = 0}^{\infty}f(n) < \infty.
$$
}

\problem{9}{Problem 26, Section 18.2, Page}
\solution{Solution}{
With the definition of Dirac measure, let $g\equiv f(x_0)$, then
$$
m(\{g = f\}) = m_{\delta_{x_0}}(\{x\in X\mid f(x) = g(x)\}) = 1 = m(X).
$$
Then $f = g$, a.e. on $X$.
$$
\int_{X}f(x)d\delta_{x_0} =\int_{X}gd\delta_{x_0} = f(x_0)\int_{X}1d\delta_{x_0} = f(x_0) < \infty.
$$
}

\problem{10}{Problem 27, Section 18.3, Page}
\solution{(i)}{
First we show that if $f$ is integrable on $X$, then the set
$$
F = \{x\mid f(x) \ne 0\}
$$
is countable. Otherwise, since $f$ is integrable, it means $f^+ $ and $f^- $ are both integrable. We now assume $f$ is nonnegative, then
$$
F = \bigcup_{n = 1}^{\infty}\{x\mid f(x) \ge \frac{1}{n}\}.
$$
If $F$ is uncountable, then there must exist $n$, s.t. $F_n = \{x\mid f(x) \ge \frac{1}{n}\}$ is uncountable, and $m(F_n) = \infty$. Thus 
$$
\int_{X}fd\eta \ge \int_{F_n}fd\eta \ge \frac{1}{n}\times\infty = \infty,
$$
which makes a contradiction. In this case, suppose there is a bijection from $F$ to a subset of $\mathbb{N}$, then this problem is the same with Problem 8.
}
\solution{(ii)}{
Since $x_0\in X$, and $\mathcal{M}$ is the $\sigma-$algebra of all subsets of $X$, then $\{x_0\}$ is measurable. For any simple function $h(x) \le |f(x)|$, suppose
$$
h(x) = \sum_{i=1}^{n}c_i1_{E_i},
$$
and $x_0 \in c_1$, then 
$$
\int_{X}h(x)d\mu_{x_0} = \sum_{i=1}^{n}c_im(E_i) = c_1 \le |f(x_0)|.
$$
Then according to the definition of integration, 
$$
\int_{X}|f(x)| \le |f(x_0)|.
$$
On the other hand, let
$$
h(x) = \left\{
\begin{aligned}
&|f(x_0)|, ~x = x_0 \\
&0, ~x\ne x_0
\end{aligned}
\right.
$$
then 
$$
\int_{X}h(x) = \int_{\{x_0\}}h(x_0) = |f(x_0)|.
$$
Hence
$$
\int_{X}|f(x)|d\mu_{x_0} = |f(x_0)|.
$$
When $f$ is arbitrary real-valued function, define $f^+ $ and $f^- $ as in the textbook, then $f = f^+-f^- $, and
$$
\int_{X}f = \int_{X}f^+ -\int_{X}f^-.
$$
}
\end{document}