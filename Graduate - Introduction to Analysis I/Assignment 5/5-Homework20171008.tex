
\documentclass{article}%
\usepackage{amsmath}
\usepackage{graphicx}
\usepackage{amsfonts}%
\usepackage{amssymb}


\setlength{\topmargin}{-0.75in}
\setlength{\textheight}{9.25in}
\setlength{\oddsidemargin}{0.0in}
\setlength{\evensidemargin}{0.0in}
\setlength{\textwidth}{6.5in}
\def\labelenumi{\arabic{enumi}.}
\def\theenumi{\arabic{enumi}}
\def\labelenumii{(\alph{enumii})}
\def\theenumii{\alph{enumii}}
\def\p@enumii{\theenumi.}
\def\labelenumiii{\arabic{enumiii}.}
\def\theenumiii{\arabic{enumiii}}
\def\p@enumiii{(\theenumi)(\theenumii)}
\def\labelenumiv{\arabic{enumiv}.}
\def\theenumiv{\arabic{enumiv}}
\def\p@enumiv{\p@enumiii.\theenumiii}
\pagestyle{plain}
\setcounter{secnumdepth}{0}
\newtheorem{theorem}{Theorem}
\newtheorem{acknowledgement}[theorem]{Acknowledgement}
\newtheorem{algorithm}[theorem]{Algorithm}
\newtheorem{axiom}[theorem]{Axiom}
\newtheorem{case}[theorem]{Case}
\newtheorem{claim}[theorem]{Claim}
\newtheorem{conclusion}[theorem]{Conclusion}
\newtheorem{condition}[theorem]{Condition}
\newtheorem{conjecture}[theorem]{Conjecture}
\newtheorem{corollary}[theorem]{Corollary}
\newtheorem{criterion}[theorem]{Criterion}
\newtheorem{definition}[theorem]{Definition}
\newtheorem{example}[theorem]{Example}
\newtheorem{exercise}[theorem]{Exercise}
\newtheorem{lemma}[theorem]{Lemma}
\newtheorem{notation}[theorem]{Notation}
\newtheorem{problem}[theorem]{Problem}
\newtheorem{proposition}[theorem]{Proposition}
\newtheorem{remark}[theorem]{Remark}
\newtheorem{solution}[theorem]{Solution}
\newtheorem{summary}[theorem]{Summary}
\newenvironment{proof}[1][Proof]{\textbf{#1.} }{\ \rule{0.5em}{0.5em}}

\begin{document}

\begin{center}
\textbf{Introduction to Analysis $I$\\Homework 5\\Sunday, October 8, 2017}\bigskip
\end{center}

\noindent\textbf{Instructions}: This and all subsequent homeworks must be submitted written in \LaTeX.

\noindent If you use results from books, Royden or others, please be explicit about what results you are using.



\begin{center}
\emph{Homework 5 is due by midnight, Saturday, October 21.}
\end{center} 
\medskip

\begin{enumerate}
\item  (Problem 24, Page 64) Let $I$ be an interval in $\mathbb{R}$ and let $f:I\to \mathbb{R}$ be increasing. Show that $f$ is measurable by first showing that, for each natural number $n$, the strictly increasing function $x\to f(x)+x/n$ is measurable, and then taking pointwise limits.


\bigskip
\textbf{Collaborators:} None
\smallskip
 
\textbf{Solution:}
Denote $f_{n}(x) = f(x)+\frac{x}{n}$. Then since $f: I \to \mathbb{R}$ is increasing, $f_n(x)$ is strictly increasing on $I \in \{[a, b], [a, b), (a, b], (a, b)\}$.

For each fixed number $c$, if $\exists x_0\in I$, s.t. $f(x_0) = c$, then the set
$$
\{x\mid f(x) < c\} = (a, x_0),
$$
and the left side is the same as the set $I$, and it is an interval in $\mathbb{R}$, hence is measurable. Thus $f_{n}$ is measurable for every $n$. 

For each $x\in I$, 
$$
\lim_{n\to\infty}f_n(x) = f(x)+\lim_{n\to\infty}\frac{x}{n} = f(x),
$$
hence $f_{n}$ converges to $f$ pointwise. Using Proposition 9, we know $f$ is measurable.
\bigskip

\item (Problem 8, Page 343)  Let $(X,\mathcal{M}, \mu)$ be a measure space.  The measure $\mu$ is said to be \textbf{semifinite} provided each measurable set of infinite measure contains measurable sets of arbitrarily large finite measure.
\begin{enumerate}
\item Show that each $\sigma$-finite measure is semifinite.
\item For $E\in \mathcal{M}$, define $\mu_1(E) = \mu(E)$, if $\mu(E)<\infty$, and if $\mu(E)=\infty$, define $\mu_1(E)=\infty$ if $E$ contains measurable sets of arbitrarily large finite measure and $\mu_1(E)= 0$ otherwise.  Show $\mu_1$ is a semifinite measure: it is called the semifinite part of $\mu$.
\item Find a measure $\mu_2$ on $\mathcal{M}$ that only takes the valures $0$ and $\infty$ and $\mu = \mu_1 + \mu_2$.
\end{enumerate} 


\bigskip
\textbf{Collaborators:} None
\smallskip
 
\textbf{Solution:}
\textbf{(a)} If $\mu$ is a $\sigma-$finite measure, then $X = \bigcup\limits_{n=1}^{\infty} X_i$, where $X_i$ has finite measure. Assume $E$ is a measurable set with infinite measure, then since $E\subset X$, $E = E\cap X = \bigcup\limits_{n=1}^{\infty} E\cap X_n$. For $\forall M > 0$, there must exist $N > 0$, s.t. $\mu\left(\bigcup\limits_{n=1}^{N}E\cap X_i\right) > M$, otherwise $\mu(E) \le \sum\limits_{n=1}^{\infty}\mu(E\cap X_n) < M_0$ for some $M_0 > 0$, which makes a contradiction with $\mu(E) = \infty$. Besides, we know from $\mu(X_i) < \infty$, that $\mu\left(\sum\limits_{n=1}^{N}(E\cap X_n)\right) < \infty$. Thus $\mu$ is semifinite.

\textbf{(b)} Suppose $E$ is a measurable set of infinite measure in the measure space $(X, \mathcal{M}, \mu_1)$. Then according to the definite of $\mu_1$, $E$ contains measurable sets of arbitrarily large finite measure. Thus $\mu_{1}$ is semifinite.

\textbf{(c)} Define $\mu_2$ like this: If $E$ can be represented as countable union of measurable sets with finite measure, then $\mu_2(E) = 0$. Otherwise $\mu_2(E) = \infty$.

First, we show that $\mu = \mu_1 + \mu_2$. If $\mu(E) < \infty$, then $E = \bigcup\limits_{n=1}^{\infty}E_n$, in which $E_n = \varnothing$ for $n \ge 2$. Thus $\mu_2(E) = 0 = \mu(E) - \mu_1(E)$. Now we assume $\mu(E) = \infty$. If $E$ can be represented as countable union of measurable sets with finite measure, that is, $E = \bigcup\limits_{n=1}^{\infty}E_n$ with $\mu(E_n) < \infty$, then with the continuity of measure, $\forall M > 0$, $\exists N > 0$, s.t. $\mu\left(\bigcup\limits_{n=1}^{N}E_n\right) > M$. Since $\mathcal{M}$ is a $\sigma-$algebra, it shows that $E$ contains measurable sets of arbitrarily large finite measure. Thus $\mu(E) = \infty = \mu_1(E) + 0 = \mu_1(E) + \mu_2(E)$. If $E$ can't be represented as countable union of measurable sets with finite measure, then $\mu(E) = \infty = \mu_1(E) +\mu_2(E)$.
% \textbf{(c)} First, when $\mu(E) < \infty$, $\mu_2(E) = \mu(E)-\mu_1(E) = 0$, and when $\mu(E) = \infty$ and $E$ does not contain measurable sets of arbitrarily large finite measure, $\mu_2(E) = \mu(E) - \mu_1(E) = \infty$. When $E$ contains measurable sets of arbitrarily large finite measure, define $\mu_2(E) = 0$. Hence $\mu = \mu_1 + \mu_2$.
% Now we consider the case when $\mu(E) = \infty$ and $E$ contains measurable sets with arbitrarily large finite measure.

% For $n = 1, 2, \cdots$, there is a measurable set $E_n \subset E$, s.t. $n \le \mu(E_n) < \infty$. Then with the Proposition of Continuity of Measure, 
% $$
% \mu_2\left(\bigcup_{n=1}^{\infty}E_n\right) = \lim_{n\to\infty}\mu_2(E_n) = \infty.
% $$
% However, since each $E_n$ is a subset of $E$, we know $\bigcup\limits_{n=1}^{\infty}E_n $ is a subset of $E$. Then use the Monotocity property of measure, 
% $$
% \mu_2(E) \ge \mu_2\left(\bigcup_{n=1}^{\infty}E_n\right) = \infty.
% $$
% Hence, when $\mu(E) = \infty$, $\mu_2(E) = \infty$. 
Now we show that $\mu_2$ is a measure. In fact, we only need to show that $\mu_2(\varnothing) = 0$, and $\mu_2$ is countably additive. First, from the definition we know $\mu(\varnothing) = 0 < \infty$, so $\mu_2(\varnothing) = 0$. Suppose $\{E_n\}_{n=1}^{\infty}$ is a sequence of disjoint measurable sets. If $\exists i$, s.t. $\mu_2(E_i) = \infty$. Then $E_i$ can't be represented as countable union of measurable sets with finite measure, thus $\bigcup\limits_{n=1}^{\infty}E_n $ can't be represented as countable union of measurable sets with finite measure. Then
$$
\mu_2\left(\bigcup_{n=1}^{\infty}E_n\right) = \infty = \sum_{n=1}^{\infty}\mu_2(E_n).
$$
Now we suppose all $\mu_2(E_n) < \infty$. In this case, since $\mu_2$ takes value only in $\{0, \infty\}$, we have $\mu_2(E_n) = 0$. In this case, all $E_n$ is a countable union of measurable sets with finite measure, then $\bigcup\limits_{n=1}^{\infty}E_n $ is a countable union of measurable sets with finite measure. Thus
$$
\mu_2\left(\bigcup_{n=1}^{\infty}E_n\right) = 0 = \sum_{n=1}^{\infty}\mu_2(E_n).
$$
\bigskip


\item (Problem 9, Page 343) Prove Proposition 3, that is, show that $\mathcal{M}_0$ is a $\sigma$-algebra, $\mu_0$ is properly defined and $(X,\mathcal{M}_0, \mu_0)$ is complete.  In what sense is $\mathcal{M}_0$ minimal?


\bigskip
\textbf{Collaborators:} None
\smallskip
 
\textbf{Solution:}
\textbf{(1)} We show $\mathcal{M}_0$ is a $\sigma-$algebra. \textbf{(1.1)} First, since $X\in \mathcal{M}, \emptyset\in\mathcal{M}$, and $X = X\cup\emptyset$, and $\mu(\emptyset) = 0$, we know $X\in\mathcal{M}_0$. \textbf{(1.2)} Suppose $E\in \mathcal{M}_0$, then $E = A\cup B$ where $B\in \mathcal{M}$ and $A\subset C$ for some $C\in\mathcal{M}$, and $\mu(C) = 0$. Then $E^c = A^c\cap B^c = (B^c\cap C^c)\cup(C\setminus A)$. Since $\mathcal{M}$ is a $\sigma-$algebra, we know $B^c, C^c, B^c\cup C^c \in \mathcal{M}$, and $C\setminus A \in \mathcal{M}$, and $(C\setminus A)\subset C$, with $\mu(C) = 0$. Hence $E^c\in\mathcal{M}_0 $. \textbf{(1.3)} Suppose $\{E_n\}$ is a sequence of sets in $\mathcal{M}_0$. We may assume $E_n = A_n\cup B_n$, where $B_n\in\mathcal{M}$, and $A_n\subset C_n$ for some $C_n\in\mathcal{M}$ with $\mu(C) = 0$. Then 
$$
\bigcup\limits_{n=1}^{\infty}E_n = \left(\bigcup\limits_{n=1}^{\infty}A_n\right) \bigcup \left(\bigcup\limits_{n=1}^{\infty}B_n \right),
$$
and $\bigcup\limits_{n=1}^{\infty}B_n\in\mathcal{M} $. With the countable addivity of measure, $\bigcup\limits_{n=1}^{\infty}A_n\subset \bigcup\limits_{n=1}^{\infty}C_n\in\mathcal{M} $, and $\mu\left(\bigcup\limits_{n=1}^{\infty}C_n\right) = \sum\limits_{n=1}^{\infty}\mu(C_n) = 0$. Thus $\bigcup\limits_{n=1}^{\infty}E_n\in\mathcal{M} $.

\textbf{(2)} We show $\mu_0$ is properly defined, which means $\mu_0$ satisfies the properties of measure. \textbf{(2.1)} First, for all $E\in\mathcal{M}_0$, suppose $E = A\cup B$ where $A\in\mathcal{M}$ and $B\subset C\in\mathcal{M}$ with $\mu(C) = 0$. Then $\mu_0(E) = \mu(A) \ge 0$. \textbf{(2.2)} Since $\emptyset\in\mathcal{M}$, and $\emptyset = \emptyset \cap \emptyset$ with $\mu(\emptyset) = 0$, we know $\mu_0(\emptyset) = \mu(\emptyset) = 0$. \textbf{(2.3)} Suppose $\{E_n\}$ is a countable collection of disjoint sets in $\mathcal{M}_0$, denote $E_n = A_n\cup B_n$, and $A_n\in\mathcal{M}$, $B_n\subset C_n\in\mathcal{M}$, with $\mu(C_n) = 0$. Then the same with \textbf{(1.3)}, we know
$$
\bigcup\limits_{n=1}^{\infty}E_n = \left(\bigcup\limits_{n=1}^{\infty}A_n\right) \bigcup \left(\bigcup\limits_{n=1}^{\infty}B_n \right),
$$
with $\bigcup\limits_{n=1}^{\infty}B_n\subset \bigcup\limits_{n=1}^{\infty}C_n\in\mathcal{M} $, and $\mu\left(\bigcup\limits_{n=1}^{\infty}C_n\right) = \sum\limits_{n=1}^{\infty}\mu(C_n) = 0$. Then
$$
\mu_0\left(\bigcup_{n=1}^{\infty}E_n\right) = \mu\left(\bigcup_{n=1}^{\infty}B_n\right) = \sum_{n=1}^{\infty}\mu(B_n) = \sum_{n=1}^{\infty}\mu_0(E_n).
$$

\textbf{(3)} We show $(X, \mathcal{M}_0, \mu_0)$ is complete. If $E\in\mathcal{M}_0$ and $\mu_0(E) = 0$, then $E = A\cup B$ where $A\in\mathcal{M}$ and $B\subset C\in \mathcal{M}$ and $\mu(C) = 0$, and $\mu(A) = \mu_0(E) = 0$. Then each subset $F\subset E$ has a form $E = \emptyset \cup F$, and $F\subset (A\cup C)$ where $\mu_0(A\cup C) = \mu(A)+\mu(C) = 0$, and $A\cup C, \emptyset \in \mathcal{M}$. Thus $F\in\mathcal{M}_0$. 


\bigskip


\item (Problem 10, Page 343) If $(X,\mathcal{M},\mu)$ is a measure space, we say that a subset $E$ of $X$ is \textbf{locally measurable} provided for each $B\in \mathcal{M}$ with $\mu(B)<\infty$, the intersection $E\cap B$ belongs to $\mathcal{M}$.  The measure $\mu$ is called \textbf{saturated} provided every locally measurable set is measurable.
\begin{enumerate}
\item Show that each $\sigma$-finite measure is saturated.
\item Show that the collection $\mathcal{C}$ of locally measurable sets is a $\sigma$-algebra.
\item Let $(X,\mathcal{M},\mu)$ be a measure space and $\mathcal{C}$ the $\sigma$ of locally measurable sets. For $E\in \mathcal{C}$, define $\overline{\mu}(E) = \mu(E)$ if $E\in \mathcal{M}$ and $\overline{\mu}(E) = \infty$ if $E\notin \mathcal{M}$.  Show that $(X,\mathcal{C},\overline{\mu})$ is a saturated measure space.
\item If $\mu$ is semifinite and $E\in \mathcal{C}$, set $\underline{\mu}(E)=\sup\{\mu(B)\mid B\in\mathcal{M},\, B\subseteq E \}$. Show that  $(X,\mathcal{C},\underline{\mu})$ is a saturated measure space and that $\underline{\mu}$ is an extension of $\mu$.  Give an example to show that $\overline{\mu}$ and $\underline{\mu}$ may be different.
\end{enumerate}


\bigskip
\textbf{Collaborators:} None
\smallskip
 
\textbf{Solution:}
\textbf{(a)} Suppose $\mu$ is a $\sigma-$finite measure. Then $X = \bigcup\limits_{n=1}^{\infty}X_n $ with $X_n$ measurable and $\mu(X_n) < \infty$. Suppose $E$ is a locally measurable set, then for all $X_n$, $E\cap X_n \in\mathcal{M}$. Thus $E = \bigcup\limits_{n=1}^{\infty}(E\cap X_n) \in\mathcal{M}$.

\textbf{(b)} \textbf{(b.1)} First, for all $B\in\mathcal{M}$, $\mu(B) < \infty$, $X\cap B = B\in\mathcal{M}$. Hence $X\in \mathcal{C}$. 
\textbf{(b.2)} Suppose $E\in\mathcal{C}$, then for each $B\in\mathcal{M}$ with $\mu(B) < \infty$, $E\cap B \in\mathcal{M}$. Then $E^c\cap B = B\setminus (E\cap B) \in\mathcal{M}$.
\textbf{(b.3)} Suppose $\{E_n\}\in \mathcal{C}$ is a sequence of local measurable sets. Then for each $B\in\mathcal{M}$ and $\mu(B) < \infty$, for all $n$, $E_n\cap B\in\mathcal{M} $. Then fix $B$, we get
$$
\left(\bigcup_{n=1}^{\infty}E_n\right)\cap B = \bigcup_{n=1}^{\infty}(E_n\cap B)\in\mathcal{M}.
$$
Hence $\mathcal{C}$ is a $\sigma-$algebra.

\textbf{(c)} \textbf{(c.1)} First we show $\overline\mu$ is a measure. First, $\overline\mu(E) \ge \mu(E)\ge 0$. As $\emptyset\in\mathcal{M}$, $\overline\mu(\emptyset) = \mu(\emptyset) = 0$. Suppose $\{E_n\}\in\mathcal{C}$ is a sequence of disjoint sets, then if $\exists i$, s.t. $E_i\notin\mathcal{M}$, then $\bigcup\limits_{n=1}^{\infty}E_n \notin\mathcal{M}$. Thus
$$
\overline\mu\left(\bigcup\limits_{n=1}^{\infty}E_n\right) = \infty = \sum_{n=1}^{\infty}\overline\mu(E_n).
$$
Now suppose $E_n\in\mathcal{M}$. Thus the countably additive property of $\overline\mu$ is the same as $\mu$.
\textbf{(c.2)} Now suppose $B\in\mathcal{C}$ and $\overline\mu(B)< \infty$, then $B\in\mathcal{M}$, otherwise $\overline\mu(B) = \infty$. Suppose $E$ is a local measurable set, then $E\cap B\in\mathcal{C}$, thus according to the definition of $\mathcal{C}$, we know $E\in\mathcal{C}$. Thus $(X, \mathcal{C}, \overline\mu)$ is a saturated measure space.

\textbf{(d)} \textbf{(d.1)} First we show $\underline{\mu}$ is a measure. $\underline{\mu}(E) \ge\mu(E)\ge 0$, $\underline{\mu}(\emptyset) = \sup\{\mu(B)\mid B\in\mathcal{M}, B\subset\emptyset\} = 0$. Suppose $\{E_n\}\in\mathcal{C}$ is a sequence of disjoint sets, then if all $E_n\in\mathcal{M}$, we have $\underline{\mu}(E_n) = \mu(E_n)$. Using the countable addivity of $\mu$ we can get the countable addivity of $\underline{\mu}$. Now suppose $E_n\notin\mathcal{M}$. Then using the countable addivity of $\mu$,
$$
\underline{\mu}\left(\bigcup_{n=1}^{\infty}E_n \right) = \sup\left\{\mu(B)\mid B\in\mathcal{M}, B\subset\bigcup_{n=1}^{\infty}E_n\right\} = \sum_{n=1}^{\infty} \sup\{\mu(B)\mid B\in\mathcal{M}, B\subset E_n\} = \sum_{n=1}^{\infty}\underline{\mu}(E_n).
$$
\textbf{(d.2)} In \textbf{(d.1)} we have shown $\underline{\mu}$ is an extension of $\mu$. Now we show $(X, \mathcal{C}, \underline{\mu})$ is a saturated measure space. Suppose $E$ is a locally measurable set, then for each $B\in\mathcal{C}$, $\underline\mu(B) < \infty$, $E\cap B\in\mathcal{C}$. For all $F\in\mathcal{F}$ with $\mu(F) < \infty$, we have $\underline{\mu}(F) = \mu(F) < \infty$. Thus $E\cap F\in\mathcal{C}$, and it means $E$ is measurable in $(X, \mathcal{C}, \underline{\mu})$. Hence, $(X, \mathcal{C}, \underline{\mu})$ is a saturated measure space.

% If $B\in\mathcal{M}$, and if $\mu(B) = \infty$,  since $\mu$ is semifinite, $\forall M > 0, \exists B_M\subset B\in\mathcal{M} $, s.t. $\mu(B_M) > M$. Then $\underline{\mu}(B) = \infty$, which makes a contradiction. Thus $\mu(B) < \infty$.
% then if $\exists n$, $\mu(E_n) = \infty$, then $\underline\mu(E_n) = \sup\{\mu(B)\mid B\in\mathcal{M}, B\subset E_n\}$. Since $\mu$ is semifinite, $\forall M > 0, \exists B\in\mathcal{M}$, s.t. $\mu(B_M) > M$. Thus $\underline{\mu}(E_n) = \infty$, and $B_{M}\subset\bigcup\limits_{n=1}^{\infty}E_n$. We have
% $$
% \underline{\mu}\left(\bigcup_{n=1}^{\infty}E_n \right) = \infty = \sum_{n=1}^{\infty}\underline{\mu}(E_n).
% $$
% If $\mu(E_n) < \infty$, 
\bigskip



\item (Problem 18, Page 373)  Let $\{u_n\}$ be a sequence of nonnegative measurable functions on $X$.  For $x\in X$, define $f(x) = \sum_{n=1}^{\infty} u_n(x)$.  Show that \[
\int_X f \, d\mu = \sum_{n=1}^{\infty} \left[\int_X u_n \, d\mu \right].\]




\end{enumerate}
\end{document}
