
\documentclass{article}%
\usepackage{amsmath}
\usepackage{graphicx}
\usepackage{amsfonts}%
\usepackage{amssymb}


\setlength{\topmargin}{-0.75in}
\setlength{\textheight}{9.25in}
\setlength{\oddsidemargin}{0.0in}
\setlength{\evensidemargin}{0.0in}
\setlength{\textwidth}{6.5in}
\def\labelenumi{\arabic{enumi}.}
\def\theenumi{\arabic{enumi}}
\def\labelenumii{(\alph{enumii})}
\def\theenumii{\alph{enumii}}
\def\p@enumii{\theenumi.}
\def\labelenumiii{\arabic{enumiii}.}
\def\theenumiii{\arabic{enumiii}}
\def\p@enumiii{(\theenumi)(\theenumii)}
\def\labelenumiv{\arabic{enumiv}.}
\def\theenumiv{\arabic{enumiv}}
\def\p@enumiv{\p@enumiii.\theenumiii}
\pagestyle{plain}
\setcounter{secnumdepth}{0}
\newtheorem{theorem}{Theorem}
\newtheorem{acknowledgement}[theorem]{Acknowledgement}
\newtheorem{algorithm}[theorem]{Algorithm}
\newtheorem{axiom}[theorem]{Axiom}
\newtheorem{case}[theorem]{Case}
\newtheorem{claim}[theorem]{Claim}
\newtheorem{conclusion}[theorem]{Conclusion}
\newtheorem{condition}[theorem]{Condition}
\newtheorem{conjecture}[theorem]{Conjecture}
\newtheorem{corollary}[theorem]{Corollary}
\newtheorem{criterion}[theorem]{Criterion}
\newtheorem{definition}[theorem]{Definition}
\newtheorem{example}[theorem]{Example}
\newtheorem{exercise}[theorem]{Exercise}
\newtheorem{lemma}[theorem]{Lemma}
\newtheorem{notation}[theorem]{Notation}
\newtheorem{problem}[theorem]{Problem}
\newtheorem{proposition}[theorem]{Proposition}
\newtheorem{remark}[theorem]{Remark}
\newtheorem{solution}[theorem]{Solution}
\newtheorem{summary}[theorem]{Summary}
\newenvironment{proof}[1][Proof]{\textbf{#1.} }{\ \rule{0.5em}{0.5em}}

\begin{document}

\begin{center}
\textbf{Introduction to Analysis $I$\\Homework 3\\Monday, September 11, 2017}\bigskip
\end{center}

\noindent\textbf{Instructions}: This and all subsequent homeworks must be submitted written in \LaTeX.

\noindent If you use results from books, Royden or others, please be explicit about what results you are using.



\begin{center}
\emph{Homework 3 is due by midnight, Friday, September 22.}
\end{center} 
\medskip

\begin{enumerate}
\item  (Problem 34, Page 53) Show that there is a continuous, strictly increasing function on the interval $[0, 1]$ that maps a
set of positive measure onto a set of measure zero.


\bigskip
\textbf{Collaborators:}\\
\smallskip
 
\textbf{Solution:}
Let $C$ be the Cantor set on $[0, 1]$, and $\varphi(x)$ be the Cantor function. Define
$$
\phi(x) = \varphi(x)+x, ~x\in C.
$$
We now show $\phi^{-1}(x)$, the inverse of $\phi(x)$, satisfies the properties in the problem.
First, since $\phi(x)$ is a continuous, strictly increasing function on $[0, 1]$, thus is continuous and strictly increasing on $C\subset [0, 1]$.

Denote $D = \phi(C)$, then according to theorems in the book, $m(D) = 1$. $\forall x_1 < x_2\in D$, if $\phi^{-1}(x_1) \ge \phi^{-1}(x_2)$, then since $\phi$ is strictly increasing, $\phi(\phi^{-1}(x_1)) = x_1+\phi^{-1}(x_1) \ge \phi(\phi^{-1}(x_2)) = x_2+\phi^{-1}(x_2)$, which means $\phi^{-1}(x_1) < \phi^{-1}(x_2)$, leading to a contradictory. Thus $\phi^{-1}$ is strictly increasing.

On the other hand, $\forall x_0 \in D, \forall \epsilon > 0$, we need to show that $\exists\delta > 0, ~\forall x_1\in D, ~|x_1-x_{0}| < \delta$, then $|\phi^{-1}(x_1)-\phi^{-1}(x)| < \epsilon$. Denote $y_1 = \phi(\min\{(\phi^{-1}(x_0)-\epsilon, ~\phi^{-1}(x_0)+\epsilon)\cap C\}), ~y_2 = \phi(\max\{(\phi^{-1}(x_0)-\epsilon, ~\phi^{-1}(x_0)+\epsilon)\cap C\})$, then let $\delta = \min{y_1-x_0, y_2-x_0}$, we have $\forall |x-x_0|< \delta$, $|\phi^{-1}(x)-\phi^{-1}(x_0)| < \epsilon$. Thus $\phi^{-1}$ is continuous on $D$.

Since $C$ is measure zero, we get a function satisfying the properties in the problem.
\bigskip

\item (Problem 37, Page 53) Let $f$ be a continuous function defined on $E$. Is it true that $f^{-1}(A)$ is always measurable if $A$ is measurable?


\bigskip
\textbf{Collaborators:}\\
\smallskip
 
\textbf{Solution:} It is not true. 

We may consider the function $\psi(x)$ defined by Proposition 21 on Page 52. It is a strictly increasing continuous function, and it maps a measurable set $A\subset C$, onto a nonmeasurable set. Thus if we consider $\psi^{-1}(x)$, it has been proved as a strictly increasing continuous function in Problem 1. Then $\psi^{-1}(A)$ is nonmeasurable.

\bigskip



\item  (Problem 39, Page 53) Let $F$ be the subset of $[0, 1]$ constructed in the same manner as the Cantor set except that
each of the intervals removed at the nth deletion stage has length $\alpha 3^{-n}$ with $0 < a < 1$. Show
that $F$ is a closed set, $[0, 1]\sim F$ dense in $[0,1]$, and $m(F) = 1- a$. Such a set F is called a \emph{generalized Cantor set}.



\bigskip
\textbf{Collaborators:}\\
\smallskip
 
\textbf{Solution:}
First, we denote $F_1 = \left[0, \frac{3-\alpha}{6}\right]\cup \left[1-\frac{3-\alpha}{6}, 1\right]$. With the same process of constructing the Cantor set, we have a collection of $F_n$. We define the generalized Cantor set as 
$$
F = \bigcap_{k=1}^{\infty}F_n.
$$
Since each $F_n$ is a closed set, we have $F$ as well closed. Then $F_n$ is the disjoint of union of $2^n$ intervals, each of length $(1-\alpha)\frac{1}{2^n}+\frac{\alpha}{3^n}$. By the finite additivity of Lebesgue measure, 
$$
m(F_k) = 1-\alpha + \left(\frac{2}{3}\right)^n\alpha.
$$
According to the continuity of measure, we have $m(F) = \lim_{k\to\infty}m(F_k) = 1-\alpha$.

Let $x < y\in [0, 1]$. If $y\notin F_k$ for one of the $k$, then since $[0, 1]\setminus F$ is open, there exists $t < y$ in $N(y)$, s.t. $t\in [0, 1]\setminus F$. If $x\notin F_k$, it is the same case. Now we assume that $x, ~y \in F$. If $x, ~y$ are not in the same subset of $F_k$, then $\exists t\in [0, 1]\setminus F$. Then we can assume that $x, ~y$ always belong to the same subset. However, the subsets make a nested set with the limit of lower bound and the upper bound be the same. According to Nested Set Theorem, we have $x = y$. Then we have proved that $[0, 1]\setminus F$ is dense.
\bigskip

\item Let $C$ be the Cantor set and let $\varphi$ be the Cantor-Lebesgue function.
\begin{enumerate}
\item Show that $C$ consists of all $x\in [0,1]$ whose  ternary expansion has coefficients equal to $0$ or $2$, i.e., if $x = \sum_{k\geq 1} c_k3^{-k}$, where each $c_k = 0, 1, \text{or } 2$, then $x\in C$ if and only if $c_k = 0 \text{ or } 2$. 
\item Show that if $x\in C$ and $x = \sum_{k\geq 1} c_k3^{-k}$, where each $c_k = 0 \text{ or } 2$, then $\varphi(x) = \sum_{k\geq 1} (\frac{1}{2} c_k) 2^{-k}$.
\end{enumerate}
\bigskip
\textbf{Collaborators:}\\
\smallskip
 
\textbf{Solution:}
\textbf{(a).} We show by induction that a number belongs to the intervals we removed in each step iff its ternary expasion has a coefficient 1. First, the interval removed at the first step can be represented as $(0.1, ~0.2)$. Thus $0.1 \notin F$, and $0.0, 0.2\in F$. Assume it holds in the first $n$ steps, then the intervals removed at the $n+1$ step has the representation $(0.a_1a_2\cdots a_n1, ~0.a_1a_2\cdots a_n2)$, in which $a_i\in\{0, 2\}$. Then each number in this interval has an expansion like
$$
0.a_1a_2\cdots a_n1a_{n+1}\cdots.
$$
Thus each number in $F$ has a coefficient 1 in its expansion.
According to the construction process, we can also know that the reverse also holds. Thus the proposition holds.

\textbf{(b)}. 
According to (a), if we count the numbers starting by $0$, then $x$ is in the $c_{k}^{th}$ set in $k^{th}$ step. Thus according to the construction of Cantor function, we have $\varphi(x) = \sum_{k}(\frac{1}{2}c_k)2^{-k}$.
\bigskip

\item  Construct a Cantor-type subset of $[0,1]$ by removing from each interval remaining at the $k^{\text{th}}$ stage, a subinterval of relative length $\theta_k$, $0 < \theta_k < 1$. Show that the remainder has measure zero if and only if $\sum_{k\geq 1} \theta_k = \infty$.  (Use the fact that for $a_k > 0$, the product $\Pi_{k = 1}^{\infty} a_k$ converges, in the sense that $\lim_{n\to \infty} \Pi_{k=1}^N a_k$ exists and  is not zero, if and only if $\sum_{k =1}^{\infty} \ln a_k$ converges.)

\bigskip
\textbf{Collaborators:}\\
\smallskip
 
\textbf{Solution:}
Denote $F_n$ as the remained set after $n$ steps. Then according to the construction, we have 
$$
m(F_n) = \prod_{i=1}^{n}(1-\theta_i).
$$
First we have $\sum_{n=1}^{\infty}-\theta_n$ and $\sum_{n=1}^{\infty}\ln(1-\theta_n)$ be both negative-term series, and the necessity of their convergence is $\lim_{n\to\infty}\theta_n = 0$. On the other hand, when $\lim_{n\to\infty} \theta_n = 0$, 
$$
\lim_{n\to\infty}\frac{\ln(1-\theta_n)}{-\theta_n} = 1,
$$
thus that $\sum{\ln(1-\theta_n)}$ converges is equivalent to the convergence of $\sum{-\theta_n}$.

Then according to the continuity of measure,
$$
\begin{aligned}
m(F) > 0 \Leftrightarrow m(\lim_{n\to\infty}F_n) > 0 \Leftrightarrow \prod_{i=1}^{\infty}(1-\theta_i) ~\text{converges} \Leftrightarrow \sum_{i=1}^{\infty} \ln(1-\theta_i) ~\text{converges} \\
\Leftrightarrow \sum_{i=1}^{\infty}\theta_i ~\text{converges}.
\end{aligned}
$$
Thus $m(F) = 0$ iff $\sum{\theta_i} = \infty$.

\bigskip





\item Let $Z$ be a set of measure zero in $\mathbb{R}$.  What is the measure of $\{x^2 \mid x\in Z\}$?



\bigskip
\textbf{Collaborators:}\\
\smallskip
 
\textbf{Solution:}
$X = \{x^2\mid x\in Z\}$ is also measure 0. 

First, if the set $Z$ is bounded, which means $Z\in [0, M]$. Since $m(Z) = 0$, $\forall \epsilon > 0$, there exists an open cover $\{O_i\} = \{(a_i, b_i)\}$, s.t. $m^*(\cup\{O_i\}\setminus Z) < \epsilon$. Thus $\{(a_i^2, b_i^2)\}$ is an open cover of $X$, and $m^*(\cup(a_i^2, b_i^2)\setminus X) \le M\sum(b_i-a_i) < M\epsilon$. Thus now $X$ is a measure-zero set. When $Z \in [-M, 0]$, with the same process above, we can know that $X$ is a measure-zero set. Then when $Z \in [-M, M]$, we know $X$ is also measure zero.

Since $Z$ is measure zero, for $\forall \epsilon > 0$ and integer $n > 0$, we have $m(Z\cap [-n, n]) < 2^{-n}\epsilon$. Thus from the discussion above we have $m(X\cap [-n^2, n^2]) < 2^{-n}\epsilon$. Since 
$$
X = \bigcup_{n=1}^{\infty}X\cap[-n, n],
$$
with the addivity of measure, 
$$
m(X) < \sum_{n=1}^{\infty}2^{-n}\epsilon = \epsilon.
$$
With the arbitrariness of $\epsilon$, we have $m(X) = 0$.
\bigskip


\item  Let $0.\alpha_1 \alpha_2 \cdots$ be the dyadic development of any $x\in[0,1]$. Let $k_1,k_2,k_3,\ldots$ be a fixed permutation of the positive integers $1,2,\ldots$, and consider the transformation $T$ which sends $x = \alpha_1\alpha_2\alpha_3\cdots$ to $Tx: = \alpha_{k_1}\alpha_{k_2}\alpha_{k_3}\cdots$. Show that if $E$ is a measurable subset of $[0,1]$ then its image under $T$, $T(E)$, is also measurable and that $m(T(E))= m(E)$.  That is, show that $T$ is a measure preserving transformation of $[0,1]$. [Consider first the special case where $E$ is a dyadic interval of the form $(s2^{-k}, (s+1)2^{-k})$ and $s = 0,1,\ldots, 2k-1$.  Then think about open sets and note that each open set can be written as a countable union of non-overlapping half-open dyadic intervals.]

\bigskip
\textbf{Collaborators:}\\
\smallskip
 
\textbf{Solution:}
Consider the intervals $E_k^i = [i\cdot2^-{k}, ~(i+1)2^{-k})$. Then
$$
T[E_k] = \bigcap_{i=1}^{2^k}F_{k_i},
$$
where 
$$
F_{k_i} = \left\{
\begin{aligned}
&[F_{k_{i-1}}.\text{left}, ~F_{k_{i-1}}.\text{mid}), ~\text{if}~ k_i = 0 \\
&[F_{k_{i-1}}.\text{mid}, ~F_{k_{i-1}}.\text{right}), ~\text{if}~ k_i = 1.
\end{aligned}
\right.
$$
Then since the finite intersection of closed sets is measurable, we know that $T[E_k]$ is measurable, and we have $m(T(E_k)) = m(E_k)$ since the probability of the numbers with infinite length be seperated into the intervals. Then since each open interval can be written as a countable union of non-overlapping half-open dyadic intervals, we know from the discussion above that $T(E)$ is measurable, and $m(T(E)) = m(E)$.

\bigskip


\end{enumerate}
\end{document}
