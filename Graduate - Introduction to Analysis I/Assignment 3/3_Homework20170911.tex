
\documentclass{article}%
\usepackage{amsmath}
\usepackage{graphicx}
\usepackage{amsfonts}%
\usepackage{amssymb}


\setlength{\topmargin}{-0.75in}
\setlength{\textheight}{9.25in}
\setlength{\oddsidemargin}{0.0in}
\setlength{\evensidemargin}{0.0in}
\setlength{\textwidth}{6.5in}
\def\labelenumi{\arabic{enumi}.}
\def\theenumi{\arabic{enumi}}
\def\labelenumii{(\alph{enumii})}
\def\theenumii{\alph{enumii}}
\def\p@enumii{\theenumi.}
\def\labelenumiii{\arabic{enumiii}.}
\def\theenumiii{\arabic{enumiii}}
\def\p@enumiii{(\theenumi)(\theenumii)}
\def\labelenumiv{\arabic{enumiv}.}
\def\theenumiv{\arabic{enumiv}}
\def\p@enumiv{\p@enumiii.\theenumiii}
\pagestyle{plain}
\setcounter{secnumdepth}{0}
\newtheorem{theorem}{Theorem}
\newtheorem{acknowledgement}[theorem]{Acknowledgement}
\newtheorem{algorithm}[theorem]{Algorithm}
\newtheorem{axiom}[theorem]{Axiom}
\newtheorem{case}[theorem]{Case}
\newtheorem{claim}[theorem]{Claim}
\newtheorem{conclusion}[theorem]{Conclusion}
\newtheorem{condition}[theorem]{Condition}
\newtheorem{conjecture}[theorem]{Conjecture}
\newtheorem{corollary}[theorem]{Corollary}
\newtheorem{criterion}[theorem]{Criterion}
\newtheorem{definition}[theorem]{Definition}
\newtheorem{example}[theorem]{Example}
\newtheorem{exercise}[theorem]{Exercise}
\newtheorem{lemma}[theorem]{Lemma}
\newtheorem{notation}[theorem]{Notation}
\newtheorem{problem}[theorem]{Problem}
\newtheorem{proposition}[theorem]{Proposition}
\newtheorem{remark}[theorem]{Remark}
\newtheorem{solution}[theorem]{Solution}
\newtheorem{summary}[theorem]{Summary}
\newenvironment{proof}[1][Proof]{\textbf{#1.} }{\ \rule{0.5em}{0.5em}}

\begin{document}

\begin{center}
\textbf{Introduction to Analysis $I$\\Homework 3\\Monday, September 11, 2017}\bigskip
\end{center}

\noindent\textbf{Instructions}: This and all subsequent homeworks must be submitted written in \LaTeX.

\noindent If you use results from books, Royden or others, please be explicit about what results you are using.



\begin{center}
\emph{Homework 3 is due by midnight, Friday, September 22.}
\end{center} 
\medskip

\begin{enumerate}
\item  (Problem 34, Page 53) Show that there is a continuous, strictly increasing function on the interval $[0, 1]$ that maps a
set of positive measure onto a set of measure zero.


\bigskip
\textbf{Collaborators:}\\
\smallskip
 
\textbf{Solution:}
Let $C$ be the Cantor set on $[0, 1]$, and $\varphi(x)$ be the Cantor function. Define
$$
\phi(x) = \varphi(x)+x, ~x\in C.
$$
We now show $\phi^{-1}(x)$, the inverse of $\phi(x)$, satisfies the properties in the problem.
First, since $\phi(x)$ is a continuous, strictly increasing function on $[0, 1]$, thus is continuous and strictly increasing on $C\subset [0, 1]$.

Denote $D = \phi(C)$, then according to theorems in the book, $m(D) = 1$. $\forall x_1 < x_2\in D$, if $\phi^{-1}(x_1) \ge \phi^{-1}(x_2)$, then since $\phi$ is strictly increasing, $\phi(\phi^{-1}(x_1)) = x_1+\phi^{-1}(x_1) \ge \phi(\phi^{-1}(x_2)) = x_2+\phi^{-1}(x_2)$, which means $\phi^{-1}(x_1) < \phi^{-1}(x_2)$, leading to a contradictory. Thus $\phi^{-1}$ is strictly increasing.

On the other hand, $\forall x_0 \in D, \forall \epsilon > 0$, since $\phi$ is continuous on $C$, then $\forall \delta > 0, ~\exists\epsilon_1 > 0, ~\forall y\in C, ~|y-\phi^{-1}(x_0)| < \epsilon_1, ~|\phi(y)-x_0| < \delta$. Denote $\epsilon_1 = \min(\epsilon_1, \epsilon)$, then $\forall y\in C, ~|y-\phi^{-1}(x_0)| < \epsilon_1, ~|\phi(y)-x_0| < \delta$. Thus according to properties of strictly increasing bijection, $\forall x\in D, ~|x-x_0| < \delta, ~|\phi^{-1}(x)-\phi^{-1}(x_0)| < \epsilon_1\le \epsilon$. It means that $\phi^{-1}$ is continuous on $D$.

Since $C$ is measure zero, we get a function satisfying the properties in the problem.
\bigskip

\item (Problem 37, Page 53) Let $f$ be a continuous function defined on $E$. Is it true that $f^{-1}(A)$ is always measurable if $A$ is measurable?


\bigskip
\textbf{Collaborators:}\\
\smallskip
 
\textbf{Solution:}

\bigskip



\item  (Problem 39, Page 53) Let $F$ be the subset of $[0, 1]$ constructed in the same manner as the Cantor set except that
each of the intervals removed at the nth deletion stage has length $\alpha 3^{-n}$ with $0 < a < 1$. Show
that $F$ is a closed set, $[0, 1]\sim F$ dense in $[0,1]$, and $m(F) = 1- a$. Such a set F is called a \emph{generalized Cantor set}.



\bigskip
\textbf{Collaborators:}\\
\smallskip
 
\textbf{Solution:}
\bigskip

\item Let $C$ be the Cantor set and let $\varphi$ be the Cantor-Lebesgue function.
\begin{enumerate}
\item Show that $C$ consists of all $x\in [0,1]$ whose  ternary expansion has coefficients equal to $0$ or $2$, i.e., if $x = \sum_{k\geq 1} c_k3^{-k}$, where each $c_k = 0, 1, \text{or } 2$, then $x\in C$ if and only if $c_k = 0 \text{ or } 2$. 
\item Show that if $x\in C$ and $x = \sum_{k\geq 1} c_k3^{-k}$, where each $c_k = 0 \text{ or } 2$, then $\varphi(x) = \sum_{k\geq 1} (\frac{1}{2} c_k) 2^{-k}$.
\end{enumerate}
\bigskip
\textbf{Collaborators:}\\
\smallskip
 
\textbf{Solution:}
\bigskip

\item  Construct a Cantor-type subset of $[0,1]$ by removing from each interval remaining at the $k^{\text{th}}$ stage, a subinterval of relative length $\theta_k$, $0 < \theta_k < 1$. Show that the remainder has measure zero if and only if $\sum_{k\geq 1} \theta_k = \infty$.  (Use the fact that for $a_k > 0$, the product $\Pi_{k = 1}^{\infty} a_k$ converges, in the sense that $\lim_{n\to \infty} \Pi_{k=1}^N a_k$ exists and  is not zero, if and only if $\sum_{k =1}^{\infty} \ln a_k$ converges.)

\bigskip
\textbf{Collaborators:}\\
\smallskip
 
\textbf{Solution:}
\bigskip





\item Let $Z$ be a set of measure zero in $\mathbb{R}$.  What is the measure of $\{x^2 \mid x\in Z\}$?



\bigskip
\textbf{Collaborators:}\\
\smallskip
 
\textbf{Solution:}
$X = \{x^2\mid x\in Z\}$ is also measure 0. 

On one hand, we can define a map 
$$
\begin{aligned}
\phi: X &\to Z, \\
x&\mapsto \sqrt{x}, \quad\text{if}~ \sqrt{x}\in Z. \\
x&\mapsto -\sqrt{x}, \quad\text{if}~ \sqrt{x}\notin Z ~\text{and}~ -\sqrt{x}\in Z.
\end{aligned}
$$
Then it is a bijection from $X$ to a subset of $Z$.

On the other hand, the map
$$
\begin{aligned}
\varphi: Z &\to X
\end{aligned}
$$
\bigskip


\item  Let $0.\alpha_1 \alpha_2 \cdots$ be the dyadic development of any $x\in[0,1]$. Let $k_1,k_2,k_3,\ldots$ be a fixed permutation of the positive integers $1,2,\ldots$, and consider the transformation $T$ which sends $x = \alpha_1\alpha_2\alpha_3\cdots$ to $Tx: = \alpha_{k_1}\alpha_{k_2}\alpha_{k_3}\cdots$. Show that if $E$ is a measurable subset of $[0,1]$ then its image under $T$, $T(E)$, is also measurable and that $m(T(E))= m(E)$.  That is, show that $T$ is a measure preserving transformation of $[0,1]$. [Consider first the special case where $E$ is a dyadic interval of the form $(s2^{-k}, (s+1)2^{-k})$ and $s = 0,1,\ldots, 2k-1$.  Then think about open sets and note that each open set can be written as a countable union of non-overlapping half-open dyadic intervals.]

\bigskip
\textbf{Collaborators:}\\
\smallskip
 
\textbf{Solution:}
\bigskip


\end{enumerate}
\end{document}
