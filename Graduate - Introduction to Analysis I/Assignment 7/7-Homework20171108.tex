
\documentclass{article}%
\usepackage{amsmath}
\usepackage{graphicx}
\usepackage{amsfonts}%
\usepackage{amssymb}
\usepackage{hyperref}

\setlength{\topmargin}{-0.75in}
\setlength{\textheight}{9.25in}
\setlength{\oddsidemargin}{0.0in}
\setlength{\evensidemargin}{0.0in}
\setlength{\textwidth}{6.5in}
\def\labelenumi{\arabic{enumi}.}
\def\theenumi{\arabic{enumi}}
\def\labelenumii{(\alph{enumii})}
\def\theenumii{\alph{enumii}}
\def\p@enumii{\theenumi.}
\def\labelenumiii{\arabic{enumiii}.}
\def\theenumiii{\arabic{enumiii}}
\def\p@enumiii{(\theenumi)(\theenumii)}
\def\labelenumiv{\arabic{enumiv}.}
\def\theenumiv{\arabic{enumiv}}
\def\p@enumiv{\p@enumiii.\theenumiii}
\pagestyle{plain}
\setcounter{secnumdepth}{0}
\newtheorem{theorem}{Theorem}
\newtheorem{acknowledgement}[theorem]{Acknowledgement}
\newtheorem{algorithm}[theorem]{Algorithm}
\newtheorem{axiom}[theorem]{Axiom}
\newtheorem{case}[theorem]{Case}
\newtheorem{claim}[theorem]{Claim}
\newtheorem{conclusion}[theorem]{Conclusion}
\newtheorem{condition}[theorem]{Condition}
\newtheorem{conjecture}[theorem]{Conjecture}
\newtheorem{corollary}[theorem]{Corollary}
\newtheorem{criterion}[theorem]{Criterion}
\newtheorem{definition}[theorem]{Definition}
\newtheorem{example}[theorem]{Example}
\newtheorem{exercise}[theorem]{Exercise}
\newtheorem{lemma}[theorem]{Lemma}
\newtheorem{notation}[theorem]{Notation}
\newtheorem{problem}[theorem]{Problem}
\newtheorem{proposition}[theorem]{Proposition}
\newtheorem{remark}[theorem]{Remark}
\newtheorem{solution}[theorem]{Solution}
\newtheorem{summary}[theorem]{Summary}
\newenvironment{proof}[1][Proof]{\textbf{#1.} }{\ \rule{0.5em}{0.5em}}

\begin{document}

\begin{center}
\textbf{Introduction to Analysis $I$\\Homework 7\\Wednesday, November  8, 
2017}\bigskip
\end{center}

\noindent\textbf{Instructions}: This and all subsequent homeworks must be submitted written in \LaTeX.

\noindent Since by now, you should all be sufficiently familiar with \LaTeX to 
compose on your own, I will forgo the formalities of the previous assignments 
and simply list the problems.  I will expect, however, that you will cite 
relevant material in Royden or other sources properly and that you will 
acknowledge your collaborators.


\begin{center}
\emph{Homework 7 is due by midnight, Saturday, November 18.}
\end{center} 
\medskip

\begin{enumerate}
\item  Problem 9,  Chapter 6 

\smallskip
\textbf{Solution:}
\smallskip



\item  Problem 10, Chapter 6
\item  Problem 13, Chapter 6

\smallskip
\textbf{Solution:}
\smallskip
First, we show that under the conditions of the Vitali Covering Lemma, $\mathcal{F}\setminus \bigcup\limits_{k=1}^n I_k $ is a Vitali covering of $E\setminus \bigcup\limits_{k=1}^n I_k$. In fact, if it is not true, then there $\exists x\in E\setminus\bigcup\limits_{k=1}^n I_k$ and $\epsilon > 0$, s.t. there does not exist an interval $I\in\mathcal{F}\setminus\bigcup\limits_{k=1}^n I_k$, s.t. $x\in I$, and $m(I) < \epsilon$. However, since $\mathcal{F}$ is a Vitali covering of $E$, then for $\epsilon_i = \frac{\epsilon}{2^i}, ~i = 1, 2, \cdots $, there exists $I_i'\in\mathcal{F} $, s.t. $x\in I_i' $ and $m(I_i') < \epsilon_i $. If there exists $i_0 $, s.t. $I_{i_0}\notin \{I_k\}_{k = 1}^{n} $, then $x\in I_{i_0} $ and $m(I_{i_0}) < \epsilon$, which makes a contradiction. Then $\{I_i'\} \in \{I_k\}_{k = 1}^{n}$, but it contradicts with $n < \infty$. Hence, our claim is proved.

Then we show that if $\{I_k\}_{k=1}^{n} $ is a finite sequence of closed intervals, then we can find a pairwise disjoint subsequence $\{I_{k_j}\}$ s.t. 
$$
m(\bigcup_{j = 1}^{m} I_{k_j}) \ge m(\bigcup_{k=1}^{n}I_k).
$$
The proof of this proposition can be seen at \href{http://www.personal.psu.edu/t20/papers/vitali-l2h/node5.html}{http://www.personal.psu.edu/t20/papers/vitali-l2h/node5.html}

Now we use these claims to construct a collection. Using Vitali Covering Lemma, for $\epsilon = \frac{3}{4}m(E)$, there exists a finite collection of disjoint intervals $\{I_k\}$, s.t. $m(E\setminus \bigcup I_k) < \epsilon$. Denote $A = \bigcup I_k $.

Since $\mathcal{F}\setminus A$ is a Vitali covering of $E\setminus I_k $, there is a finite set of intervals $J_k $, s.t. $J_k\subset E\setminus A $ (Otherwise just follow the proof of Vitali Covering Lemma on the textbook), and 
$$
m(E\setminus (A\cup\bigcup_{k=1}^m J_k)) < \frac{1}{12}m(E\setminus A).
$$
Then using claim 2, there is a pairwise disjoint subset $\{J_{k_i}\}$ s.t. 
$$
m(\bigcup J_{k_i}) \ge \frac{1}{3}m(\bigcup J_k).
$$
Then denote $B = \bigcup J_{k_i} $, and 
$$
m(E\setminus (A\cup B)) < \frac{2}{3}m(\bigcup J_k) +\frac{1}{12}m(E\setminus A) \le (\frac{2}{3}+\frac{1}{12})m(E\setminus A) = \frac{3}{4}m(E\setminus A).
$$
By constructing like above recursively, we can get a sequence of subsets $\{A_i\}$, s.t. 
$$
m(E\setminus \bigcup_{i=1}^n A_i) \le (\frac{3}{4})^n m(E).
$$
By countable addivity,
$$
m(E\setminus \bigcup_{i=1}^\infty A_i) = 0.
$$

\item  Problem 16, Chapter 6

\smallskip
\textbf{Solution:}
\smallskip
First, we show that if $f_1, ~f_2$ are differentiable functions, then $f = f_1+f_2$ is differentiable. In fact,
$$
\begin{aligned}
\overline{D}f(x) &= \lim_{h\to 0}\sup_{0 < |t| < h}\frac{f(x+t)-f(x)}{t} = \lim_{h\to 0}\sup_{0 < |t| < h}(\frac{f_1(x+t)-f_1(x)}{t}+\frac{f_2(x+t)-f_2(x)}{t}) \\
&\le \lim_{h\to 0}\sup_{0 < |t| < h}\frac{f_1(x+t)-f_1(x)}{t} + \lim_{h\to 0}\sup_{0 < |t| < h}\frac{f_2(x+t)-f_2(x)}{t}
\end{aligned}
$$
$$
\begin{aligned}
\underline{D}f(x) &= \lim_{h\to 0}\inf_{0 < |t| < h}\frac{f(x+t)-f(x)}{t} = \lim_{h\to 0}\inf_{0 < |t| < h}(\frac{f_1(x+t)-f_1(x)}{t}+\frac{f_2(x+t)-f_2(x)}{t}) \\
&\ge \lim_{h\to 0}\inf_{0 < |t| < h}\frac{f_1(x+t)-f_1(x)}{t} + \lim_{h\to 0}\inf_{0 < |t| < h}\frac{f_2(x+t)-f_2(x)}{t}
\end{aligned}
$$
But since $f_1$ and $f_2$ are both differentiable, 
$$
\lim_{h\to 0}\inf_{0 < |t| < h}\frac{f_i(x+t)-f_i(x)}{t} = \lim_{h\to 0}\sup_{0 < |t| < h}\frac{f_i(x+t)-f_i(x)}{t}
$$
holds for $i = 1, 2$. Thus $\overline{D}f(x) = \underline{D}f(x)$, which means $f$ is differentiable.

Let
$$
g^+ = \max\{g(x), ~0\}, ~g^- = \min\{g(x), ~0\},
$$
Then $g^+ \ge 0, ~g^- \le 0$, and $g(x) = g^+(x) + g^-(x)$. Define
$$
f_1(x) = \int_{a}^{x}g^+, ~f_2(x) = \int_{a}^{x}g^-,
$$
then $f_1, ~f_2$ are both monotone functions, using Lebesgue's Theorem, $f_1, ~f_2 $ are both differentiable a.e. on $(a, b)$, respectively differentiable on $(a, b)\setminus E_1 $ and $(a, b) \setminus E_2 $ where $m(E_1) = m(E_2) = 0$. Then using the conclusion above, $f = f_1 + f_2 $ is differentiable on $(a, b)\setminus (E_1\cup E_2)$, and hence differentiable a.e. on $(a, b)$.





\item  Problem 26, Chapter 6
\item  Problem 29, Chapter 6
\item  Problem 35, Chapter 6

\end{enumerate}
\end{document}
